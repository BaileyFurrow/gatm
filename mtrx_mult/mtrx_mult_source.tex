\documentclass[../gatm.tex]{subfiles}

\begin{document}

\section{Matrix Multiplication}

\newcommand{\indsize}{\scriptsize}
\newcommand{\colind}[2]{\displaystyle\smash{\mathop{#1}^{\raisebox{.5\normalbaselineskip}{$#2$}}}}
\newcommand{\rowind}[1]{\mbox{$#1$}}

You've all seen a bunch of numbers organized in a table. Sometimes a table is just a table, but sometimes we will call it a \textbf{matrix}.

What makes a matrix different from a table? Although they encapsulate the same information, we can meaningfully \textit{multiply} matrices. This lesson's purpose is to explain why the matrix multiplication rule makes sense and when it is useful.

Consider a region with four towns, creatively named $A,B,C,D$. There are modes of transport between these towns; a path can go from a town to any town, including the same town. These paths are shown in Figure~\ref{fig:four_town_scenario}. At each ``step,'' you can take any path from one town to the next. For example, you might start on $A$, then take either of the two paths to $D$. But you cannot start on $D$ and go directly to $B$. When there is more than one path between towns, we could also just draw a line and label it with a number: this is shown in Figure~\ref{fig:scenario_with_nums}. Note how each town has a ``path'' going to itself. Taking this means you don't go anywhere.

\begin{asydef}
pair A = (0,1);
pair C = (0,-1);
pair B = (2,0);
pair D = (-2,0);
\end{asydef}

\begin{figure}
\begin{minipage}{0.4\textwidth}
\begin{asy}[width=\textwidth]

label("$A$", A);
label("$B$", B);
label("$C$", C);
label("$D$", D);

// A<->B
draw((0.30000,0.85000)--(1.70000,0.15000), ArcArrows);

// A<->C
draw((0.10000,0.70000)--(0.10000,-0.70000), ArcArrows);
draw((-0.10000,0.70000)--(-0.10000,-0.70000), ArcArrows);

// A<->D
draw((-0.25528,0.76056)--(-1.65528,0.06056), ArcArrows);
draw((-0.34472,0.93944)--(-1.74472,0.23944), ArcArrows);

// C<->D
draw((-1.70000,-0.15000)--(-0.30000,-0.85000), ArcArrows);

// C<->B
draw((1.70000,-0.15000)--(0.30000,-0.85000), ArcArrows);

/*
def connect_self(p1, dx1=-1, dy1=1, dx2=0, dy2=1):
	return "draw((%5.5f, %5.5f)..(%5.5f, %5.5f)..(%5.5f, %5.5f),ArcArrow);" % (p1[0] + dx1 * 0.2, p1[1] + dy1 * 0.2, p1[0] + dx2 * 0.7, p1[1] + dy2 * 0.7, p1[0] + (1 if dy2 == 0 else -1) * dx1 * 0.2, p1[1] + (1 if dx2 == 0 else -1) + dy1 * 0.2)

connect_self((0,1), -1, 1, 0, 1);
connect_self((2,0), 1, 1, 1, 0);
connect_self((0,-1), 1, -1, 0, -1);
connect_self((-2,0), -1, -1, -1, 0);
*/

// A<->A, B<->B ...
draw((-0.2,1.2)..(0,1.7)..(0.2,1.2),ArcArrow);
draw((2.20000, 0.20000)..(2.70000, 0.00000)..(2.20000, -0.20000),ArcArrow);
draw((0.20000, -1.20000)..(0.00000, -1.70000)..(-0.20000, -1.20000),ArcArrow);
draw((-2.20000, -0.20000)..(-2.70000, 0.00000)..(-2.20000, 0.20000),ArcArrow);
\end{asy}

\caption{Four town transportation scenario.}
\label{fig:four_town_scenario}
\end{minipage}\hfill
\begin{minipage}{0.4\textwidth}
\begin{asy}[width=\textwidth]

label("$A$", A);
label("$B$", B);
label("$C$", C);
label("$D$", D);

// A<->B
draw((0.30000,0.85000)--(1.70000,0.15000), ArcArrows);

// A<->C
draw((0,0.70000)--(0,-0.70000), ArcArrows);
label("$2$", (0,0), E);

// A<->D
draw((-0.30000,0.85000)--(-1.70000,0.15000), ArcArrows);
label("$2$", ((-0.30000,0.85000)+(-1.70000,0.15000))/2, NW);

// C<->D
draw((-1.70000,-0.15000)--(-0.30000,-0.85000), ArcArrows);

// C<->B
draw((1.70000,-0.15000)--(0.30000,-0.85000), ArcArrows);

/*
def connect_self(p1, dx1=-1, dy1=1, dx2=0, dy2=1):
	return "draw((%5.5f, %5.5f)..(%5.5f, %5.5f)..(%5.5f, %5.5f),ArcArrow);" % (p1[0] + dx1 * 0.2, p1[1] + dy1 * 0.2, p1[0] + dx2 * 0.7, p1[1] + dy2 * 0.7, p1[0] + (1 if dy2 == 0 else -1) * dx1 * 0.2, p1[1] + (1 if dx2 == 0 else -1) + dy1 * 0.2)

connect_self((0,1), -1, 1, 0, 1);
connect_self((2,0), 1, 1, 1, 0);
connect_self((0,-1), 1, -1, 0, -1);
connect_self((-2,0), -1, -1, -1, 0);
*/

// A<->A, B<->B ...
draw((-0.2,1.2)..(0,1.7)..(0.2,1.2),ArcArrow);
draw((2.20000, 0.20000)..(2.70000, 0.00000)..(2.20000, -0.20000),ArcArrow);
draw((0.20000, -1.20000)..(0.00000, -1.70000)..(-0.20000, -1.20000),ArcArrow);
draw((-2.20000, -0.20000)..(-2.70000, 0.00000)..(-2.20000, 0.20000),ArcArrow);
\end{asy}

\caption{The scenario, with numbers instead of duplicate lines.}
\label{fig:scenario_with_nums}
\end{minipage}
\end{figure}

Let us consider the \textbf{transportation matrix}, also known as an \textbf{adjacency matrix}, in this scenario. The number in the $i$\textsuperscript{th} row and $j$\textsuperscript{th} column of the matrix $A$, which we'll call $a_{ij}$, gives the number of ways to walk directly from town $i$ to town $j$. This last fact is reflected in the matrix: $a_{42}=a_{24}=0$.
% huge thanks to https://tex.stackexchange.com/questions/331691/how-can-i-get-a-matrix-with-row-and-column-labels-that-can-also-be-aligned-with
%print(' \\\\\n'.join(" & ".join("a_{%s%s}" % (i,j) for j in range(1, 5)) for i in range(1,5)) + " \\\\" )

$$
  A = \left[\begin{array}{cccc}
  a_{11} & a_{12} & a_{13} & a_{14} \\
a_{21} & a_{22} & a_{23} & a_{24} \\
a_{31} & a_{32} & a_{33} & a_{34} \\
a_{41} & a_{42} & a_{43} & a_{44} \\
  \end{array}\right] =
  \rotatebox[origin=c]{90}{from}\;\,
  \begin{array}{@{}c@{}}
    \rowind{A} \\ \rowind{B} \\ \rowind{C} \\ \rowind{D}
  \end{array}
  \mathop{\left[
  \begin{array}{cccc}
     \colind{1}{A}  &  \colind{1}{B}  &  \colind{2}{C}  & \colind{2}{D} \\
1 & 1 & 1 & 0 \\
2 & 1 & 1 & 1 \\
2 & 0 & 1 & 1 \\
  \end{array}
  \right]}^{
  \begin{array}{@{}c@{}}
    \rowind{\text{to}} \\ \mathstrut
  \end{array}
  }
$$

This matrix is \textbf{symmetric}, meaning there are no one-way paths. In mathematical terms, we have $a_{ij}=a_{ji}$ for all valid $i,j$. In visual terms, the matrix is symmetric about the \textbf{main diagonal}, going from top left to bottom right. Furthermore, the main diagonal is all $1$s, because we allow staying in the town you start in.

Suppose there's a shuttle bus that only goes one way, from town $A$ to $B$ to $C$ to $D$ and then back to $A$ again. The transportation matrix for this scenario---again, allowing staying still---is shown in Figure~\ref{fig:adjacency_b}.

\begin{figure}[h]
\begin{minipage}{0.4\textwidth}
\begin{center}
$$B=\rotatebox[origin=c]{90}{from}\;\,
  \begin{array}{@{}c@{}}
    \rowind{A} \\ \rowind{B} \\ \rowind{C} \\ \rowind{D}
  \end{array}
  \mathop{\left[
  \begin{array}{cccc}
     \colind{1}{A}  &  \colind{1}{B}  &  \colind{0}{C}  & \colind{0}{D} \\
0 & 1 & 1 & 0 \\
0 & 0 & 1 & 1 \\
1 & 0 & 0 & 1 \\
  \end{array}
  \right]}^{
  \begin{array}{@{}c@{}}
    \rowind{\text{to}} \\ \mathstrut
  \end{array}
  }$$
\end{center}
\end{minipage}\hfill
\begin{minipage}{0.4\textwidth}
\begin{center}
\begin{asy}[width=\textwidth]

label("$A$", A);
label("$B$", B);
label("$C$", C);
label("$D$", D);

// A->B
draw((0.30000,0.85000)--(1.70000,0.15000), ArcArrow);

// D->A
draw((-1.70000,0.15000)--(-0.30000,0.85000), ArcArrow);

// C->D
draw((-0.30000,-0.85000)--(-1.70000,-0.15000), ArcArrow);

// C<->B
draw((1.70000,-0.15000)--(0.30000,-0.85000), ArcArrows);

/*
def connect_self(p1, dx1=-1, dy1=1, dx2=0, dy2=1):
	return "draw((%5.5f, %5.5f)..(%5.5f, %5.5f)..(%5.5f, %5.5f),ArcArrow);" % (p1[0] + dx1 * 0.2, p1[1] + dy1 * 0.2, p1[0] + dx2 * 0.7, p1[1] + dy2 * 0.7, p1[0] + (1 if dy2 == 0 else -1) * dx1 * 0.2, p1[1] + (1 if dx2 == 0 else -1) + dy1 * 0.2)

connect_self((0,1), -1, 1, 0, 1);
connect_self((2,0), 1, 1, 1, 0);
connect_self((0,-1), 1, -1, 0, -1);
connect_self((-2,0), -1, -1, -1, 0);
*/

// A<->A, B<->B ...
draw((-0.2,1.2)..(0,1.7)..(0.2,1.2),ArcArrow);
draw((2.20000, 0.20000)..(2.70000, 0.00000)..(2.20000, -0.20000),ArcArrow);
draw((0.20000, -1.20000)..(0.00000, -1.70000)..(-0.20000, -1.20000),ArcArrow);
draw((-2.20000, -0.20000)..(-2.70000, 0.00000)..(-2.20000, 0.20000),ArcArrow);
\end{asy}
\end{center}
\end{minipage}
\begin{minipage}{0.4\textwidth}
\caption{Transportation matrix $B$.}
\label{fig:adjacency_b}
\end{minipage}\hfill
\begin{minipage}{0.4\textwidth}
\caption{Graph of matrix $B$.}
\label{fig:directed}
\end{minipage}
\end{figure}

Because there are one-way connections, $B$ is not symmetric. For example, $b_{12}=1$, but $b_{21}=0$. The graph for this matrix would therefore be ``directed''; it would have arrows indicating the direction of each street. This is shown in Figure~\ref{fig:directed}.

Now, suppose you wanted to know the total number of ways to go from town to town in one step, by path or by bus. To find the total, you add the matrices in the obvious way: term by term, or $(a+b)_{ij}=a_{ij}+b_{ij}$. This is shown in Figure~\ref{fig:mtrx_add}.

\begin{figure}[h]
\begin{minipage}{0.57\textwidth}
$$A+B=\left[\begin{array}{cccc}
1 & 1 & 2 & 2 \\
1 & 1 & 1 & 0 \\
2 & 1 & 1 & 1 \\
2 & 0 & 1 & 1 \\
\end{array}\right] + \left[\begin{array}{cccc}
1 & 1 & 0 & 0 \\
0 & 1 & 1 & 0 \\
0 & 0 & 1 & 1 \\
1 & 0 & 0 & 1 \\
\end{array}\right] = \left[\begin{array}{cccc}
2 & 2 & 2 & 2 \\
1 & 2 & 2 & 0 \\
2 & 1 & 2 & 2 \\
3 & 0 & 1 & 2 \\
\end{array}\right]$$
\caption{Matrix addition of $A$ and $B$.}
\label{fig:mtrx_add}
\end{minipage}\hfill
\begin{minipage}{0.3\textwidth}
$$A+B=\left[\begin{array}{cccc}
1 & 2 & 2 & 2 \\
1 & 1 & 2 & 0 \\
2 & 1 & 1 & 2 \\
3 & 0 & 1 & 1 \\
\end{array}\right]$$
\caption{$A+B$, with $1$s on the diagonal.}
\label{fig:mtrx_replace}
\end{minipage}
\end{figure}

Okay, so that's a little silly: we've counted two different ways to stay still, namely ``not taking a path'' and ``not going anywhere on the bus.'' We should rewrite the matrix, putting ones on the diagonal, as in Figure~\ref{fig:mtrx_replace}. Despite this minor issue, it's still true in general that this most naïve way of adding matrices is also the most convenient and useful. Just don't blindly follow a math recipe without considering its meaning!

But now comes a surprise: the most useful way to multiply matrices is not the obvious way. Why not? You'll see several different examples in the coming weeks. For now, think: what would it mean in terms of transportation if we just multiplied corresponding numbers like $a_{13}b_{13}$?\footnote{This ``obvious'' product is actually known as the Hadamard product.} It would be meaningless, as far as I can tell.

Instead, we want multiplication of the two matrices $B$ and $A$ to represent taking one step by walking and then one step by bus. Similarly, multiplication of matrix $A$ by itself will represent the number of ways to go from town to town in two steps by walking.

So, what rule of matrix multiplication will make that happen? To get from town $A$ to $C$ in two steps, for example, we have to go from town $A$ to one of the four towns, then from that town to town $C$. The total number of ways to do this is

$$\underbrace{a_{11}}_{A\to A}\cdot \underbrace{a_{13}}_{A\to C}+\underbrace{a_{12}}_{A\to B}\cdot \underbrace{a_{23}}_{B\to C}+\underbrace{a_{13}}_{A\to C}\cdot \underbrace{a_{33}}_{C\to C}+\underbrace{a_{14}}_{A\to D}\cdot \underbrace{a_{43}}_{D\to C}=\sum_{j=1}^{4}a_{1j}a_{j3}.$$

And that's how we'll eventually define matrix multiplication. More formally, we can say that to determine the $ij$ entry of the product $XY$ of matrices $X$ and $Y$, use the following formula:

$$(XY)_{ij}=x_{i1}y_{1j}+x_{i2}y_{2j}+\cdots + x_{in}y_{nj} = \sum_{k=1}^n x_{ik} y_{kj},$$

where $X$ has $n$ columns and $Y$ has $n$ rows. This is all fine and dandy if you're say, programming a computer to do matrix multiplication, but we should find a more intuitive way to interpret this definition.

Suppose we're multiplying two matrices $X$ and $Y$. For convenience, let's make them $3\times 2\quad$\footnote{Note that the first dimension is rows and the second is columns, as is the usual order.} and $2\times 3$:

\begin{center}
$$\underbrace{\left[\,\begin{array}{ccc}
1                      & 2 & 3                      \\ \hline
\multicolumn{1}{|l}{4} & 5 & \multicolumn{1}{l|}{6} \\ \hline
\end{array}\,\right]}_{X}\underbrace{\left[\,\begin{array}{|c|c}
\cline{1-1}
7  & 8  \\
9  & 10 \\
11 & 12 \\ \cline{1-1}
\end{array}\,\right]}_{Y}=\left[\begin{array}{cc}
1\cdot 7 + 2\cdot 9 + 3\cdot 11 & 1\cdot 8 + 2\cdot 10 + 3\cdot 12 \\
4\cdot 7 + 5\cdot 9 + 6\cdot 11 & 4\cdot 7 + 9\cdot 10 + 11\cdot 12 \\
\end{array}
\right]=\left[\begin{array}{cc}
58 & 64 \\
\boxed{139} & 250 \\
\end{array}\right].$$
\end{center}

Observe the boxed numbers. To get $139$, we multiplied the boxed rows and columns term by term. That is, we did $4\cdot 7 + 5\cdot 9 + 6\cdot 11$. We can think of this as the dot product of two vectors:

$${<} 4, 5, 6{>} \bullet {<} 7, 9, 11{>} = 139.$$

In our example, to find the top left number $a_{11}$, we'd do ${<}1,2,3{>} \bullet {<}7,9,11{>}$. In general, to find $(XY)_{ij}$, we find the dot product of the $i$th \textbf{row vector} of $X$ and $j$th \textbf{column vector} of $Y$.

With the ability to multiply matrices more easily, let's try some problems.

\begin{enumerate}
\item The three-post snap group can be represented by a set of graphs, each with three towns. The posts are the towns and the elastic bands are the roads. For example, \label{prob:adjacency_matrices_map_subgroup}

$$A=\rotatebox[origin=c]{90}{from}\;\,
  \begin{array}{@{}c@{}}
    \rowind{1} \\ \rowind{2} \\ \rowind{3}
  \end{array}
  \mathop{\left[
  \begin{array}{ccc}
     \colind{1}{1}  &  \colind{0}{2}  &  \colind{0}{3} \\
0 & 0 & 1 \\
0 & 1 & 0 \\
  \end{array}
  \right]}^{
  \begin{array}{@{}c@{}}
    \rowind{\text{to}} \\ \mathstrut
  \end{array}
  }\quad \longleftrightarrow\vcenter{\hbox{
\begin{asy}
size(50);
pair n1 = (0,0);
pair n2 = (1,-sqrt(3));
pair n3 = (2,0);
draw((1.20000,-1.38564)--(1.80000,-0.34641), ArcArrows);
draw((0.26000, -0.26000)..(0.00000, -0.91000)..(-0.26000, -0.26000),ArcArrow);
label("$1$", n1);
label("$2$", n2);
label("$3$", n3);
draw((-0.6,0.4)--(2.3,0.4)--(2.3,-2.1)--(-0.6,-2.1)--cycle);
\end{asy}
}}
$$
\newcounter{mtrx_mult_problem_i}
\newcounter{mtrx_mult_problem_ii}

\begin{enumerate}
\item Draw the graphs and transportation matrices for this group.
\item Try a few multiplications and notice the isomorphism to the snap group.
\end{enumerate}
\item Using $3\times 3$ matrices $A$ and $B$ from this section, compute
\begin{multicols}{4}
\begin{enumerate}
\item $AA=A^2$
\item $AB$
\item $BA$
\item $B^2$
\setcounter{mtrx_mult_problem_ii}{\value{enumii}}
\end{enumerate}
\end{multicols}
\begin{enumerate}
\setcounter{enumii}{\value{mtrx_mult_problem_ii}}
\item Which one ($AB$ and $BA$) represents taking a step by walking, then by bus?
\item Use your calculator to check your computations of $A^2$, $AB$, $BA$, and $B^2$.
\end{enumerate}
\item Write a $3\times 3$ matrix $T$ that shows the following scenario: you can go from town $B$ to $C$, $C$ to $D$, and $D$ to $B$ by train, in exactly one way each, and not backwards.
\begin{enumerate}
\item Why can't you add this matrix to matrices $A$ or $B$?
\item Rewrite matrix $T$ so that it \textit{can} be meaningfully added to matrices $A$ and $B$. What did you do to its dimensions?
\end{enumerate}
\setcounter{mtrx_mult_problem_i}{\value{enumi}}
\end{enumerate}

It's time for a review of \textbf{sigma notation}! Sigma notation represents a sum. It is defined as
$$S=\sum_{k=m}^{n} f(k)=f(m)+f(m+1)+\cdots + f(n),$$
for function $f$ and integers $m,n$. $k$ is the index over which the summation is taking place. It takes on all integer values between $m$ and $n$, inclusive. You might read the sum portion like so: ``The summation of $f$ of $k$ from $k$ equals $m$ to $n$ is equal to $f$ equals $S$.''

\begin{enumerate}
\setcounter{enumi}{\value{mtrx_mult_problem_i}}
\item Evaluate the following:
\begin{multicols}{6}
\begin{enumerate}
\item $\displaystyle\sum_{k=1}^4 k$
\item $\displaystyle\sum_{k=0}^5 k^2$
\item $\displaystyle\sum_{k=1}^{10} 3$
\item $\displaystyle\sum_{k=1}^n k$
\item $\displaystyle\sum_{k=1}^n n$
\item $\displaystyle\sum_{k=1}^n 1$
\end{enumerate}
\end{multicols}

\item The matrix $C^T$ whose rows are the same as the respective columns of matrix $C$ is called the \textbf{transpose} of $C$. For example,

$$C=\left[\begin{array}{cc}
1 & 2 \\
3 & 4 \\
\end{array}\right],\: C^T=\left[\begin{array}{cc}
1 & 3 \\
2 & 4 \\
\end{array}\right].$$

\begin{enumerate}
\item Let the elements of $C$ be $c_{ij}$ and the elements of $C^T$ be $c'_{ij}$. Write a formula for $C^T$ in terms of these elements.
\item Write $\left[\begin{array}{ccc}
2 & 1 & 5 \\
4 & -2 & 0 \\
\end{array}\right]^T$.
\end{enumerate}
\item Fill in the blanks: Multiplying an $m\times n$ matrix by a(n) $\underline{\phantom{egg}} \times k$ matrix gives a(n) $\underline{\phantom{egg}}\times\underline{\phantom{egg}}$ matrix.
\item Dogs can eat cats, rats, or mice; cats can eat rats or mice; rats can eat mice.
\begin{enumerate}
\item Make a matrix $E$ showing what can eat what.
\item Draw a directed graph.
\item Calculate and interpret $E^2$, $E^3$, $E^4$.
\end{enumerate}
\setcounter{mtrx_mult_problem_i}{\value{enumi}}
\end{enumerate}

The following table shows the amount of each ingredient a bakery uses in making one batch of sourdough bread and biscuits. Of course, the units vary depending on the ingredient. Let's call this matrix $S$ for sourdough.

\begin{center}
\begin{blockarray}{lccccccccc}
Sourdough & Flour & Starter & Yeast & Water & Salt & Soda & Sugar & Butter \\ \cline{1-9}
\\
\begin{block}{l[cccccccc]c}
Bread     & $5$     & $1$       & $0$     & $\frac{4}{3}$   & $1$    & $1$    & $0$     & $0$     & \multirow{2}{*}{$=S$} \\
Biscuits  & $5$     & $1$       & $1$     & $\frac{5}{4}$   & $\frac{3}{4}$  & $0$    & $\frac{1}{3}$   & $2$ & \\
\end{block} 
\\ \cline{1-9}
\end{blockarray}
\end{center}

The bakery wants to know how much the ingredients cost for one batch of bread, and how much for one batch of biscuits. The unit cost of the ingredients is given in the following table. Let's call this matrix $C$ for cost.

\begin{center}
\begin{blockarray}{lccccccccc}
 & Flour & Starter & Yeast & Water & Salt & Soda & Sugar & Butter \\ \cline{1-9}
\\
\begin{block}{l[cccccccc]c}
\$ per unit    & $5$     & $20$       & $10$     & $0$   & $1$    & $2$    & $5$     & $12$  & $= C$    \\
\end{block} 
\\ \cline{1-9}
\end{blockarray}
\end{center}

\begin{enumerate}
\setcounter{enumi}{\value{mtrx_mult_problem_i}}
\item \begin{enumerate}
\item Unfortunately, if you try to multiply $S$ and $C$ as given, it won't work. Why not?
\item What do you need to do to $C$ so they can be multiplied? Explain the dimensions of each matrix.
\item Once you've fixed matrix $C$, do the multiplication. What are the dimensions of your answer?
\end{enumerate}
\item Matrix multiplication is not necessarily commutative, even when the dimensions of the matrices suggest it might be. How do we know? Be specific.
\item Matrix multiplication is associative, though. Prove that $(PX)T=P(XT)$ for $$P=\left[\begin{array}{cc} m & n \\ p & q \\ \end{array}\right],\: X=\left[\begin{array}{cc} x & y \\ z & w \\ \end{array}\right],\: Y=\left[\begin{array}{cc} r & s \\ t & u \\ \end{array}\right].$$
\item Prove that matrix multiplication is distributive: $P(X+T)=PX+PT$.
\item When does $PX=XP$? Don't worry if you get some messy equations in your answer.
\item Cook's Seafood Restaurant in Menlo Park sells fish and chips. The Captain's order is two pieces of fish and one order of chips, while the Regular order is one piece of fish and one order of chips.
\begin{enumerate}
\item Write a matrix representing these facts, with clear labels on your rows and columns.
\item The restaurant management estimates their cost at $0.75$ for each piece of fish and $0.50$ for each order of chips. Represent this as a matrix, then use matrix multiplication to calculate the cost of the two possible orders.
\item For a party, Cook's provides $10$ Captain's orders and $5$ Regular orders. Write this as a matrix and use matrix multiplication to find how many pieces of fish and orders of chips are provided.
\item Now use matrix multiplication to find out the cost of the party.
\end{enumerate}
\item We will find matrices to be particularly useful for solving systems of linear equations. For instance, $$\begin{cases}3x+4y&=5 \\ 6x+4y &= 8\end{cases}\quad\longleftrightarrow\quad\left[\begin{array}{cc} 3 & 4 \\ 6 & 7 \\ \end{array}\right]\left[\begin{array}{c} x \\ y \\ \end{array}\right]=\left[\begin{array}{c} 5 \\ 8 \\ \end{array}\right].$$ Rewrite $$\begin{cases}2x+3y+4z&=5 \\ 5x-4y+2z &= 2 \\ x+2y &= 7\end{cases}$$ as a matrix equation in this way.
\item \begin{enumerate}
\item What is the transpose of the $3\times 3$ matrix $M$ from the previous problem?
\item Use $M^T$ to rewrite the system in the previous problem.
\item What is the transpose of the transpose matrix, $(M^T)^T$?
\end{enumerate}
\end{enumerate}

\end{document}

\documentclass[../gatm_answers.tex]{subfiles}

\begin{document}

\section{Matrices Generate Groups}

\begin{outer_problem}[start=1]
\item Analyze this group with the following elements, following the form of Example 1. What makes this group fundamentally different from the example?
$$I=\left[\begin{array}{cc} 1 & 0 \\ 0 & 1 \end{array}\right], A=\left[\begin{array}{cc} 0 & -1 \\ 1 & 0 \end{array}\right], B=\left[\begin{array}{cc} -1 & 0 \\ 0 & -1 \end{array}\right], C=\left[\begin{array}{cc} -1 & 0 \\ 0 & 1 \end{array}\right]$$
\end{outer_problem}

\begin{iinner_problem}[start=1]
\item Specify the elements of the matrix group, unless they are all given.
\end{iinner_problem}

They are all given.

\begin{iinner_problem}
\item Describe what each matrix does to the plane.
\end{iinner_problem}

$I$ does nothing. $A$ rotates by $90^\circ=\frac{\pi}{2}$ counterclockwise. $B$ reflects over the origin, or rotates $180^\circ=\pi$ counterclockwise. $C$ rotates by $90^\circ=\frac{\pi}{2}$ clockwise, or $270^\circ=\frac{3\pi}{2}$ counterclockwise.

\begin{iinner_problem}
\item Construct a group table; you can use a calculator.
\end{iinner_problem}

This is pretty simple. Everything is a rotation by a factor of $90^\circ=\frac{\pi}{2}$.

$$\begin{array}{c|c|c|c|c|}
\cdot & I & A & B & C \\
I & I & A & B & C \\
A & A & B & C & I \\
B & B & C & I & A \\
C & C & I & A & B \\
\end{array}$$

\begin{iinner_problem}
\item Decide which symmetry group your matrix is isomorphic to.
\end{iinner_problem}

This is the cyclic group $C_4$, which is (up to isomorphism) the rotation group of the square.

\begin{outer_problem}
\item The matrix $\left[\begin{array}{cc} -\frac{1}{2} & -\frac{\sqrt{3}}{2} \\ \frac{\sqrt{3}}{2} & -\frac{1}{2}\end{array}\right]$ generates a group of order $3$. Enumerate the elements of this group and analyze per the example.
\end{outer_problem}

\begin{iinner_problem}[start=1]
\item Specify the elements of the matrix group, unless they are all given.
\end{iinner_problem}

Let the given matrix be $M=\begin{bmatrix} -\frac{1}{2} & -\frac{\sqrt{3}}{2} \\ \frac{\sqrt{3}}{2} & -\frac{1}{2}\end{bmatrix}$. Then

$$M^2=\begin{bmatrix} -\frac{1}{2} & \frac{\sqrt{3}}{2} \\ -\frac{\sqrt{3}}{2} & -\frac{1}{2}\end{bmatrix};$$
$$M^3=I=\begin{bmatrix} 1 & 0 \\ 0 & 1 \end{bmatrix}.$$

\begin{iinner_problem}
\item Describe what each matrix does to the plane.
\end{iinner_problem}

$M$ rotates by $120^\circ=\frac{2\pi}{3}$ counterclockwise. $M^2$ rotates by $240^\circ=\frac{4\pi}{3}$ counterclockwise, or $120^\circ=\frac{2\pi}{3}$ clockwise. $I$ does nothing.

\begin{iinner_problem}
\item Construct a group table; you can use a calculator.
\end{iinner_problem}

These are all rotations by a factor of $120^\circ=\frac{2\pi}{3}$.

$$\begin{array}{c|c|c|c|}
\cdot & I & M & M^2 \\
I & I & M & M^2 \\
M & M & M^2 & I \\
M^2 & M^2 & I & M \\
\end{array}$$

\begin{iinner_problem}
\item Decide which symmetry group your matrix is isomorphic to.
\end{iinner_problem}

This is the cyclic group $C_3$, which is (up to isomorphism) the rotation group of the triangle.

\begin{outer_problem}
\item The matrices $\left[\begin{array}{cc} -\frac{1}{2} & -\frac{\sqrt{3}}{2} \\ \frac{\sqrt{3}}{2} & -\frac{1}{2}\end{array}\right]$ and $\left[\begin{array}{cc} 1 & 0 \\ 0 & -1 \end{array}\right]$ generate a group of order $6$, of which the group in problem 2 is a subgroup. Enumerate the elements of the group and analyze per the example.
\end{outer_problem}

\begin{iinner_problem}[start=1]
\item Specify the elements of the matrix group, unless they are all given.
\end{iinner_problem}

We know that the first matrix is a rotation by $120^\circ = \frac{2\pi}{3}$, and the second matrix is a reflection about the $x$-axis, since it flips the $y$ coordinate. Thus, let the first matrix be $r$ and the second matrix be $f$. Note how understanding transformations helps us find the other matrices without much work.

The six elements are shown below.

\begin{align*}
r &= \begin{bmatrix} -\frac{1}{2} & -\frac{\sqrt{3}}{2} \\ \frac{\sqrt{3}}{2} & -\frac{1}{2} \end{bmatrix} \\
f &= \begin{bmatrix} 1 & 0 \\ 0 & -1 \end{bmatrix} \\
r^2 &= \begin{bmatrix} -\frac{1}{2} & \frac{\sqrt{3}}{2} \\ -\frac{\sqrt{3}}{2} & -\frac{1}{2} \end{bmatrix} \\
fr &= \begin{bmatrix} -\frac{1}{2} & \frac{\sqrt{3}}{2} \\ \frac{\sqrt{3}}{2} & \frac{1}{2} \end{bmatrix} \\
fr^2 &= \begin{bmatrix} -\frac{1}{2} & -\frac{\sqrt{3}}{2} \\ -\frac{\sqrt{3}}{2} & \frac{1}{2} \end{bmatrix} \\
I &= f^2 = r^3 = \begin{bmatrix} 1 & 0 \\ 0 & 1 \end{bmatrix} \\
\end{align*}

\begin{iinner_problem}
\item Describe what each matrix does to the plane.
\end{iinner_problem}

$r$ is a rotation by $120^\circ = \frac{2\pi}{3}$ counterclockwise. $f$ is a reflection about the $x$-axis. $r^2$ is a rotation by $240^\circ = \frac{4\pi}{3}$ counterclockwise, or $120^\circ = \frac{2\pi}{3}$ clockwise. $fr$ is a reflection about the line $\theta = 120^\circ = \frac{2\pi}{3}$. $fr^2$ is a reflection about the line $\theta=240^\circ=\frac{4\pi}{3}$. $I$ does nothing.

\begin{iinner_problem}
\item Construct a group table; you can use a calculator.
\end{iinner_problem}

Here you go.

$$\begin{array}{c|c|c|c|c|c|c|}
\cdot & I & r & r^2 & f & fr & fr^2 \\
I & I & r & r^2 & f & fr & fr^2 \\
r & r & r^2 & I & fr^2 & f & fr \\
r^2 & r^2 & I & r & fr & fr^2 & f \\
f & f & fr & fr^2 & I & r & r^2 \\
fr & fr^2 & f & fr & r^2 & I & r \\
fr^2 & fr & fr^2 & f & r & r^2 & I \\
\end{array}$$

\begin{iinner_problem}
\item Decide which symmetry group your matrix is isomorphic to.
\end{iinner_problem}

This is the dihedral group of order $6$, or the symmetry group of the triangle.

\begin{iinner_problem}
\item What other sets of two matrices could have generated this group?
\end{iinner_problem}

The sets $\{r, fr\}$, $\{r, fr^2\}$, $\{r^2, f\}$, $\{r^2, fr\}$, $\{r^2, fr^2\}$, $\{f,fr\}$, $\{fr,fr^2\}$, and $\{f,fr^2\}$. In fact, any two non-identity elements can together generate the group, except for $\{r, r^2\}$.

\begin{outer_problem}
\item The matrix $\left[\begin{array}{cc} \frac{\sqrt{5}-1}{4} & -\frac{\sqrt{10+2\sqrt{5}}}{4} \\ \frac{\sqrt{10+2\sqrt{5}}}{4} & \frac{\sqrt{5}-1}{4} \end{array}\right]$ generates a group of order $5$! Enumerate the elements of the group and analyze per the example; you can use a calculator.
\end{outer_problem}

\begin{outer_problem}
\item Let $A=\twomat{\cos\frac{2\pi}{n}}{-\sin\frac{2\pi}{n}}{\sin\frac{2\pi}{n}}{\cos\frac{2\pi}{n}}$, $B=\twomat{\cos\frac{2\pi}{n}}{\sin\frac{2\pi}{n}}{\sin\frac{2\pi}{n}}{-\cos\frac{2\pi}{n}}$, $C=\twomat{1}{0}{0}{-1}$, and $n$ be an integer. What group is generated by the following sets of generators? Describe them geometrically.
\end{outer_problem}

\begin{inner_problem}[start=1]
\item $\{A\}$
\end{inner_problem}

\begin{inner_problem}
\item $\{B\}$
\end{inner_problem}

\begin{inner_problem}
\item $\{A,B\}$
\end{inner_problem}

\begin{inner_problem}
\item $\{B,C\}$
\end{inner_problem}

\begin{outer_problem}
\item Do the previous problem, but replacing $\frac{2\pi}{n}$ with $k$, where $k$ is an integer number of radians.
\end{outer_problem}

\begin{inner_problem}[start=1]
\item $\{A\}$
\end{inner_problem}

\begin{inner_problem}
\item $\{B\}$
\end{inner_problem}

\begin{inner_problem}
\item $\{A,B\}$
\end{inner_problem}

\begin{inner_problem}
\item $\{B,C\}$
\end{inner_problem}

\begin{outer_problem}
\item Given $C=\twomat{1}{0}{0}{-1}$ and $D=\twomat{1}{1}{0}{-1}$, what is the order of the group generated by the following sets of generators?
\end{outer_problem}

\begin{inner_problem}[start=1]
\item $\{C\}$
\end{inner_problem}

\begin{inner_problem}
\item $\{D\}$
\end{inner_problem}

\begin{inner_problem}
\item $\{C,D\}$
\end{inner_problem}

\begin{outer_problem}
\item What matrix could generate the cyclic group of order $n$, $C_n$?
\end{outer_problem}

\begin{outer_problem}
\item What two matrices could generate the dihedral group of order $2n$, $D_n$?\end{outer_problem}

\begin{outer_problem}
\item Look at Problem~\ref{prob:adjacency_matrices_map_subgroup} on page~\pageref{prob:adjacency_matrices_map_subgroup}. The adjacency matrices map to a subgroup of the full cube symmetry group. What rotations/reflections do they map to?
\end{outer_problem}

\begin{outer_problem}
\item Given $P=\threemat{0}{-1}{0}{1}{0}{0}{0}{0}{1}$, $Q=\threemat{0}{1}{0}{0}{0}{1}{1}{0}{0}$, and $R=\threemat{-1}{0}{0}{0}{1}{0}{0}{0}{1}$, try understanding the groups generated by:
\end{outer_problem}

\begin{inner_problem}[start=1]
\item ${P}$
\end{inner_problem}

\begin{inner_problem}
\item ${Q}$
\end{inner_problem}

\begin{inner_problem}
\item ${R}$
\end{inner_problem}

\begin{inner_problem}
\item ${P,Q}$
\end{inner_problem}

\begin{inner_problem}
\item ${P,R}$
\end{inner_problem}

\begin{inner_problem}
\item ${Q,R}$
\end{inner_problem}

\begin{inner_problem}
\item ${P,Q,R}.$
\end{inner_problem}

\begin{outer_problem}
\item The matrix $\twomat{1}{1}{0}{1}$ produces a shear. What is its inverse---what undoes the shear?
\end{outer_problem}

\begin{outer_problem}
\item The complex numbers, excluding zero, form a group under multiplication. What set of matrices is isomorphic to the same group under multiplication?
\end{outer_problem}

\begin{outer_problem}
\item Does the set of all $2\times 2$ matrices form a group under multiplication? Why or why not?
\end{outer_problem}

\end{document}
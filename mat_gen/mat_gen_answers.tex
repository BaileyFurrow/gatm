\documentclass[../gatm_answers.tex]{subfiles}

\begin{document}

\section{Matrices Generate Groups}

\begin{enumerate}
\item Analyze this group with the following elements, following the form of Example 1. What makes this group fundamentally different from the example?
$$I=\left[\begin{array}{cc} 1 & 0 \\ 0 & 1 \end{array}\right], A=\left[\begin{array}{cc} 0 & -1 \\ 1 & 0 \end{array}\right], B=\left[\begin{array}{cc} -1 & 0 \\ 0 & -1 \end{array}\right], C=\left[\begin{array}{cc} -1 & 0 \\ 0 & 1 \end{array}\right]$$
\item The matrix $\left[\begin{array}{cc} -\frac{1}{2} & -\frac{\sqrt{3}}{2} \\ \frac{\sqrt{3}}{2} & -\frac{1}{2}\end{array}\right]$ generates a group of order $3$. Enumerate the elements of this group and analyze per the example.
\item The matrices $\left[\begin{array}{cc} -\frac{1}{2} & -\frac{\sqrt{3}}{2} \\ \frac{\sqrt{3}}{2} & -\frac{1}{2}\end{array}\right]$ and $\left[\begin{array}{cc} 1 & 0 \\ 0 & -1 \end{array}\right]$ generate a group of order $6$, of which the group in problem 2 is a subgroup. Enumerate the elements of the group and analyze per the example.
\begin{enumerate}
\setcounter{enumii}{5}
\item What other two sets of two matrices could have generated this group?
\end{enumerate}
\item The matrix $\left[\begin{array}{cc} \frac{\sqrt{5}-1}{4} & -\frac{\sqrt{10+2\sqrt{5}}}{4} \\ \frac{\sqrt{10+2\sqrt{5}}}{4} & \frac{\sqrt{5}-1}{4} \end{array}\right]$ generates a group of order $5$! Enumerate the elements of the group and analyze per the example; you can use a calculator.
\item Let $A=\twomat{\cos\frac{2\pi}{n}}{-\sin\frac{2\pi}{n}}{\sin\frac{2\pi}{n}}{\cos\frac{2\pi}{n}}$, $B=\twomat{\cos\frac{2\pi}{n}}{\sin\frac{2\pi}{n}}{\sin\frac{2\pi}{n}}{-\cos\frac{2\pi}{n}}$, $C=\twomat{1}{0}{0}{-1}$, and $n$ be an integer. What group is generated by the following sets of generators? Describe them geometrically. \begin{multicols}{4}%
\begin{enumerate}%
\item $\{A\}$
\item $\{B\}$
\item $\{A,B\}$
\item $\{B,C\}$
\end{enumerate}%
\end{multicols}%
\item Do the previous problem, but replacing $\frac{2\pi}{n}$ with $k$, where $k$ is an integer number of radians.
\item Given $C=\twomat{1}{0}{0}{-1}$ and $D=\twomat{1}{1}{0}{-1}$, what is the order of the group generated by the following sets of generators?
\begin{multicols}{3}
\begin{enumerate}
\item $\{C\}$
\item $\{D\}$
\item $\{C,D\}$
\end{enumerate}
\end{multicols}
\item What matrix could generate the cyclic group of order $n$, $C_n$?
\item What two matrices could generate the dihedral group of order $2n$, $D_n$?
\item Look at Problem~\ref{prob:adjacency_matrices_map_subgroup} on page~\pageref{prob:adjacency_matrices_map_subgroup}. The adjacency matrices map to a subgroup of the full cube symmetry group. What rotations/reflections do they map to?
\item Given $P=\threemat{0}{-1}{0}{1}{0}{0}{0}{0}{1}$, $Q=\threemat{0}{1}{0}{0}{0}{1}{1}{0}{0}$, and $R=\threemat{-1}{0}{0}{0}{1}{0}{0}{0}{1}$, try understanding the groups generated by:
\begin{multicols}{4}
\begin{enumerate}
\item ${P}$
\item ${Q}$
\item ${R}$
\item ${P,Q}$
\item ${P,R}$
\item ${Q,R}$
\item ${P,Q,R}.$
\end{enumerate}
\end{multicols}
\item The matrix $\twomat{1}{1}{0}{1}$ produces a shear. What is its inverse---what undoes the shear?
\item The complex numbers, excluding zero, form a group under multiplication. What set of matrices is isomorphic to the same group under multiplication?
\item Does the set of all $2\times 2$ matrices form a group under multiplication? Why or why not?
\end{enumerate}

\end{document}
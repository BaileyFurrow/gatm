\documentclass[../gatm_answers.tex]{subfiles}

\begin{document}

\section{Matrices Generate Groups}

\begin{outer_problem}[start=1]
\item Analyze this group with the following elements, following the form of Example 1. What makes this group fundamentally different from the example?
$$I=\left[\begin{array}{cc} 1 & 0 \\ 0 & 1 \end{array}\right], A=\left[\begin{array}{cc} 0 & -1 \\ 1 & 0 \end{array}\right], B=\left[\begin{array}{cc} -1 & 0 \\ 0 & -1 \end{array}\right], C=\left[\begin{array}{cc} -1 & 0 \\ 0 & 1 \end{array}\right]$$
\end{outer_problem}

\begin{iinner_problem}[start=1]
\item Specify the elements of the matrix group, unless they are all given.
\end{iinner_problem}

They are all given.

\begin{iinner_problem}
\item Describe what each matrix does to the plane.
\end{iinner_problem}

$I$ does nothing. $A$ rotates by $90^\circ=\frac{\pi}{2}$ counterclockwise. $B$ reflects over the origin, or rotates $180^\circ=\pi$ counterclockwise. $C$ rotates by $90^\circ=\frac{\pi}{2}$ clockwise, or $270^\circ=\frac{3\pi}{2}$ counterclockwise.

\begin{iinner_problem}
\item Construct a group table; you can use a calculator.
\end{iinner_problem}

This is pretty simple. Everything is a rotation by a factor of $90^\circ=\frac{\pi}{2}$.

$$\begin{array}{c|c|c|c|c|}
\cdot & I & A & B & C \\
I & I & A & B & C \\
A & A & B & C & I \\
B & B & C & I & A \\
C & C & I & A & B \\
\end{array}$$

\begin{iinner_problem}
\item Decide which symmetry group your matrix is isomorphic to.
\end{iinner_problem}

This is the cyclic group $C_4$, which is (up to isomorphism) the rotation group of the square.

\begin{outer_problem}
\item The matrix $\left[\begin{array}{cc} -\frac{1}{2} & -\frac{\sqrt{3}}{2} \\ \frac{\sqrt{3}}{2} & -\frac{1}{2}\end{array}\right]$ generates a group of order $3$. Enumerate the elements of this group and analyze per the example.
\end{outer_problem}

\begin{iinner_problem}[start=1]
\item Specify the elements of the matrix group, unless they are all given.
\end{iinner_problem}

Let the given matrix be $M=\begin{bmatrix} -\frac{1}{2} & -\frac{\sqrt{3}}{2} \\ \frac{\sqrt{3}}{2} & -\frac{1}{2}\end{bmatrix}$. Then

$$M^2=\begin{bmatrix} -\frac{1}{2} & \frac{\sqrt{3}}{2} \\ -\frac{\sqrt{3}}{2} & -\frac{1}{2}\end{bmatrix};$$
$$M^3=I=\begin{bmatrix} 1 & 0 \\ 0 & 1 \end{bmatrix}.$$

\begin{iinner_problem}
\item Describe what each matrix does to the plane.
\end{iinner_problem}

$M$ rotates by $120^\circ=\frac{2\pi}{3}$ counterclockwise. $M^2$ rotates by $240^\circ=\frac{4\pi}{3}$ counterclockwise, or $120^\circ=\frac{2\pi}{3}$ clockwise. $I$ does nothing.

\begin{iinner_problem}
\item Construct a group table; you can use a calculator.
\end{iinner_problem}

These are all rotations by a factor of $120^\circ=\frac{2\pi}{3}$.

$$\begin{array}{c|c|c|c|}
\cdot & I & M & M^2 \\
I & I & M & M^2 \\
M & M & M^2 & I \\
M^2 & M^2 & I & M \\
\end{array}$$

\begin{iinner_problem}
\item Decide which symmetry group your matrix is isomorphic to.
\end{iinner_problem}

This is the cyclic group $C_3$, which is (up to isomorphism) the rotation group of the triangle.

\begin{outer_problem}
\item The matrices $\left[\begin{array}{cc} -\frac{1}{2} & -\frac{\sqrt{3}}{2} \\ \frac{\sqrt{3}}{2} & -\frac{1}{2}\end{array}\right]$ and $\left[\begin{array}{cc} 1 & 0 \\ 0 & -1 \end{array}\right]$ generate a group of order $6$, of which the group in problem 2 is a subgroup. Enumerate the elements of the group and analyze per the example.
\end{outer_problem}

\begin{iinner_problem}[start=1]
\item Specify the elements of the matrix group, unless they are all given.
\end{iinner_problem}

We know that the first matrix is a rotation by $120^\circ = \frac{2\pi}{3}$, and the second matrix is a reflection about the $x$-axis, since it flips the $y$ coordinate. Thus, let the first matrix be $r$ and the second matrix be $f$. Note how understanding transformations helps us find the other matrices without much work.

The six elements are shown below.

\begin{align*}
r &= \begin{bmatrix} -\frac{1}{2} & -\frac{\sqrt{3}}{2} \\ \frac{\sqrt{3}}{2} & -\frac{1}{2} \end{bmatrix} \\
f &= \begin{bmatrix} 1 & 0 \\ 0 & -1 \end{bmatrix} \\
r^2 &= \begin{bmatrix} -\frac{1}{2} & \frac{\sqrt{3}}{2} \\ -\frac{\sqrt{3}}{2} & -\frac{1}{2} \end{bmatrix} \\
fr &= \begin{bmatrix} -\frac{1}{2} & \frac{\sqrt{3}}{2} \\ \frac{\sqrt{3}}{2} & \frac{1}{2} \end{bmatrix} \\
fr^2 &= \begin{bmatrix} -\frac{1}{2} & -\frac{\sqrt{3}}{2} \\ -\frac{\sqrt{3}}{2} & \frac{1}{2} \end{bmatrix} \\
I &= f^2 = r^3 = \begin{bmatrix} 1 & 0 \\ 0 & 1 \end{bmatrix} \\
\end{align*}

\begin{iinner_problem}
\item Describe what each matrix does to the plane.
\end{iinner_problem}

$r$ is a rotation by $120^\circ = \frac{2\pi}{3}$ counterclockwise. $f$ is a reflection about the $x$-axis. $r^2$ is a rotation by $240^\circ = \frac{4\pi}{3}$ counterclockwise, or $120^\circ = \frac{2\pi}{3}$ clockwise. $fr$ is a reflection about the line $\theta = 120^\circ = \frac{2\pi}{3}$. $fr^2$ is a reflection about the line $\theta=240^\circ=\frac{4\pi}{3}$. $I$ does nothing.

\begin{iinner_problem}
\item Construct a group table; you can use a calculator.
\end{iinner_problem}

Here you go.

$$\begin{array}{c|c|c|c|c|c|c|}
\cdot & I & r & r^2 & f & fr & fr^2 \\ \hline
I & I & r & r^2 & f & fr & fr^2 \\ \hline
r & r & r^2 & I & fr^2 & f & fr \\ \hline
r^2 & r^2 & I & r & fr & fr^2 & f \\ \hline
f & f & fr & fr^2 & I & r & r^2 \\ \hline
fr & fr^2 & f & fr & r^2 & I & r \\ \hline
fr^2 & fr & fr^2 & f & r & r^2 & I \\ \hline
\end{array}$$

\begin{iinner_problem}
\item Decide which symmetry group your matrix is isomorphic to.
\end{iinner_problem}

This is the dihedral group of order $6$, or the symmetry group of the triangle.

\begin{iinner_problem}
\item What other sets of        matrices could have generated this group?
\end{iinner_problem}

The sets $\{r, fr\}$, $\{r, fr^2\}$, $\{r^2, f\}$, $\{r^2, fr\}$, $\{r^2, fr^2\}$, $\{f,fr\}$, $\{fr,fr^2\}$, and $\{f,fr^2\}$. In fact, any two non-identity elements can together generate the group, except for $\{r, r^2\}$.

\begin{outer_problem}
\item The matrix $\left[\begin{array}{cc} \frac{\sqrt{5}-1}{4} & -\frac{\sqrt{10+2\sqrt{5}}}{4} \\ \frac{\sqrt{10+2\sqrt{5}}}{4} & \frac{\sqrt{5}-1}{4} \end{array}\right]$ generates a group of order $5$! Enumerate the elements of the group and analyze per the example; you can use a calculator.
\end{outer_problem}

\begin{iinner_problem}[start=1]
\item Specify the elements of the matrix group, unless they are all given.
\end{iinner_problem}

We'd expect this to be a rotation matrix of some multiple of $72^\circ$. Thus, let's call it $r$ for now. It isn't immediately clear, however, how to compute $\cos 72^\circ$. All available sum and difference expressions seem useless. We'll defer this computation to part (b). Here are the elements of the matrix group:

\begin{align*}
r &= \begin{bmatrix} \frac{\sqrt{5}-1}{4} & -\frac{\sqrt{10+2\sqrt{5}}}{4} \\ \frac{\sqrt{10+2\sqrt{5}}}{4} & \frac{\sqrt{5}-1}{4} \end{bmatrix} \\
r^2 &= \begin{bmatrix} -\frac{1+\sqrt{5}}{4} & -\frac{\sqrt{10-2\sqrt{5}}}{4} \\ \frac{\sqrt{10-2\sqrt{5}}}{4} & -\frac{1+\sqrt{5}}{4} \end{bmatrix} \\
r^3 &= \begin{bmatrix} -\frac{1+\sqrt{5}}{4} & \frac{\sqrt{10-2\sqrt{5}}}{4} \\ -\frac{\sqrt{10-2\sqrt{5}}}{4} & -\frac{1+\sqrt{5}}{4} \end{bmatrix} \\
r^4 &= \begin{bmatrix} \frac{\sqrt{5}-1}{4} & \frac{\sqrt{10+2\sqrt{5}}}{4} \\ -\frac{\sqrt{10+2\sqrt{5}}}{4} & \frac{\sqrt{5}-1}{4} \end{bmatrix} \\
I &= r^5 = \begin{bmatrix} 1 & 0 \\ 0 & 1 \end{bmatrix} \\
\end{align*}

Note that there are many equivalent ways to write the entries of these matrices. For example, $$-\frac{\sqrt{10-2\sqrt{5}}}{4} = -\sqrt{2\left(5+\sqrt{5}\right)}+\sqrt{10\left(5+\sqrt{5}\right)}.$$

The latter is what WolframAlpha gives; I prefer the former form.

\begin{iinner_problem}
\item Describe what each matrix does to the plane.
\end{iinner_problem}

We'd guess that $r$ is a rotation of $\frac{360^\circ}{5} = 72^\circ$, but to prove this we need to find $\cos 72^\circ$ and $\sin 72^\circ$.

Consider $\cos 72^\circ$. Because of symmetry around $2.5\cdot 72^\circ=180^\circ$, we have $\cos (2\cdot 72^\circ) = \cos (3\cdot 72^\circ)$. By the double-angle and triple-angle (which we found in the complex numbers section) formulae,

$$\cos 2x = 2\cos^2 x - 1;$$
$$\cos 3x = 4\cos^3 x - 3\cos x.$$

Let $c = \cos 72^\circ$. Then we have

\begin{align*}
2c^2 - 1 &= 4c^3 - 3c \\
4c^3 - 2c^2 - 3c + 1 &= 0 \\
(4c^2 + 2c - 1)(c-1) &= 0, \\
\end{align*}

We know $\cos 72^\circ \neq \cos 0^\circ = 1$, so

\begin{align*}
4c^2 + 2c - 1 &= 0 \\
c &= \frac{-1\pm \sqrt{5}}{4}.
\end{align*}

Since $0 < 72^\circ < 90^\circ$, we have $c > 0$, so $c=\frac{\sqrt{5}-1}{4}$. To find $\sin 72^\circ$ we use the Pythagorean identity and choose the positive root:

$$\sin 72^\circ = \sqrt{1 - c^2} = \sqrt{1 - \frac{5 - 2\sqrt{5} + 1}{16}}$$
$$=\sqrt{\frac{16 - 6 + 2\sqrt{5}}{16}}$$
$$=\frac{\sqrt{10+2\sqrt{5}}}{4}.$$

Indeed, we have

$$\begin{bmatrix} \cos 72^\circ & -\sin 72^\circ \\ \sin 72^\circ & \cos 72^\circ \end{bmatrix} = \begin{bmatrix} c & -s \\ s & c \end{bmatrix} = \begin{bmatrix} \frac{\sqrt{5}-1}{4} & -\frac{\sqrt{10+2\sqrt{5}}}{4} \\ \frac{\sqrt{10+2\sqrt{5}}}{4} & \frac{\sqrt{5}-1}{4} \end{bmatrix} = r.$$

\begin{iinner_problem}
\item Construct a group table; you can use a calculator.
\end{iinner_problem}

Here it is:

$$\begin{array}{c|c|c|c|c|c|}
\cdot & I & r & r^2 & r^3 & r^4 \\ \hline
I & I & r & r^2 & r^3 & r^4 \\ \hline
r & r & r^2 & r^3 & r^4 & I \\ \hline
r^2 & r^2 & r^3 & r^4 & I & r^2 \\ \hline
r^3 & r^3 & r^4 & I & r & r^2 \\ \hline
r^4 & r^4 & I & r & r^2 & r^3 \\ \hline
\end{array}$$

\begin{iinner_problem}
\item Decide which symmetry group your matrix is isomorphic to.
\end{iinner_problem}

This the cyclic group of order $5$, or the rotation group of the regular pentagon.

\begin{iinner_problem}
\item What other sets of matrices could have generated this group?
\end{iinner_problem}

Any matrix in this group, except the identity matrix, would generate the whole group, since $5$ is a prime number.

\begin{outer_problem}
\item Let $A=\twomat{\cos\frac{2\pi}{n}}{-\sin\frac{2\pi}{n}}{\sin\frac{2\pi}{n}}{\cos\frac{2\pi}{n}}$, $B=\twomat{\cos\frac{2\pi}{n}}{\sin\frac{2\pi}{n}}{\sin\frac{2\pi}{n}}{-\cos\frac{2\pi}{n}}$, $C=\twomat{1}{0}{0}{-1}$, and $n$ be an integer. What group is generated by the following sets of generators? Describe them geometrically.
\end{outer_problem}

\begin{inner_problem}[start=1]
\item $\{A\}$
\end{inner_problem}

$A$ is a rotation matrix rotating by $\frac{2\pi}{n}$ radians, which is the angle subtended by one of the sides of an $n$-gon:

\begin{center}
\begin{asy}[width=0.3\textwidth]
int n = 11;

for (int i = 0; i < n; ++i) {
	draw(expi(2 * pi * i / n)--expi(2 * pi * (i + 1) / n));
	draw(expi(2 * pi * i / n)--(0,0));
}

real r = 0.45;

path subtend_arc = (0.3,0)..(expi(pi / n) * 0.3)..(expi(2 * pi / n) * 0.3);
draw(subtend_arc);
label("$\frac{2\pi}{11}$", subtend_arc, expi(pi / n));
\end{asy}
\captionof{figure}{A rotation of $\frac{2\pi}{n}$ radians is a symmetry of the $n$-gon. Here, $n=11$.}
\end{center}

\begin{inner_problem}
\item $\{B\}$
\end{inner_problem}

$B=\twomat{\cos\frac{2\pi}{n}}{\sin\frac{2\pi}{n}}{\sin\frac{2\pi}{n}}{-\cos\frac{2\pi}{n}}$ initially appears to be a rotation matrix, but the right column is negated. What could this mean?!

Well, notice that $BC = A$. Since $C$ is just a reflection over the $x$-axis, $C^2=I$, so we have $BCC=AC$ and thus $B=AC$. In geometric terms, $B$ is a reflection over the $x$-axis, followed by a rotation of $2\pi/n$ radians counterclockwise; recall that our matrices transform right-to-left. But a non-zero rotation followed by a reflection is just a reflection about a different axis! So $\{B\}$ generates the cyclic group of order $2$, which is the rotation group of the rectangle or the symmetry group of the line segment.

To confirm this, we can show that $B^2 = I$:

$$B^2 = \twomat{\cos\frac{2\pi}{n}}{\sin\frac{2\pi}{n}}{\sin\frac{2\pi}{n}}{-\cos\frac{2\pi}{n}}\twomat{\cos\frac{2\pi}{n}}{\sin\frac{2\pi}{n}}{\sin\frac{2\pi}{n}}{-\cos\frac{2\pi}{n}} = \twomat{\cos^2\frac{2\pi}{n}+\sin^2\frac{2\pi}{n}}{\cos\frac{2\pi}{n}\sin\frac{2\pi}{n}-\sin\frac{2\pi}{n}\cos\frac{2\pi}{n}}{\sin\frac{2\pi}{n}\cos\frac{2\pi}{n}-\cos\frac{2\pi}{n}\sin\frac{2\pi}{n}}{\sin^2\frac{2\pi}{n}+\cos^2\frac{2\pi}{n}}$$
$$ = \twomat{1}{0}{0}{1} = I.$$

\begin{inner_problem}
\item $\{A,B\}$
\end{inner_problem}

Thinking of these as transformations, we see that $B$ is a reflection across the line $\theta=\pi/n$ (not $2\pi/n$!), which is a symmetry of the $n$-gon\footnote{To be pedantic, the $n$-gon centered on the origin and with a vertex on the $x$-axis.}. Combined with the rotation of $2\pi/n$, this generates the dihedral group of order $2n$: the symmetry group of the $n$-gon.

\begin{inner_problem}
\item $\{B,C\}$
\end{inner_problem}

Since $A=BC$ and $A^{n-1}B=C$, this problem's set can generate $A$ and the previous problem's set can generate $C$. Thus, they are the same; $\{B,C\}$ generates the dihedral group of order $2n$: the symmetry group of the $n$-gon.

\begin{outer_problem}
\item Given $C=\twomat{1}{0}{0}{-1}$ and $D=\twomat{1}{1}{0}{-1}$, what is the order of the group generated by the following sets of generators?
\end{outer_problem}

\begin{inner_problem}[start=1]
\item $\{C\}$
\end{inner_problem}

This has order $2$, since $C$ is just a reflection over the $x$-axis. In algebraic terms, $C^2=I$.

\begin{inner_problem}
\item $\{D\}$
\end{inner_problem}

Interestingly, $D^2=I$, so this again has order $2$. Truly succulent!

\begin{inner_problem}
\item $\{C,D\}$
\end{inner_problem}

What new do we get from a set of two matrices? Well, consider $CD=J$ (how fun) and $DC=K$ (less fun) which are just $J=\twomat{1}{1}{0}{1}$ and $K=\twomat{1}{-1}{0}{1}$, respectively. Let's analyze products of $J$ and $K$.

Well, $JK = KJ = I$. More interesting stuff happens when we multiply them repeatedly.

\begin{align*}
J^2 &= \twomat{1}{2}{0}{1} \\
J^3 &= \twomat{1}{3}{0}{1} \\
K^2 &= \twomat{1}{-2}{0}{1} \\
K^3 &= \twomat{1}{-3}{0}{1} \\
J^3K^2 &= \twomat{1}{1}{0}{1} = J \\
K^3J^2 &= \twomat{1}{-1}{0}{1} = K. \\
\end{align*}

Interesting! It seems a product of $J$'s and $K$'s leads to a matrix of the form $\twomat{1}{n}{0}{1}$, where $n$ is an integer. We can also multiply this by $C$, which gives us the matrices $\twomat{1}{m}{0}{-1}$, where $m$ is an integer. The order of this group is countably infinite, since we can enumerate all of the elements in a list.

\begin{outer_problem}
\item What matrix could generate the cyclic group of order $n$, $C_n$?
\end{outer_problem}

The matrix $\twomat{\cos \frac{2\pi}{n}}{-\sin \frac{2\pi}{n}}{\sin \frac{2\pi}{n}}{\cos \frac{2\pi}{n}}$ could do so.

\begin{outer_problem}
\item What two matrices could generate the dihedral group of order $2n$, $D_n$?\end{outer_problem}

\begin{outer_problem}
\item Look at Problem~\ref{prob:adjacency_matrices_map_subgroup} on page~\pageref{prob:adjacency_matrices_map_subgroup}. The adjacency matrices map to a subgroup of the full cube symmetry group. What rotations/reflections do they map to?
\end{outer_problem}

\begin{outer_problem}
\item Given $P=\threemat{0}{-1}{0}{1}{0}{0}{0}{0}{1}$, $Q=\threemat{0}{1}{0}{0}{0}{1}{1}{0}{0}$, and $R=\threemat{-1}{0}{0}{0}{1}{0}{0}{0}{1}$, try understanding the groups generated by:
\end{outer_problem}

Fair warning: these problems are challenging. With more advanced tools of abstract algebra, however, they are much easier. Nonetheless, I will present ``elementary'' solutions.

\begin{inner_problem}[start=1]
\item $\{P\}$
\end{inner_problem}

We have $P^2 = \threemat{-1}{0}{0}{0}{-1}{0}{0}{0}{1}$, $P^3 = \threemat{0}{1}{0}{-1}{0}{0}{0}{0}{1}$, $P^4 = \threemat{1}{0}{0}{0}{1}{0}{0}{0}{1} = I$. Thus, $\{P\}$ generates the cyclic group of order $4$; the rotation group of the square.

\begin{inner_problem}
\item $\{Q\}$
\end{inner_problem}

We have $Q^2 = \threemat{0}{0}{1}{1}{0}{0}{0}{1}{0}$ and $Q^3 = \threemat{1}{0}{0}{0}{1}{0}{0}{0}{1}=I$. Thus, $\{Q\}$ generates the cyclic group of order $3$; the rotation group of the equilateral triangle.

\begin{inner_problem}
\item $\{R\}$
\end{inner_problem}

Clearly, $R^2 = I$, so $R$ generates the cyclic group of order $2$; the rotation group of the rectangle.

\begin{inner_problem}
\item $\{P,Q\}$
\end{inner_problem}

This is where the complexity begins.

\textbf{Approach 1: Purely in Matrices}

This way is kind of silly, so I would suggest you read Approach 2.

To wrap our heads around this group, we consider what left-multiplying by $P$ and $Q$ does to a matrix's entries. Multiplying $P$ by some matrix $M$, we get

$$\threemat{0}{-1}{0}{1}{0}{0}{0}{0}{1}\threemat{a}{b}{c}{d}{e}{f}{g}{h}{i} = \threemat{-d}{-e}{-f}{a}{b}{c}{g}{h}{i}.$$

Thus, $P$ swaps the top two rows and negates the topmost row, in that order. Multiplying $Q$ by $M$, we get

$$\threemat{0}{1}{0}{0}{0}{1}{1}{0}{0}\threemat{a}{b}{c}{d}{e}{f}{g}{h}{i} = \threemat{d}{e}{f}{g}{h}{i}{a}{b}{c}.$$

Thus, $Q$ cycles the rows ``upward,'' where row 1 goes to row 3, row 2 goes to row 1, and row 3 goes to row 2. So $P$ and $Q$ are just operations on rows; columns don't matter.

Consider the sequence of multiplications

$$\underbrace{PQQQPQP\cdots PPP}_{\text{some sequence}} Q^3 = \underbrace{PQQQPQP\cdots PPP}_{\text{some sequence}} I.$$

On one hand, every matrix that can be generated by $P$ and $Q$ can be written in this form. On the other hand, we can treat the sequence of $P$ and $Q$ as (left-to-right) sequential row operations on the identity matrix $I$.

Think of the identity matrix as the ordered triple $(R_1,R_2,R_3)$, where $R_i$ is row vector $i$. Then $P$ is the function $f(r_1,r_2,r_3) = (-r_2,r_1,r_3)$ and $Q$ is the function $f(r_1,r_2,r_3) = (r_2,r_3,r_1)$.

The order of this group is clearly finite; a quick upper bound is the number of permutations of $(R_1,R_2,R_3)$, along with any combination of negations of elements. This is

$$3!\cdot 2^3 = 48.$$

We can reduce this upper bound by constructing an invariant: something that neither of these operations change. This invariant is rather simple:

$$K = \begin{cases}1 & \text{cyclic order preserved} \\ 0 & \text{cyclic order not preserved}\end{cases} + \begin{cases}-1 & \text{even number of negated rows} \\ 0 & \text{odd number of negated rows} \end{cases}.$$

``Cyclic order'' is what I'm calling the property that you can start at $R_1$ and continue reading right, looping back if necessary, to read $R_1,R_2,R_3$. Cyclic order is not preserved if you read $R_1,R_3,R_2$.

The invariant for the identity is $K=0$. For all attainable row triples, the invariant is $K=0$. That's because $Q$ just cycles the elements, changing nothing about the invariant, while $P$ changes the parity of negated rows \textit{and} either restores or removes ``cyclic order.'' This will produce canceling effects in the two terms of $K$.\footnote{Verify this yourself if you don't believe me, I'm getting tired.}

Half of row triples have $K=0$. A simple way to see this is that cyclic orderedness and negation of rows can be chosen independently (among all possible triples), and since there are two possibilities for each, there is a $1/4$ chance of any particular state. Since $2$ states are zero, we have $2/4=1/2$ of row triples that have $K=0$ and can be attained. That's $1/2\cdot 48 = 24$ total row triples for a new upper bound.

Let's see if we can construct all $24$ of these triples using $P$ and $Q$. There are two cases: cyclic order preserved and even number of negated rows, and non-cyclic order preserved and odd number of negated rows.

If we can construct $\pm R_1, \pm R_2, \pm R_3$ (where the number of -s is even, totaling $4$ cases) and $\pm R_1, \pm R_3, \pm R_2$ (where the number of -s is odd, totaling $4$ cases), then applying $Q$ iteratively will give us all $(4+4)\cdot 3 = 24$ possible cases. Let's see if this is possible.

Case 1: $\pm R_1, \pm R_2, \pm R_3$, even number of negations.

Subcase a: $0$ negations. This is just the identity, or $Q^3$.
Subcase b: $2$ negations.

Ssubcase i: $-R_1, -R_2, R_3$.
We can get this by applying $P$ twice: this is just $P^2$.
Ssubcase ii: $-R_1, R_2, -R_3$.
We cycle the elements until $R_3$ and $R_1$ are first, then apply $P^2$, then cycle back to the original order. In this case, it is $QP^2Q^2$:
$$(R_1,R_2,R_3)\rightarrow^{Q^2} (R_3,R_1,R_2)\rightarrow^{P^2} (-R_3,-R_1,R_2)\rightarrow^{Q} (-R_1,R_2,-R_3).$$
Subcase iii: $R_1, -R_2, -R_3$.
We apply the same concept as ssubcase ii. In this case, it is $Q^2P^2Q$.
$$(R_1,R_2,R_3)\rightarrow^{Q} (R_2,R_3,R_1) \rightarrow^{P^2} (-R_2,-R_3,R_1) \rightarrow^{Q^2} (R_1, -R_2, -R_3).$$

Case 2: $\pm R_1, \pm R_3, \pm R_2$, odd number of negations.

Subcase a: $1$ negation.
Ssubcase i: $-R_1, R_3, R_2$.
We cycle to $R_3, R_1, R_2$, then apply $P$. This is just $PQ^2$.
Ssubcase ii: $R_1, -R_3, R_2$.
We cycle to $R_2, R_3, R_1$, then apply $P$, then cycle to $R_1, -R_3, R_2$. This is just $Q^2PQ$.
Ssubcase iii: $R_1, R_3, -R_2$.
We apply $P$, then cycle to $R_1, R_3, -R_2$. This is just $QP$.

Subcase b: $3$ negations.
We can get this by taking subcase iii, then left-multiplying by $P^2$, which negates the first two rows. This is just $P^2QP$.

In matrix and row form, these results are summarized like so:

$$I = Q^3 = \threemat{1}{0}{0}{0}{1}{0}{0}{0}{1} \leftrightarrow (R_1,R_2,R_3)$$
$$P^2 = \threemat{-1}{0}{0}{0}{-1}{0}{0}{0}{1} \leftrightarrow (-R_1,-R_2,R_3)$$
$$QP^2Q^2 = \threemat{-1}{0}{0}{0}{1}{0}{0}{0}{1} \leftrightarrow (-R_1,R_2,-R_3)$$
$$Q^2P^2Q = \threemat{1}{0}{0}{0}{-1}{0}{0}{0}{-1} \leftrightarrow (R_1,-R_2,-R_3)$$
$$PQ^2 = \threemat{-1}{0}{0}{0}{0}{1}{0}{1}{0} \leftrightarrow (-R_1,R_3,R_2)$$
$$Q^2PQ = \threemat{1}{0}{0}{0}{0}{-1}{0}{1}{0} \leftrightarrow (R_1,-R_3,R_2)$$
$$QP = \threemat{1}{0}{0}{0}{0}{1}{0}{-1}{0} \leftrightarrow (R_1,R_3,-R_2)$$
$$P^2QP = \threemat{-1}{0}{0}{0}{0}{-1}{0}{-1}{0} \leftrightarrow (-R_1,-R_3,-R_2).$$

Left-multiplying each of these by $I$ (nothing), $Q$, and $Q^2$ yield all $24$ elements of our group. Since this is a lower bound, and we found $24$ to also be an upper bound, this group has order $24$.

The underlying structure is a bit unclear still, and honestly this solution method is very tedious and prone to error. I would't be surprised if I made a silly error somewhere. This next approach will make it a bit clearer though.

\textbf{Approach 2: Geometric Visualization}

We observe matrix $P$ and notice it looks rather like a rotation matrix in 2D, but taken to 3D in the manner we discussed in Mapping the Plane with Matrices. In particular, it is a rotation counterclockwise by $\sin^{-1}1 = 90^\circ = \frac{\pi}{2}$ about the $z$-axis. (Note that rotating about the origin in 2D is rotating about the $z$-axis in 3D.)

What transformation is matrix $Q$ then? Well just as we could apply a matrix to the points $(1,0)$ and $(0,1)$ to understand its behavior in 2D, we can apply a matrix to the points $(1,0,0)$, $(0,1,0)$, and $(0,0,1)$. This tells us that $(1,0,0)$ is mapped to $(0,0,1)$, $(0,1,0)$ is mapped to $(1,0,0)$, and $(0,0,1)$ is mapped to $(1,0,0)$ (we get this by reading off the column vectors). Drawing out this motion makes it clear that it is a rotation, as shown in Figure~\ref{fig:clearly_a_rotation}.

\begin{center}
\begin{asy}[width=0.4\textwidth]
import three;

currentprojection=perspective(2.5,5,4);

draw((-1,0,0)--(1.5,0,0), Arrow3);
draw((0,-1,0)--(0,1.5,0), Arrow3);
draw((0,0,-1)--(0,0,1.5), Arrow3);

triple A = (1,0,0);
triple B = (0,1,0);
triple C = (0,0,1);

dot(A);
dot(B);
dot(C);

triple middle = (A+B+C)/3;

path3 arcA = A..(2*middle-B)..C;
path3 arcB = C..(2*middle-A)..B;
path3 arcC = B..(2*middle-C)..A;

draw(point(arcA,0.2)..point(arcA,1)..point(arcA,1.8), ArcArrow3);
draw(point(arcB,0.2)..point(arcB,1)..point(arcB,1.8), ArcArrow3);
draw(point(arcC,0.2)..point(arcC,1)..point(arcC,1.8), ArcArrow3);

label("$x$", (1.5,0,0), W);
label("$y$", (0,1.5,0), E);
label("$z$", (0,0,1.5), N);

draw((-1,-1,-1)--(1.5,1.5,1.5), dashed);
label("$l$", (1,1,1), N);
\end{asy}
\label{fig:clearly_a_rotation}
\captionof{figure}{$Q$ is a rotation about an interesting axis $l$, the one going between $(0,0,0)$ and $(1,1,1)$.}
\end{center}

Let's try to find an object which has these two rotations as symmetries. An obvious one is the cube, centered at $(0,0,0)$. Arbitrarily, let's let the vertices be $(\pm 1, \pm 1, \pm 1)$. Then these two matrix rotations look like so:

\begin{asydef}
	import three;

	triple A = (1,1,1);
	triple B = (-1,1,1);
	triple C = (-1,-1,1);
	triple D = (1,-1,1);
	triple T = (0,0,-2);
	triple Ap = A + T;
	triple Bp = B + T;
	triple Cp = C + T;
	triple Dp = D + T;

	void drawTheCube() {
		draw(A--B--C--D--cycle);
		draw(A--Ap);
		draw(B--Bp);
		draw(C--Cp);
		draw(D--Dp);
		draw(Ap--Bp--Cp--Dp--cycle);
	}
\end{asydef}

\begin{minipage}{0.3\textwidth}
\begin{asy}[width=0.9\textwidth]
	drawTheCube();

	label("$A$", A, N);
	label("$B$", B, E);
	label("$C$", C, N);
	label("$D$", D, W);
	label("$A'$", Ap, S);
	label("$B'$", Bp, E);
	label("$C'$", Cp, S);
	label("$D'$", Dp, W);
\end{asy}
\captionof{figure}{Rotation $I$.}
\end{minipage}\hfill
\begin{minipage}{0.3\textwidth}
\begin{asy}[width=0.9\textwidth]
	drawTheCube();

	label("$D$", A, N);
	label("$A$", B, E);
	label("$B$", C, N);
	label("$C$", D, W);
	label("$D'$", Ap, S);
	label("$A'$", Bp, E);
	label("$B'$", Cp, S);
	label("$C'$", Dp, W);
\end{asy}
\captionof{figure}{Rotation $P$.}
\end{minipage}\hfill
\begin{minipage}{0.3\textwidth}
\begin{asy}[width=0.9\textwidth]
	drawTheCube();

	label("$A$", A, N);
	label("$D$", B, E);
	label("$D'$", C, N);
	label("$A'$", D, W);
	label("$B$", Ap, S);
	label("$C$", Bp, E);
	label("$C'$", Cp, S);
	label("$B'$", Dp, W);
\end{asy}
\captionof{figure}{$Q$.}
\end{minipage}

So since both generators are valid rotations of the cube, we at least know that this group is a subgroup of the rotation group of the cube, which we previously found to be $S_4$, the permutation group on $4$ elements. To figure out which subgroup, we recall the methodology we used to prove that the rotation group of the cube was isomorphic to $S_4$. We labeled opposite vertices with the same number, going $1$ through $4$. Then each valid rotation of the cube corresponds with a permutation of $(1,2,3,4)$ on a chosen face. In this case, we will use the top face, and consider the vertices counterclockwise starting from the corner facing the camera.

We find which permutations $P$ and $Q$ correspond to:

\begin{minipage}{0.3\textwidth}
\begin{asy}[width=0.9\textwidth]
	drawTheCube();

	label("$1$", A, N);
	label("$2$", B, E);
	label("$3$", C, N);
	label("$4$", D, W);
	label("$3$", Ap, S);
	label("$4$", Bp, E);
	label("$1$", Cp, S);
	label("$2$", Dp, W);
\end{asy}
\captionof{figure}{$I\leftrightarrow (1,2,3,4)$, according to the top face.}
\end{minipage}\hfill
\begin{minipage}{0.3\textwidth}
\begin{asy}[width=0.9\textwidth]
	drawTheCube();

	label("$4$", A, N);
	label("$1$", B, E);
	label("$2$", C, N);
	label("$3$", D, W);
	label("$2$", Ap, S);
	label("$3$", Bp, E);
	label("$4$", Cp, S);
	label("$1$", Dp, W);
\end{asy}
\captionof{figure}{$P\leftrightarrow (4,1,2,3)$.}
\end{minipage}\hfill
\begin{minipage}{0.3\textwidth}
\begin{asy}[width=0.9\textwidth]
	drawTheCube();

	label("$1$", A, N);
	label("$4$", B, E);
	label("$2$", C, N);
	label("$3$", D, W);
	label("$3$", Ap, S);
	label("$4$", Bp, E);
	label("$1$", Cp, S);
	label("$2$", Dp, W);
\end{asy}
\captionof{figure}{$Q\leftrightarrow (1,4,2,3)$.}
\end{minipage}

So $P$ is a cyclic permutation of all four elements (which makes sense, since the top face is just rotating $90^\circ$), and $Q$ is a cyclic permutation of the last three elements. This makes sense; $P$ has period $4$ and $Q$ has period $3$. What permutations can we generate with these two operations though?

We show that we can swap any two adjacent elements. If this is possible, then any permutation can definitely be reached.

First, we show we can swap the first two elements. This is done by the sequence $P^3Q^2$:

$$(1,2,3,4)\rightarrow^{Q^2}(1,3,4,2)\rightarrow^{P^3}(2,1,3,4).$$

Then, to swap elements in positions $i$ and $i+1$, where $1\leq i \leq 3$, we 1. cycle the elements so that the elements once in positions $i$ and $i+1$ are now in positions $1$ and $2$; 2. perform the aforementioned swap; 3. cycle the elements back to their original positions.

For example, suppose we want to swap the second and third elements in $(1,2,3,4)$. Then we start with $P^3$, which makes the sequence $(2,3,4,1)$. Continuing with the swap, $P^3Q^2$, takes us to $(3,2,4,1)$. Then $P$ takes us to $(1,3,2,4)$. Overall, this transformation is $PP^3Q^2P^3$, which actually simplifies to $Q^2P^3$.

A bit more formally, the swap between positions $i$ and $i+1$ is

$$P^{i-1}P^3Q^2P^{5-i}.$$

(Note that $P^0=I$.) In our example, $i=2$.

So we can swap any two elements. This means we can get any permutation\footnote{This is intuitive but we haven't actually proved it. Perhaps I'll add the proof later? Nah.}.

So $\{P,Q\}$ generates the rotation group of the cube, which is the permutation group on $4$ elements $S_4$.

\begin{inner_problem}
\item $\{P,R\}$
\end{inner_problem}

This isn't too hard to do with matrices, but a geometric way is more fun imo. Observe the matrices' resemblance to 2D transformation matrices (we've already observed this for $P$):

$$P=\threemat{0}{-1}{0}{1}{0}{0}{0}{0}{1};\quad R = \threemat{-1}{0}{0}{0}{1}{0}{0}{0}{1}.$$

$P$ is thus a rotation counterclockwise by $90^\circ$, and $R$ is a reflection over the $y$-axis, since the $x$ coordinate is being negated. This makes clear that it is the dihedral group of the square, $D_4$, which has order $8$.

\begin{inner_problem}
\item $\{Q,R\}$
\end{inner_problem}

This is a bit trickier, because $Q$ is not reducible to some plain 2D transformation matrix. Luckily, left-multiplying by $R$ \textit{is} a row operation, so we recall the idea of naming the rows of the identity matrix $R_1$, $R_2$, and $R_3$:

$$\begin{blockarray}{lccc}
\begin{block}{l[ccc]}
R_1 & 1 & 0 & 0 \\
R_2 & 0 & 1 & 0 \\
R_3 & 0 & 0 & 1 \\
\end{block}
\end{blockarray}.$$

We represent $I$ as the ordered triple $(R_1,R_2,R_3)$. Then left-multiplying by $Q$ is the function $f(r_1,r_2,r_3)=(r_3,r_1,r_2)$, and left-multiplying by $R$ is the function $f(r_1,r_2,r_3)=(-r_1,r_2,r_3)$.

It's pretty clear what's going on now. Since we can cycle the rows as we please, and negate any one of them, we have $3$ orders\footnote{Not $6$ orders, since the ``cyclic order'' is preserved. That is, something like $(R_1,R_3,R_2)$ is not achievable.} and $2^3$ possible negation patterns, we have $3\cdot 2^3 = 24$ total elements.

But what is the structure of this group? Is it again, the rotation group of the cube? Let's keep in mind the fact that this group has order $24$ and enter the geometric realm.

We know that $R$ is a reflection of the $x$-coordinate, and thus through the $xy$-plane. $Q$ and $R$ are indeed symmetries of the cube, but since the full symmetry group of the cube (including reflections) has $48$ elements, we need something a bit more restrictive.

This may seem like a leap in logic, but bear with me for a moment. Have you ever looked carefully at a standard volleyball, like the one in Figure~\ref{fig:standard_volleyball}?\footnote{Not to brag, but a funny memory I had in $9$\textsuperscript{th} grade P.E. was looking at a volleyball and explaining to someone---I forget who---how it's similar to a cube. Little did I know how useful that memory would be a year and a half later!} It has six sides, but the sides have an extra feature on them: the seams, which are directional.

\begin{asydef}
	import three;

	real x_proj = 5;
	real y_proj = 2;
	real z_proj = 4;

	triple v1 = (1,1,1) / sqrt(3);
	triple v2 = (-1,1,1) / sqrt(3);
	triple v3 = (-1,-1,1) / sqrt(3);
	triple v4 = (1,-1,1) / sqrt(3);
	triple O = (0,0,0);

	path3 p1 = arc(O,v1,v4);
	path3 p2 = arc(O,v2,v3);
	path3 p3 = arc(O,point(p1,2/3),point(p2,2/3));
	path3 p4 = arc(O,point(p1,4/3),point(p2,4/3));

	// p1 through p4 comprise one-sixth of the volleyball

	transform3[] transforms = {
		rotate(180,(1,0,0)),
		rotate(120,(-1,-1,1)),
		rotate(240,(-1,-1,1)),
		identity4,
		rotate(120,(1,1,1)),
		rotate(240,(1,1,1))};

	int steps = 100;

	void succulentlyDraw(path3 p) {
		for (int i = 0; i < steps; ++i) {
			real frac_along = i / steps * 2;

			path3 seg = point(p,frac_along)--point(p,frac_along + 2/steps);
			triple pt = point(p,frac_along);
			if (x_proj*pt.x + y_proj*pt.y + z_proj*pt.z < 0) {
				draw(seg, dotted+gray(0.75));
			} else {
				draw(seg);
			}
		}
	}

	void drawSurroundingBoundary() {
		triple on_horizon = unit((1/x_proj, 1/y_proj, -2/z_proj));
		triple on_horizon2 = unit((-2/x_proj, 1/y_proj, 1/z_proj));
		draw(arc(O, on_horizon, on_horizon2), dotted);
		draw(arc(O, on_horizon2, -on_horizon), dotted);
		draw(arc(O, -on_horizon, -on_horizon2), dotted);
		draw(arc(O, -on_horizon2, on_horizon), dotted);
	}
\end{asydef}

\begin{minipage}{0.4\textwidth}
\begin{asy}[width=0.9\textwidth]
import three;

currentprojection=orthographic(x_proj,y_proj,z_proj);
for (int i = 0; i < 6; ++i) {
	succulentlyDraw(transforms[i]*p1);
	succulentlyDraw(transforms[i]*p2);
	succulentlyDraw(transforms[i]*p3);
	succulentlyDraw(transforms[i]*p4);
}

drawSurroundingBoundary();
\end{asy}
\label{fig:standard_volleyball}
\captionof{figure}{A standard volleyball, with its interesting striations.}
\end{minipage}\hfill
\begin{minipage}{0.4\textwidth}
\begin{asy}[width=0.9\textwidth]
	import three;

	currentprojection=orthographic(x_proj,y_proj,z_proj);
	transform3 rt = rotate(90,(0,0,1));
	for (int i = 0; i < 6; ++i) {
		succulentlyDraw(transforms[i]*rt*p1);
		succulentlyDraw(transforms[i]*rt*p2);
		succulentlyDraw(transforms[i]*rt*p3);
		succulentlyDraw(transforms[i]*rt*p4);
	}

	draw((0,0,-1.5)--(0,0,2),dashed);
	dot((0,0,-1));
	dot((0,0,1));

	triple cntr = (0,0,1.5);
	real r = 0.2;
	draw(shift(cntr)*((r,0,0)..((r,r,0) / sqrt(2))..(0,r,0)),Arrow3);

	drawSurroundingBoundary();
\end{asy}
\label{fig:bad_rotation}
\captionof{figure}{This rotation of $90^\circ$ is a symmetry of the cube, but \textit{not} of the volleyball.}
\end{minipage}

This is similar to a cube, in that it has six sides, but it has restricted symmetries. For example, we can't rotate it $90^\circ$ about an axis going straight through two opposite faces, since then the seams would not line up. This is shown in Figure~\ref{fig:bad_rotation}.

The motivation for analyzing this shape is that it \textit{does} accept the matrices $R$ and $Q$ as symmetries. $R$, or flipping the volleyball over a midplane between two opposite sides, is a symmetry, as shown in Figure~\ref{fig:symmetry_R}. $Q$, or rotating the volleyball $120^\circ$ around a vertex-vertex axis, is also a symmetry, as shown in Figure~\ref{fig:symmetry_Q};

\begin{minipage}{0.4\textwidth}
\begin{asy}[width=0.9\textwidth]
	y_proj = 2.5;
	z_proj = 3.5;

	import three;

	currentprojection=orthographic(x_proj,y_proj,z_proj);

	for (int i = 0; i < 6; ++i) {
		succulentlyDraw(transforms[i]*p1);
		succulentlyDraw(transforms[i]*p2);
		succulentlyDraw(transforms[i]*p3);
		succulentlyDraw(transforms[i]*p4);
	}

	path3 midplane_boundary = (-1.5,0,1.5)--(1.5,0,1.5)--(1.5,0,-1.5)--(-1.5,0,-1.5)--cycle;
	path3 intersection_c = (0,0,1)..(1,0,0)..(0,0,-1)..(-1,0,0)..cycle;

	draw(surface(midplane_boundary^^intersection_c), evenodd+opacity(0.3)+gray(0.5));
	draw(midplane_boundary, dotted);

	draw(intersection_c, dashed);
	drawSurroundingBoundary();
	draw((-1,0,0)--(-2,0,0),ArcArrow3);
	draw((0,1,0)--(0,1.5,0),ArcArrow3);
	draw((0,0,1)--(0,0,2),ArcArrow3);

	draw(O--(-1,0,0),opacity(0.6)+dotted);
	draw(O--(0,1,0),opacity(0.6)+dotted);
	draw(O--(0,0,1),opacity(0.6)+dotted);

	dot((-1,0,0));
	dot((0,1,0));
	dot((0,0,1));

	label("$y$",(-2,0,0),N);
	label("$x$",(0,1.5,0),E);
	label("$z$",(0,0,2),N);
\end{asy}
\label{fig:symmetry_R}
\captionof{figure}{$R$, a reflection through the $yz$-plane, is a symmetry of the volleyball.}
\end{minipage}\hfill
\begin{minipage}{0.4\textwidth}
\begin{asy}[width=0.9\textwidth]
	y_proj = 1;
	import three;

	currentprojection=orthographic(x_proj,y_proj,z_proj);

	for (int i = 0; i < 6; ++i) {
		succulentlyDraw(transforms[i]*p1);
		succulentlyDraw(transforms[i]*p2);
		succulentlyDraw(transforms[i]*p3);
		succulentlyDraw(transforms[i]*p4);
	}

	draw((-1.2,-1.2,-1.2)--(2,2,2),dashed);

  triple cntr_ish = (1.4,1.4,1.4);
	triple arc_vert1 = cntr_ish + (0,0,0.4);

	draw(arc_vert1..(rotate(-60,cntr_ish)*arc_vert1)..(rotate(-120,cntr_ish)*arc_vert1), Arrow3);

	dot((arc_vert1 + rotate(180,cntr_ish)*arc_vert1)/2);

	dot(v1);
	dot(-v1);
	dot(O);

	drawSurroundingBoundary();
	label("$l$", (2,2,2), E);
\end{asy}
\label{fig:symmetry_Q}
\captionof{figure}{$Q$, a rotation around the axis $l$, is a symmetry of the volleyball.}
\end{minipage}

Indeed, since there are two possible directions for each set of seams, it seems like this should have half the symmetries of the cube. To make this concrete, we observe that there are six faces. Each face can be oriented to face the top, and there are two choices---not four---for its orientation, since it only has bilateral symmetry and not four-fold symmetry like a square. This gives $6\cdot 2$ choices. But we can also mirror the volleyball into the ``mirror world,'' which multiplies the number of choices by $2$. Thus, there are $6\cdot 2\cdot 2=24$ symmetries of the volleyball.

We know that $Q$ and $R$ are symmetries of the volleyball, and, via our matrix logic, generate a group of order $24$. But if they generated a group other than the symmetry group of the volleyball---let's call it $V$ for short---then $V$ isn't closed, since some elements $Q,R\in V$ generate a different group of the same order.

So $\{Q,R\}$ generates the symmetries of a volleyball! I think this is wonderful. For the curious, this is known as \textbf{pyritohedral} symmetry. Spicy. As an abstract group, this is $A_4\times C_2$, where $A_4$ is the alternating group on $4$ elements.

\begin{inner_problem}
\item $\{P,Q,R\}.$
\end{inner_problem}

While this may look terrifying at first, we can reuse lots of information from the previous problems. We know that $P,Q,R$ are all symmetries of the cube (including reflections). Thus, $\{P,Q,R\}$ generates a subgroup of the symmetry group of the cube, which we previously found has order $48$ (and abstract structure $S_4\times C_2$)

We also know that $\{P,Q\}$ generates the rotational group of the cube, which has order $24$. Thus, since $R$ is a plain old reflection, it takes every element of the rotational group to the ``mirror world,'' for a total of $24\cdot 2 = 48$ elements. The only subgroup with order $48$ of a group of order $48$ is the group itself. Therefore, $\{P,Q,R\}$ generates the full symmetry group of the cube, which has order $48$.

\begin{outer_problem}
\item The matrix $\twomat{1}{1}{0}{1}$ produces a shear. What is its inverse---what undoes the shear?
\end{outer_problem}

The matrix $\twomat{1}{-1}{0}{1}$ undoes the shear, since the product of these two matrices is the identity matrix.

\begin{outer_problem}
\item The complex numbers, excluding zero, form a group under multiplication. What set of matrices is isomorphic to the same group under multiplication?
\end{outer_problem}

As we found previously, the set of matrices of the form $\twomat{a}{-b}{b}{a}$ where $a,b\in\mathbb{R}$ under multiplication is isomorphic, since we have the one-to-one correspondence $\twomat{a}{-b}{b}{a}\leftrightarrow a+bi$.

\begin{outer_problem}
\item Does the set of all $2\times 2$ matrices form a group under multiplication? Why or why not?
\end{outer_problem}

No, because there is no matrix $M$ such that $M\twomat{0}{0}{0}{0} = \twomat{1}{0}{0}{1}$; it cannot satisfy invertibility.

\end{document}

\documentclass[../gatm.tex]{subfiles}

\begin{document}

\section{Matrices Generate Groups}
\setcounter{problem_i}{0}

As we have seen, the groups that we examined in the first couple sections of this class have representations with matrices under the operation of matrix multiplication.

Recall that the rotation group of the equilateral triangle could be generated by one element---repeatedly applying a rotation of $120^\circ$. We called this the cyclic group of order $3$, $C_3$ for short. It took two generators to produce the dihedral group of the equilateral triangle---either a rotation and a reflection or two reflections. We called this the dihedral group of order $6$, $D_3$ for short.

In the following problems, you will be examining some of these groups, writing group tables, and determining to which symmetry group each matrix group is isomorphic. Look for patterns. Try to discover the characteristics of each matrix that tell you what it ``does'' geometrically.

For Problems 1-4:
\begin{enumerate}[label=(\alph*)]
\item Specify the elements of the matrix group, unless they are all given.
\item Describe what each matrix does to the plane.
\item Construct a group table; you can use a calculator.
\item Decide which symmetry group your matrix is isomorphic to.
\end{enumerate}
\vspace{1cm}

Let's see an example of this on the following matrices:
$$I=\left[\begin{array}{cc} 1 & 0 \\ 0 & 1 \end{array}\right],\, A=\left[\begin{array}{cc} -1 & 0 \\ 0 & -1 \end{array}\right],\,  B=\left[\begin{array}{cc} 1 & 0 \\ 0 & -1 \end{array}\right],\, C=\left[\begin{array}{cc} -1 & 0 \\ 0 & 1 \end{array}\right].$$

\newcounter{mg_enumi}

\begin{minipage}{0.6\textwidth}
\begin{enumerate}[label=(\alph*)]
\item They are given.
\item $I$ is the identity transformation. $A$ rotates $180^\circ$ (alternatively, it reflects through the origin). $B$ reflects over the $x$ axis. $C$ reflects over the $y$ axis.
\item This group is isomorphic to the symmetry group for the rectangle, otherwise known as the dihedral group of the rectangle, $D_2$ for short.
\setcounter{mg_enumi}{\value{enumi}}
\end{enumerate}
\end{minipage}\hfill
\begin{minipage}{0.25\textwidth}
\begin{enumerate}[label=(\alph*)]
\setcounter{enumi}{\value{mg_enumi}}
\item $\begin{array}{c|c|c|c|c|}
& I & A & B & C \\ \hline
I & I & A & B & C \\ \hline
A & A & I & C & B \\ \hline
B & B & C & I & A \\ \hline
C & C & B & A & I \\ \hline
\end{array}$
\end{enumerate}
\end{minipage}
\vspace{2cm}

Now you can try this for yourself!
\begin{enumerate}
\item Analyze this group with the following elements, following the form of Example 1. What makes this group fundamentally different from the example?
$$I=\left[\begin{array}{cc} 1 & 0 \\ 0 & 1 \end{array}\right], A=\left[\begin{array}{cc} 0 & -1 \\ 1 & 0 \end{array}\right], B=\left[\begin{array}{cc} -1 & 0 \\ 0 & -1 \end{array}\right], C=\left[\begin{array}{cc} -1 & 0 \\ 0 & 1 \end{array}\right]$$
\item The matrix $\left[\begin{array}{cc} -\frac{1}{2} & -\frac{\sqrt{3}}{2} \\ \frac{\sqrt{3}}{2} & -\frac{1}{2}\end{array}\right]$ generates a group of order $3$. Enumerate the elements of this group and analyze per the example.
\item The matrices $\left[\begin{array}{cc} -\frac{1}{2} & -\frac{\sqrt{3}}{2} \\ \frac{\sqrt{3}}{2} & -\frac{1}{2}\end{array}\right]$ and $\left[\begin{array}{cc} 1 & 0 \\ 0 & -1 \end{array}\right]$ generate a group of order $6$, of which the group in problem 2 is a subgroup. Enumerate the elements of the group and analyze per the example.
\begin{enumerate}
\setcounter{enumii}{4}
\item What other sets of two matrices could have generated this group?
\end{enumerate}
\item The matrix $\left[\begin{array}{cc} \frac{\sqrt{5}-1}{4} & -\frac{\sqrt{10+2\sqrt{5}}}{4} \\ \frac{\sqrt{10+2\sqrt{5}}}{4} & \frac{\sqrt{5}-1}{4} \end{array}\right]$ generates a group of order $5$! Enumerate the elements of the group and analyze per the example; you can use a calculator.
\item Let $A=\twomat{\cos\frac{2\pi}{n}}{-\sin\frac{2\pi}{n}}{\sin\frac{2\pi}{n}}{\cos\frac{2\pi}{n}}$, $B=\twomat{\cos\frac{2\pi}{n}}{\sin\frac{2\pi}{n}}{\sin\frac{2\pi}{n}}{-\cos\frac{2\pi}{n}}$, $C=\twomat{1}{0}{0}{-1}$, and $n$ be an integer. What group is generated by the following sets of generators? Describe them geometrically. \begin{multicols}{4}%
\begin{enumerate}%
\item $\{A\}$
\item $\{B\}$
\item $\{A,B\}$
\item $\{B,C\}$
\end{enumerate}%
\end{multicols}%
\item Given $C=\twomat{1}{0}{0}{-1}$ and $D=\twomat{1}{1}{0}{-1}$, what is the order of the group generated by the following sets of generators?
\begin{multicols}{3}
\begin{enumerate}
\item $\{C\}$
\item $\{D\}$
\item $\{C,D\}$
\end{enumerate}
\end{multicols}
\item What matrix could generate a group isomorphic to the cyclic group of order $n$, $C_n$?
\item What set of two matrices could generate a group isomorphic to the dihedral group of order $2n$, $D_n$?
\item Look at Problem~\ref{prob:adjacency_matrices_map_subgroup} on page~\pageref{prob:adjacency_matrices_map_subgroup}. The adjacency matrices map to a subgroup of the full cube symmetry group. What rotations/reflections do they map to?
\item Given $P=\threemat{0}{-1}{0}{1}{0}{0}{0}{0}{1}$, $Q=\threemat{0}{1}{0}{0}{0}{1}{1}{0}{0}$, and $R=\threemat{-1}{0}{0}{0}{1}{0}{0}{0}{1}$, try understanding the groups generated by:
\begin{multicols}{4}
\begin{enumerate}
\item $\{P\}$
\item $\{Q\}$
\item $\{R\}$
\item $\{P,Q\}$
\item $\{P,R\}$
\item $\{Q,R\}$
\item $\{P,Q,R\}.$
\end{enumerate}
\end{multicols}
\item The matrix $\twomat{1}{1}{0}{1}$ produces a shear. What is its inverse---what undoes the shear?
\item The complex numbers, excluding zero, form a group under multiplication. What set of matrices is isomorphic to the same group under multiplication?
\item Does the set of all $2\times 2$ matrices form a group under multiplication? Why or why not?
\end{enumerate}

The following analyses are more in-depth than Problems 1-4.

\subsection*{Analysis 1}

Analyze the group generated by $A=\twomat{\frac{\sqrt{2}}{2}}{-\frac{\sqrt{2}}{2}}{\frac{\sqrt{2}}{2}}{\frac{\sqrt{2}}{2}}$ under multiplication.

The elements are as follows:

$$A= \twomat{\frac{\sqrt{2}}{2}}{-\frac{\sqrt{2}}{2}}{\frac{\sqrt{2}}{2}}{\frac{\sqrt{2}}{2}},\,
A^2 = \twomat{0}{-1}{1}{0},\,
A^3 = \twomat{\frac{\sqrt{2}}{2}}{-\frac{\sqrt{2}}{2}}{\frac{\sqrt{2}}{2}}{-\frac{\sqrt{2}}{2}},\,
A^4 = \twomat{-1}{0}{0}{-1}$$
$$A^5 = \twomat{-\frac{\sqrt{2}}{2}}{\frac{\sqrt{2}}{2}}{-\frac{\sqrt{2}}{2}}{-\frac{\sqrt{2}}{2}},\,
A^6 = \twomat{0}{1}{-1}{0},\,
A^7 = \twomat{\frac{\sqrt{2}}{2}}{\frac{\sqrt{2}}{2}}{-\frac{\sqrt{2}}{2}}{\frac{\sqrt{2}}{2}},\,
A^8 = \twomat{1}{0}{0}{1} = I.$$

In order, these are rotations of $0$, $\frac{\pi}{4}$, $\frac{\pi}{2}$, $\frac{3\pi}{4}$, $\pi$, $\frac{5\pi}{4}$, $\frac{3\pi}{2}$, and $\frac{7\pi}{4}$ radians counterclockwise. $A$, $A^3$, $A^5$, and $A^7$---$A$ to any power relatively prime to $8$---are all generators of the group\footnote{Can you figure out why?}.

This is the cyclic group of order $8$, $C_8$. It is isomorphic to the rotation group of the regular octagon. $A^8=\twomat{1}{0}{0}{1}$ is the identity element.

$$\begin{array}{c|c|c|c|c|c|c|c|c|}
\cdot & I & A & A^2 & A^3 & A^4 & A^5 & A^6 & A^7 \\ \hline
I & I & A & A^2 & A^3 & A^4 & A^5 & A^6 & A^7 \\ \hline
A & A & A^2 & A^3 & A^4 & A^5 & A^6 & A^7 & I \\ \hline
A^2 & A^2 & A^3 & A^4 & A^5 & A^6 & A^7 & I & A \\ \hline
A^3 & A^3 & A^4 & A^5 & A^6 & A^7 & I & A & A^2 \\ \hline
A^4 & A^4 & A^5 & A^6 & A^7 & I & A & A^2 & A^3 \\ \hline
A^5 & A^5 & A^6 & A^7 & I & A & A^2 & A^3 & A^4 \\ \hline
A^6 & A^6 & A^7 & I & A & A^2 & A^3 & A^4 & A^5 \\ \hline
A^7 & A^7 & I & A & A^2 & A^3 & A^4 & A^5 & A^6 \\ \hline
\end{array}$$

\subsection*{Analysis 2}

Analyze the group generated by $B=\twomat{1}{0}{0}{-1}$ and $C=\twomat{0}{-1}{1}{0}$.

The generated matrices are as follows. The third, last row is all duplicates.

$$C=\twomat{0}{-1}{1}{0},\, C^2=\twomat{-1}{0}{0}{-1},\, C^3=\twomat{0}{1}{-1}{0},\, C^4=\twomat{1}{0}{0}{1}=I$$
$$BC=\twomat{0}{-1}{-1}{0},\, BC^2=\twomat{-1}{0}{0}{1},\, BC^3 = \twomat{0}{1}{1}{0},\, B = \twomat{1}{0}{0}{-1}$$
$$CB=\twomat{0}{1}{1}{0}=BC^3,\, C^2B=\twomat{-1}{0}{0}{-1}=C^2,\, C^3B=\twomat{0}{-1}{-1}{0}=BC,\, B^2=\twomat{1}{0}{0}{1}=I$$

$I$ is the identity. $C$ rotates $90^\circ$, $C^2$ rotates $180^\circ$, and $C^3$ rotates $270^\circ$. $B$ reflects over the $x$ axis, $BC$ reflects over line $y=x$, $BC^2$ reflects over the $y$ axis, and $BC^3$ reflects over the line $y=-x$. This group is $D_4$, the symmetry group of the square. It contains the subgroups $C_2$ and $C_4$ once each, and four copies of the subgroup $D_2$.

The group table is shown below. I've used $BC$ instead of $C^3B$, etc., so that each reflection or rotation is denoted by a unique notation.

$$\begin{array}{c|c|c|c|c|c|c|c|c|}
& I & C & C^2 & C^3 & B & BC & BC^2 & BC^3 \\ \hline
I & I & C & C^2 & C^3 & B & BC & BC^2 & BC^3 \\ \hline
C & C & C^2 & C^3 & I & BC^3 & B & BC & BC^2 \\ \hline
C^2 & C^2 & C^3 & I & C & BC^2 & BC^3 & B & BC \\ \hline
C^3 & C^3 & I & C & C^2 & BC & BC^2 & BC^3 & B \\ \hline
B & B & BC & BC^2 & BC^3 & I & C & C^2 & C^3 \\ \hline
BC & BC & BC^2 & BC^3 & B & C^3 & I & C & C^2 \\ \hline
BC^2 & BC^2 & BC^3 & B & BC & C^2 & C^3 & I & C \\ \hline
BC^3 & BC^3 & B & BC & BC^2 & C & C^2 & C^3 & I \\ \hline
\end{array}$$


\end{document}

\documentclass[../gatm_answers.tex]{subfiles}

\begin{document}

\section{Rotation and Reflection Groups}

\begin{outer_problem}[start=1]
\item Notice that the original dihedral group had twice as many elements as the rotation group. Why?
\end{outer_problem}

\noindent(Answers may vary.)

There are a couple ways to think about this, but an intuitive way is to consider a ``mirror world'' of reflection and the ``normal world'' where the orientation in normal. Here, orientation is not absolute orientation, but the difference between clockwise and counterclockwise. For chemistry nerds, it is like chirality. Rotation preserves orientation, but reflection does not. Instead, it takes us between these two ``worlds.'' Thus, it allows twice the number of elements.

\begin{outer_problem}
\item Make and justify a conjecture extending this observation to the dihedral groups of other shapes like rectangles, squares, and hexagons, as well as the symmetry group of the cube.
\end{outer_problem}

\noindent(Answers may vary.)

Conjecture: The dihedral groups of a shape has twice the order of its rotation group.

Informal Justification: A shape can be flipped or not, and it can have whatever rotational isometries applied to it whether it's flipped or not. Thus, the dihedral group allows for twice the number of elements as the rotation group.

\begin{outer_problem}
\item Let $r$ be a $180^{\circ}$ rotation, $x$ be a reflection over the $x$-axis, and $y$ be a reflection over the $y$-axis. Write a table for the dihedral group of the rectangle, recalling that the allowed isometries are reflections and rotations. How does this table differ from the dihedral group of the equilateral triangle? \label{prob:rectangle_group}
\end{outer_problem}

\begin{figure}[h]
	\begin{center}
		\begin{minipage}[b]{\textwidth}
			\centering
			\begin{tabular}{c|cccc}
				\hline
				$\cdot$ & $I$ & $r$ & $x$ & $y$ \\ \hline
				\rowcolor{light-gray}
				$I$ & $I$ & $r$ & $x$ & $y$ \\
				$r$ & $r$ & $I$ & $y$ & $x$ \\
				\rowcolor{light-gray}
				$x$ & $x$ & $y$ & $I$ & $r$ \\
				$y$ & $y$ & $x$ & $r$ & $I$ \\ \hline
			\end{tabular}
			\vspace*{0.5\baselineskip}
		\end{minipage}
	\end{center}
	\vspace*{-\baselineskip}%just for kicks (no caption i guess makes it 1 not 2? note to check this later)
\end{figure}

\noindent The table is shown above. The four elements are shown acting on a rectangle with ``P'' painted on it in Figure~\ref{fig:p_rectangle} to show the transformation a bit better.

\begin{figure}[h]
	\begin{center}
		\begin{minipage}[b]{\textwidth}
			\centering
			\begin{asy}[width=0.5\textwidth]
			path rect = (0,0)--(2,0)--(2,1)--(0,1)--cycle;
			
			path P_outer = shift(1-0.35/2,0.5-0.5/2)*scale(0.1)*((0,0)--(1,0)--(1,2.5)--(2.5,2.5)..(3.5,3.75)..(2.5,5)--(0,5)--cycle);
			path P_inner = shift(1-0.35/2,0.5-0.5/2)*scale(0.1)*((1,3.25)--(2,3.25)..(2.5,3.75)..(2,4.25)--(1,4.25)--cycle);
			
			pair lp = (1,-0.5);
			
			draw(rect);
			filldraw(P_outer, gray(0.5));
			filldraw(P_inner, white);
			label("$I$", lp);
			
			transform tr = shift(5,0)*scale(-1,1);
			
			draw(tr*rect);
			filldraw(tr*P_outer, gray(0.5));
			filldraw(tr*P_inner, white);
			label("$y$", shift(3,0)*lp);
			
			draw((4,1.3)--(4,-0.3),dashed);
			
			tr = shift(0,-1.5)*scale(1,-1);
			
			draw(tr*rect);
			filldraw(tr*P_outer, gray(0.5));
			filldraw(tr*P_inner, white);
			label("$x$", shift(0,-2.5)*lp);
			
			draw((-0.3,-2)--(2.3,-2),dashed);
			
			tr = shift(5,-1.5)*rotate(180);
			
			draw(tr*rect);
			filldraw(tr*P_outer, gray(0.5));
			filldraw(tr*P_inner, white);
			label("$r$", shift(3,-2.5)*lp);
			\end{asy}
		\end{minipage}
	\end{center}
	\vspace*{-2\baselineskip}
	\begin{center}
		\begin{minipage}[t]{\textwidth}
			\captionof{figure}{A rectangle AMBULATES and FLIPS around.}
			\label{fig:p_rectangle}
		\end{minipage}
	\end{center}
	\vspace*{-2\baselineskip}
\end{figure}

\noindent This differs from the dihedral group of the equilateral triangle, $D_3$, in several ways. The most obvious is that there are only $4$ elements. Also, all elements besides $I$ in this group have a period of $2$, while $D_3$ has two elements with a period of $3$.

\begin{outer_problem}
\item Write a table for the rotation group of the square, with $4$ elements and $16$ entries. Compare this table to Problem~\ref{prob:rectangle_group}.
\end{outer_problem}

\begin{figure}[h]
	\begin{center}
		\begin{minipage}[b]{\textwidth}
			\centering
			\begin{tabular}{c|cccc}
				\hline
				$\cdot$ & $I$ & $r$ & $r^2$ & $r^3$ \\ \hline
				\rowcolor{light-gray}
				$I$ & $I$ & $r$ & $r^2$ & $r^3$ \\
				$r$ & $r$ & $r^2$ & $r^3$ & $I$ \\
				\rowcolor{light-gray}
				$r^2$ & $r^2$ & $r^3$ & $I$ & $r$ \\
				$r^3$ & $r^3$ & $I$ & $r$ & $r^2$ \\ \hline
			\end{tabular}
			\vspace*{0.5\baselineskip}
		\end{minipage}
	\end{center}
	\vspace*{-2\baselineskip}
\end{figure}% why tf cant this go BELOW the next sentence? well i dont have enough time to fix so above it goes

\noindent The elements are $I=r_0$, $r=r_{90}$, $r^2=r_{180}$, and $r^3=r_{270}$. The table is shown above.

While this has the same order as the rectangle's dihedral group, it has a different structure. There are two elements with period $4$ ($r$, $r^3$) and one element with period $2$ ($r^2$).\\

\noindent\textbf{\textit{For each of the following problems, find the following:}}

\begin{enumerate}[label=(\alph*)]
\item The number of elements; this is known as the \textbf{order}. More formally known as \textbf{cardinality}
\item If order $< 10$, name the set of elements; otherwise, explain how you know the order
\item A smallest possible \textbf{generating set}; in other words, a list of elements which generate a group\footnote{There may be multiple generating sets of the same size.}
\item Whether the group is \textbf{commutative}; in other words, whether its operation $\cdot$ satisfies $X\cdot Y=Y\cdot X$ for all $X,Y$
\end{enumerate}

\begin{outer_problem}
\item Rectangle under rotation
\end{outer_problem}

\begin{inner_problem}[start=1]
\item Number of elements
\end{inner_problem}

\noindent This group has two elements, the identity and the rotation of $180^\circ$.

\begin{inner_problem}
\item If order $< 10$, the set of elements; otherwise, an explanation of how you know the order
\end{inner_problem}

\noindent As stated, they are the identity $I$ and the rotation $r$ of $180^\circ$, as shown in Figure~\ref{fig:rect_rot}.

\begin{figure}[h]
	\begin{center}
		\begin{minipage}[b]{\textwidth}
			\centering
			\begin{asy}[width=0.5\textwidth]
			pair A = (0,0);
			pair B = (2,0);
			pair C = (2,1);
			pair D = (0,1);
			
			draw(A--B--C--D--cycle);
			
			dot(A);
			dot(B);
			dot(C);
			dot(D);
			
			label("$A$", A, SW);
			label("$B$", B, SE);
			label("$C$", C, NE);
			label("$D$", D, NW);
			
			label("$I$", (1,-0.5));
			
			transform so = shift(3,0);
			
			pair Ap = so*A;
			pair Bp = so*B;
			pair Cp = so*C;
			pair Dp = so*D;
			
			draw(Ap--Bp--Cp--Dp--cycle);
			
			label("$C'$", Ap, SW);
			label("$D'$", Bp, SE);
			label("$A'$", Cp, NE);
			label("$B'$", Dp, NW);
			
			label("$r$", so*(1,-0.5));
			\end{asy}
		\end{minipage}
	\end{center}
	\vspace*{-2\baselineskip}
	\begin{center}
		\begin{minipage}[t]{\textwidth}
			\captionof{figure}{Rectangle under rotation.}
			\label{fig:rect_rot}
		\end{minipage}
	\end{center}
	\vspace*{-2\baselineskip}
\end{figure}

\begin{inner_problem}
\item A smallest possible \textbf{generating set}
\end{inner_problem}

\noindent The smallest possible generating set is the singleton $\{r\}$.

\begin{inner_problem}
\item Whether the group is \textbf{commutative}
\end{inner_problem}

\noindent The group is commutative, since it's only comprised of rotations, which commute.

\begin{outer_problem}
\item Rectangle under reflection
\end{outer_problem}

\noindent We already considered this in Problem 3.

\begin{inner_problem}[start=1]
\item Number of elements
\end{inner_problem}

\noindent There are $4$ elements in this group.

\begin{inner_problem}
\item If order $< 10$, the set of elements; otherwise, an explanation of how you know the order
\end{inner_problem}

\noindent The elements are the identity $I$, rotation $r$ by $180^\circ$, reflection $x$ over the $x$ axis, and reflection $y$ over the $y$ axis.

\begin{inner_problem}
\item A smallest possible \textbf{generating set}
\end{inner_problem}

\noindent (Answers may vary.)

$\{r,x\}$, $\{r,y\}$, and $\{x,y\}$ all generate the group. No single element, however, can generate the group.

\begin{inner_problem}
\item Whether the group is \textbf{commutative}
\end{inner_problem}

\noindent This group is commutative.

\begin{outer_problem}
\item Square under rotation
\end{outer_problem}

\noindent Again, we have considered this group before.

\begin{inner_problem}[start=1]
\item Number of elements
\end{inner_problem}

\noindent There are $4$ elements.

\begin{inner_problem}
\item If order $< 10$, the set of elements; otherwise, an explanation of how you know the order
\end{inner_problem}

\noindent The elements are rotations $I=r_0$, $r=r_{90}$, $r^2=r_{180}$, and $r^3=r_{270}$.

\begin{inner_problem}
\item A smallest possible \textbf{generating set}
\end{inner_problem}

\noindent (Answers may vary.)

Both $\{r\}$ and $\{r^3\}$ generate the group, because $1,3$ are coprime to $4$.

\begin{inner_problem}
\item Whether the group is \textbf{commutative}
\end{inner_problem}

\noindent The group is commutative, since it consists of all rotations.

\begin{outer_problem}
\item Square under reflection
\end{outer_problem}

\begin{inner_problem}[start=1]
\item Number of elements
\end{inner_problem}

\noindent There are $8$ elements in this group. We can quickly see this by noting that it is the dihedral group of the square, which has twice the order of the rotation group of the square. We just found that had $4$ elements, and $2\cdot 4=8$.

\begin{inner_problem}
\item If order $< 10$, the set of elements; otherwise, an explanation of how you know the order
\end{inner_problem}

\noindent The elements are as follows:

Rotations $I=r_0$, $r=r_{90}$, $r^2=r_{180}$, and $r^3=r_{270}$;
reflections $f=$ flip over the $x$-axis, $fr=r$ then $f$, $fr^2$ and $fr^3$.

Recall that rotations can be generated by a sequence of two reflections.

Each of these elements is shown in Figure~\ref{fig:square_reflect}.

\begin{figure}[h]
	\begin{center}
		\begin{minipage}[b]{\textwidth}
			\centering
			\begin{asy}[width=0.7\textwidth]
			string[][] elements = {
			{"I","A","B","C","D"},
			{"r","D","A","B","C"},
			{"r^2","C","D","A","B"},
			{"r^3","B","C","D","A"},
			{"f","D","C","B","A"},
			{"fr","C","B","A","D"},
			{"fr^2","B","A","D","C"},
			{"fr^3","A","D","C","B"}
			};
			
			pair getTL(int i) {
				return (2*(i%4),-2.5*(i#4));
			}
			
			pair A = (0,0);
			pair B = (1,0);
			pair C = (1,1);
			pair D = (0,1);
			pair labl = (0.5,-0.4);
			string d = "$";
			
			path sq = A--B--C--D--cycle;
			
			for (int i = 0; i < 8; ++i) {
				string[] element = elements[i];
				pair tl = getTL(i);
				transform tlshf = shift(tl);
			
				draw(tlshf*sq);
				label(d+element[1]+d, tlshf*A, SW);
				label(d+element[2]+d, tlshf*B, SE);
				label(d+element[3]+d, tlshf*C, NE);
				label(d+element[4]+d, tlshf*D, NW);
				label(d+element[0]+d, tlshf*labl);
			
				if (i >= 4) {
					// flip element; let's draw a flip line!
			
					if (i % 2 == 0) { // across x or y axis
						draw(tlshf*rotate(i%4 * 45, (0.5,0.5))*((-0.3,0.5)--(1.3,0.5)),dashed);
					} else {
						draw(tlshf*rotate((i-1)%4 * 45, (0.5,0.5))*((1.3,-0.3)--(-0.3,1.3)),dashed);
					}
				}
			}
			\end{asy}
		\end{minipage}
	\end{center}
	\vspace*{-2\baselineskip}
	\begin{center}
		\begin{minipage}[t]{\textwidth}
			\captionof{figure}{Reflections of a square.}
			\label{fig:square_reflect}
		\end{minipage}
	\end{center}
	\vspace*{-2\baselineskip}
\end{figure}

\begin{inner_problem}
\item A smallest possible \textbf{generating set}
\end{inner_problem}

\noindent(Answers may vary.)

Any pair of a rotation and flip will generate the set, except for $\{r^2,fr^2\}$ and $\{r^2,f\}$; these will produce the rectangle group instead. Any pair of two flips, except for $\{f,fr^2\}$, will also work. As an example of both of these categories, both $\{r^2,fr^3\}$ and $\{f,fr\}$ will generate the group.

\begin{inner_problem}
\item Whether the group is \textbf{commutative}
\end{inner_problem}

\noindent This group is not commutative. For example, $fr=fr$, but $rf=fr^3$.

\begin{outer_problem}
\item Square prism under rotation
\end{outer_problem}

\noindent This group is isomorphic to the dihedral group of the square in Problem 8.

\begin{inner_problem}[start=1]
\item Number of elements
\end{inner_problem}

\noindent This is a bit more difficult than the previous questions, because we need to understand what elements are possible. We can rotate the prism about its central axis, which an action analogous to just rotating a square: $4$ elements. But we can also rotate the prism $180^\circ$ on a axis through the middle (pictures are shown in the next subpart). This switches the top square face with the bottom face, giving another $4$ elements. In total, we have $8$ elements.

\begin{inner_problem}
\item If order $< 10$, the set of elements; otherwise, an explanation of how you know the order
\end{inner_problem}

\noindent The set of elements are shown in Figure~\ref{fig:this_took_me_so_long} below. Let $a$ be a rotation of $90^\circ$ counterclockwise---as viewed from the top---around the central axis, going through the centers of both square faces; let $b$ be a rotation of $180^\circ$ around an axis going through the centers of faces $\square ABB'A'$ and $\square DCC'D'$.

\begin{asydef}
import three;

string d = "$"; // $

void draw_rect_prism(string[] labels, triple loc=(0,0,0), bool midplane_oof) {
	triple A = (0,0,0);
	triple B = (0,1,0);
	triple C = (-1,1,0);
	triple D = (-1,0,0);

	transform3 sd = shift(0,0,-2);

	triple Ap = sd*A;
	triple Bp = sd*B;
	triple Cp = sd*C;
	triple Dp = sd*D;

	if (!midplane_oof) {
		draw(A--B--C--D--A);
		draw(A--Ap--Bp--B);
		draw(Bp--Cp--C);
		draw(Ap--Dp--D, dashed);
		draw(Dp--Cp, dashed);
	} else {
		real rat = 0.9;

		draw(A--(A+D)/2);
		draw(A--B--(B+C)/2);
		draw(D--C);
		draw(A--Ap--Bp--B);
		draw(C--Cp);
		draw(Ap--Dp,dashed);
		draw(Bp--(Bp+Cp)/2);
		draw(D--Dp,dashed);
		draw(Ap--Dp--Cp,dashed);

		draw((Bp+Cp)/2--point(Bp--Cp,rat),dashed);
		draw(point(Bp--Cp,rat)--Cp);
		draw((B+C)/2--point(B--C,rat),dashed);
		draw(point(B--C,rat)--C);
		draw((A+D)/2--point(A--D,rat),dashed);
		draw(point(A--D,rat)--D);
	}

	label(d + labels[0] + d, A, NW);
	label(d + labels[1] + d, B, N);
	label(d + labels[2] + d, C, NE);
	label(d + labels[3] + d, D, N);

	label(d + labels[4] + d, Ap, SW);
	label(d + labels[5] + d, Bp, SE);
	label(d + labels[6] + d, Cp, SE);
	label(d + labels[7] + d, Dp, S);
}

void draw_rect_prism_c(string elem, string[] labels, bool midplane_oof) {
	draw_rect_prism(labels, (0,0,0), midplane_oof);
	if (substr(elem,0,1) == "a") { // Vertical rotation
		int deg = length(elem) == 1 ? 1 : (int) substr(elem,2,1);
		real ang = deg*pi/2;

		dot((-0.5,0.5,0));
		dot((-0.5,0.5,-2));

		draw((-0.5,0.5,0.2) -- (-0.5,0.5,-2.2), dotted);
		draw(((-0.3,0.5,0)..(-0.5,0.5,0) + 0.2 * (cos(ang/2), sin(ang/2), 0)..(-0.5,0.5,0) + 0.2 * (cos(ang), sin(ang), 0)), Arrow3);
	}

	triple[] diag_axes = {(0,0,-1),(0,1,-1),(-1,1,-1),(-1,0,-1)};
	triple[] cntr_axes = {(0,0.5,-1),(-0.5,1,-1),(-1,0.5,-1),(-0.5,0,-1)};

	if (substr(elem,0,1) == "b") { // Horizontal rotation
		int deg = length(elem) == 1 ? 0 : (length(elem) == 2 ? 1 : (int) substr(elem,3,1));
		triple a1,a2;

		path3 rotate180 = shift(0,0.5,-1) * ((0,-0.2,0)..(0,0,0.2)..(0,0.2,0));

		if (deg % 2 == 0) { // Center rotation
			a1 = cntr_axes[deg # 2];
			a2 = cntr_axes[2+deg # 2];
		} else { // Diagonal rotation
			a1 = diag_axes[(deg - 1)#2];
			a2 = diag_axes[2+(deg-1)#2];
		}

		dot(a1);
		dot(a2);

		draw((1.4*a1-0.4*a2)--(1.4*a2-0.4*a1),dotted);
		draw(rotate((45 * (deg-2*(deg%2))), (-0.5,0.5,0), (-0.5,0.5,1)) * shift((deg%2) * (sqrt(2)/2 - 0.5), 0, 0) * rotate180, Arrow3);

	}
}

string[] elems = {"I","a","a^2","a^3","b","ba","ba^2","ba^3"};
string[][] labels_list = {
{"A","B","C","D","A'","B'","C'","D'"},
{"D","A","B","C","D'","A'","B'","C'"},
{"C","D","A","B","C'","D'","A'","B'"},
{"B","C","D","A","B'","C'","D'","A'"},
{"B'","A'","D'","C'","B","A","D","C"},
{"A'","D'","C'","B'","A","D","C","B"},
{"D'","C'","B'","A'","D","C","B","A"},
{"C'","B'","A'","D'","C","B","A","D"}
};

void draw_elem(int indx, bool midplane_oof = false) {
	draw_rect_prism_c(elems[indx],labels_list[indx],midplane_oof);
	label(d + elems[indx] + d, (-0.5,0.5,-2.8));
}
\end{asydef}

% I don't want to manually do this, but Asymptote forces me to
% yeah asymptote sux bro we shoulda just used HTML, CSS, MathJax, and SVG to make this now THAT would have been a genius move
\begin{figure}[h]
	\begin{center}
		\begin{minipage}[b]{0.3\textwidth}
			\centering
			\begin{asy}[width=0.8\textwidth]
				import three;
				currentprojection = orthographic(5,3,4);
				draw_elem(0);
			\end{asy}
		\end{minipage}
		\hfill
		\begin{minipage}[b]{0.3\textwidth}
			\centering
			\begin{asy}[width=0.8\textwidth]
				import three;
				currentprojection = orthographic(5,3,4);
				draw_elem(1);
			\end{asy}
		\end{minipage}
		\hfill
		\begin{minipage}[b]{0.3\textwidth}
			\centering
			\begin{asy}[width=0.8\textwidth]
				import three;
				currentprojection = orthographic(5,3,4);
				draw_elem(2);
			\end{asy}
		\end{minipage}
	\end{center}

	\begin{center}
		\begin{minipage}[b]{0.3\textwidth}
			\centering
			\begin{asy}[width=0.8\textwidth]
				import three;
				currentprojection = orthographic(5,3,4);
				draw_elem(3);
			\end{asy}
		\end{minipage}
		\hfill
		\begin{minipage}[b]{0.3\textwidth}
			\centering
			\begin{asy}[width=0.8\textwidth]
				import three;
				currentprojection = orthographic(5,3,4);
				draw_elem(4);
			\end{asy}
		\end{minipage}
		\hfill
		\begin{minipage}[b]{0.3\textwidth}
			\centering
			\begin{asy}[width=0.9\textwidth]
				import three;
				currentprojection = orthographic(5,3,4);
				draw_elem(5);
			\end{asy}
		\end{minipage}
	\end{center}
\end{figure}

\begin{figure}[h]
	\begin{center}
		\begin{minipage}[b]{0.45\textwidth}
			\centering
			\begin{asy}[width=0.65\textwidth]
				import three;
				currentprojection = orthographic(5,3,4);
				draw_elem(6);
			\end{asy}
		\end{minipage}
		\hfill
		\begin{minipage}[b]{0.45\textwidth}
			\centering
			%im so fcking done with asymptote WHY CANT THIS THING RESIZE OMFG
			%after two hours on just this one figure im legit at my wits end
			%mostly cuz it takes about 30 minutes for this to build
			\begin{asy}[width=0.7\textwidth]
				import three;
				currentprojection = orthographic(5,3,4);
				draw_elem(7);
			\end{asy}
		\end{minipage}
	\end{center}
	\vspace*{-2\baselineskip}
	\begin{center}
		\begin{minipage}[t]{\textwidth}
			\captionof{figure}{The elements of the rotation group of the rectangular prism.}
			\label{fig:this_took_me_so_long}
		\end{minipage}
	\end{center}
	\vspace*{-2\baselineskip}
\end{figure}

\begin{inner_problem}
\item A smallest possible \textbf{generating set}
\end{inner_problem}

\noindent(Answers may vary.)
The elements with $b$ in their name are equivalent to the reflections in the dihedral group of the square. Thus, we need a ``reflection'' $ba^n$ and a rotation $a^m$, or two separate reflections. All such pairs work except for $\{a^2,ba^2\}$, $\{a^2,b\}$ and $\{b,ba^2\}$. An example from each category: $\{a,b\}$, $\{b,ba\}$.

\begin{inner_problem}
\item Whether the group is \textbf{commutative}
\end{inner_problem}

\noindent This group is not commutative. For example, $ba=ba$, but $ab=ba^3$.

\begin{outer_problem}
\item Square prism under reflection
\end{outer_problem}

\begin{inner_problem}[start=1]
\item Number of elements
\end{inner_problem}

\noindent If the previous group---the rotation group of the square prism---had $8$ elements, then this group should have $16$ elements.

\begin{inner_problem}
\item If order $< 10$, the set of elements; otherwise, an explanation of how you know the order
\end{inner_problem}

\noindent We know the order because the previous group has $8$ elements, and dihedral groups have twice the number of elements of the rotation group, this group has $16$ elements.

\begin{inner_problem}
\item A smallest possible \textbf{generating set}
\end{inner_problem}

\noindent(Answers may vary significantly.)

Since we could generate the previous group with (most) pairs of $\{ba^n,a^m\}$, or (most) pairs of $\{a^n,a^m\}$, we could just add another element $c$ which is a true geometric reflection about, say, the midplane $P$ between $\square DCD'C'$ and $\square ABB'A$ as shown in Figure~\ref{fig:midplane_elem_c}.

Thus, $\{a,b,c\}$ can generate the group. You can prove that two generators are impossible, but the proof either requires making the group table or some more sophisticated abstract algebra. I will give the latter for those who are well-versed in group theory already, but it will probably be inaccessible to most.

The rotation group generated by $\{a,b\}$ is $D_4$. Define a new element $d$ which is the reflection through the midplane $P$ between the two square faces.\footnote{For the curious, $d=cba^2$.} This is crudely shown in Figure~\ref{fig:midplane_elem_d}; I couldn't be bothered to make a nicer figure. Then the reflection group generated by $\{d\}$ is $Z_2$. Furthermore, the operation sets $\{a,b\}$ and $\{d\}$ are separable, in that $a^xb^yd=da^xb^y\quad$\footnote{This can be shown concretely by simply showing geometrically that $ad=da$ and $bd=db$.}. Thus, the group $G$ described in this problem is (isomorphic to) the direct product:

$$G\cong D_4\times Z_2.$$

\begin{figure}[h]
	\begin{center}
		\begin{minipage}[b]{\textwidth}
			\centering
			$$\vcenter{\hbox{
			\begin{asy}
			size(200);
			import three;
			currentprojection = orthographic(5,4,4);
			draw_elem(0,true);
			
			path3 midplane_boundary = (-0.5,-0.3,0.3)--(-0.5,1.3,0.3)--(-0.5,1.3,-2.3)--(-0.5,-0.3,-2.3)--cycle;
			
			draw(midplane_boundary,dotted);
			draw(surface(midplane_boundary),opacity(0.4)+gray(0.4));
			
			label("$M$", point(midplane_boundary,2), S);
			
			dot((-0.5,0,0));
			dot((-0.5,1,0));
			dot((-0.5,1,-2));
			dot((-0.5,0,-2));
			\end{asy}
			}}
			\mathop{\longrightarrow}^\text{reflection}
			\vcenter{\hbox{
			\begin{asy}
			size(200);
			import three;
			
			string[] vertices = {"D","C","B","A","D'","C'","B'","A'"};
			
			currentprojection = orthographic(5,4,4);
			draw_rect_prism_c("oof", vertices, false);
			label(d + "$c$" + d, (-0.5,0.5,-2.8));
			\end{asy}
			}}
			$$
		\end{minipage}
	\end{center}
	\vspace*{-\baselineskip}
	\begin{center}
		\begin{minipage}[t]{\textwidth}
			\captionof{figure}{3D reflection over the midplane $M$ is $c$.}
			\label{fig:midplane_elem_c}
		\end{minipage}
	\end{center}

	\begin{center}
		\begin{minipage}[b]{\textwidth}
			\centering
			$$
			\vcenter{\hbox{
			\begin{asy}
			size(200);
			import three;
			currentprojection = orthographic(5,4,4);
			draw_elem(0);
			
			path3 midplane_boundary = (0.3,-0.3,-1)--(-1.3,-0.3,-1)--(-1.3,1.3,-1)--(0.3,1.3,-1)--cycle;
			
			draw(midplane_boundary,dotted);
			draw(surface(midplane_boundary),opacity(0.4)+gray(0.4));
			
			label("$P$", point(midplane_boundary,0), W);
			
			dot((0,0,-1));
			dot((-1,1,-1));
			dot((-1,0,-1));
			dot((0,1,-1));
			\end{asy}
			}}
			\mathop{\longrightarrow}^\text{reflection}
			\vcenter{\hbox{
			\begin{asy}
			size(200);
			import three;
			
			string[] vertices = {"A'","B'","C'","D'","A","B","C","D"};
			
			currentprojection = orthographic(5,4,4);
			draw_rect_prism_c("oof", vertices, false);
			label(d + "$d$" + d, (-0.5,0.5,-2.8));
			\end{asy}
			}}
			$$
		\end{minipage}
	\end{center}
	\vspace*{-\baselineskip}
	\begin{center}
		\begin{minipage}[t]{\textwidth}
			\captionof{figure}{$d$ is the reflection through midplane $P$.}
			\label{fig:midplane_elem_d}
		\end{minipage}
	\end{center}
	\vspace*{-2\baselineskip}
\end{figure}

\noindent We wish to show that $Z_2\times Z_2\times Z_2$ is a quotient of this group. That is, we wish to find a normal subgroup $N$ such that

$$G/N = Z_2\times Z_2\times Z_2.$$

\noindent If this is true, then the minimal generating set of $G$ has at least cardinality $3$. All that remains is to find $N$ and $G/N$.

It suffices to show that $Z_2\times Z_2\triangleleft D_4$, since then $Z_2\times Z_2\times Z_2\triangleleft D_4\times Z_2.$ We have $|Z_2\times Z_2| = 2^2 = 4$, so we want $|D_4/N|=4$. We know $|D_4|=8$, so by Lagrange's theorem, $|N|=2$.

\noindent A normal subgroup of $D_4$ is $N=\{1,a^2\}$. It is normal because for $x\in\{0,1,2,3\}$ and $y\in\{0,1\}$:

\begin{align*}
(b^xa^y)a^2(b^xa^y)^{-1} &= (b^xa^y)a^2(a^{-y}b^{-x}) \\
&= b^xa^{2+y-y}b^{-x} \\
&= b^xa^2b^{-x} \\
&= b^xb^{-x}a^2 \\
&= a^2 \in \{1,a^2\}.
\end{align*}

\noindent The corresponding quotient group is

$$D_4/N = \{\{1,a^2\}, \{a,a^3\}, \{b,ba^2\}, \{ba, ba^3\}\}.$$

\noindent We have the isomorphism $\{b^xa^{y},b^xa^{y+2}\}\leftrightarrow (x,y)$ under the operation of element-wise addition modulo $2$. After all,

$$\{b^{x_1}a^{y_1},b^{x_1}a^{y_1+2}\}\cdot \{b^{x_2}a^{y_2},b^{x_1}a^{y_2+2}\} = \{b^{x_1+x_2}a^{y_1+y_2},b^{x_1+x_2}a^{y_1+y_2+2}\}.$$

\noindent Therefore, $$D_4/N \cong Z_2 \times Z_2,$$ so $$Z_2 \times Z_2 \times Z_2 \triangleleft D_4\times Z_2=G.$$ Since the minimal generating set of $Z_2\times Z_2\times Z_2$ is $3$, $G$'s generating set is at least $3$. But we've already found the set $\{a,b,c\}$ which generates $G$!\footnote{$\{a,b,d\}$ also generates $G$.} Thus, it is minimal.

\begin{inner_problem}
\item Whether the group is \textbf{commutative}
\end{inner_problem}

\noindent As we found in the previous problem, the rotation group of the square prism is not commutative, and since that's a subgroup of this group, this group certainly isn't commutative.

\begin{outer_problem}
\item Regular pentagon under rotation
\end{outer_problem}

\begin{inner_problem}[start=1]
\item Number of elements
\end{inner_problem}

\noindent This is just the cyclic group of order $5$, so there are $5$ elements.

\begin{inner_problem}
\item If order $< 10$, the set of elements; otherwise, an explanation of how you know the order
\end{inner_problem}

\noindent The elements are rotations of $I=r_0$, $r=r_{72}$, $r^2=r_{144}$, $r^3=r_{216}$, $r^4=r_{288}$. They are shown below.

\begin{figure}[h]
	\begin{center}
		\begin{minipage}[b]{\textwidth}
			\centering
			\begin{asy}[width=0.9\textwidth]
			void draw_pentagon(string[] labels, pair pos=(0,0)) {
				pair prev_pt;
				for (int i = 0; i < 6; ++i) {
					real ang = pi/2 + i * 2 * pi / 5;
					pair cist = (cos(ang),sin(ang));
					pair vertex = pos + (cos(ang),sin(ang));
			
					if (i != 0) {
						draw(vertex--prev_pt);
						label("$" + labels[i % 5] + "$", vertex, cist);
					}
			
					prev_pt = vertex;
				}
			}
			
			string[] vertices = {"A","B","C","D","E"};
			string[] labels = {"I","r","r^2","r^3","r^4"};
			
			real r = 0.5;
			
			for (int i = 0; i < 5; ++i) {
				draw_pentagon(vertices, (3*i,0));
				label("$"+labels[i]+"$", (3*i,-1), S);
			
				real ang = i * 2 * pi / 5;
			
				if (i != 0) draw(shift(3*i,0)*((r,0)..r*(cos(ang/2),sin(ang/2))..r*(cos(ang),sin(ang))), Arrow);
			
				vertices.insert(0, vertices[4]);
				vertices.delete(5);
			}
			\end{asy}
		\end{minipage}
	\end{center}
	\vspace*{-\baselineskip}
\end{figure}

\noindent Pentagons should always wear helmets, lest they want to damage their vertices.

\begin{inner_problem}
\item A smallest possible \textbf{generating set}
\end{inner_problem}

\noindent Any rotation by itself $\{r^n\}$ works, since $5$ is a prime.

\begin{inner_problem}
\item Whether the group is \textbf{commutative}
\end{inner_problem}

\noindent The group is indeed commutative, since all operations are rotations.

\begin{outer_problem}
\item Regular pentagon under reflection
\end{outer_problem}

\begin{inner_problem}[start=1]
\item Number of elements
\end{inner_problem}

\noindent This is the dihedral group of the pentagon, which has $2\cdots 5=10$ elements.

\begin{inner_problem}
\item If order $< 10$, the set of elements; otherwise, an explanation of how you know the order
\end{inner_problem}

\noindent We know the order because it should have twice the number of elements as the rotation group, which has $5$ elements, giving $10$ elements total.

\begin{inner_problem}
\item A smallest possible \textbf{generating set}
\end{inner_problem}

\noindent We can either do a rotation and a reflection or two reflections. Since $5$ is prime, all pairs work (unlike for the square). Let $f$ is a flip over the vertical axis. Examples of each are $\{r,f\}$ and $\{f,fr\}$.

\begin{inner_problem}
\item Whether the group is \textbf{commutative}
\end{inner_problem}

\noindent The group is not commutative. For example, $fr=fr$, but $rf=fr^4$.

\begin{outer_problem}
\item Regular pentagonal prism under rotation
\end{outer_problem}

\noindent This is isomorphic to the dihedral group of the pentagon, which is Problem 12. The reason is the same as for Problem 9's dependence on 8, thus I will not explain it.

\begin{outer_problem}
\item Regular pentagonal prism under reflection
\end{outer_problem}

\noindent This is akin to Problem 10.

\begin{inner_problem}[start=1]
\item Number of elements
\end{inner_problem}

\noindent$2\cdot 10=20.$

\begin{inner_problem}
\item If order $< 10$, the set of elements; otherwise, an explanation of how you know the order
\end{inner_problem}

\noindent We know the order because the rotation group of the pentagonal prism has $10$ elements, so its dihedral group has $20$ elements.

\begin{inner_problem}
\item A smallest possible \textbf{generating set}
\end{inner_problem}

\noindent If $a$ is a rotation of $72^\circ$ about the central axis, $b$ is a rotation of $180^\circ$ about a horizontal axis, and $d$ is a reflection across the midplane between the two pentagonal faces, then $\{a,b,d\}$ generates the set, since $\{a,b\}$ generates all rotations and $d$ turns them into their mirror images. But this isn't the right answer.

Are there any smaller generating sets? The previous trick asserting no using more advanced abstract algebra doesn't actually work.\footnote{If you understand it, it's because $Z_2\times Z_2\times Z_2$ isn't a quotient of this group, since this group $D_5 \times Z_2$ has order $20$ which is not divisible by $8$.} We have $ad=da$ and $bd=db$ (you can verify this geometrically). So to have a two element subgroup we likely need something like $a^nd$ and $ba^m$ for some integers $n,m$, so that we can potentially generate $a$, $b$ and $d$.

Let's try $ad$ and $b$. Taking successive powers of $ad$, we get

\begin{align*}
ad &= ad \\
(ad)^2 &= a^2 \\
(ad)^3 &= a^3d \\
(ad)^4 &= a^4 \\
(ad)^5 &= a^5d = d \\
(ad)^6 &= a
\end{align*}

\noindent We've just generated $d$ and $a$ from $ad$ alone! Since we have $b$ already, we have created $\{a,b,d\}$ from $\{ad,b\}$. Thus, the smallest generating set has size $2$. (We can't have size $1$ because then the group would be cyclic and thus commutative, which this group certainly isn't.)

This is a hard problem. Don't worry if you didn't get it.

\begin{inner_problem}
\item Whether the group is \textbf{commutative}
\end{inner_problem}

\noindent The dihedral group of the pentagon is a subgroup of this group, and is not commutative, so this group is not commutative.

\begin{outer_problem}
\item Regular pentagonal pyramid under rotation
\end{outer_problem}

\noindent This is just isomorphic to the rotation group of the pentagon, or Problem 11.

\begin{outer_problem}
\item Regular pentagonal pyramid under reflection
\end{outer_problem}

\noindent This is just isomorphic to the reflection group of the pentagon, or Problem 12.

\begin{outer_problem}
\item Regular tetrahedron (triangular pyramid) under rotation
\end{outer_problem}

\noindent This is isomorphic to a subgroup of $S_4$, the snap group of order $24$.

\begin{inner_problem}[start=1]
\item Number of elements
\end{inner_problem}

\noindent The snap group includes reflections, but this does not: thus, this group has $\frac{4!}{2}=12$ elements.

\begin{inner_problem}
\item If order $< 10$, the set of elements; otherwise, an explanation of how you know the order
\end{inner_problem}

\noindent We know the order because this is the rotation group of a tetrahedron, and the reflection group of a tetrahedron has $24$ elements, so this must have half that.

\begin{inner_problem}
\item A smallest possible \textbf{generating set}
\end{inner_problem}

\noindent Another difficult problem!

Let's figure out where the rotation axes actually are. There are $4$ axes going through a vertex---let's call these \textit{vertex} axes $v_i$. There are also $3$ axes going through the midpoints of opposite edges: let's call these \textit{edge} axes $e_i$. These axes are enumerated and shown in Figure~\ref{fig:tetra_rot_axes}.

\begin{asydef}
void draw_axis(string name, pair dir, triple a, triple b, real stretch=0.3) {
	triple start = (1+stretch)*a-stretch*b;
	triple end = (1+stretch)*b-stretch*a;

	draw(start--end,dotted);
	label("$" + name + "$", start, dir);
}
\end{asydef}

\begin{figure}[h]
	\begin{center}
		\begin{minipage}[b]{\textwidth}
			\centering
			\begin{asy}[width=0.5\textwidth]
			currentprojection=orthographic(2,5,5);
			triple A = (0,0,0);
			triple B = (1,0,0);
			triple C = (0.5,sqrt(3)/2,0);
			triple D = (A+B+C)/3 + (0,0,sqrt(6)/3);
			
			draw(B--C--D--A);
			draw(A--B,dashed);
			draw(D--B);
			draw(A--C);
			
			label("$A$", A, E);
			label("$B$", B, W);
			label("$C$", C, S);
			label("$D$", D, N);
			
			draw_axis("v_A", E, A, (B+C+D)/3);
			draw_axis("v_B", W, B, (A+C+D)/3);
			draw_axis("v_C", S, C, (A+B+D)/3);
			draw_axis("v_D", N, D, (A+B+C)/3);
			
			draw_axis("e_1", E, (D+C)/2, (A+B)/2);
			draw_axis("e_2", S, (B+C)/2, (A+D)/2);
			draw_axis("e_3", SE, (A+C)/2, (B+D)/2);
			
			triple[] cows = {A,B,C,D};
			
			for (int i = 0; i < 4; ++i) {
				for (int j = i; j < 4; ++j) {
					dot((cows[i]+cows[j])/2);
				}
			}
			
			for (int i = 0; i < 4; ++i) {
				dot((cows[0]+cows[1]+cows[2]+cows[3]-cows[i])/3);
			}
			
			\end{asy}
		\end{minipage}
	\end{center}
	\vspace*{-2\baselineskip}
	\begin{center}
		\begin{minipage}[t]{\textwidth}
			\captionof{figure}{Regular tetrahedron's succulent rotation axes.}
			\label{fig:tetra_rot_axes}
		\end{minipage}
	\end{center}
	\vspace*{-2\baselineskip}
\end{figure}

\noindent We can rotate by $120^\circ$ or $240^\circ$ (counterclockwise as viewed from the vertex) about any $v_i$, but only by $180^\circ$ about any $e_i$. Along with the identity, this gives all $2\cdot 4+3+1=12$ elements.

To make manipulating these elements easier, treat them as moving elements in a list. We name this list with indices as shown in Figure~\ref{fig:tetra_indices}. Thus, the identity element $I$ is $(A,B,C,D)$. A rotation of $240^\circ$ around $v_A$ swaps vertices in positions $(3\quad 4)$ then $(2\quad 3)$, so $v_A=(A,D,B,C)$ as shown in Figure~\ref{fig:v_a_repr}.

If we take a look at an edge rotation, say $e_1$, you will see it also swaps two vertices: in this case, $(3\quad 4)$ and $(1\quad 2)$. In general, any edge rotation or vertex rotation will swap two vertices---you can see this by plain symmetry or if you want, working it out for each rotation.

\begin{figure}[h]
	\begin{center}
		\begin{minipage}[b]{0.45\textwidth}
			\centering
			\begin{asy}[width=0.5\textwidth]
			currentprojection=orthographic(2,5,5);
			triple A = (0,0,0);
			triple B = (1,0,0);
			triple C = (0.5,sqrt(3)/2,0);
			triple D = (A+B+C)/3 + (0,0,sqrt(6)/3);
			
			draw(B--C--D--A);
			draw(A--B,dashed);
			draw(D--B);
			draw(A--C);
			
			label("$1$", A, E);
			label("$2$", B, W);
			label("$3$", C, S);
			label("$4$", D, N);
			
			\end{asy}
		\end{minipage}
		\hfill
		\begin{minipage}[b]{0.45\textwidth}
			\begin{asy}[width=0.7\textwidth]
			currentprojection=orthographic(2,5,5);
			triple A = (0,0,0);
			triple B = (1,0,0);
			triple C = (0.5,sqrt(3)/2,0);
			triple D = (A+B+C)/3 + (0,0,sqrt(6)/3);
			
			draw(B--C--D--A);
			draw(A--B,dashed);
			draw(D--B);
			draw(A--C);
			
			label("$1=A$", A, E);
			label("$2=D$", B, W);
			label("$3=B$", C, S);
			label("$4=C$", D, N);
			
			draw_axis("v_A", E, A, (B+C+D)/3,0.6);
			dot((B+C+D)/3);
			\end{asy}
		\end{minipage}
	\end{center}
	\vspace*{-2\baselineskip}
	\begin{center}
		\begin{minipage}[t]{0.45\textwidth}
			\captionof{figure}{Regular tetrahedron's indices.}
			\label{fig:tetra_indices}
		\end{minipage}
		\hfill
		\begin{minipage}[t]{0.45\textwidth}
			\captionof{figure}{$v_A=(A,D,B,C)$.}
			\label{fig:v_a_repr}
		\end{minipage}
	\end{center}
	\vspace*{-2\baselineskip}
\end{figure}

\noindent We now have a more abstract representation of this group: namely, it is the group of \textit{even permutations} of $(A,B,C,D)$. Even permutations are permutations made by swapping two pairs at a time. For example, $(B,A,D,C)$ is even, but $(A,B,D,C)$ is not. The group operation is composing two permutations by chaining them together. Note that the identity, $(A,B,C,D)$ is considered even, just as $0$ is considered even.

One element is clearly not enough, because this group is not cyclic. Can we do it in two elements though?

Consider two vertex rotations, which cycle (without loss of generality) the first three vertices and the last three vertices. That is, $a=(3,1,2,4)$ and $b=(1,4,2,3)$. Can we get every even permutation with combinations of $a$ and $b$? Let's try list them out:

\begin{alignat*}{9}
a &=(3,1,2,4), \qquad &&a^2&&=(2,3,1,4), \qquad &&a^3=I&&=(1,2,3,4), \qquad  &&b&&=(1,4,2,3), \\
b^2&=(1,3,4,2), &&\cancel{b^3}=I&&=a^3, &&ab&&=(2,1,4,3), &&ab^2&&=(4,1,3,2), \\
a^2b&=(4,2,1,3), &&a^2b^2&&=(3,4,1,2), &&\cancel{ba}&&=a^2b^2, &&b^2a&&=(3,2,4,1), \\
ba^2&=(2,4,3,1), &&\cancel{b^2a^2}&&=ab, &&\cancel{bab}&&=a^2, &&\cancel{b^2ab}&&=ba^2, \\
\cancel{bab^2}&=a^2b, &&b^2ab^2&&=(4,3,2,1)=ba^2b &&
\end{alignat*}

\noindent We have successfully generated all $12$ elements with the set $\{a,b\}$. Thus, a two element generating set is sufficient! Interestingly, this means you can turn a tetrahedron however you want by holding it at two corners and twisting it with each.

For the curious, this group is known as the alternating group $A_4$.

\begin{inner_problem}
\item Whether the group is \textbf{commutative}
\end{inner_problem}

\noindent The group is clearly not commutative, since $ab\neq ba$.

\begin{outer_problem}
\item Regular tetrahedron under reflection
\end{outer_problem}

\noindent This is just the snap group of order $4$, $S_4$.

\begin{inner_problem}[start=1]
\item Number of elements
\end{inner_problem}

\noindent As we found in the first problem, $S_4$ has $4!=24$ elements.

\begin{inner_problem}
\item If order $< 10$, the set of elements; otherwise, an explanation of how you know the order
\end{inner_problem}

\noindent As we found in the first problem, $S_4$ has $4!=24$ elements.

\begin{inner_problem}
\item A smallest possible \textbf{generating set}
\end{inner_problem}

\noindent This is tricky.

The obvious thing to do is keep $\{a,b\}$ from the previous problem and add some reflection $c$. Then $\{a,b,c\}$ has all $24$ elements, since $\{a,b\}$ makes $12$ elements and $c$ makes a copy of each ``in the mirror world.'' This is not, however, the right answer.

A generating of $2$ elements is actually possible! There are several ways to see this, but I find a permutation argument easiest to follow.

$S_4$ is not just the reflection group of the tetrahedron, but also the group of all permutations of $(1,2,3,4)$. Consider the permutation $j=(4,1,2,3)$, which cycles all the elements, and the permutation $k=(2,1,3,4)$, which swaps the first elements. Then

$$j=(4,1,2,3),\quad j^2=(3,4,1,2),\quad j^3=(2,3,4,1),\quad j^4=I=(1,2,3,4).$$

\noindent We can flip any two adjacent elements (as well as the first and last elements) by doing the following:
\begin{enumerate}
\item Cycle using powers of $j$ until the two elements in question are the first two elements.
\item Swap them with an application of $k$.
\item Cycle back to the starting position with powers of $j$.
\end{enumerate}
In more mathematical terms, we can swap indices $i$ and $i+1$, where $1\leq i \leq 3$, with the following element.

$$j^{i-1}kj^{5-i}.$$

\noindent Intuitively, if you can swap any two adjacent elements, you can make any permutations. The proof of this is pretty standard and outside the scope of this answer key.

For fun, let's see what the elements $j$ and $k$ actually are, operating on the tetrahedron.

\begin{figure}[h]
	\begin{center}
		\begin{minipage}{0.3\textwidth}
			\centering
			\begin{asy}[width=0.9\textwidth]
			currentprojection=orthographic(2,5,5);
			triple A = (0,0,0);
			triple B = (1,0,0);
			triple C = (0.5,sqrt(3)/2,0);
			triple D = (A+B+C)/3 + (0,0,sqrt(6)/3);
			
			draw(B--C--D--A);
			draw(A--B,dashed);
			draw(D--B);
			draw(A--C);
			
			label("$1$", A, E);
			label("$2$", B, W);
			label("$3$", C, S);
			label("$4$", D, N);
			\end{asy}
		\end{minipage}
		\hfill
		\begin{minipage}{0.3\textwidth}
			\centering
			\begin{asy}[width=0.9\textwidth]
			currentprojection=orthographic(2,5,5);
			triple A = (0,0,0);
			triple B = (1,0,0);
			triple C = (0.5,sqrt(3)/2,0);
			triple D = (A+B+C)/3 + (0,0,sqrt(6)/3);
			
			draw(B--C--D--A);
			draw(A--B,dashed);
			draw(D--B);
			draw(A--C);
			
			label("$4$", A, E);
			label("$1$", B, W);
			label("$2$", C, S);
			label("$3$", D, N);
			
			draw_axis("e_1", E, (D+C)/2, (A+B)/2);
			dot((D+C)/2);
			dot((A+B)/2);
			\end{asy}
		\end{minipage}
		\hfill
		\begin{minipage}{0.3\textwidth}
			\centering
			\begin{asy}[width=0.9\textwidth]
			currentprojection=orthographic(2,5,5);
			triple A = (0,0,0);
			triple B = (1,0,0);
			triple C = (0.5,sqrt(3)/2,0);
			triple D = (A+B+C)/3 + (0,0,sqrt(6)/3);
			
			draw(B--C--D);
			triple inter1 = point(D--A,0.375);
			draw(D--inter1,dashed);
			draw(inter1--A);
			triple inter2 = point(C--A,0.48);
			draw(C--inter2,dashed);
			draw(inter2--A);
			draw(A--B,dashed);
			draw(D--B);
			
			label("$2$", A, E);
			label("$1$", B, W);
			label("$3$", C, S);
			label("$4$", D, N);
			
			path3 midplane = (0.5,-0.2,-0.2)--(0.5,sqrt(3)/2+0.2,-0.2)--(0.5,sqrt(3)/2+0.2,1)--(0.5,-0.2,1)--cycle;
			
			draw(midplane, dotted);
			draw(surface(midplane), opacity(0.4)+gray(0.3));
			
			dot((A+B)/2);
			\end{asy}
		\end{minipage}
	\end{center}
	\vspace*{-2\baselineskip}
	\begin{center}
		\begin{minipage}[t]{0.3\textwidth}
			\centering
			$I$
		\end{minipage}
		\hfill
		\begin{minipage}[t]{0.3\textwidth}
			\centering
			$j$
		\end{minipage}
		\hfill
		\begin{minipage}[t]{0.3\textwidth}
			\centering
			$k$
		\end{minipage}
	\end{center}
	\vspace*{-2\baselineskip}
	\begin{center}
		\begin{minipage}[t]{\textwidth}
			\captionof{figure}{The two elements $j$ and $k$ generate the full symmetry group of the tetrahedron.}
		\end{minipage}
	\end{center}
	\vspace*{-2\baselineskip}
\end{figure}

\noindent Thus, the true minimal generating set is $\{j,k\}$ as described.

\begin{inner_problem}
\item Whether the group is \textbf{commutative}
\end{inner_problem}

\noindent This group is certainly not commutative, since the previous group from Problem 17 was not commutative and is a subgroup of this group.

\begin{outer_problem}
\item Cube under rotation
\end{outer_problem}

\noindent There are a couple ways to analyze this. My favorite one is to choose a face to make the top face, which can be done in $6$ ways, then choose which rotation that face should be in, which can be done in $4$ ways.

\begin{inner_problem}[start=1]
\item Number of elements
\end{inner_problem}

\noindent Since we choose a front face in $6$ ways, and its rotations in $4$ ways, we have $6\cdot 4=24$ total rotations.

\begin{inner_problem}
\item If order $< 10$, the set of elements; otherwise, an explanation of how you know the order
\end{inner_problem}

\noindent The order is found above.

\begin{inner_problem}
\item A smallest possible \textbf{generating set}
\end{inner_problem}

\begin{asydef}
triple A = (0,0,0); // Vertices of a cube
triple B = (0,1,0);
triple C = (-1,1,0);
triple D = (-1,0,0);

transform3 sd = shift(0,0,-1);

triple Ap = sd * A;
triple Bp = sd * B;
triple Cp = sd * C;
triple Dp = sd * D;

void drawCube(picture pic, string[] labels) {
    draw(pic,A--B--C--Cp--Bp--Ap--A--B--Bp);
    draw(pic,B--C);
    draw(pic,A--D--C);
    draw(pic,Ap--Dp--D,dashed); // behind the cube
    draw(pic,Dp--Cp,dashed);

    if (labels.length > 0) {
        label(pic, d + labels[0] + d, A, NW);
        label(pic, d + labels[1] + d, B, N);
        label(pic, d + labels[2] + d, C, NE);
        label(pic, d + labels[3] + d, D, N);
        label(pic, d + labels[4] + d, Ap, SW);
        label(pic, d + labels[5] + d, Bp, S);
        label(pic, d + labels[6] + d, Cp, SE);
        label(pic, d + labels[7] + d, Dp, S);
    }
}
\end{asydef}

\begin{figure}[h]
	\begin{center}
		\begin{minipage}[b]{\textwidth}
			\centering
			\begin{asy}[width=0.3\textwidth]
			draw(A--Cp,dotted);
			draw(B--Dp,dotted);
			draw(Ap--C,dotted);
			draw(Bp--D,dotted);
			
			dot(A);
			dot(B);
			dot(C);
			dot(D);
			dot(Ap);
			dot(Bp);
			dot(Cp);
			dot(Dp);
			
			string[] labels = {"1","2","3","4","3","4","1","2"};
			drawCube(currentpicture, labels);
			\end{asy}
		\end{minipage}
	\end{center}
	\vspace*{-2\baselineskip}
	\begin{center}
		\begin{minipage}[t]{\textwidth}
			\captionof{figure}{Marking opposite pairs of vertices.}
			\label{fig:cube_vertices_label}
		\end{minipage}
	\end{center}
	\vspace*{-2\baselineskip}
\end{figure}

\noindent This is tough until you make a key observation. If we label space-diagonally opposite vertices (that is, vertices which don't share a face) with the same number, as shown in Figure~\ref{fig:cube_vertices_label}, then we can easily enumerate valid rotations.

The front face starts off saying ``1,2,3,4.'' I claim that the $4!$ permutations of these four labels on the front face yields every rotation, and only rotations. This can be manually verified, but the higher-level argument isn't too bad.

First, note that you will always see the numbers ``1,2,3,4`` in \textit{some order} on the front face; you cannot see two of one number because all numbers are placed on diagonals of each other, and never share a side.

Second, note that the list of four numbers on the front face uniquely determines the other labels, since each has exactly one pair on the back face. For example, if there is a $3$ in the closest corner to the camera, then there \textit{must} be a $3$ in the furthest corner of the camera.

Third, we demonstrate that the permutation of labels can always be represented as a rotation. There are six fundamentally different types of label squares under rotation \textit{rotation}, as shown in Figure~\ref{fig:six_labelings}. But all appear somewhere as a face on the cube, as shown in Figure~\ref{fig:labelings_appear}.

\begin{figure}[h]
	\begin{center}
		\begin{minipage}[b]{0.6\textwidth}
			\centering
			\begin{asy}[width=0.8\textwidth]
			int[] type_list = {
			1,2,4,3,2,3,1,4,3,4,2,1,4,1,3,2,1,4,3,2,3,2,1,4
			};
			string[] name_list = { "$\alpha$","$\beta$","$\gamma$","$\delta$","$\epsilon$","$\zeta$" };
			
			void draw_sq(pair loc, int a, int b, int c, int d) {
			
			transform ls = shift(2*loc);
			
			pair A = ls*(0,0);
			pair B = ls*(1,0);
			pair C = ls*(1,-1);
			pair D = ls*(0,-1);
			
			draw(A--B--C--D--cycle);
			
			label("$" + (string) a + "$", A, NW);
			label("$" + (string) b + "$", B, NE);
			label("$" + (string) c + "$", C, SE);
			label("$" + (string) d + "$", D, SW);
			
			}
			
			for (int i = 0; i < 6; ++i) {
				real x = i%3;
				real y = -(i # 3);
				draw_sq((x,y), type_list[4*i],type_list[4*i+1],type_list[4*i+2],type_list[4*i+3]);
				label(name_list[i], (2*x+0.5,2*y-0.5));
			}
			\end{asy}
		\end{minipage}
		\hfill
		\begin{minipage}[b]{0.3\textwidth}
			\begin{asy}[width=0.9\textwidth]
			string[] labels = {"1","2","3","4","3","4","1","2"};
			drawCube(currentpicture, labels);
			
			label(YZ()*"$\alpha$", (0,0.5,-0.5));
			label(XZ()*"$\beta$", (-0.5,1,-0.5));
			label(YZ()*"$\gamma$", (-1,0.5,-0.5));
			label(XZ()*"$\delta$", (-0.5,0,-0.5));
			label(XY()*"$\epsilon$", (-0.5,0.5,0));
			label(XY()*"$\zeta$", (-0.5,0.5,-1));
			\end{asy}
		\end{minipage}
	\end{center}
	\vspace*{-2\baselineskip}
	\begin{center}
		\begin{minipage}[t]{0.6\textwidth}
			\captionof{figure}{The six different labelings of a square.}
			\label{fig:six_labelings}
		\end{minipage}
		\hfill
		\begin{minipage}[t]{0.3\textwidth}
			\captionof{figure}{The six different labelings indeed appear on the cube!}
			\label{fig:labelings_appear}
		\end{minipage}
	\end{center}
	\vspace*{-2\baselineskip}
\end{figure}

\noindent We have demonstrated that every permutation of the front face labels 1. creates a unique orientation of the cube and 2. that orientation is a rotations. Since there are $24$ unique permutations and $24$ unique rotations, every rotation has exactly one corresponding permutation and vice versa. We can now construct an isomorphism! The set of label permutations under the operation of composing permutations (as we did with the tetrahedron) and the set of rotations under the operation of composing rotations are \textit{isomorphic}. In symbols, $S_4\cong G$, our group.

So the group of rotations of a cube is actually $S_4$, the permutation group of $4$ elements. I find this incredible.

Back to the main question: what is the minimal generating set? In the previous question, we found that the permutations $(4,1,2,3)$---cycling all elements forward---and $(2,1,3,4)$---swapping the first two elements---generate $S_4$. For the cube, those are two rotations $a$ and $b$ as shown in Figure~\ref{fig:cube_generating_rotations}.

\begin{inner_problem}
\item Whether the group is \textbf{commutative}
\end{inner_problem}

\noindent This group is not commutative, since $S_4$ is not commutative.

\begin{figure}[h]
	\begin{center}
		\begin{minipage}[b]{0.45\textwidth}
			\centering
			\begin{asy}[width=0.7\textwidth]
			string[] labels = {"4","1","2","3","2","3","4","1"};
			drawCube(currentpicture, labels);
			
			dot((-0.5,0.5,0));
			dot((-0.5,0.5,-1));
			
			draw((-0.5,0.5,0.2)--(-0.5,0.5,-1.2),dotted);
			triple c = (-0.5,0.5,0);
			real r = 0.2;
			
			draw((c+r*(0,-1,0))..(c+(r*sqrt(2)/2,-r*sqrt(2)/2,0))..(c+r*(1,0,0)),Arrow3);
			
			\end{asy}
		\end{minipage}
		\hfill
		\begin{minipage}[b]{0.45\textwidth}
			\centering
			\begin{asy}[width=0.7\textwidth]
			string[] labels = {"2","1","3","4","3","4","2","1"};
			drawCube(currentpicture, labels);
			
			dot((-0.5,0,-0.5));
			dot((-0.5,1,-0.5));
			
			draw((-0.5,-0.2,-0.5)--(-0.5,1.2,-0.5),dotted);
			triple c = (-0.5,1,-0.5);
			real r = 0.2;
			
			draw((c+r*(0,0,1))..(c+r*(-1,0,0))..(c+r*(0,0,-1)),Arrow3);
			\end{asy}
		\end{minipage}
	\end{center}
	\vspace*{-2\baselineskip}
	\begin{center}
		\begin{minipage}[t]{0.45\textwidth}
			\centering
			$a$
		\end{minipage}
		\hfill
		\begin{minipage}[t]{0.45\textwidth}
			\centering
			$b$
		\end{minipage}
	\end{center}
	\vspace*{-2\baselineskip}
	\begin{center}
		\begin{minipage}[t]{\textwidth}
			\captionof{figure}{The two rotations $a$ and $b$.}
			\label{fig:cube_generating_rotations}
		\end{minipage}
	\end{center}
	\vspace*{-2\baselineskip}
\end{figure}

\begin{outer_problem}
\item Cube under reflection
\end{outer_problem}

\begin{inner_problem}[start=1]
\item Number of elements
\end{inner_problem}

\noindent There are $24$ elements in the rotation group of the cube, so naturally there are $48$ elements in the reflection group.

\begin{inner_problem}
\item If order $< 10$, the set of elements; otherwise, an explanation of how you know the order
\end{inner_problem}

\noindent For each of the $24$ rotations of the cube, there is also a reflected version over some plane. This gives $2\cdot 24=48$ total elements in this group.

\begin{inner_problem}
\item A smallest possible \textbf{generating set}
\end{inner_problem}

\noindent If $c$ is a reflection about, say, the origin of the cube, then $\{a,b,c\}$ (where $a,b$ are the rotations from before) would generate the whole group, since $\{a,b\}$ generates all rotations and $\{c\}$ generates their respective reflections. But can we do it in two?

As usual it seems, the answer is yes! The proof is not mine, because I couldn't figure it out, but due to math.SE user \textbf{verret}. It does require some more advanced concepts, so it is probably inaccessible to most.

The group we've been analyzing is $S_4\times Z_2$. Let $S_4$ be permuting elements $\{1,2,3,4\}$ and $Z_2$ be permuting elements $\{5,6\}$ (note that $Z_2=S_2$). Then given two elements $g=(4,1,2,3,5,6)$ in our notation, meaning that indices $(1,2,3,4)$ are cycled, and $h=(3,1,2,4,6,5)$, meaning that indices $(1,2,3)$ and $(5,6)$ are cycled, we can construct the group.

Note that $h^2=(2,3,1,5,6)$ is in $S_4$, since it does not permute indices $5,6$. It has a period of $3$, and thus generates a subgroup of order $3$. Furthermore, $h^3$ only permutes $(5,6)$. Furthermore, $g$ is an element in $S_4$ and has a period of $4$. Thus, since $\gcd(3,4)=1$, by Lagrange's theorem we know that $\{h^2,g\}$ generates a subgroup of $S_4$ of at least order $3\cdot 4=12$.

The only such subgroup, besides $S_4$ itself, is the alternating group $A_4$. But $g$ is outside of $A_4$, since it is an odd permutation:

$$(1,2,\boxed{3,4})\to (1,\boxed{2,4},3)\to (\boxed{1,4},2,3) \to (4,1,2,3).$$

Thus, $\{h^2,g\}$ does not generate $A_4$, and must generate $S_4$. Adding $h^3$, the generator for $Z_2$, to this set gives the full $S_4,Z_2$. The minimal generating set is therefore $\{g,h\}$ as defined.

For the curious, using our vertex ``labeling'' convention as before, the elements $g$ and $h$ are shown in Figure~\ref{fig:elements_g_and_h}.

\begin{inner_problem}
\item Whether the group is \textbf{commutative}
\end{inner_problem}

\noindent The subgroup of rotations of the cube, $S_4$, is not commutative, so this group is definitely not commutative.

\pagebreak
%if this key wasnt 177 pages long i would try to fix the huge whitespace on this page. note to check this later

\begin{figure}[h]
	\begin{center}
		\begin{minipage}[b]{0.3\textwidth}
			\centering
			\begin{asy}[width=0.9\textwidth]
				string[] labels = {"A","B","C","D","A'","B'","C'","D'"};
				drawCube(currentpicture, labels);
			\end{asy}
		\end{minipage}
		\hfill
		\begin{minipage}[b]{0.3\textwidth}
			\centering
			\begin{asy}[width=0.9\textwidth]
			string[] labels = {"D","A","B","C","D'","A'","B'","C'"};
			drawCube(currentpicture, labels);
			
			dot((-0.5,0.5,0));
			dot((-0.5,0.5,-1));
			
			draw((-0.5,0.5,0.2)--(-0.5,0.5,-1.2),dotted);
			triple c = (-0.5,0.5,0);
			real r = 0.2;
			
			draw((c+r*(0,-1,0))..(c+(r*sqrt(2)/2,-r*sqrt(2)/2,0))..(c+r*(1,0,0)),Arrow3);
			
			\end{asy}
		\end{minipage}
		\hfill
		\begin{minipage}[b]{0.3\textwidth}
			\centering
			\begin{asy}[width=0.9\textwidth]
				string[] labels = {"A'","A","B","B'","D'","D","C","C'"};
				drawCube(currentpicture, labels);
			\end{asy}
		\end{minipage}
	\end{center}
	\vspace*{-2\baselineskip}
	\begin{center}
		\begin{minipage}[t]{0.3\textwidth}
			\centering
			$I$
		\end{minipage}
		\hfill
		\begin{minipage}[t]{0.3\textwidth}
			\centering
			$g$
		\end{minipage}
		\hfill
		\begin{minipage}[t]{0.3\textwidth}
			\centering
			$h$
		\end{minipage}
	\end{center}
	\vspace*{-2\baselineskip}
	\begin{center}
		\begin{minipage}[t]{\textwidth}
			\captionof{figure}{Elements $g$ and $h$. Note that $h$ is not solely a reflection about a mirror plane, but actually a combination of a rotation and reflection: a so-called rotoreflection!}
			\label{fig:elements_g_and_h}
		\end{minipage}
	\end{center}
	\vspace*{-2\baselineskip}
\end{figure}
% ASSIGNING FOURTEEN PAGES OF WORK IN ONE TEXTBOOK PAGE SHOULD BE CRIMINALIZED AND NEVER EVER DONE AGAIN

\end{document}

\documentclass[../gatm.tex]{subfiles}

\begin{document}

\section{Glossary}
\setcounter{problem_i}{0}

\begin{description}[align=left]

\item [adjacency matrix] a square matrix connecting vertices of a graph; a.k.a. transportation matrix

\item [bijection] one-to-one correspondence

\item [binary operation] (of a group) an operation acting on two elements

\item[cardinality] size of a group; number of elements. Interchangeable term with order.

\item[closure] a condition where operation on elements of the group always produces an element of that same group

\item[collinear] on the same line

\item[complex conjugate] a complex number where its imaginary part is negated

\item[Demoivre's theorem] $$(r_1 (\cos \theta + i \sin \theta)) (r_2 (\cos \phi + i \sin \phi)) = (r_1r_2) (\cos(\theta + \phi) + i \sin(\theta + \phi))$$

\item[dihedral group] a group of symmetries of a regular polygon formed by rotations and reflections

\item[eigenspace] the linear subspace consisting of all eigenvectors associated with a particular eigenvalue

\item[eigenvector] a vector which when operated on by a given operator gives a scalar multiple of itself

\item[eigenvalue] the scalar multiple that is associated with the eigenvector

\item[Euler's totient function] a function $\phi(n)$ that tells how many numbers are relatively prime to $n$

\item[generating set] the list of generator elements

\item[generator] elements that can generate the entire group by a series of operations

\item[group] a set of elements, finite or infinite, formed by a certain binary operation that satisfies the four fundamental properties: closure, associativity, identity, and inverse

\item[identity element] an element, when acted on other element via binary operation, that outputs the same element

\item[image] output of a transformation

\item[isometry] a linear transformation preserving length

\item[isomorphism] possessing bijection, or one-to-one correspondence, between two groups

\item[linear mapping] all lines are mapped to lines with fixed origin point

\item[linearly independent] (of an eigenvector) two vectors that are not multiples of each other; have different directions

\item[matrix decomposition] decomposing a transformation matrix into simpler units of transformations

\item[order] size of a group; number of elements. Interchangeable term with cardinality.

\item[period] number of times an element of a group has to be operated on itself to yield an identity element of that group

\item[permutation] an order of things in which they can be arranged

\item[preimage] input of a transformation

\item[shear] a linear transformation where all points along a particular line remain fixed, while other points are shifted parallel to the fixed line by a distance proportional to their perpendicular distance from the fixed line

\item[unit vector] a vector with a magnitude of 1

\end{description}

\end{document}
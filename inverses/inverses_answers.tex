\documentclass[../gatm_answers.tex]{subfiles}

\begin{document}

\section{Inverses}

\begin{outer_problem}[start=1]
\item
\end{outer_problem}

\begin{inner_problem}[start=1]
\item With real numbers, one of the important purposes of division is that it lets you solve equations like $ax=b$ for $x$. Solve this by division (difficult!).
\end{inner_problem}

We divide both sides by $a$, to get $x=\frac{b}{a}$.

\begin{inner_problem}
\item If division didn't exist, you could still solve this equation by multiplication. The number you'd multiply by is called the ``\textbf{multiplicative inverse}'' of $a$. What is the property that defines this special number?
\end{inner_problem}

The property defining the number is that it is the unique number which, when multiplying by $a$, yields $1$. That is, if the number is $c$, then $ac=1$.

\begin{inner_problem}
\item The multiplicative inverse of $a$ is often written $a^{-1}$. Why does this notation make sense?
\end{inner_problem}

Since $a=a^1$, we have $a^{-1}a^1=a^{1-1}=a^0=1$.

\begin{outer_problem}
\item
\end{outer_problem}

\begin{inner_problem}[start=1]
\item For fixed $a,b$, you might think that the equation $ax=b$ has only one solution, but sometimes it can have zero or infinitely many. Give an example of both cases.
\end{inner_problem}

\begin{inner_problem}
\item How does the existence of a unique solution relate to the idea of multiplicative invertibility?
\end{inner_problem}

\begin{inner_problem}
\item Are there any other possible numbers of solutions?
\end{inner_problem}

\begin{outer_problem}
\item
\end{outer_problem}

\begin{inner_problem}[start=1]
\item Define ``one-to-one'' function.
\end{inner_problem}

\begin{inner_problem}
\item Is $f(x)=ax$ a one-to-one function for all real $a$? (Hint: Look for the silly exception(s)!)
\end{inner_problem}

\begin{outer_problem}
\item Would your answers to the previous numbers change if you were talking about complex numbers instead of just real numbers? Why or why not?
\end{outer_problem}

\begin{outer_problem}
\item
\end{outer_problem}

\begin{inner_problem}[start=1]
\item Find all solutions of $5x\equiv 7$ in clock arithmetic.
\end{inner_problem}

\begin{inner_problem}
\item Find all solutions of $2x\equiv 6$ in clock arithmetic.
\end{inner_problem}

\begin{inner_problem}
\item Find all solutions of $6x\equiv 6$ in clock arithmetic.
\end{inner_problem}

\begin{inner_problem}
\item Find all solutions of $2x\equiv 7$ in clock arithmetic.
\end{inner_problem}

\begin{inner_problem}
\item What are all possible numbers of solutions that $ax\equiv b$ can have in clock arithmetic?
\end{inner_problem}

\begin{outer_problem}
\item How does the number of solutions to $ax\equiv b$ relate to the idea of multiplicative inverse? (Hint: You can try solving for $a=5,2,6,3$ and $b=1$. What numbers would be $5^{-1}$, $2^{-1}$, $6^{-1}$, and $3^{-1}$ in clock arithmetic?)
\end{outer_problem}

\begin{outer_problem}
\item
\end{outer_problem}

\begin{inner_problem}[start=1]
\item In clock arithmetic, for what values of $a$ is $f(x)\equiv ax$ a one-to-one function?
\end{inner_problem}

\begin{inner_problem}
\item How does this relate to whether $ax=1$ has exactly one solution?
\end{inner_problem}

\begin{outer_problem}
\item How does this all relate to groups?
\end{outer_problem}

\begin{inner_problem}[start=1]
\item The clock numbers are a group under clock addition. Name that group!
\end{inner_problem}

\begin{inner_problem}
\item They are not a group under multiplication. Why?
\end{inner_problem}

\begin{inner_problem}
\item A subset of four of the clock numbers form a group under the operation of multiplication. Find them, and write a group table.
\end{inner_problem}

\begin{inner_problem}
\item Describe this group. What is the inverse of each element?
\end{inner_problem}

\begin{inner_problem}
\item What symmetry group is it isomorphic to?
\end{inner_problem}

\begin{outer_problem}
\item If the numbers on an advanced Mars clock went from $0$ to $4$,
\end{outer_problem}

\begin{inner_problem}[start=1]
\item They would form a group under addition. Make a group table!
\end{inner_problem}

\begin{inner_problem}
\item What group is this isomorphic to?
\end{inner_problem}

\begin{inner_problem}
\item A subset of four of these numbers forms a group under multiplication. Find them and write a group table.
\end{inner_problem}

\begin{inner_problem}
\item Describe this multiplication group.
\end{inner_problem}

\begin{inner_problem}
\item What symmetry group is it isomorphic to?
\end{inner_problem}

\begin{outer_problem}
\item
\end{outer_problem}

\begin{inner_problem}[start=1]
\item \label{prob:needed_for_matrix_undo3}Find all solutions $(x,y)$ of $\twomat{1}{2}{3}{4}\left[\begin{array}{c} x \\ y \end{array}\right]=\left[\begin{array}{c} 5 \\ 6 \end{array}\right]$, by multiplying out the left side and rewriting this as a system of equations.
\end{inner_problem}

\begin{inner_problem}
\item \label{prob:needed_for_matrix_undo4}Find all solutions $(x,y)$ of $\twomat{1}{2}{2}{4}\left[\begin{array}{c} x \\ y \end{array}\right]=\left[\begin{array}{c} 5 \\ 6 \end{array}\right]$
\end{inner_problem}

\begin{inner_problem}
\item Find all solutions $(x,y)$ of $\twomat{1}{2}{2}{4}\left[\begin{array}{c} x \\ y \end{array}\right]=\left[\begin{array}{c} 5 \\ 10 \end{array}\right]$
\end{inner_problem}

\begin{inner_problem}
\item What are all possible numbers of solutions that $AX=B$ can have in matrices? Use your knowledge of the properties of systems of equations.
\end{inner_problem}

\begin{outer_problem}
\item Now, let's relate the two matrices from the previous problem to the transformations we know.
\end{outer_problem}

\begin{inner_problem}[start=1]
\item Contrast the mapping properties of $\twomat{1}{2}{3}{4}$ and $\twomat{1}{2}{2}{4}$.
\end{inner_problem}

\begin{inner_problem}
\item Find the determinants of these matrices. What do you notice?
\end{inner_problem}

\begin{inner_problem}
\item When is $f(X)=AX$ a one-to-one function? That is, in mapping the plane, when does each point in the image have exactly one preimage?
\end{inner_problem}

\begin{inner_problem}
\item Compare how you find the number of solutions of the real number equation $ax=b$ with how you find the number of solutions of the matrix equation $AX=B$.
\end{inner_problem}

\begin{outer_problem}
\item Let $K=\twomat{5}{7}{8}{-3}$.
\end{outer_problem}

\begin{inner_problem}[start=1]
\item Find all solutions to $K\left[\begin{array}{c} x \\ y \end{array}\right]=\left[\begin{array}{c} 10 \\ 2 \end{array}\right]$.
\end{inner_problem}

\begin{inner_problem}
\item If we knew a matrix which was the inverse of $K$, written $K^{-1}$, we could write the following equation:

$$K^{-1}K\left[\begin{array}{c} x \\ y \end{array}\right]=K^{-1}\left[\begin{array}{c} 5 \\ 10 \end{array}\right].$$

What would the left side reduce to?
\end{inner_problem}

\begin{outer_problem}
\item Consider the following matrix inverses:

$$\twomat{3}{4}{2}{-5}^{-1}=\twomat{\frac{3}{23}}{\frac{4}{23}}{\frac{2}{23}}{-\frac{5}{23}}$$
$$\twomat{1}{0}{0}{-1}^{-1}=\twomat{1}{0}{0}{-1}$$
$$\twomat{3}{1}{2}{4}^{-1}=\frac{1}{10}\twomat{4}{-1}{-2}{3}$$
$$\twomat{1}{2}{3}{4} = \twomat{-2}{1}{3}{-1}$$
\end{outer_problem}

\begin{inner_problem}[start=1]
\item Look for a pattern in these inverses.
\end{inner_problem}

\begin{inner_problem}
\item Describe the inverse of an arbitrary matrix: $\twomat{a}{c}{b}{d}^{-1}=\frac{1}{\phantom{000000}}\twomat{}{}{}{}.$ Use the word determinant in your answer.
\end{inner_problem}

\begin{inner_problem}
\item We've been writing the inverse of matrix $A$ as $A^{-1}$. Why does this notation make sense?
\end{inner_problem}

\begin{outer_problem}
\item Now, see what happens when you multiply the following matrices:
\end{outer_problem}

\begin{inner_problem}[start=1]
\item $-\frac{1}{2}\twomat{2}{3}{4}{5}\twomat{5}{-3}{-4}{2}$
\end{inner_problem}

\begin{inner_problem}
\item $\frac{1}{71}\twomat{5}{7}{8}{-3}\twomat{3}{7}{8}{-5}$
\end{inner_problem}

\begin{inner_problem}
\item $\twomat{a}{c}{b}{d}\frac{1}{ad-bc}\twomat{d}{-c}{-b}{a}$
\end{inner_problem}

\begin{inner_problem}
\item $\frac{1}{ad-bc}\twomat{d}{-c}{-b}{a}\twomat{a}{c}{b}{d}$
\end{inner_problem}

\begin{outer_problem}
\item For another approach to finding the inverse of a matrix, solve the following for $w,x,y,z$ in terms of $a,b,c,d$ by converting the matrix equations into a set of four linear equations:

$$\twomat{w}{y}{x}{z}\twomat{a}{c}{b}{d}=\twomat{1}{0}{0}{1}.$$
\end{outer_problem}

\begin{outer_problem}
\item Rewrite each system of equations in matrix form. Use your calculator to calculate a matrix inverse, solve the system, and finally, check your answer. Remember to make clear in your work when you have used a calculator.
\end{outer_problem}

\begin{inner_problem}[start=1]
\item $\begin{cases} 2x+3y &= 5 \\ 4x+5y &= 7 \end{cases}$
\end{inner_problem}

\begin{inner_problem}
\item $\begin{cases} 37x+12y &= 65 \\ 93x+40y &= 156\end{cases}$
\end{inner_problem}

\begin{inner_problem}
\item $\begin{cases} 2x+5y+3z &= 5 \\ 3x+2y+4z &= 7 \\ 13x+16y+18z &= 4\end{cases}$
\end{inner_problem}

\begin{inner_problem}
\item $\begin{cases} w + 2x + 3y + 4z &= 7 \\ 3w-x-2y-5z&=5 \\ 5w+3x-y-4z&=3 \\ 7w+9x+5y-2z&=2\end{cases}$
\end{inner_problem}

\begin{inner_problem}
\item $\begin{cases} 2x+5y+2z &= 1 \\ 3x+2y+4z &= 1 \\ 13x+16y+18z &= 5 \end{cases}$
\end{inner_problem}

\begin{inner_problem}
\item When can you use matrix inverses to solve a system of equations?
\end{inner_problem}

\begin{outer_problem}
\item You can fit a polynomial to any set of points in the plane, so long as the points pass the Vertical Line Test.
\end{outer_problem}

\begin{inner_problem}[start=1]
\item What is the least degree polynomial through%
\end{inner_problem}

\begin{iinner_problem}[start=1]
\item One point?
\end{iinner_problem}

\begin{iinner_problem}
\item Two points?
\end{iinner_problem}

\begin{iinner_problem}
\item Three points?
\end{iinner_problem}

\begin{iinner_problem}
\item $n$ points?
\end{iinner_problem}

\begin{outer_problem}
\item Find a polynomial of least degree that passes through $(0,3)$, $(1,5)$, $(2,-3)$, $(3,4)$, and $(4,7)$.
\end{outer_problem}

\end{document}

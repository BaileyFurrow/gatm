\documentclass[../gatm_answers.tex]{subfiles}

\begin{document}

\section{Inverses}

\begin{enumerate}
\setcounter{enumi}{\value{problem_i}}
\item \begin{enumerate}
\item With real numbers, one of the important purposes of division is that it lets you solve equations like $ax=b$ for $x$. Solve this by division (difficult!).
\item If division didn't exist, you could still solve this equation by multiplication. The number you'd multiply by is called the ``\textbf{multiplicative inverse}'' of $a$. What is the property that defines this special number?
\item The multiplicative inverse of $a$ is often written $a^{-1}$. Why does this notation make sense?
\end{enumerate}
\item \begin{enumerate}
\item You might think that the equation $ax=b$ has only one solution, but sometimes it can have zero or infinitely many. Give an example of both cases.
\item How does the existence of a unique solution relate to the idea of multiplicative invertibility?
\item Are there any other possible numbers of solutions?
\end{enumerate}
\item \begin{enumerate}
\item Define ``one-to-one'' function.
\item Is $f(x)=ax$ a one-to-one function for all real $a$? (Hint: Look for the silly exception(s)!)
\end{enumerate}
\item Would your answers to the previous numbers change if you were talking about complex numbers instead of just real numbers? Why or why not?
\setcounter{problem_i}{\value{enumi}}
\end{enumerate}

\begin{enumerate}
\setcounter{enumi}{\value{problem_i}}
\item \begin{enumerate}
\item Find all solutions of $5x\equiv 7$ in clock arithmetic.
\item Find all solutions of $2x\equiv 6$ in clock arithmetic.
\item Find all solutions of $6x\equiv 6$ in clock arithmetic.
\item Find all solutions of $2x\equiv 7$ in clock arithmetic.
\item What are all possible numbers of solutions that $ax\equiv b$ can have in clock arithmetic?
\end{enumerate}
\item How does the number of solutions to $ax\equiv b$ relate to the idea of multiplicative inverse? (Hint: You can try solving for $a=5,2,6,3$ and $b=1$. What numbers would be $5^{-1}$, $2^{-1}$, $6^{-1}$, and $3^{-1}$ in clock arithmetic?)
\item \begin{enumerate}
\item In clock arithmetic, for what values of $a$ is $f(x)\equiv ax$ a one-to-one function?
\item How does this relate to whether $ax=1$ has exactly one solution?
\end{enumerate}
\item How does this all relate to groups?
\begin{enumerate}
\item The clock numbers are a group under clock addition. Name that group!
\item They are not a group under multiplication. Why?
\item A subset of four of the clock numbers form a group under the operation of multiplication. Find them, and write a group table.
\item Describe this group. What is the inverse of each element?
\item What symmetry group is it isomorphic to?
\end{enumerate}
\item If the numbers on an advanced Mars clock went from $0$ to $4$,
\begin{enumerate}
\item They would form a group under addition. Make a group table!
\item What group is this isomorphic to?
\item A subset of four of these numbers forms a group under multiplication. Find them and write a group table.
\item Describe this multiplication group.
\item What symmetry group is it isomorphic to?
\end{enumerate}
\setcounter{problem_i}{\value{enumi}}
\end{enumerate}

\begin{enumerate}
\setcounter{enumi}{\value{problem_i}}
\item \begin{enumerate}
\item \label{prob:needed_for_matrix_undo3}Find all solutions $(x,y)$ of $\twomat{1}{2}{3}{4}\left[\begin{array}{c} x \\ y \end{array}\right]=\left[\begin{array}{c} 5 \\ 6 \end{array}\right]$, by multiplying out the left side and rewriting this as a system of equations.
\item \label{prob:needed_for_matrix_undo4}Find all solutions $(x,y)$ of $\twomat{1}{2}{2}{4}\left[\begin{array}{c} x \\ y \end{array}\right]=\left[\begin{array}{c} 5 \\ 6 \end{array}\right]$
\item Find all solutions $(x,y)$ of $\twomat{1}{2}{2}{4}\left[\begin{array}{c} x \\ y \end{array}\right]=\left[\begin{array}{c} 5 \\ 10 \end{array}\right]$
\item What are all possible numbers of solutions that $AX=B$ can have in matrices? Use your knowledge of the properties of systems of equations.
\end{enumerate}
\item Now, let's relate the two matrices from the previous problem to the transformations we know.
\begin{enumerate}
\item Contrast the mapping properties of $\twomat{1}{2}{3}{4}$ and $\twomat{1}{2}{2}{4}$.
\item Find the determinants of these matrices. What do you notice?
\item When is $f(X)=AX$ a one-to-one function? That is, in mapping the plane, when does each point in the image have exactly one preimage?
\item Compare how you find the number of solutions of the real number equation $ax=b$ with how you find the number of solutions of the matrix equation $AX=B$.
\end{enumerate}
\item Let $K=\twomat{5}{7}{8}{-3}$.\begin{enumerate}
\item Find all solutions to $K\left[\begin{array}{c} x \\ y \end{array}\right]=\left[\begin{array}{c} 10 \\ 2 \end{array}\right]$.
\item If we knew a matrix which was the inverse of $K$, written $K^{-1}$, we could write the following equation:

$$K^{-1}K\left[\begin{array}{c} x \\ y \end{array}\right]=K^{-1}\left[\begin{array}{c} 5 \\ 10 \end{array}\right].$$

What would the left side reduce to?
\end{enumerate}
\setcounter{problem_i}{\value{enumi}}
\end{enumerate}

\begin{enumerate}
\setcounter{enumi}{\value{problem_i}}
\item \begin{enumerate}
\item \label{prob:why_did_herreshoff_do_this1}Now look at your results from Problem~\ref{prob:needed_for_matrix_undo1} on page~\pageref{prob:needed_for_matrix_undo1}. Multiply the matrices that undid $\twomat{3}{4}{2}{-5}$. Did you get $\frac{1}{23}\twomat{-5}{-4}{-2}{3}$?
\item \label{prob:why_did_herreshoff_do_this2}Do the same for Problem~\ref{prob:needed_for_matrix_undo2} on page~\pageref{prob:needed_for_matrix_undo2} and Problems~\ref{prob:needed_for_matrix_undo3} and~\ref{prob:needed_for_matrix_undo4} on page~\pageref{prob:needed_for_matrix_undo3}.
\item You should have found a problem in inverting the matrix in~\ref{prob:needed_for_matrix_undo4}. What is it? Answer geometrically, making reference to the matrix's mapping.
\end{enumerate}
\item \begin{enumerate}
\item Look for a pattern in the answers to Problems \label{prob:why_did_herreshoff_do_this1} and \label{prob:why_did_herreshoff_do_this2}.
\item Describe the inverse of an arbitrary matrix: $\twomat{a}{c}{b}{d}^{-1}=\frac{1}{\phantom{000000}}\twomat{}{}{}{}.$ Use the word determinant in your answer.
\item We've been writing the inverse of matrix $A$ as $A^{-1}$. Why does this notation make sense?
\end{enumerate}
\item Now, see what happens when you multiply the following matrices:
\begin{multicols}{2}
\begin{enumerate}
\item $-\frac{1}{2}\twomat{2}{3}{4}{5}\twomat{5}{-3}{-4}{2}$
\item $\frac{1}{71}\twomat{5}{7}{8}{-3}\twomat{3}{7}{8}{-5}$
\item $\twomat{a}{c}{b}{d}\frac{1}{ad-bc}\twomat{d}{-c}{-b}{a}$
\item $\frac{1}{ad-bc}\twomat{d}{-c}{-b}{a}\twomat{a}{c}{b}{d}$
\end{enumerate}
\end{multicols}
\item For another approach to finding the inverse of a matrix, solve the following for $w,x,y,z$ in terms of $a,b,c,d$ by converting the matrix equations into a set of four linear equations:

$$\twomat{w}{y}{x}{z}\twomat{a}{c}{b}{d}=\twomat{1}{0}{0}{1}.$$
\setcounter{problem_i}{\value{enumi}}
\end{enumerate}

\begin{enumerate}
\setcounter{enumi}{\value{problem_i}}
\item Rewrite each system of equations in matrix form. Use your calculator to calculate a matrix inverse, solve the system, and finally, check your answer. Remember to make clear in your work when you have used a calculator.
\begin{multicols}{3}
\begin{enumerate}
\item $\begin{cases} 2x+3y &= 5 \\ 4x+5y &= 7 \end{cases}$
\item $\begin{cases} 37x+12y &= 65 \\ 93x+40y &= 156\end{cases}$
\item $\begin{cases} 2x+5y+3z &= 5 \\ 3x+2y+4z &= 7 \\ 13x+16y+18z &= 4\end{cases}$
\item $\begin{cases} w + 2x + 3y + 4z &= 7 \\ 3w-x-2y-5z&=5 \\ 5w+3x-y-4z&=3 \\ 7w+9x+5y-2z&=2\end{cases}$
\item $\begin{cases} 2x+5y+2z &= 1 \\ 3x+2y+4z &= 1 \\ 13x+16y+18z &= 5 \end{cases}$
\item When can you use matrix inverses to solve a system of equations?
\end{enumerate}
\end{multicols}
\item You can fit a polynomial to any set of points in the plane, so long as the points pass the Vertical Line Test.
\begin{enumerate}
\item What is the least degree polynomial through%
\begin{multicols}{4}%
\begin{enumerate}%
\item One point?
\item Two points?
\item Three points?
\item $n$ points?
\end{enumerate}%
\end{multicols}%
\item Find a polynomial of least degree that passes through $(0,3)$, $(1,5)$, $(2,-3)$, $(3,4)$, and $(4,7)$.
\end{enumerate}
\setcounter{problem_i}{\value{enumi}}
\end{enumerate}

\end{document}
\documentclass[../gatm_answers.tex]{subfiles}

\begin{document}

\section{Infinite Groups}

Note: an \textbf{injection} $f$ is a function taking $A$ into $B$ such that for all $a\in A$, $f(a)\in B$ and $f(a)$ is unique. In other words, there are no two $a_1,a_2\in A$, $a_1\neq a_2$ such that $f(a_1)=f(a_2)$.

\begin{outer_problem}[start=1]
\item Where have you come across the roots \textit{iso}- and -\textit{morphic} before?
\end{outer_problem}

(Answers may vary.)

\textit{Iso}- is a root meaning equal. You might have seen it in isometry, isometric (paper), isomer, isosceles, isotonic, isotropy, and isotope. \textit{Morph} means ``form'' or ``shape.'' You might have seen it in metamorphosis, amorphous, anthropomorphism, or morpheme.

\begin{outer_problem}
\item Can two groups be isomorphic if they have different orders?
\end{outer_problem}

No. Suppose we have groups $A$ and $B$ such that $|A|>|B|$ ($A$ is bigger than $B$). Then we can't have a one-to-one correspondence between the elements of $A$ and $B$, because there will always be elements in $A$ without a ``partner'' in $B$. Thus, they cannot be isomorphic.

\begin{outer_problem}
\item The rotation group for the hexagon $H$ has six elements: the identity, and rotations of $\frac{\pi}{3}$, $\frac{2\pi}{3}$, $\pi$, $\frac{4\pi}{3}$, $\frac{5\pi}{3}$ radians. A rotation of $\frac{\pi}{3}$ generates the group.
\end{outer_problem}

\begin{inner_problem}[start=1]
\item Which other rotation generates the group?
\end{inner_problem}

The other rotation which generates the group is $\frac{5\pi}{3}$, because $5$ is coprime with $6$. This is necessary because otherwise a subgroup of the full $C_6$ is formed. For example, $\frac{2\pi}{3}$ generates 

$$\left\{0,\frac{2\pi}{3},\frac{4\pi}{3}\right\},$$

which is merely $C_3$. Lame!

\begin{inner_problem}
\item What is the period of each element?
\end{inner_problem}

\begin{align*}
0\text{ or }I &: 1 \\
\frac{\pi}{3} &: 6 \\
\frac{2\pi}{3} &: 3 \\
\pi &: 2 \\
\frac{4\pi}{3} &: 3 \\
\frac{5\pi}{3} &: 6 \\
\end{align*}

\begin{outer_problem}
\item $H$ has the same number of elements as the dihedral group $D_3$. 
\end{outer_problem}

\begin{inner_problem}[start=1]
\item Are the two groups isomorphic? How do you know?
\end{inner_problem}

No, the two groups are not isomorphic, although they are the same size. An easy way to see this is that $D_3$ has three reflections, which have period $2$, but $H$ only has one element of period $2$.

\begin{inner_problem}
\item What is the period of each element of $D_3$?
\end{inner_problem}

\begin{align*}
I\text{ or }I &: 1 \\
r &: 3 \\
r^2  &: 3 \\
f &: 2 \\
fr &: 2 \\
fr^2 &: 2 \\
\end{align*}

\begin{inner_problem}
\item What can you say if the sets of the periods of the elements of each group are not the same?
\end{inner_problem}

If the periods of each group can't be paired up, then the elements cannot be paired up either; after all, isomorphism is a structure-preserving operation. Thus, the two groups are not isomorphic.

\begin{inner_problem}
\item Which subgroups of the cyclic group $C_6$ and $D_3$ are isomorphic?
\end{inner_problem}

One is $C_2$, which is $\{0,\pi\}$ in $C_6$ and $\{I,\text{any reflection}\}$ in $D_3$. The other non-trivial one is $C_3$, which is $\left\{0,\frac{(3\pm 1)\pi}{3}\right\}$ in $C_6$ and $\{I,\text{any rotation}\}$ in $D_3$. Both also have the trivial subgroup $\{I\}$ of just the identity element.

\begin{outer_problem}
\item Could an infinite group be isomorphic to a finite group?
\end{outer_problem}

No, because their sizes are not the same; a one-to-one correspondence cannot be constructed.

\begin{outer_problem}
\item Do you think all infinite groups are isomorphic to each other? Find a counterexample if you can.
\end{outer_problem}

Not all infinite groups are isomorphic. For example, the set of rotations about the origin has only one element of period $2$, namely $r_{180^\circ}$. But the set of reflections about the origin has infinitely many elements of period $2$. Both, however, are infinite in size.

\begin{outer_problem}
\item Make guesses to the relative sizes of the following pairs of sets. You may use shorthand like $|a| < |b|$, $|a| > |b|$, $|a| = |b|$. After you have made your guesses, we will analyze some of the cases and you can find out how good your intuition was.
\end{outer_problem}

(Answers may vary, but the ``correct'' answers are shown.)

\begin{inner_problem}[start=1]
\item natural numbers, $\mathbb{N}$ vs. positive even numbers, $2\mathbb{N}$
$$|\mathbb{N}|=|2\mathbb{N}|$$
\item natural numbers, $\mathbb{N}$ vs. positive rational numbers, $\mathbb{Q}^+$
$$|\mathbb{N}|=\left|\mathbb{Q}^+\right|$$
\item natural numbers, $\mathbb{N}$ vs. real numbers between zero and one, $[0,1)$
$$|\mathbb{N}|<|[0,1)|$$
\item real numbers, $\mathbb{R}$ vs. complex numbers, $\mathbb{C}$
$$|\mathbb{R}|=|\mathbb{C}|$$
\item real numbers, $\mathbb{R}$ vs. points on a line
$$|\mathbb{R}|=|\text{points on a line}|$$
\item points on a line vs. points on a line segment
$$|\text{points on a line}|=|\text{points on a line segment}|$$
\item points on a line vs. points on a plane
$$|\text{points on a line}|=|\text{points on a plane}|$$
\item rational numbers, $\mathbb{Q}$ vs. Cantor set (look this up or ask your teacher)
$$|Q|<|\mathcal{C}|$$
\end{inner_problem}

\begin{outer_problem}
\item Now, please return to problem 7 and revise your answers. Justify each answer by producing a one-to-one correspondence, or showing the impossibility of doing so. Part (h) is an optional challenge.
\end{outer_problem}

\begin{inner_problem}[start=1]
\item natural numbers, $\mathbb{N}$ vs. positive even numbers, $2\mathbb{N}$
\end{inner_problem}

This one is pretty straightforward. We have the following injection from $\mathbb{N}$ to $2\mathbb{N}$:

$$s\in \mathbb{N} \to 2s\in 2\mathbb{N}.$$

We have the following injection from $2\mathbb{N}$ to $\mathbb{N}$:

$$s\in \mathbb{N} \to \frac{s}{2}\in \mathbb{N}.$$

Since we can go both ways, we have $|\mathbb{N}| = |2\mathbb{N}|$, even though $\mathbb{N}\subset 2\mathbb{N}$ ($\mathbb{N}$ is a subset of $2\mathbb{N}$).\footnote{We didn't explicitly state it because it's pretty intuitive, but this is using the Cantor-Schröder-Bernstein theorem (CSB).}

\begin{inner_problem}
\item natural numbers, $\mathbb{N}$ vs. positive rational numbers, $\mathbb{Q}^+$
\end{inner_problem}

Surprisingly, we can make a one-to-one correspondence. If we list out the positive rationals in reduced form ($\frac{p}{q}$ with $p,q$ coprime), ordered by increasing denominator, we can create the correspondence:

$$\renewcommand{\arraystretch}{1.6}\begin{array}{cccccccccccc}
\mathbb{Q}^+ & \dfrac{0}{1} & \dfrac{1}{1} & \dfrac{1}{2} & \dfrac{2}{1} & \dfrac{1}{3} & \dfrac{3}{1} & \dfrac{1}{4} & \dfrac{2}{3} & \dfrac{3}{2} & \dfrac{4}{1} & \cdots \\
& \updownarrow & \updownarrow & \updownarrow & \updownarrow & \updownarrow & \updownarrow & \updownarrow & \updownarrow & \updownarrow & \updownarrow & \cdots \\
\mathbb{N} & 1 & 2 & 3 & 4 & 5 & 6 & 7 & 8 & 9 & 10 & \cdots \\
\end{array}$$

More details of this construction are given later in the textbook chapter.

\begin{inner_problem}
\item natural numbers, $\mathbb{N}$ vs. real numbers between zero and one, $[0,1)$
\end{inner_problem}

A one-to-one correspondence cannot exist between these two sets. The classic proof of this is Cantor's diagonal argument, which is given in the textbook. 

\begin{inner_problem}
\item real numbers, $\mathbb{R}$ vs. complex numbers, $\mathbb{C}$
\end{inner_problem}

This is a somewhat tough problem. The key is to represent complex numbers $a+bi$ as the ordered pair $(a,b)$ where $a,b\in \mathbb{R}$.

Here is the route we will take:

\begin{enumerate}
\item Construct a one-to-one correspondence between $[0,1)$ and $\mathbb{R}$
\item Use (1) to construct a similar correspondence between $[0,1)^2$ and $\mathbb{R}^2$. That is, we will construct a correspondence between ordered pairs of reals in $[0,1)$ and ordered pairs of reals.
\item We find an injection from $[0,1)^2$ into $[0,1)$.
\item We find an injection from $[0,1)$ into $[0,1)^2$. This shows there is a one-to-one correspondence between $[0,1)$ and $[0,1)^2$.
\item We ``chain'' the correspondences together:

$$\mathbb{R} \leftrightarrow [0,1) \leftrightarrow [0,1)^2 \leftrightarrow \mathbb{R}^2.$$
\end{enumerate}

Step 1: The most straight forward way to do this is in two functions. Let $a(x)=\tan \left(\pi \left(x-\frac{1}{2}\right)\right)$ and $b(x)=$

\begin{inner_problem}
\item real numbers, $\mathbb{R}$ vs. points on a line
\end{inner_problem}

\begin{inner_problem}
\item points on a line vs. points on a line segment
\end{inner_problem}

\begin{inner_problem}
\item points on a line vs. points on a plane
\end{inner_problem}

\begin{inner_problem}
\item rational numbers, $\mathbb{Q}$ vs. Cantor set (look this up or ask your teacher)
\end{inner_problem}

\begin{outer_problem}
\item Here’s a list of infinite sets, each with an operation. For each pair, answer: i.~Does it form a group? ii.~Which previous group(s) is it isomorphic to?
\end{outer_problem}

\begin{inner_problem}
\item natural numbers, addition
\end{inner_problem}

\begin{inner_problem}
\item integers, addition
\end{inner_problem}

\begin{inner_problem}
\item even integers, addition
\end{inner_problem}

\begin{inner_problem}
\item odd integers, addition
\end{inner_problem}

\begin{inner_problem}
\item rational numbers, addition
\end{inner_problem}

\begin{inner_problem}
\item real numbers, addition
\end{inner_problem}

\begin{inner_problem}
\item complex numbers, addition
\end{inner_problem}

\begin{inner_problem}
\item integers, multiplication
\end{inner_problem}

\begin{inner_problem}
\item integer powers of $2$, multiplication
\end{inner_problem}

\begin{inner_problem}
\item rational numbers, multiplication
\end{inner_problem}

\begin{inner_problem}
\item rational numbers excluding $0$, multiplication
\end{inner_problem}

\begin{inner_problem}
\item complex numbers, multiplication
\end{inner_problem}

\begin{inner_problem}
\item rotation by a rational number of degrees
\end{inner_problem}

\begin{inner_problem}
\item rotation by a rational number of radians
\end{inner_problem}

\begin{inner_problem}
\item rotation by an integer number of radians
\end{inner_problem}

\begin{outer_problem}
\item Can an irrational number taken to an irrational power ever be rational? Consider the potential example $a = \sqrt{2}^{\sqrt{2}}$. To help you answer this question, let $b = a^{\sqrt{2}}$. Simplify $b$, and explain why we don’t need to know whether $a$ is rational or irrational.
\end{outer_problem}


\end{document}
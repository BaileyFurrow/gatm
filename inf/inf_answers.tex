\documentclass[../gatm_answers.tex]{subfiles}

\begin{document}

\section{Trigonometry Review}

\begin{outer_problem}[start=1]
\item Where have you come across the roots \textit{iso}- and -\textit{morphic} before?
\end{outer_problem}

\begin{outer_problem}
\item Can two groups be isomorphic if they have different orders?
\end{outer_problem}

\begin{outer_problem}
\item The rotation group for the hexagon $H$ has six elements: the identity, and rotations of $\frac{\pi}{3}$, $\frac{2\pi}{3}$, $\pi$, $\frac{4\pi}{3}$, $\frac{5\pi}{3}$ radians. A rotation of $\frac{\pi}{3}$ generates the group.
\end{outer_problem}

\begin{inner_problem}[start=1]
\item Which other rotation generates the group?
\end{inner_problem}

\begin{inner_problem}
\item What is the period of each element?
\end{inner_problem}

\begin{outer_problem}
\item $H$ has the same number of elements as the dihedral group $D_3$. 
\end{outer_problem}

\begin{inner_problem}[start=1]
\item Are the two groups isomorphic? How do you know?
\end{inner_problem}

\begin{inner_problem}
\item What is the period of each element of $D_3$?
\end{inner_problem}

\begin{inner_problem}
\item What can you say if the sets of the periods of the elements of each group are not the same?
\end{inner_problem}

\begin{inner_problem}
\item Which subgroups of the cyclic group $C_6$ and $D_3$ are isomorphic?
\end{inner_problem}

\begin{outer_problem}
\item Could an infinite group be isomorphic to a finite group?
\end{outer_problem}

\begin{outer_problem}
\item Do you think all infinite groups are isomorphic to each other? Find a counterexample if you can.
\end{outer_problem}

\begin{outer_problem}
\item Make guesses to the relative sizes of the following pairs of sets. You may use shorthand like $a < b$, $a > b$, $a = b$. After you have made your guesses, we will analyze some of the cases and you can find out how good your intuition was.
\end{outer_problem}

(Answers may vary, but the ``correct'' answers are shown.)

\begin{inner_problem}[start=1]
\item natural numbers, $\mathbb{N}$ vs. positive even numbers, $2\mathbb{N}$
$$|\mathbb{N}|=|2\mathbb{N}|$$
\item natural numbers, $\mathbb{N}$ vs. positive rational numbers, $\mathbb{Q}^+$
$$|\mathbb{N}|=\left|\mathbb{Q}^+\right|$$
\item natural numbers, $\mathbb{N}$ vs. real numbers between zero and one, $[0,1)$
$$|\mathbb{N}|<|[0,1)|$$
\item real numbers, $\mathbb{R}$ vs. complex numbers, $\mathbb{C}$
$$|\mathbb{R}|=|\mathbb{C}|$$
\item real numbers, $\mathbb{R}$ vs. points on a line
$$|\mathbb{R}|=|\text{points on a line}|$$
\item points on a line vs. points on a line segment
$$|\text{points on a line}|=|\text{points on a line segment}|$$
\item points on a line vs. points on a plane
$$|\text{points on a line}|=|\text{points on a plane}|$$
\item rational numbers, $\mathbb{Q}$ vs. Cantor set (look this up or ask your teacher)
$$|Q|<|\mathcal{C}|$$
\end{inner_problem}

\begin{outer_problem}
\item Now, please return to problem 7 and revise your answers. Justify each answer by producing a one-to-one correspondence, or showing the impossibility of doing so. Part (h) is an optional challenge.
\end{outer_problem}

\begin{inner_problem}[start=1]
\item natural numbers, $\mathbb{N}$ vs. positive even numbers, $2\mathbb{N}$
\end{inner_problem}

\begin{inner_problem}
\item natural numbers, $\mathbb{N}$ vs. positive rational numbers, $\mathbb{Q}^+$
\end{inner_problem}

\begin{inner_problem}
\item natural numbers, $\mathbb{N}$ vs. real numbers between zero and one, $[0,1)$
\end{inner_problem}

\begin{inner_problem}
\item real numbers, $\mathbb{R}$ vs. complex numbers, $\mathbb{C}$
\end{inner_problem}

\begin{inner_problem}
\item real numbers, $\mathbb{R}$ vs. points on a line
\end{inner_problem}

\begin{inner_problem}
\item points on a line vs. points on a line segment
\end{inner_problem}

\begin{inner_problem}
\item points on a line vs. points on a plane
\end{inner_problem}

\begin{inner_problem}
\item rational numbers, $\mathbb{Q}$ vs. Cantor set (look this up or ask your teacher)
\end{inner_problem}

\begin{outer_problem}
\item Here’s a list of infinite sets, each with an operation. For each pair, answer: i.~Does it form a group? ii.~Which previous group(s) is it isomorphic to?
\end{outer_problem}

\begin{inner_problem}
\item natural numbers, addition
\end{inner_problem}

\begin{inner_problem}
\item integers, addition
\end{inner_problem}

\begin{inner_problem}
\item even integers, addition
\end{inner_problem}

\begin{inner_problem}
\item odd integers, addition
\end{inner_problem}

\begin{inner_problem}
\item rational numbers, addition
\end{inner_problem}

\begin{inner_problem}
\item real numbers, addition
\end{inner_problem}

\begin{inner_problem}
\item complex numbers, addition
\end{inner_problem}

\begin{inner_problem}
\item integers, multiplication
\end{inner_problem}

\begin{inner_problem}
\item integer powers of $2$, multiplication
\end{inner_problem}

\begin{inner_problem}
\item rational numbers, multiplication
\end{inner_problem}

\begin{inner_problem}
\item rational numbers excluding $0$, multiplication
\end{inner_problem}

\begin{inner_problem}
\item complex numbers, multiplication
\end{inner_problem}

\begin{inner_problem}
\item rotation by a rational number of degrees
\end{inner_problem}

\begin{inner_problem}
\item rotation by a rational number of radians
\end{inner_problem}

\begin{inner_problem}
\item rotation by an integer number of radians
\end{inner_problem}

\begin{outer_problem}
\item Can an irrational number taken to an irrational power ever be rational? Consider the potential example $a = \sqrt{2}^{\sqrt{2}}$. To help you answer this question, let $b = a^{\sqrt{2}}$. Simplify $b$, and explain why we don’t need to know whether $a$ is rational or irrational.
\end{outer_problem}


\end{document}
\documentclass[../gatm_answers.tex]{subfiles}

\begin{document}

\section{Infinite Groups}

Note: an \textbf{injection} $f$ is a function taking $A$ into $B$ such that for all $a\in A$, $f(a)\in B$ and $f(a)$ is unique. In other words, there are no two $a_1,a_2\in A$, $a_1\neq a_2$ such that $f(a_1)=f(a_2)$.

\begin{outer_problem}[start=1]
\item Where have you come across the roots \textit{iso}- and -\textit{morphic} before?
\end{outer_problem}

(Answers may vary.)

\textit{Iso}- is a root meaning equal. You might have seen it in isometry, isometric (paper), isomer, isosceles, isotonic, isotropy, and isotope. \textit{Morph} means ``form'' or ``shape.'' You might have seen it in metamorphosis, amorphous, anthropomorphism, or morpheme.

\begin{outer_problem}
\item Can two groups be isomorphic if they have different orders?
\end{outer_problem}

No. Suppose we have groups $A$ and $B$ such that $|A|>|B|$ ($A$ is bigger than $B$). Then we can't have a one-to-one correspondence between the elements of $A$ and $B$, because there will always be elements in $A$ without a ``partner'' in $B$. Thus, they cannot be isomorphic.

\begin{outer_problem}
\item The rotation group for the hexagon $H$ has six elements: the identity, and rotations of $\frac{\pi}{3}$, $\frac{2\pi}{3}$, $\pi$, $\frac{4\pi}{3}$, $\frac{5\pi}{3}$ radians. A rotation of $\frac{\pi}{3}$ generates the group.
\end{outer_problem}

\begin{inner_problem}[start=1]
\item Which other rotation generates the group?
\end{inner_problem}

The other rotation which generates the group is $\frac{5\pi}{3}$, because $5$ is coprime with $6$. This is necessary because otherwise a subgroup of the full $C_6$ is formed. For example, $\frac{2\pi}{3}$ generates 

$$\left\{0,\frac{2\pi}{3},\frac{4\pi}{3}\right\},$$

which is merely $C_3$. Lame!

\begin{inner_problem}
\item What is the period of each element?
\end{inner_problem}

\begin{align*}
0\text{ or }I &: 1 \\
\frac{\pi}{3} &: 6 \\
\frac{2\pi}{3} &: 3 \\
\pi &: 2 \\
\frac{4\pi}{3} &: 3 \\
\frac{5\pi}{3} &: 6 \\
\end{align*}

\begin{outer_problem}
\item $H$ has the same number of elements as the dihedral group $D_3$. 
\end{outer_problem}

\begin{inner_problem}[start=1]
\item Are the two groups isomorphic? How do you know?
\end{inner_problem}

No, the two groups are not isomorphic, although they are the same size. An easy way to see this is that $D_3$ has three reflections, which have period $2$, but $H$ only has one element of period $2$.

\begin{inner_problem}
\item What is the period of each element of $D_3$?
\end{inner_problem}

\begin{align*}
I\text{ or }I &: 1 \\
r &: 3 \\
r^2  &: 3 \\
f &: 2 \\
fr &: 2 \\
fr^2 &: 2 \\
\end{align*}

\begin{inner_problem}
\item What can you say if the sets of the periods of the elements of each group are not the same?
\end{inner_problem}

If the periods of each group can't be paired up, then the elements cannot be paired up either; after all, isomorphism is a structure-preserving operation. Thus, the two groups are not isomorphic.

\begin{inner_problem}
\item Which subgroups of the cyclic group $C_6$ and $D_3$ are isomorphic?
\end{inner_problem}

One is $C_2$, which is $\{0,\pi\}$ in $C_6$ and $\{I,\text{any reflection}\}$ in $D_3$. The other non-trivial one is $C_3$, which is $\left\{0,\frac{(3\pm 1)\pi}{3}\right\}$ in $C_6$ and $\{I,\text{any rotation}\}$ in $D_3$. Both also have the trivial subgroup $\{I\}$ of just the identity element.

\begin{outer_problem}
\item Could an infinite group be isomorphic to a finite group?
\end{outer_problem}

No, because their sizes are not the same; a one-to-one correspondence cannot be constructed.

\begin{outer_problem}
\item Do you think all infinite groups are isomorphic to each other? Find a counterexample if you can.
\end{outer_problem}

Not all infinite groups are isomorphic. For example, the set of rotations about the origin has only one element of period $2$, namely $r_{180^\circ}$. But the set of reflections about the origin has infinitely many elements of period $2$. Both, however, are infinite in size.

\begin{outer_problem}
\item Make guesses to the relative sizes of the following pairs of sets. You may use shorthand like $|a| < |b|$, $|a| > |b|$, $|a| = |b|$. After you have made your guesses, we will analyze some of the cases and you can find out how good your intuition was.
\end{outer_problem}

(Answers may vary, but the ``correct'' answers are shown.)

\begin{inner_problem}[start=1]
\item natural numbers, $\mathbb{N}$ vs. positive even numbers, $2\mathbb{N}$
$$|\mathbb{N}|=|2\mathbb{N}|$$
\item natural numbers, $\mathbb{N}$ vs. positive rational numbers, $\mathbb{Q}^+$
$$|\mathbb{N}|=\left|\mathbb{Q}^+\right|$$
\item natural numbers, $\mathbb{N}$ vs. real numbers between zero and one, $[0,1)$
$$|\mathbb{N}|<|[0,1)|$$
\item real numbers, $\mathbb{R}$ vs. complex numbers, $\mathbb{C}$
$$|\mathbb{R}|=|\mathbb{C}|$$
\item real numbers, $\mathbb{R}$ vs. points on a line
$$|\mathbb{R}|=|\text{points on a line}|$$
\item points on a line vs. points on a line segment
$$|\text{points on a line}|=|\text{points on a line segment}|$$
\item points on a line vs. points on a plane
$$|\text{points on a line}|=|\text{points on a plane}|$$
\item rational numbers, $\mathbb{Q}$ vs. Cantor set (look this up or ask your teacher)
$$|Q|<|\mathcal{C}|$$
\end{inner_problem}

\begin{outer_problem}
\item Now, please return to problem 7 and revise your answers. Justify each answer by producing a one-to-one correspondence, or showing the impossibility of doing so. Part (h) is an optional challenge.
\end{outer_problem}

\begin{inner_problem}[start=1]
\item natural numbers, $\mathbb{N}$ vs. positive even numbers, $2\mathbb{N}$
\end{inner_problem}

This one is pretty straightforward. We have the following injection from $\mathbb{N}$ to $2\mathbb{N}$:

$$s\in \mathbb{N} \to 2s\in 2\mathbb{N}.$$

We have the following injection from $2\mathbb{N}$ to $\mathbb{N}$:

$$s\in \mathbb{N} \to \frac{s}{2}\in \mathbb{N}.$$

Since we can go both ways, we have $|\mathbb{N}| = |2\mathbb{N}|$, even though $\mathbb{N}\subset 2\mathbb{N}$ ($\mathbb{N}$ is a subset of $2\mathbb{N}$).\footnote{We didn't explicitly state it because it's pretty intuitive, but this is using the Cantor-Schröder-Bernstein theorem (CSB).}

\begin{inner_problem}
\item natural numbers, $\mathbb{N}$ vs. positive rational numbers, $\mathbb{Q}^+$
\end{inner_problem}

Surprisingly, we can make a one-to-one correspondence. If we list out the positive rationals in reduced form ($\frac{p}{q}$ with $p,q$ coprime), ordered by increasing denominator, we can create the correspondence:

$$\renewcommand{\arraystretch}{1.6}\begin{array}{cccccccccccc}
\mathbb{Q}^+ & \dfrac{0}{1} & \dfrac{1}{1} & \dfrac{1}{2} & \dfrac{2}{1} & \dfrac{1}{3} & \dfrac{3}{1} & \dfrac{1}{4} & \dfrac{2}{3} & \dfrac{3}{2} & \dfrac{4}{1} & \cdots \\
& \updownarrow & \updownarrow & \updownarrow & \updownarrow & \updownarrow & \updownarrow & \updownarrow & \updownarrow & \updownarrow & \updownarrow & \cdots \\
\mathbb{N} & 1 & 2 & 3 & 4 & 5 & 6 & 7 & 8 & 9 & 10 & \cdots \\
\end{array}$$

More details of this construction are given later in the textbook chapter. In any case, $\left|\mathbb{Q}^+\right| = |\mathbb{N}|.$

\begin{inner_problem}
\item natural numbers, $\mathbb{N}$ vs. real numbers between zero and one, $[0,1)$
\end{inner_problem}

A one-to-one correspondence cannot exist between these two sets, so $\left|\mathbb{N}\right| < \left|[0,1)\right|$. The classic proof of this is Cantor's diagonal argument, which is given in the textbook. 

\begin{inner_problem}
\item real numbers, $\mathbb{R}$ vs. complex numbers, $\mathbb{C}$
\end{inner_problem}

This is a pretty tough problem to do in a logically sound way. The key is to represent complex numbers $a+bi$ as the ordered pair $(a,b)$ where $a,b\in \mathbb{R}$. The set of all $(a,b)$ is denoted $\mathbb{R}^2$.

Here is the route we will take:

\begin{enumerate}
\item Construct a one-to-one correspondence between the interval $[0,1)$ and $\mathbb{R}$.
\item Use (1) to construct a similar correspondence between $[0,1)^2$ and $\mathbb{R}^2$. That is, we will construct a correspondence between ordered pairs of reals in $[0,1)$ and ordered pairs of any reals.
\item We find an injection from $[0,1)$ into $[0,1)^2$. 
\item We find an injection from $[0,1)^2$ into $[0,1)$. This shows there is a one-to-one correspondence between $[0,1)$ and $[0,1)^2$.
\item We ``chain'' the correspondences together:

$$\mathbb{R} \leftrightarrow [0,1) \leftrightarrow [0,1)^2 \leftrightarrow \mathbb{R}^2.$$
\end{enumerate}

Step 1: The most straight forward way to do this is to show there is an injection from $[0,1)$ into $\mathbb{R}$, and vice versa.\footnote{Again, this uses the Cantor-Schröder-Bernstein theorem.} We have $f(x)=x$ as a straightforward injection from $[0,1)$ into $\mathbb{R}$, and

$$g(x)=\frac{1}{1+e^{-x}}$$

as an injection $g\, : \, \mathbb{R} \to [0,1)$. Thus, there exists a one-to-one correspondence $H$ between $\mathbb{R}$ and $[0,1)$.

Step 2: If $H$ is the function from Step 1, we have

$$J(a,b)=(H(a),H(b))$$

as a one-to-one correspondence between $\mathbb{R}^2$ and $[0,1)^2$.

Step 3: An injection from $[0,1)$ into $[0,1)^2$ is straightforward:

$$k_1(x)=(x, 0).$$

Step 4: An injection from $[0,1)^2$ into $[0,1)$ is the more challenging portion. The basic idea is to interleave digits like so:

$$(0.123456789..., 0.314159265)\mathop{\to} ^ {k_2} 0.132134415569728695...$$

The main issue with this construction is that $0.5=0.4999...$ gives two different outputs, so this mapping isn't even a function:

$$(0.5,0.0)\to 0.50$$
$$(0.499...,0.0)\to 0.409090... \neq 0.50.$$

The easiest thing to do here is arbitrarily choose one of these mappings. In particular, we represent a number with an infinite sequence of trailing zeroes $0.a_1a_2\cdots a_n00000...$ with the numerically equivalent

$$0.a_1a_2\cdots (a_n-1)9999....$$

Now, our function $k_2$ is a true injection, since $k(a,b)\in [0,1)$ for all $(a,b)\in [0,1)^2$ and $k(a_1,b_1)\neq k(a_2,b_2)$ for $(a_1,b_1)\neq (a_2,b_2)$.

Step 5: We have constructed an injection $k_1$ from $[0,1)\to [0,1)^2$ and an injection $k_2$ from $[0,1)^2\to [0,1)$. Thus, there exists a one-to-one correspondence $K$ between $[0,1)$ and $[0,1)^2$.

We chain the correspondences, finally proving that there exists a one-to-one correspondence between $\mathbb{R}$ and $\mathbb{R}^2$:

$$\mathbb{R} \mathop{\leftrightarrow} ^{H} [0,1) \mathop{\leftrightarrow} ^ {K} [0,1)^2 \mathop{\leftrightarrow} ^ {J} \mathbb{R}^2.$$

Thus, $|\mathbb{R}|=\left|\mathbb{R}^2\right|$.

\begin{inner_problem}
\item real numbers, $\mathbb{R}$ vs. points on a line
\end{inner_problem}

This is pretty straightforward if you think of points on a line as points on a number line. We arbitrarily choose a point on the line for $0$ and a point for $1$. In this regime, each point on the line corresponds with a unique real number. Thus, $|\mathbb{R}| = |\text{points on a line}|$.

\begin{inner_problem}
\item points on a line vs. points on a line segment
\end{inner_problem}

The simplest way to do this is, once again, to show there is an injection going both ways. We can go from segment $\to$ line by observing that a segment is just a subset of a line. We can go from line to segment by representing each point as a real number $\mathbb{R}$ as we already did, then taking the function

$$f(x)=\frac{1}{1+e^{-x}}$$

which turns that point into a real number in the interval $(0,1)$. This can be mapped onto the line segment by simply choosing one endpoint to be $0$ and the other to be $1$.

$|\text{points on a line segment}| = |\text{points on a line}|$

\begin{inner_problem}
\item points on a line vs. points on a plane
\end{inner_problem}

We can represent points on a line, as usual, with $\mathbb{R}$. We can represent points on a plane by arbitrarily choosing non-collinear points for $(0,0)$, $(1,0)$ and $(0,1)$ and letting this be a coordinate space where points $(a,b)$ are expressed as

$$a<1,0>+b<0,1>.$$

Note that the two vectors don't have to be perpendicular. This shows that we can represent points on a plane by $\mathbb{R}^2$. But we've already proved $\left|\mathbb{R}^2\right| = |\mathbb{R}|$! Thus,

$$\left|\text{points on a plane}\right| = |\text{points on a line}|.$$

\begin{inner_problem}
\item rational numbers, $\mathbb{Q}$ vs. Cantor set (look this up or ask your teacher)
\end{inner_problem}

The Cantor set $\mathcal{C}$ is formed by iteratively deleting the open middle third of segments, starting with the unit segment. Formally, we can construct this like so\footnote{This construction, taken literally, is actually scaling the set down by $\frac{1}{3}$ and copying it at each step. The net result, however, is equivalent.}:

$$C_1=[0,1]$$
$$C_n=\frac{1}{3} C_{n-1} \cup \left(\frac{1}{3} C_{n-1} + \frac{2}{3}\right) \quad \text{for }n\geq 2$$
$$\mathcal{C}=\bigcap_{n=1}^{\infty} C_n. \quad \text{(Intersection of all intervals }C_n\text{.)}$$

More intuitively, the construction is shown in Figure~\ref{fig:construct_cantor}.
\begin{center}
\begin{asy}[width=0.7\textwidth]
real[] segments = {0,1};
real[] inter = {};

label("$0$", (0,0), N);
label("$1$", (1,0), N);

label("$\frac{1}{3}$", (1/3,-0.09), N);
label("$\frac{2}{3}$", (2/3,-0.09), N);

for (int i = 0; i < 5; ++i) {
	real y = -0.09 * i;
	
	label("$C_" + (string) (i+1) + "$", (-0.05, y));
	
	for (int j = 0; j < segments.length; j += 2) {
		pair s1 = (segments[j], y);
		pair s2 = (segments[j+1], y);
		
		dot(s1);
		dot(s2);
		
		draw(s1--s2);
	}
	
	for (int j = 0; j < segments.length; j += 2) {
		real s1 = segments[j];
		real s2 = segments[j+1];
		
		inter.push(s1);
		inter.push(2/3 * s1 + 1/3 * s2);
		inter.push(1/3 * s1 + 2/3 * s2);
		inter.push(s2);
	}
	
	for (int j = 0; j < inter.length; ++j) {
		segments[j] = inter[j];
	}
	
	inter.delete(0,inter.length-1);
}

label("$\vdots$", (0.5, 5 * -0.09));
label("$\stackrel{\downarrow}{\mathcal{C}}$", (-0.05, 5*-0.09)); 

label("$L$", (1/6, -0.09), N);
label("$R$", (5/6, -0.09), N);

label("$L'$", (1/18, -2 * 0.09), N);
label("$R'$", (5/18, -2 * 0.09), N);
\end{asy}
\captionof{figure}{The construction of the Cantor set $\mathcal{C}$.}
\label{fig:construct_cantor}
\end{center}

How do we attack this problem? It's not immediately clear how to tell whether a  number $x$ is in $\mathcal{C}$.\footnote{A number $x$ is in $\mathcal{C}$ if and only if it has a ternary (base-$3$) representation consisting of only $0$s and $2$s. For example, $1/27$ is in the Cantor set because $1/27 = 0.001_3 = 0.000\overline{2}_3$.} We should turn it into something we know how to deal with.

Consider how we would choose a random point in $\mathcal{C}$. Starting at $C_1$, we can either go to the left segment or the right segment (marked as $L$ and $R$ in Figure~\ref{fig:construct_cantor}). If we choose $L$, then we once again have $2$ choices: to go to $L'$ or $R'$. This continues ad infinitum. Thus, we can correspond each element $x\in \mathcal{C}$ with a binary number in the interval $[0,1]$.\footnote{Note it is inclusive because $0.1111..._2=1$ and $0.0000..._2=0$.}

As an example, suppose we choose segments in the sequence $LRRLLLRLLLL...$. Then the corresponding binary number is

$$0.0110001\overline{0}_2=\frac{39}{128}\in [0,1].$$

Thus, we have a one-to-one correspondence between the elements of $\mathcal{C}$ and $[0,1]$. The question is asking the relative sizes of $\mathbb{Q}$ and $\mathcal{C}$. We already know (by Cantor's diagonal argument or otherwise) that $|\mathbb{Q}| < |[0,1]|$. Therefore, we have

$$|\mathbb{Q}| < |[0,1]| = |\mathcal{C}| \Longrightarrow |\mathbb{Q}| < |\mathcal{C}|.$$

\begin{outer_problem}
\item Here’s a list of infinite sets, each with an operation. For each pair, answer: i.~Does it form a group? ii.~Which previous group(s) is it isomorphic to?
\end{outer_problem}

\begin{inner_problem}[start=1]
\item natural numbers, addition
\end{inner_problem}

\begin{iinner_problem}[start=1]
\item Does it form a group?
\end{iinner_problem}

Nope! It cannot satisfy the identity, since for $x+I=I+x=I$ to be true for all $x$ we need $I=0$. If you're a fan of the standard ISO 80000, and include $0\in\mathbb{N}$, then it still doesn't form a group, since it can't satisfy the invertibility property. For example, the inverse of $1$ should be $-1$ so that $1+(-1)=0$, but $-1 \not\in \mathbb{N}$.

\begin{iinner_problem}
\item Which previous group(s) is it isomorphic to?
\end{iinner_problem}

oof

\begin{inner_problem}
\item integers, addition
\end{inner_problem}

\begin{iinner_problem}[start=1]
\item Does it form a group?
\end{iinner_problem}

It does form a group. The identity element is $0$, and it satisfies all necessary properties:

Identity: $x+0=0+x=x$.

Closure: If $x,y\in \mathbb{Z}$, then $x+y\in\mathbb{Z}$.

Associativity: We have $x+(y+z)=(x+y)+z$ for all $x,y,z\in\mathbb{Z}$.

Inverse: The inverse of $x$ is $-x$, since $x+(-x)=(-x)+x=0$.

\begin{iinner_problem}
\item Which previous group(s) is it isomorphic to?
\end{iinner_problem}

None, I wonder why.

\begin{inner_problem}
\item even integers, addition
\end{inner_problem}

\begin{iinner_problem}[start=1]
\item Does it form a group?
\end{iinner_problem}

It does form a group. The identity element is $0$, and it satisfies all necessary properties:

Identity: $x+0=0+x=x$.

Closure: If $x,y\in 2\mathbb{Z}$, then $x+y=2s+2t=2(s+t)\in 2\mathbb{Z}$.

Associativity: We have $x+(y+z)=(x+y)+z$ for all $x,y,z \in 2\mathbb{Z}$.

Inverse: The inverse of $x$ is $-x$, since $x+(-x)=(-x)+x=0$.

\begin{iinner_problem}
\item Which previous group(s) is it isomorphic to?
\end{iinner_problem}

It is isomorphic to integers under addition, because we can simply correspond $2n\in 2\mathbb{Z}$ with $n\in\mathbb{Z}$. All the group structure is preserved, since $2m+2n\in 2\mathbb{Z}$ corresponds with $m+n\in\mathbb{Z}$.

\begin{inner_problem}
\item odd integers, addition
\end{inner_problem}

\begin{iinner_problem}[start=1]
\item Does it form a group?
\end{iinner_problem}

This does not form a group, since it cannot satisfy the identity property. There is no odd integer $I$ such that $x+I=I+x=x$.

\begin{iinner_problem}
\item Which previous group(s) is it isomorphic to?
\end{iinner_problem}

oof

\begin{inner_problem}
\item rational numbers, addition
\end{inner_problem}

\begin{iinner_problem}[start=1]
\item Does it form a group?
\end{iinner_problem}

Yes, this forms a group with identity element $0$. It satisfies all necessary properties:

Identity: $\frac{p}{q}+0=0+\frac{p}{q}=\frac{p}{q}$.

Closure: If $\frac{p_1}{q_1},\frac{p_2}{q_2}\in \mathbb{Q}$, then

$$\frac{p_1}{q_1}+\frac{p_2}{q_2}=\frac{p_1q_2+p_2q_1}{q_1q_2}\in \mathbb{Q}.$$

Associativity: We have $x+(y+z)=(x+y)+z$ for all $x,y,z \in 2\mathbb{Z}$.

Inverse: The inverse of $\frac{p}{q}$ is $-\frac{p}{q}$, since

$$\frac{p}{q}+\left(-\frac{p}{q}\right)=\left(-\frac{p}{q}\right)+\frac{p}{q}=0.$$

\begin{iinner_problem}
\item Which previous group(s) is it isomorphic to?
\end{iinner_problem}

None. It's not isomorphic to integers under addition because for each element $x\in \mathbb{Q}$, there exists an element $y=\frac{x}{2}\in \mathbb{Q}$ such that

$$y+y=x.$$

This is impossible for any odd integers (analogously, for the group of even integers under addition, impossible for any elements not divisible by $4$).

\begin{inner_problem}
\item real numbers, addition
\end{inner_problem}

\begin{iinner_problem}[start=1]
\item Does it form a group?
\end{iinner_problem}

Yes, this forms a group with identity element $0$. It satisfies all necessary properties:

Identity: $x+0=0+x=x$.

Closure: If $x,y\in \mathbb{R}$, then $x+y\in \mathbb{R}$.

Associativity: We have $x+(y+z)=(x+y)+z$ for all $x,y,z \in \mathbb{R}$.

Inverse: The inverse of $x$ is $-x$, since $x+(-x)=(-x)+x=0$.

\begin{iinner_problem}
\item Which previous group(s) is it isomorphic to?
\end{iinner_problem}

None. After all, $\mathbb{R}$ is uncountable, while the groups we've seen so far are countable.

\begin{inner_problem}
\item complex numbers, addition
\end{inner_problem}

\begin{iinner_problem}[start=1]
\item Does it form a group?
\end{iinner_problem}

Yes, this forms a group with identity element $0=0+0i$. It satisfies all necessary properties:

Identity: $x+0=0+x=x$.

Closure: If $x,y\in \mathbb{C}$, then $x+y\in \mathbb{C}$.

Associativity: We have $x+(y+z)=(x+y)+z$ for all $x,y,z \in \mathbb{C}$.

Inverse: The inverse of $x$ is $-x$, since $x+(-x)=(-x)+x=0$.

\begin{iinner_problem}
\item Which previous group(s) is it isomorphic to?
\end{iinner_problem}

Assuming the axiom of choice\footnote{The axiom of choice states that for every indexed family of sets $(S_i)_{i\in I}$, where $S_i\neq \emptyset$, there exists an indexed family of elements $(x_i)_{i\in I}$  such that $x_i\in S_i$ for all $i\in I$. Intuitively, this means that given a list of non-empty sets, you can select exactly one item from each set.}, it is actually isomorphic to $\mathbb{R}$ under addition. Since $\mathbb{C}$ is uncountable, this is the only candidate.

Proving they are isomorphic is tough\footnote{In fact, it is impossible to construct an ``explicit'' isomorphism because $\mathbb{R} \not \cong \mathbb{C}$ is consistent with the axiom of choice.} without the introduction of vector spaces (specifically, $\mathbb{Q}$-vector spaces). I was originally going to put it in, but I couldn't get it under a satisfactory length. If you're really curious, check out https://math.stackexchange.com/a/1511685/677124 for a mildly accessible view of the subject... if you already understand the basics of vector spaces. In summary, both $\mathbb{R}$ and $\mathbb{R}^2$ are vector spaces over the rational numbers $\mathbb{Q}$, and since $|\mathbb{R}|=|\mathbb{R}^2|$ they are isomorphic as vector spaces. This also implies that they are isomorphic as additive groups.

\begin{inner_problem}
\item integers, multiplication
\end{inner_problem}

\begin{iinner_problem}[start=1]
\item Does it form a group?
\end{iinner_problem}

No, this does not form a group. The identity element would be $1$, so that $1\cdot x = x\cdot 1 = x$, but $1\cdot 0 = 0\neq 1$, so it cannot satisfy invertibility. Even if we removed $0$, for any $p\neq \pm 1$ there is no integer $q$ such that $pq=1$.

\begin{iinner_problem}
\item Which previous group(s) is it isomorphic to?
\end{iinner_problem}

oof

\begin{inner_problem}
\item integer powers of $2$, multiplication
\end{inner_problem}

\begin{iinner_problem}[start=1]
\item Does it form a group?
\end{iinner_problem}

Yes! Let the group be called $\mathcal{W}=\{2^x\, :\, x\in \mathbb{Z}\}$ for fun. The identity element is $2^0=1$. The group properties are satisfied:

Identity: $2^x\cdot 1=1\cdot 2^x=2^x$.

Closure: $2^x\cdot 2^y = 2^{x+y} \in \mathcal{W}$.

Associativity: We have $2^{x}(2^{y}\cdot 2^{z})=(2^{x}\cdot 2^{y})2^{z}=2^{x+y+z}$ for all $x,y,z \in \mathbb{Z}$.

Inverse: The inverse of $2^x$ is $2^{-x}$, since $2^{x}2^{-x}=2^{-x}2^{x}=2^0=1$.

\begin{iinner_problem}
\item Which previous group(s) is it isomorphic to?
\end{iinner_problem}

It is isomorphic to integers under addition. $2^n\in \mathcal{W}$ corresponds with $n\in \mathbb{Z}$, since we have $$2^m\cdot 2^n=2^{m+n}\leftrightarrow m+n.$$

\begin{inner_problem}
\item rational numbers, multiplication
\end{inner_problem}

\begin{iinner_problem}[start=1]
\item Does it form a group?
\end{iinner_problem}

The rational numbers under multiplication do not form a group, because if the identity element is $1$, then $1\cdot 0 = 0\neq 1$, so it cannot satisfy invertibility.

\begin{iinner_problem}
\item Which previous group(s) is it isomorphic to?
\end{iinner_problem}

oof

\begin{inner_problem}
\item rational numbers excluding $0$, multiplication
\end{inner_problem}

\begin{iinner_problem}[start=1]
\item Does it form a group?
\end{iinner_problem}

Yes! The rational numbers excluding $0$, written $\mathbb{Q} \setminus {0}$, form a group under multiplication with identity element $1$. The group properties are satisfied:

Identity: $x\cdot 1=1\cdot x=x$.

Closure: If $x,y\in \mathbb{Q} \setminus {0}$, then $x+y\in \mathbb{Q} \setminus {0}$; the product of two nonzero rational numbers is rational and nonzero.

Associativity: We have $x(yz)=(xy)z$ for all $x,y,z \in \mathbb{Q} \setminus {0}$.

Inverse: The inverse of $x\in \mathbb{Q} \setminus {0}$ is $\frac{1}{x}$, since $x\left(\frac{1}{x}\right)=\left(\frac{1}{x}\right)x=1$ and $x\neq 0$.

\begin{iinner_problem}
\item Which previous group(s) is it isomorphic to?
\end{iinner_problem}

It is not isomorphic to any. The only candidates are other groups with countably infinite order, aka addition of rational numbers and addition of integers.

It cannot be addition of rational numbers, because for every element $k=\frac{p}{q}$ there exists an element $r=\frac{p}{2q}$ such that $k=2r$. This property is not true for rational numbers under multiplication, because the corresponding $r$ would be $\sqrt{\frac{p}{q}}$, which is only valid for perfect (rational) squares $\frac{p}{q}$. For the same reason, it cannot be addition of integers, since odd integers also cannot have this property.

\begin{inner_problem}
\item complex numbers, multiplication
\end{inner_problem}

\begin{iinner_problem}[start=1]
\item Does it form a group?
\end{iinner_problem}

No, because $0$ prevents the group from satisfying invertibility.

\begin{iinner_problem}
\item Which previous group(s) is it isomorphic to?
\end{iinner_problem}

oof

\begin{inner_problem}
\item rotation by a rational number of degrees
\end{inner_problem}

\begin{iinner_problem}[start=1]
\item Does it form a group?
\end{iinner_problem}

Yes! The identity is $0$ and it can simply by thought of adding rationals modulo $360$. The group properties are satisfied:

Identity: $x\cdot 1=1\cdot x=x$.

Closure: If $x,y\in\mathbb{Q}$ then $x+y\in \mathbb{Q}$ and it 

Associativity: We have $x(yz)=(xy)z$ for all $x,y,z \in \mathbb{Q} \setminus {0}$.

Inverse: The inverse of $x\in \mathbb{Q} \setminus {0}$ is $\frac{1}{x}$, since $x\left(\frac{1}{x}\right)=\left(\frac{1}{x}\right)x=1$ and $x\neq 0$.

\begin{iinner_problem}
\item Which previous group(s) is it isomorphic to?
\end{iinner_problem}

None. The easiest way to see this is that the element $\frac{360}{n}^\circ$, where $n$ is an integer, has period $n$, so we can construct elements of arbitrary periods. No previous groups have elements of arbitrary periods.

\begin{inner_problem}
\item rotation by a rational number of radians
\end{inner_problem}

\begin{iinner_problem}[start=1]
\item Does it form a group?
\end{iinner_problem}

Yes! The identity element is $0$ and it is totally equivalent to the rational numbers under addition. That's because for any rational radian rotations $r_{p_1/q_1}$ and $r_{p_2/q_2}$, where $p_1,q_1,p_2,q_2\in \mathbb{Z}$, $q_1,q_2\neq 0$ and $p_1/q_1\neq p_2/q_2$, we have $r_{p_1/q_1}\cong r_{p_2/q_2}$. The easiest way to understand this is that for two such rotations to be equal, there must be some integer $k$ such that

$$\frac{p_1}{q_1} = \frac{p_2}{q_2} + 2\pi k.$$

There are two cases to consider: $k=0$ and $k\neq 0$. If $k=0$, then $\frac{p_1}{q_1} = \frac{p_2}{q_2}$, which violates our assumption. If $k\neq 0$, then since $\pi$ is irrational, the RHS is irrational while the LHS is rational. Since an irrational and rational cannot be equal, such an integer $k$ cannot exist and $r_{p_1/q_1}\not\cong r_{p_2/q_2}$.

The group properties are straightforward and identical to the rationals under addition.

\begin{iinner_problem}
\item Which previous group(s) is it isomorphic to?
\end{iinner_problem}

As explained, it is isomorphic to the rational numbers under addition.

\begin{inner_problem}
\item rotation by an integer number of radians
\end{inner_problem}

\begin{iinner_problem}[start=1]
\item Does it form a group?
\end{iinner_problem}

Yes. The identity element is $0$ and it is equivalent to the integers under addition. We proceed in a similar method to the previous problem: consider two integer radian rotations $r_a$ and $r_b$ where $a,b\in \mathbb{Z}$ and $a\neq b$.

Suppose $r_a\cong r_b$. Then there is some integer $k$ such that

$$a=b+2\pi k.$$

If $k=0$, then $a=b$, contradicting our assumption that $a\neq b$. If $k\neq 0$, then the RHS is irrational while the LHS is rational. This is impossible to satisfy, so $k$ does not exist and $r_a\not\cong r_b$.

The group properties are straightforward and identical to the integers under addition.

\begin{iinner_problem}
\item Which previous group(s) is it isomorphic to?
\end{iinner_problem}

As explained, it is isomorphic to the integers under addition.

\begin{outer_problem}
\item Can an irrational number taken to an irrational power ever be rational? Consider the potential example $a = \sqrt{2}^{\sqrt{2}}$. To help you answer this question, let $b = a^{\sqrt{2}}$. Simplify $b$, and explain why we don’t need to know whether $a$ is rational or irrational.
\end{outer_problem}

We have

$$b = a^{\sqrt{2}} = \left(\sqrt{2}^{\sqrt{2}}\right)^{\sqrt{2}} = \sqrt{2} ^ {\sqrt{2}\cdot \sqrt{2}} = \sqrt{2}^2 = 2.$$

We know $\sqrt{2}$ is irrational. Suppose $a$ is rational. Then we have an example of an irrational number raised to an irrational power which \textit{is} rational! But if $a$ is not rational, then must be irrational, so $b$ is an example of an irrational number ($a$) raised to an irrational power ($\sqrt{2}$) which is rational! Thus, no matter the case, there exists an irrational number raised to an irrational power which is rational. Interestingly, we don't know whether $a$ is rational or irrational, only that one of $a,b$ satisfies the problem.

Logically, let $A=$ $a$ is rational and $C=$ there exists an irrational number raised to an irrational power which is rational. Then $A\to C$ and $\lnot A\to C$. Since $A \lor \lnot A$, we know that $C\lor C$, so $C$ is true.

As an aside, $a$ is the irrational one (and is in fact transcendental) by the Gelfond-Schneider theorem. Proving this theorem requires some pretty advanced analysis, yet we are able to derive related results with simple logic.

\end{document}
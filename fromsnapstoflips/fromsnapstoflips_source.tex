\documentclass[../gatm.tex]{subfiles}

\begin{document}

\section{From Snaps to Flips}

\newcommand{\degree}{\ensuremath{^\circ}}

\begin{asydef}
pair unitify(pair p) {
return p / (sqrt(p.x * p.x + p.y * p.y));
}

pair cis(real angle) {
return (cos(angle), sin(angle));
}

void drawTriangle(pair offset, real v_stretch, string[] order, bool showaxes = false, bool[] shownaxes = {true, true, true}, bool labelaxes = true) {
pair A = offset + (sqrt(3), 0);
pair B = offset + (0, 1);
pair C = offset + (0, -1);
pair O = offset + (sqrt(3) / 3, 0);

draw(A--B--C--cycle);

real egg = v_stretch - 1;

label(order[0], (O + egg * A) / v_stretch);
label(order[1], (O + egg * B) / v_stretch);
label(order[2], (O + egg * C) / v_stretch);

void drawExtLine(pair A, pair B, real extra = 0.2, string labelstr = "") {
	pair C = (A + extra * (A - B)), D = (B + extra * (B - A));
	draw(C -- D, dashed);
	label(labelstr, C, unitify(A - B));
}

if (showaxes) {
pair[] vertices = {A, B, C};
string[] axisnames = {"$A$", "$B$", "$C$"};

for (int i = 0; i < 3; ++i) {
	if (!shownaxes[i]) continue;
	pair vertex = vertices[i];
	string axisname = axisnames[i];
	pair sum = 0;
	
	for (int j = 0; j < 3; ++j) {
		if (j == i) continue;
		sum += vertices[j];
	}
	
	sum /= 2;
	drawExtLine(vertex, sum, labelaxes ? axisname : "");
}
}

}

string _1 = "1";
string _2 = "2";
string _3 = "3";

string[][] orders = {
{_1, _2, _3},
{_1, _3, _2},
{_3, _2, _1},
{_2, _1, _3},
{_3, _1, _2},
{_2, _3, _1}
};

string[] names = {"$I$", "$A$", "$B$", "$C$", "$D$", "$E$"};

\end{asydef}
%
\begin{figure}[h]
\begin{minipage}{0.3\textwidth}
\begin{center}
\begin{asy}
size(0,60);

pair offset = (0,0);
string[] order = {"1", "2", "3"};

drawTriangle(offset, 2, order);
\end{asy}
\end{center}
\caption{The paper triangle.}
\label{paper_triangle}
\end{minipage}
%
\begin{minipage}{0.3\textwidth}
\begin{center}
\begin{asy}
size(0,60);

pair offset = (0,0);
string[] order = {"", "", ""};

drawTriangle(offset, 2, order, true);
\end{asy}
\end{center}
\caption{Its axes of reflection.}
\label{triangle_reflections}
\end{minipage}
%
\begin{minipage}{0.3\textwidth}
\begin{center}
\begin{asy}
size(0,60);

drawTriangle((0, 0), 2, orders[4], false);
label(names[4], (sqrt(3) / 2, -1.4));
drawTriangle((3, 0), 2, orders[2], false);
label(names[2], (3 + sqrt(3) / 2, -1.4));

label("$\stackrel{A}{\Longrightarrow}$", (2.4, 0.2));
\end{asy}
\end{center}
\caption{$AD = B$; Notice the RTL evaluation.}
\label{aid_is_b}
\end{minipage}
\end{figure}

\begin{figure}[h]
\begin{center}
\begin{asy}
unitsize(0.7cm);

drawTriangle((-4, 0), 2, orders[0], false);
label("$\stackrel{\mbox{\small flip!}}{\Longrightarrow}$", (-1, 0));

for (int i = 0; i < 6; ++i) {
	bool[] shownaxes = {false, false, false};
	
	if (1 <= i && i <= 3) {
		shownaxes[i - 1] = true;
	}
	
	if (i == 4 || i == 5) {
		real angle = (i == 4) ? 2 * pi / 3 : 4 * pi / 3;
		real radius = 0.8;
		pair center = (2.4 * i + sqrt(3) / 2, 1.7);
		pair end = (center + radius * cis(angle));
		
		draw((center + radius * cis(0)) .. (center + radius * cis(angle / 2)) .. end -- end, EndArrow);
		label((i == 4) ? "$\phantom{\degree}120\degree$" : "$\phantom{\degree}240\degree$", center);
	}
	
	drawTriangle((2.4 * i, 0), 2, orders[i], true, shownaxes, false);
	label(names[i], (2.4 * i + sqrt(3) / 2, -1.4));
}
\end{asy}
\end{center}
\caption{The six ending positions.}
\label{triangle_isos}
\end{figure}

% Brandon to Timothy: stop using dashes wrong. NEVER PUT spaces around an em dash. this hurts my eyes.
% VOCAB: DIHEDRAL GROUP, ISOMETRY, ISOMORPHIC (probs gonna change?), GENERATOR

\noindent You can use a paper/cardboard triangle to help visualize the next concept: cut out an equilateral triangle, label its front vertices $1$, $2$, and $3$ as shown in Figure ~\ref{paper_triangle}, and place it down in the shown orientation. Consider the possible ways to move this triangle From this starting position, you can reflect the triangle over one of three axes: $A$, $B$, or $C$, as shown in Figure ~\ref{triangle_reflections}. You could also rotate the triangle $120\degree{}$ or $240\degree{}$ counterclockwise. The final possible positions are shown in Figure ~\ref{triangle_isos}.

Notice that each position corresponds to a different operation which preserves the triangle's location. For example, $I$ means ``leave the triangle alone,'' $A$ means ``flip the triangle about the $A$ axis,'' and $D$ means ``rotate the triangle $120\degree{}$ counterclockwise.'' We can combine these operations to form other operations by writing them in sequence. Unlike most cases, however, we evaluate them right-to-left (RTL) rather than left-to-right (LTR). For example, $AD=B$, as shown in Figure ~\ref{aid_is_b}.

These six positions form another group: the \textbf{dihedral group} of order $3$, or $D_3$. If we split ``dihedral'' into ``di-'' and ``-hedral,'' we see it means ``two faces''; these are the two faces of our paper triangle. Let's study the properties of this group.

\subsection{Problems}

% https://tex.stackexchange.com/questions/209092/how-do-i-make-every-column-the-same-width
\newcolumntype{C}{>{\centering\arraybackslash}p{1em}}

\begin{figure}[H]
\begin{minipage}{0.6\textwidth}
\begin{enumerate}
\item The six positions or ``operations'' are considered to be isometries. Isometries are ways of mapping the triangle to itself, preserving shape and location. Are there any others on this triangle?
\item As with the snap group, we can make a group table for the flip group. Fill out a table like the one in Figure ~\ref{sbstable} in your notebook. Like the snap group table, the top row indicates what operation is done first and the left column indicates what's done second, so that $XY$ is in the $X$\textsuperscript{th} row and $Y$\textsuperscript{th} column.
\newcounter{enumLast}
\setcounter{enumLast}{\theenumi}
\end{enumerate}
\end{minipage}\hfill% makes the subsequent minipage right justified
\begin{minipage}{0.35\textwidth}\centering
\begin{tabular}{C|C|C|C|C|C|C|}
$\cdot$ & $I$ & $A$ & $B$ & $C$ & $D$ & $E$ \\ \hline
$I$    &   &   &   &   &   &   \\ \hline
$A$    &   &   &   &   & $B$  &   \\ \hline
$B$    &   &   &   &   &   &   \\ \hline
$C$    &   &   &   &   &   &   \\ \hline
$D$    &   &   &   &   &   &   \\ \hline
$E$    &   &   &   &   &   &   \\ \hline
\end{tabular}
\caption{Unfilled $D_3$ group table.}
\label{sbstable}
\end{minipage}
\end{figure}
%
\begin{enumerate}
\setcounter{enumi}{\theenumLast}
\item What is the relationship between the tables for the snap group $S_3$ and the flip group $D_3$?
\item $S_3$ and $D_3$ are said to be \textbf{isomorphic}. Groups $A$ with operation $\bullet$ and $B$ with operation $\star$ are isomorphic if you can find a one-to-one correspondence between the two groups' elements. This means we can find some pairing of elements between the two groups $A_1\leftrightarrow B_1, A_2\leftrightarrow B_2, \cdots, A_n \leftrightarrow B_n$ such that $A_j \bullet A_k = A_l \leftrightarrow B_j \star B_k = B_m$.
\item \begin{enumerate}
\item Make a table for only the rotations of $D_3$, a subgroup of $D_3$.
\item Which subgroup of the snap group $S_3$ is isomorphic to the subgroup in (a)?
\end{enumerate}
\item What shape's dihedral (rotation and reflection) group is isomorphic to (a) the two post snap group $S_2$, (b) one post $S_1$, (c) four posts $S_4$ (hint: it's not a square), and (d) five posts $S_5$?
\item Find an combination of $A$ and $D$ that yields $C$.
\item We call $A$ and $D$ \textbf{generators} of the group because every element of the group is expressible as some combination of $A$s and $D$s. For convenience, let's call $A$ ``$f$'' since it's a flip, and call $D$ ``$r$'' meaning a $120\degree$ rotation counterclockwise. Then, for example, $fr^2$ is a rotation of $2\cdot 120\degree = 240\degree$, followed by a flip across the $A$ axis, equivalent to our original $C$. Make a new table using $I$, $r$, $r^2$, $f$, $fr$, and $fr^2$ as elements, like the one in Figure ~\ref{alttable}. \textit{Note that the element order is different!}
\item What other pairs of elements could you have used to generate that table?
\item You should notice the $3\times 3$ table of a group you've already described in the top-left corner of your table. What is it, and what are the two possible generators of this three-element group?
\item Explain why each element of the flip group $D_3$ has the period it has.
\item Some pairs of elements of the flip group are two-element subgroups. What are they?
\item One of the elements forms a one-element subgroup. What is it?
\setcounter{enumLast}{\theenumi}
\end{enumerate}
A \textbf{group} $G$ is a set of elements together with a \textbf{binary operation} that meets the following criteria:
\begin{enumerate}[label=(\alph*)]
\item Identity: There is an element $I\in G$ such that for all $X\in G$, $X\bullet I = I\bullet X = X$.
\item Closure: If $X$, $Y$ are elements of the group, then $X\bullet Y$ is also an element of the group.
\item Invertibility: Each element $X$ has an inverse $X^{-1}$ such that $X\bullet X^{-1} = X^{-1}\bullet X = I$.
\item Associativity: For all elements $X$, $Y$, and $Z$, $X\bullet (Y\bullet Z) = (X\bullet Y) \bullet Z$.
\end{enumerate}
\begin{enumerate}
\setcounter{enumi}{\theenumLast}
\item Addition of two numbers is a \textbf{binary operation}, while addition of three numbers is not. In logic, $\land$ (and) and $\lor$ (or) are binary operations, but $\lnot$ (not) is not. Define binary operation in your own words, and name some other binary operations.
\item In your original flip group table, what is
\begin{enumerate}
\item The identity element?
\item The inverse of $A$?
\item The inverse of $E$?
\end{enumerate}
\end{enumerate}

\begin{figure}
\begin{minipage}{0.4\textwidth}
\begin{center}
\begin{tabular}{C|C|C|C|C|C|C|}
$\cdot$ & $I$ & $r$ & $r^2$ & $f$ & $fr$ & $fr^2$ \\ \hline
$I$    &   &   &   &   &   &   \\ \hline
$r$    &   &   &   & $fr^2$  &   &   \\ \hline
$r^2$    &   &   &   &   &   &   \\ \hline
$f$    &   &   &   &   &   &   \\ \hline
$fr$    &   &   &   &   &   &   \\ \hline
$fr^2$    &   &   &   &   &   &   \\ \hline
\end{tabular}
\end{center}
\caption{Unfilled alternate $D_3$ table.}
\label{alttable}
\end{minipage}%
\begin{minipage}{0.6\textwidth}
\begin{center}
\begin{asy}
size(0,60);

drawTriangle((0, 0), 2, orders[0], false);
label(names[0], (sqrt(3) / 2, -1.4), (0,0), basealign);
drawTriangle((3, 0), 2, orders[4], false);
label("$r$", (3 + sqrt(3) / 2, -1.4), (0,0), basealign);
drawTriangle((6, 0), 2, orders[5], false);
label("$r^2$", (6 + sqrt(3) / 2, -1.4), (0,0), basealign);
drawTriangle((9, 0), 2, orders[2], false);
label("$fr^2=C$", (9 + sqrt(3) / 2, -1.4), (0,0), basealign);

label("$\stackrel{r}{\Longrightarrow}$", (2.4, 0.2));
label("$\stackrel{r}{\Longrightarrow}$", (5.4, 0.2));
label("$\stackrel{f}{\Longrightarrow}$", (8.4, 0.2));
\end{asy}
\end{center}
\caption{$fr^2=C$. Again, notice the RTL evaluation.}
\label{fr2}
\end{minipage}
\end{figure}

\end{document}
\documentclass[../gatm.tex]{subfiles}

\begin{document}

\section{From Snaps to Flips}

\begin{asydef}
pair unitify(pair p) {
return p / (sqrt(p.x * p.x + p.y * p.y));
}

void drawTriangle(pair offset, real v_stretch, string[] order, bool showaxes = false) {
pair A = offset + (sqrt(3), 0);
pair B = offset + (0, 1);
pair C = offset + (0, -1);
pair O = offset + (sqrt(3) / 3, 0);

draw(A--B--C--cycle);

real egg = v_stretch - 1;

label(order[0], (O + egg * A) / v_stretch);
label(order[1], (O + egg * B) / v_stretch);
label(order[2], (O + egg * C) / v_stretch);

void drawExtLine(pair A, pair B, real extra = 0.2, string labelstr = "") {
	pair C = (A + extra * (A - B)), D = (B + extra * (B - A));
	draw(C -- D, dashed);
	label(labelstr, C, unitify(A - B));
}

if (showaxes) {
pair[] vertices = {A, B, C};
string[] axisnames = {"$A$", "$B$", "$C$"};

for (int i = 0; i < 3; ++i) {
	pair vertex = vertices[i];
	string axisname = axisnames[i];
	pair sum = 0;
	
	for (int j = 0; j < 3; ++j) {
		if (j == i) continue;
		sum += vertices[j];
	}
	
	sum /= 2;
	drawExtLine(vertex, sum, axisname);
}
}

}

\end{asydef}

\begin{figure}[h]
\begin{minipage}{0.3\textwidth}
\begin{center}
\begin{asy}
unitsize(1cm);

pair offset = (0,0);
string[] order = {"1", "2", "3"};

drawTriangle(offset, 2, order);
\end{asy}
\end{center}
\caption{The paper triangle.}
\label{paper_triangle}
\end{minipage}
\begin{minipage}{0.4\textwidth}
\begin{center}
\begin{asy}
unitsize(1cm);

pair offset = (0,0);
string[] order = {"", "", ""};

drawTriangle(offset, 2, order, true);
\end{asy}
\end{center}
\caption{The triangles axes of reflection.}
\label{triangle_reflections}
\end{minipage}
\end{figure}

\begin{figure}

\begin{asy}
unitsize(0.7cm);
string _1 = "1";
string _2 = "2";
string _3 = "3";

string[][] orders = {
{_1, _2, _3},
{_1, _3, _2},
{_3, _2, _1},
{_2, _1, _3},
{_3, _1, _2},
{_2, _3, _1}
};

string[] names = {"$I$", "$A$", "$B$", "$C$", "$D$", "$E$"};

drawTriangle((-4, 0), 2, orders[0], false);
label("$\stackrel{\mbox{\small flip!}}{\Longrightarrow}$", (-1, 0));
for (int i = 0; i < 6; ++i) {
	drawTriangle((2.4 * i, 0), 2, orders[i], false);
	label(names[i], (2.4 * i + sqrt(3) / 2, -1.4));
}
\end{asy}

\caption{The six triangle isometries.}
\label{triangle_isos}
\end{figure}

Cut out a paper equilateral triangle and label its vertices $1$, $2$, and $3$ in permanent marker as shown in Figure ~\ref{paper_triangle}, and place it down on a table in the shown orientation. This triangle has several \textit{isometries} to itself, which are ways of mapping the triangle to itself. For example, you could \textit{reflect} it around the three axes $A$, $B$ and $C$, as shown in Figure ~\ref{triangle_reflections}. You could also rotate it by $120^{\circ}$ or $240^{\circ}$.

There are six isometries of the triangle to consider. These are shown in Figure ~\ref{triangle_isos}.

\end{document}
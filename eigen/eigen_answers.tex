\documentclass[../gatm_answers.tex]{subfiles}

\begin{document}

\section{Eigenvectors and Eigenvalues}

\begin{outer_problem}[start=1]
\item Consider the matrix equation $\twomat{0}{1}{6}{1}\twovec{x}{y} = \twovec{y}{6x+y}=\twovec{x'}{y'}$. We wish to find an eigenvector $\twovec{x}{y}$.
\end{outer_problem}

\begin{inner_problem}[start=1]
\item On graph paper, draw what the matrix $\twomat{0}{1}{6}{1}$ does to the vectors $\twovec{1}{0}$ and $\twovec{0}{1}$.
\end{inner_problem}

$\twovec{1}{0}$ goes to $\twovec{0}{6}$ (dashed lines) and $\twovec{0}{1}$ goes to $\twovec{1}{1}$ (solid lines):

\begin{center}
\begin{asy}[width=0.25\textwidth]
import graph;

draw((0,0)--(1,0),dashed,Arrow);
draw((0,0)--(0,6),dashed,Arrow);

draw((0,0)--(0,1),Arrow);
draw((0,0)--(1,1),Arrow);

xaxis("$x$", p=gray(0.7), ticks=Ticks, xmin=-2, xmax = 2);
yaxis("$y$", p=gray(0.7), ticks=Ticks, axis=XEquals(-2));

\end{asy}
\captionof{figure}{The mapping of the matrix.}
\end{center}

\begin{inner_problem}
\item In your picture, draw a rough line through the origin where you think a family of eigenvectors may be.
\end{inner_problem}

This is the line where vectors should change direction. Thus, it should be roughly where the diagram is ``flipped,'' though this definitely is not a pure reflection.

\begin{center}
\begin{asy}[width=0.25\textwidth]
import graph;

draw((0,0)--(1,0),dashed,Arrow);
draw((0,0)--(0,6),dashed,Arrow);

draw((0,0)--(0,1),Arrow);
draw((0,0)--(1,1),Arrow);

xaxis("$x$", p=gray(0.7), ticks=Ticks, xmin=-2, xmax = 2);
yaxis("$y$", p=gray(0.7), ticks=Ticks, axis=XEquals(-2));

draw((-0.5,-1.5)--(2,6), dotted);
label("$l$", (2,6), NE);

\end{asy}
\captionof{figure}{The mapping of the matrix, with the suspected eigenvectors indicated by $l$.}
\end{center}

\begin{inner_problem}
\item Try some lattice points, say $\stwovec{1}{1}$, $\stwovec{1}{2}$, $\stwovec{1}{3}$, $\stwovec{1}{4}$, $\stwovec{1}{5}$. What does the matrix transform each vector into?
\end{inner_problem}

These points are transformed as follows:

$$\twomat{0}{1}{6}{1}\twovec{1}{1} = \twovec{1}{7};$$
$$\twomat{0}{1}{6}{1}\twovec{1}{2} = \twovec{2}{8};$$
$$\twomat{0}{1}{6}{1}\twovec{1}{3} = \twovec{3}{9};$$
$$\twomat{0}{1}{6}{1}\twovec{1}{4} = \twovec{4}{10};$$
$$\twomat{0}{1}{6}{1}\twovec{1}{5} = \twovec{5}{11}.$$

\begin{inner_problem}
\item Which of these is an eigenvector?
\end{inner_problem}

$\twovec{1}{3}$ is an eigenvector, since the image is $\twovec{3}{9}$, which is just the original vector times $3$.

\begin{inner_problem}
\item Does it lie near the line you drew earlier?
\end{inner_problem}

Well, it lies \textit{on} the line I drew because I'm using the computer, but it should lie close. Here it is superimposed on the previous graph:

\begin{center}
\begin{asy}[width=0.25\textwidth]
import graph;

draw((0,0)--(1,0),dashed,Arrow);
draw((0,0)--(0,6),dashed,Arrow);

draw((0,0)--(0,1),Arrow);
draw((0,0)--(1,1),Arrow);

xaxis("$x$", p=gray(0.7), ticks=Ticks, xmin=-2, xmax = 2);
yaxis("$y$", p=gray(0.7), ticks=Ticks, axis=XEquals(-2));

draw((-0.5,-1.5)--(2,6), dotted);
label("$l$", (2,6), NE);

\end{asy}
\captionof{figure}{The mapping of the matrix, with the suspected eigenvectors indicated by $l$.}
\end{center}

\begin{outer_problem}
\item This guess-and-check process for finding eigenvectors is terrible, so let's develop a procedure to find the eigenvalues and eigenvectors for any $2\times 2$ matrix. We will use the same example.

\begin{align*}
\twomat{0}{1}{6}{1}\twovec{x}{y} &= \lambda\twovec{x}{y} & \text{Definition of eigenvector} \\
&= \lambda\twomat{1}{0}{0}{1}\twovec{x}{y} \\
\Longrightarrow \left(\twomat{0}{1}{6}{1}-\lambda\twomat{1}{0}{0}{1}\right)\twovec{x}{y}&=\twovec{0}{0} & \text{Subtraction and factoring} \\
\Longrightarrow \twomat{-\lambda}{1}{6}{1-\lambda}\twovec{x}{y} &= \twovec{0}{0}
\end{align*}
\end{outer_problem}

\begin{inner_problem}[start=1]
\item If $\stwovec{x}{y}\neq \stwovec{0}{0}$, then $$\det \twomat{-\lambda}{1}{6}{1-\lambda}=0.$$ Why? Think inverses.
\end{inner_problem}

If we left-multiply both sides of the last equation above by the inverse of the $2\times 2$ matrix, we'll get

$$\twovec{x}{y} = \twomat{-\lambda}{1}{6}{1-\lambda}^{-1}\twovec{0}{0} = \twovec{0}{0}.$$

But we assumed $\twovec{x}{y} \neq \twovec{0}{0}$, so the inverse must not exist---that's the only other case. Thus,

$$\det \twomat{-\lambda}{1}{6}{1-\lambda}=0.$$

\begin{inner_problem}
\item Find the above determinant in terms of $\lambda$ and solve for the eigenvalues.
\end{inner_problem}

We have $$\det \twomat{-\lambda}{1}{6}{1-\lambda}=-(1-\lambda)\lambda-1(6)=\lambda^2 - \lambda - 6.$$

This is just a quadratic in $\lambda$, which factors as

$$(\lambda + 2)(\lambda - 3) = 0.$$

Thus, $\lambda = -2,3$.

\begin{inner_problem}
\item One eigenvalue is $\lambda=3$. We solve for the associated eigenvector like so:
\begin{align*}
\twovec{0}{0} &= \twomat{-\lambda}{1}{6}{1-\lambda}\twovec{x}{y} \\
&= \twomat{-3}{1}{6}{-2}\twovec{x}{y} \\
\Longrightarrow \twovec{0}{0} &= \twovec{-3x+y}{6x-2y} \\
\Longrightarrow y&=3x \rightarrow \twovec{x}{y}=s\twovec{1}{3}\quad \text{(for some }s\text{)}
\end{align*}
Solve for the other eigenvector using the other eigenvalue from part (b).
\end{inner_problem}

The other eigenvalue is $-2$.

\begin{align*}
\twovec{0}{0} &= \twomat{-\lambda}{1}{6}{1-\lambda}\twovec{x}{y} \\
&= \twomat{2}{1}{6}{3}\twovec{x}{y} \\
\Longrightarrow \twovec{0}{0} &= \twovec{2x+y}{6x+3y} \\
\Longrightarrow y&=-2x \rightarrow \twovec{x}{y}=s\twovec{1}{-2}.
\end{align*}

\begin{inner_problem}
\item Check your work by multiplying the original matrix by the eigenvector!
\end{inner_problem}

$$\twomat{0}{1}{6}{1}s\twovec{1}{-2} = s\twovec{0\cdot 1 - 2\cdot 1}{6\cdot 1 - 2\cdot 1} = s\twovec{-2}{4} = -2\cdot\left(s\twovec{1}{-2}\right).$$

Indeed, the image of $s\twovec{1}{-2}$ is the pre-image scaled by $-2$.

\begin{outer_problem}
\item Solve for the eigenvectors and eigenvalues of the following matrices:
\end{outer_problem}

\begin{inner_problem}[start=1]
\item $\twomat{3}{24}{4}{7}$
\end{inner_problem}

\begin{align*}
\twomat{3}{24}{4}{7}\twovec{x}{y} &= \lambda\twovec{x}{y} & \text{Definition of eigenvector} \\
&= \lambda\twomat{1}{0}{0}{1}\twovec{x}{y} \\
\Longrightarrow \left(\twomat{3}{24}{4}{7}-\lambda\twomat{1}{0}{0}{1}\right)\twovec{x}{y}&=\twovec{0}{0} & \text{Subtraction and factoring} \\
\Longrightarrow \twomat{3-\lambda}{24}{4}{7-\lambda}\twovec{x}{y} &= \twovec{0}{0} \\
\det \twomat{3-\lambda}{24}{4}{7-\lambda} &= 0 \\
(3-\lambda)(7-\lambda)-24(4) &= 0 \\
\lambda^2 - 10\lambda - 75 &= 0 \\
(\lambda+5)(\lambda-15) &= 0 \\
\lambda &= -5,15.\\
\end{align*}

Thus, the eigenvalues are $-5$ and $15$. We now find the corresponding eigenvectors.

$-5$:

\begin{align*}
\twovec{0}{0} &= \twomat{3-\lambda}{24}{4}{7-\lambda}\twovec{x}{y} \\
&= \twomat{8}{24}{4}{12}\twovec{x}{y} \\
\Longrightarrow \twovec{0}{0} &= \twovec{8x+24y}{4x+12y} \\
\Longrightarrow y&=-3x \rightarrow \twovec{x}{y}=s\twovec{1}{-3}.
\end{align*}

The first eigenvalue-eigenvector pair is $\left\{-5, \twovec{1}{-3}\right\}$.

$15$:

\begin{align*}
\twovec{0}{0} &= \twomat{3-\lambda}{24}{4}{7-\lambda}\twovec{x}{y} \\
&= \twomat{-12}{24}{4}{-8}\twovec{x}{y} \\
\Longrightarrow \twovec{0}{0} &= \twovec{-12x+24y}{4x-8y} \\
\Longrightarrow y&=2x \rightarrow \twovec{x}{y}=s\twovec{1}{2}.
\end{align*}

The second eigenvalue-eigenvector pair is $\left\{15, \twovec{1}{2}\right\}$.

\begin{inner_problem}
\item $\twomat{3}{1}{2}{4}$
\end{inner_problem}

\begin{align*}
\twomat{3}{1}{2}{4}\twovec{x}{y} &= \lambda\twovec{x}{y} & \text{Definition of eigenvector} \\
&= \lambda\twomat{1}{0}{0}{1}\twovec{x}{y} \\
\Longrightarrow \left(\twomat{3}{1}{2}{4}-\lambda\twomat{1}{0}{0}{1}\right)\twovec{x}{y}&=\twovec{0}{0} & \text{Subtraction and factoring} \\
\Longrightarrow \twomat{3-\lambda}{1}{2}{4-\lambda}\twovec{x}{y} &= \twovec{0}{0} \\
\det \twomat{3-\lambda}{1}{2}{4-\lambda} &= 0 \\
(3-\lambda)(4-\lambda)-1(2) &= 0 \\
(\lambda-2)(\lambda-5) &= 0 \\
\lambda &= 2,5.\\
\end{align*}

Thus, the eigenvalues are $2$ and $5$. We now find the corresponding eigenvectors.

$2$:

\begin{align*}
\twovec{0}{0} &= \twomat{3-\lambda}{1}{2}{4-\lambda}\twovec{x}{y} \\
&= \twomat{1}{1}{2}{2}\twovec{x}{y} \\
\Longrightarrow \twovec{0}{0} &= \twovec{x+y}{2x+2y} \\
\Longrightarrow y&=-x \rightarrow \twovec{x}{y}=s\twovec{1}{-1}.
\end{align*}

The first eigenvalue-eigenvector pair is $\left\{2, \twovec{1}{-1}\right\}$.

$5$:

\begin{align*}
\twovec{0}{0} &= \twomat{3-\lambda}{1}{2}{4-\lambda}\twovec{x}{y} \\
&= \twomat{-2}{1}{2}{-1}\twovec{x}{y} \\
\Longrightarrow \twovec{0}{0} &= \twovec{-2x+y}{2x-y} \\
\Longrightarrow y&=2x \rightarrow \twovec{x}{y}=s\twovec{1}{2}.
\end{align*}

The second eigenvalue-eigenvector pair is $\left\{5, \twovec{1}{2}\right\}$.

\begin{inner_problem}
\item $\twomat{1}{-1}{4}{6}$
\end{inner_problem}

\begin{align*}
\twomat{1}{-1}{4}{6}\twovec{x}{y} &= \lambda\twovec{x}{y} & \text{Definition of eigenvector} \\
&= \lambda\twomat{1}{0}{0}{1}\twovec{x}{y} \\
\Longrightarrow \left(\twomat{1}{-1}{4}{6}-\lambda\twomat{1}{0}{0}{1}\right)\twovec{x}{y}&=\twovec{0}{0} & \text{Subtraction and factoring} \\
\Longrightarrow \twomat{1-\lambda}{-1}{4}{6-\lambda}\twovec{x}{y} &= \twovec{0}{0} \\
\det \twomat{1-\lambda}{-1}{4}{6-\lambda} &= 0 \\
(1-\lambda)(6-\lambda)-(-1)(4) &= 0 \\
(\lambda-2)(\lambda-5) &= 0 \\
\lambda &= 2,5.\\
\end{align*}

Thus, the eigenvalues are $2$ and $5$. Interestingly, these are the same eigenvalues as the previous problem. We now find the corresponding eigenvectors.

$2$:

\begin{align*}
\twovec{0}{0} &= \twomat{1-\lambda}{-1}{4}{6-\lambda}\twovec{x}{y} \\
&= \twomat{-1}{-1}{4}{4}\twovec{x}{y} \\
\Longrightarrow \twovec{0}{0} &= \twovec{x+y}{4x+4y} \\
\Longrightarrow y&=-x \rightarrow \twovec{x}{y}=s\twovec{1}{-1}.
\end{align*}

The first eigenvalue-eigenvector pair is $\left\{2, \twovec{1}{-1}\right\}$.

$5$:

\begin{align*}
\twovec{0}{0} &= \twomat{1-\lambda}{-1}{4}{6-\lambda}\twovec{x}{y} \\
&= \twomat{-4}{-1}{4}{1}\twovec{x}{y} \\
\Longrightarrow \twovec{0}{0} &= \twovec{-4x-y}{4x+y} \\
\Longrightarrow y&=-4x \rightarrow \twovec{x}{y}=s\twovec{1}{-4}.
\end{align*}

The second eigenvalue-eigenvector pair is $\left\{5, \twovec{1}{-4}\right\}$.

\begin{outer_problem}
\item Fill in the blanks: The image of an eigenvector will have the same \underline{\phantom{00000}} when acted on by the transformation \underline{\phantom{00000}} for which it is an eigenvector. The image of the eigenvector is simply the eigenvector itself multiplied by its corresponding \underline{\phantom{00000}}.
\end{outer_problem}

The image of an eigenvector will have the same \underline{direction} when acted on by the transformation \underline{matrix} for which it is an eigenvector. The image of the eigenvector is simply the eigenvector itself multiplied by its corresponding \underline{eigenvalue}.

\begin{outer_problem}
\item
\end{outer_problem}

\begin{inner_problem}[start=1]
\item If the transformation matrix were a reflection over a line $y=x\tan\theta$, in what directions would the two eigenvectors point? Think geometrically.
\end{inner_problem}

Geometrically, the eigenvectors would be 1. along the line $y=x\tan\theta$ and 2. perpendicular to the line. Observe the figure below if you're confused.
\begin{center}
\begin{asy}[width=0.4\textwidth]
import graph;

real theta = pi / 3.4;

draw(-expi(theta)--expi(theta)*1.5,dotted);
label("$l$", expi(theta), NW);

pair e1 = expi(theta) / 3;
pair e2 = rotate(90) * e1;

draw((0,0) -- e1, Arrow);
draw((0,0) -- e2, Arrow);

label("$e_1$", e1, NE);
label("$e_2$", e2, NW);

xaxis("x");
yaxis('y');
\end{asy}
\captionof{figure}{The eigenvectors $e_1,e_2$ of reflection over the line $l$.}
\end{center}

\begin{inner_problem}
\item What would the angle between them be?
\end{inner_problem}

The angle between them is $90^\circ$, since one is along a line and the other is perpendicular to that line.

\begin{inner_problem}
\item What would their eigenvalues be?
\end{inner_problem}

Referring to the above figure, $e_1$ would have an eigenvalue of $1$, since its magnitude and direction is completely preserved, while $e_2$ would have an eigenvalue of $-1$, since it is multiplied by $-1$ to be inverted like that.

\begin{outer_problem}
\item Recall that multiplication by $\twomat{\cos 2\theta}{\sin 2\theta}{\sin 2\theta}{-\cos 2\theta}$ results in a reflection over $y=x\tan \theta$.
\end{outer_problem}

\begin{inner_problem}[start=1]
\item Write a matrix that results in a reflection over the line $y=\frac{\sqrt{3}}{3}x.$
\end{inner_problem}

The angle here is $$\tan^{-1} \frac{\sqrt{3}}{3} = \tan^{-1} \frac{1}{\sqrt{3}} = \tan^{-1} \frac{1/2}{1/\sqrt{3}} = 30^\circ.$$

Thus, the matrix is

$$\twomat{\cos 2\cdot 30^\circ}{\sin 2\cdot 30^\circ}{\sin 2\cdot 30^\circ}{-\cos 2\cdot 30^\circ} = \twomat{\frac{1}{2}}{\frac{\sqrt{3}}{2}}{\frac{\sqrt{3}}{2}}{-\frac{1}{2}}.$$

\begin{inner_problem}
\item Find the eigenvalues of that matrix, and the corresponding eigenvectors.
\end{inner_problem}

We find the eigenvalues:

$$\twomat{\frac{1}{2}-\lambda}{\frac{\sqrt{3}}{2}}{\frac{\sqrt{3}}{2}}{-\frac{1}{2}-\lambda}\twovec{x}{y}=\twovec{0}{0}$$
$$\left(\frac{1}{2}-\lambda \right)\left(-\frac{1}{2}-\lambda\right) - \frac{\sqrt{3}}{2}\cdot \frac{\sqrt{3}}{2} = \lambda^2 - \frac{1}{4} - \frac{3}{4} = \lambda^2-1$$
$$\lambda = \pm 1.$$

We then find the corresponding eigenvectors:

$1$:

\begin{align*}
\twovec{0}{0} &= \twomat{\frac{1}{2}-\lambda}{\frac{\sqrt{3}}{2}}{\frac{\sqrt{3}}{2}}{-\frac{1}{2}-\lambda}\twovec{x}{y} \\
&= \twomat{-\frac{1}{2}}{\frac{\sqrt{3}}{2}}{\frac{\sqrt{3}}{2}}{-\frac{3}{2}}\twovec{x}{y} \\
\Longrightarrow \twovec{0}{0} &= \twovec{-\frac{1}{2}x+\frac{\sqrt{3}}{2}y}{\frac{\sqrt{3}}{2}x-\frac{3}{2}y} \\
\Longrightarrow y&=\frac{x}{\sqrt{3}} \rightarrow \twovec{x}{y}=s\twovec{\sqrt{3}}{1}.
\end{align*}

$-1$:

\begin{align*}
\twovec{0}{0} &= \twomat{\frac{1}{2}-\lambda}{\frac{\sqrt{3}}{2}}{\frac{\sqrt{3}}{2}}{-\frac{1}{2}-\lambda}\twovec{x}{y} \\
&= \twomat{\frac{3}{2}}{\frac{\sqrt{3}}{2}}{\frac{\sqrt{3}}{2}}{\frac{1}{2}}\twovec{x}{y} \\
\Longrightarrow \twovec{0}{0} &= \twovec{\frac{3}{2}x+\frac{\sqrt{3}}{2}y}{\frac{\sqrt{3}}{2}x+\frac{1}{2}y} \\
\Longrightarrow y&=-x\sqrt{3} \rightarrow \twovec{x}{y}=s\twovec{1}{-\sqrt{3}}.
\end{align*}

Thus, the eigenvalue-eigenvector pairs are $\left\{1, \twovec{\sqrt{3}}{1}\right\}$ and $\left\{1, \twovec{1}{-\sqrt{3}}\right\}$.

\begin{inner_problem}
\item Do your calculations agree with your answers to the previous problem?
\end{inner_problem}

Yes, they do. Here are the graphs of those two eigenvectors:

\begin{center}
\begin{asy}[width=0.38\textwidth]
import graph;

real theta = pi / 3.4;

draw(-expi(theta)--expi(theta)*1.8,dotted);
label("$\theta=30^\circ$", expi(theta)*1.8, NW);

pair e1 = expi(theta);
pair e2 = rotate(90) * e1;

draw((0,0) -- e1, Arrow);
draw((0,0) -- e2, Arrow);

label("$e_1$", e1, NW);
label("$e_2$", e2, NW);

xaxis("x", ticks=Ticks);
yaxis('y', ticks=Ticks);
\end{asy}
\captionof{figure}{The calculated eigenvectors with the line $\theta=30^\circ$.}
\end{center}

The corresponding eigenvalues also match up.

\begin{inner_problem}
\item What are the relationships between the two eigenvectors and between the two eigenvalues?
\end{inner_problem}

The two eigenvectors are $90^\circ$ displaced from one another. The two eigenvalues are opposites.

\begin{outer_problem}
\item
\end{outer_problem}

\begin{inner_problem}[start=1]
\item Write a matrix which results in a $60^\circ$ rotation counterclockwise.
\end{inner_problem}

This is just $\twomat{\cos 60^\circ}{-\sin 60^\circ}{\sin 60^\circ}{\cos 60^\circ}=\twomat{\frac{1}{2}}{-\frac{\sqrt{3}}{2}}{\frac{\sqrt{3}}{2}}{\frac{1}{2}}$.

\begin{inner_problem}
\item Find the eigenvalues. What do you find strange?
\end{inner_problem}

We solve the equation $$\det \twomat{\frac{1}{2} - \lambda}{-\frac{\sqrt{3}}{2}}{\frac{\sqrt{3}}{2}}{\frac{1}{2} - \lambda} = 0.$$

\begin{align*}
\left(\frac{1}{2} - \lambda\right)\left(\frac{1}{2} - \lambda\right)-\left(-\frac{\sqrt{3}}{2}\right)\cdot \frac{\sqrt{3}}{2} &= 0 \\
\lambda^2 - \lambda + \frac{1}{4} + \frac{3}{4} &= 0 \\
\lambda^2 - \lambda + 1 &= 0 \\
\lambda &= \frac{1\pm\sqrt{-3}}{2} \\
\lambda &= \frac{1}{2} + \frac{\sqrt{3}}{2}i, \frac{1}{2} - \frac{\sqrt{3}}{2}i.\\
\end{align*}

The eigenvalues are complex! They have magnitude $1$, however, like the eigenvalues of the reflection.

\begin{inner_problem}
\item Find the eigenvectors for those eigenvalues. What's strange about them?
\end{inner_problem}

$\frac{1}{2} + \frac{\sqrt{3}}{2}i:$

\begin{align*}
\twomat{\frac{1}{2} - \lambda}{-\frac{\sqrt{3}}{2}}{\frac{\sqrt{3}}{2}}{\frac{1}{2} - \lambda}\twovec{x}{y} &= \twomat{-\frac{\sqrt{3}}{2}i}{-\frac{\sqrt{3}}{2}}{\frac{\sqrt{3}}{2}}{-\frac{\sqrt{3}}{2}i}\twovec{x}{y} &= \twovec{0}{0} \\
\twovec{-\frac{\sqrt{3}}{2}ix-\frac{\sqrt{3}}{2}y}{\frac{\sqrt{3}}{2}x-\frac{\sqrt{3}}{2}iy} &= \twovec{0}{0} \\
\Longrightarrow y &= -ix \rightarrow \twovec{x}{y} = s\twovec{1}{-i}.
\end{align*}

Weird!

$\frac{1}{2} - \frac{\sqrt{3}}{2}i:$

\begin{align*}
\twomat{\frac{1}{2} - \lambda}{-\frac{\sqrt{3}}{2}}{\frac{\sqrt{3}}{2}}{\frac{1}{2} - \lambda}\twovec{x}{y} &= \twomat{\frac{\sqrt{3}}{2}i}{-\frac{\sqrt{3}}{2}}{\frac{\sqrt{3}}{2}}{\frac{\sqrt{3}}{2}i}\twovec{x}{y} &= \twovec{0}{0} \\
\twovec{\frac{\sqrt{3}}{2}ix-\frac{\sqrt{3}}{2}y}{\frac{\sqrt{3}}{2}x+\frac{\sqrt{3}}{2}iy} &= \twovec{0}{0} \\
\Longrightarrow y &= ix \rightarrow \twovec{x}{y} = s\twovec{1}{i}.
\end{align*}

Fascinating! The eigenvalue-eigenvector pairs are $\left\{\frac{1}{2} + \frac{\sqrt{3}}{2}i, \twovec{1}{-i}\right\}$ and $\left\{\frac{1}{2} - \frac{\sqrt{3}}{2}i, \twovec{1}{i}\right\}$.

\begin{inner_problem}
\item Explain what's going on.
\end{inner_problem}

There are no vectors which don't change orientation under a rotation, so the solutions we get can't be real. Nonetheless, a quadratic always has two roots if the discriminant is nonzero, so we get two solutions.

\begin{inner_problem}
\item What are the relationships between the two eigenvectors and between the two eigenvalues?
\end{inner_problem}

The eigenvalues are each other's complex conjugates. The eigenvectors, graphed in the complex plane, form a $90^\circ$ between each other.

\begin{outer_problem}
\item The matrix $\twomat{1}{2}{0}{1}$ is a shear parallel to the $x$ axis.
\end{outer_problem}

\begin{inner_problem}[start=1]
\item What vectors don't change direction when multiplied by this matrix?
\end{inner_problem}

Only vectors parallel to the $x$-axis don't change direction, i.e. $\twovec{s}{0}$ for any real $s$.

\begin{inner_problem}
\item What would you expect the eigenvectors to be?
\end{inner_problem}

We'd expect there to only be one family of eigenvectors, $\twovec{1}{0}$.

\begin{inner_problem}
\item Find the eigenvectors and eigenvalues of this matrix.
\end{inner_problem}

We want $\det \twomat{1-\lambda}{2}{0}{1-\lambda}=0$, which simplifies to $(\lambda -1)^2 = 0$. Thus, there is only one eigenvalue: $1$. This makes sense.

The eigenvector is the solution to $\twomat{0}{2}{0}{0}\twovec{x}{y} = \twovec{0}{0}$, which gives $\twovec{2y}{0}=\twovec{0}{0}$. Thus, $y=0$, and the eigenvectors are the family $s\twovec{1}{0}$.

\begin{inner_problem}
\item What is different this time?
\end{inner_problem}

There is only one eigenvector and eigenvalue!

\begin{inner_problem}
\item Can you represent every vector as sums of eigenvectors?
\end{inner_problem}

In this case, you cannot represent every vector as a sum of eigenvectors. After all, any sum of the one eigenvector cannot have a nonzero $y$ coordinate.

\begin{outer_problem}
\item The matrices below result in some stretches. Find the eigenvectors and eigenvalues for both.
\end{outer_problem}

\begin{inner_problem}[start=1]
\item $\left[\begin{smallmatrix} 2 & 0 \\ 0 & 5 \end{smallmatrix}\right]$
\end{inner_problem}

The characteristic polynomial\footnote{The polynomial involving $\lambda$ determining the eigenvalues.} is just $(2-\lambda)(5-\lambda)$, which gives eigenvalues $\lambda=2,5$.

$2$:

$$\twomat{2-2}{0}{0}{5-2}\twovec{x}{y} = \twovec{0}{0}$$
$$\twovec{0}{3y} = \twovec{0}{0}.$$

Thus, $y=0$ and the family of eigenvectors is $\twovec{1}{0}$.

$5$:

$$\twomat{2-5}{0}{0}{5-5}\twovec{x}{y} = \twovec{0}{0}$$
$$\twovec{-3x}{0} = \twovec{0}{0}.$$

Thus, $x=0$ and the family of eigenvectors is $\twovec{0}{1}$.

\begin{inner_problem}
\item $\left[\begin{smallmatrix} 3 & 0 \\ 0 & 3 \end{smallmatrix}\right]$
\end{inner_problem}

The characteristic polynomial here is $(3-\lambda)(3-\lambda)$, yielding $\lambda = 3$. We find the eigenvector:

$$\twomat{3-3}{0}{0}{3-3}\twovec{x}{y} = \twovec{0}{0}$$
$$\twovec{0}{0} = \twovec{0}{0}.$$

Thus, all $\twovec{x}{y}$ are eigenvectors. This makes sense! After all, all vectors are scaled up by a factor of $3$, and no vectors change direction.

\begin{outer_problem}
\item Note that most $2\times 2$ matrices have two eigenvectors. How many would you expect to find for an $n\times n$ matrix?
\end{outer_problem}

You'd expect there to be $n$ eigenvectors in an $n\times n$ matrix. One way to rationalize this further is that the characteristic polynomial of an $n\times n$ matrix is degree $n$, which usually has $n$ roots.

\begin{outer_problem}
\item Assuming that $p,q,r,s,t,u,x,y$ are real, what conditions would you impose on them in the matrices (i)~$\twomat{3}{p}{q}{4}$, (ii)~$\twomat{x}{-2}{3}{y}$, and (iii)~$\twomat{r}{s}{t}{u}$ to have...
\end{outer_problem}

\begin{inner_problem}[start=1]
\item ... two real eigenvalues?
\end{inner_problem}

\begin{iinner_problem}[start=1]
\item $\twomat{3}{p}{q}{4}$
\end{iinner_problem}

The characteristic polynomial here is $(3-\lambda)(4-\lambda)-pq$. Expanded out, this is $\lambda^2 - 7\lambda + 12 - pq$. We want the discriminant to be greater than $0$ to have two real eigenvalues, so

\begin{align*}
b^2-4ac=7^2-4(1)(12-pq) &> 0 \\
48-4pq &< 49 & \text{Manipulate, flip the inequality}\\
4pq &> -1 \text{Subtract 48 from both sides}\\
pq &> -\frac{1}{4}. \\
\end{align*}

This is our restriction; we must have $pq>-\frac{1}{4}$.

\begin{iinner_problem}
\item $\twomat{x}{-2}{3}{y}$
\end{iinner_problem}

The characteristic polynomial here is $(x-\lambda)(y-\lambda)-(-2)(3)$. This expands out to $\lambda^2 - (x+y)\lambda + xy + 6$. Again, we want the discriminant to be greater than $0$ to have two real eigenvalues, so

\begin{align*}
b^2-4ac=(x+y)^2-4(1)(xy+6) &> 0 \\
x^2 + y^2 + 2xy - 4xy - 24 &> 0 \\
x^2 - 2xy + y^2 &> 24 \\
(x-y)^2 &> 24. \\
\end{align*}

Thus, our restriction is $(x-y)^2 > 24$, or equivalently, $|x-y| > \sqrt{24} = 2\sqrt{6}.$

This is always true by the Trivial Inequality\footnote{The \textit{Trivial Inequality} states $x^2\geq 0$ for all real $x$.}. Thus, there are always two real eigenvalues for this matrix.

\begin{iinner_problem}
\item $\twomat{r}{s}{t}{u}$
\end{iinner_problem}

The characteristic polynomial here is $(r-\lambda)(u-\lambda)-st$, which expands out to

$$\lambda^2 - (u+r)\lambda + ru - st.$$

Again, we want the discriminant to be greater than $0$ to have two real eigenvalues, so
\begin{align*}
b^2-4ac=(u+r)^2 - 4(ru-st)  &> 0 \\
u^2 + r^2 + 2ru - 4ru - 4st &> 0 \\
u^2 - 2ru + r^2 &> 4st \\
(u-r)^2 &> 4st. \\
\end{align*}

There isn't a great way to simplify this, but $(u-r)^2 > 4st$ is a potential answer. The first line of the equations above also gives us a potentially simpler interpretation:

$$(u+r)^2 > 4(ru-st) = 4\det \twomat{r}{s}{t}{u}.$$

Thus, the sum of the top-left to bottom-right diagonal squared must be greater than four times the determinant. Wordy!

\begin{inner_problem}
\item ... two complex eigenvalues?
\end{inner_problem}

\begin{iinner_problem}[start=1]
\item $\twomat{3}{p}{q}{4}$
\end{iinner_problem}

This is identical to problem (a) part i, but we want the discriminant to be smaller than $0$. The proof is identical, just with a flipped inequality sign, so the answer is

$$pq < -\frac{1}{4}.$$

\begin{iinner_problem}
\item $\twomat{x}{-2}{3}{y}$
\end{iinner_problem}

This is identical to problem (a) part ii, but we want the discriminant to be smaller than $0$. The proof is identical, just with a flipped inequality sign, so the answer is

$$(x-y)^2 < 24.$$

\begin{iinner_problem}
\item $\twomat{r}{s}{t}{u}$
\end{iinner_problem}

This is identical to problem (a) part iii, but we want the discriminant to be smaller than $0$. The proof is identical, just with a flipped inequality sign, so the answer is

$$(u-r)^2 < 4st \quad\text{or}\quad (u+r)^2 < 4\det \twomat{r}{s}{t}{u}.$$

\begin{inner_problem}
\item ... only one eigenvalue?
\end{inner_problem}

\begin{iinner_problem}[start=1]
\item $\twomat{3}{p}{q}{4}$
\end{iinner_problem}

This is identical to the previous two iterations of this matrix, but with an equality sign:

$$pq = -\frac{1}{4}.$$

\begin{iinner_problem}
\item $\twomat{x}{-2}{3}{y}$
\end{iinner_problem}

This is identical to the previous two iterations of this matrix, but with an equality sign:

$$(x-y)^2 = 24 \rightarrow x-y = \pm 2\sqrt{6}.$$

\begin{iinner_problem}
\item $\twomat{r}{s}{t}{u}$
\end{iinner_problem}

This is identical to the previous two iterations of this matrix, but with an equality sign:

$$(u-r)^2 = 4st \quad\text{or}\quad (u+r)^2 = 4\det \twomat{r}{s}{t}{u}.$$

\begin{outer_problem}
\item
\end{outer_problem}

\begin{inner_problem}[start=1]
\item Write a $3\times 3$ matrix showing a rotation of $\theta$ around the $z$ axis.
\end{inner_problem}

We already did this a couple sections ago. The matrix is

$$\threemat{\cos\theta}{-\sin\theta}{0}{\sin\theta}{\cos\theta}{0}{0}{0}{1}.$$

\begin{inner_problem}
\item Name the real eigenvector (this shouldn't require any work).
\end{inner_problem}

The real eigenvector is the $z$-axis, since it doesn't move (observe the figure below if you're confused). Explicitly, this is the family of eigenvectors

$$s\begin{bmatrix} 0 \\ 0 \\ 1 \end{bmatrix}.$$

\begin{center}
\begin{asy}[width=0.5\textwidth]
import graph3;

currentprojection=perspective(4,5,3.5);

real theta = 27; // degrees

triple v1 = unit((-1,-1,1));
triple v2 = unit((-1,2,0));
triple v3 = unit((0.5,-0.3,-0.8));
triple v4 = (0,0,1);

triple O = (0,0,0);

draw(O--v1, Arrow3);
draw(O--v2, Arrow3);
draw(O--v3, Arrow3);
draw(O--v4, Arrow3);

label("$v_1$", v1, NE);
label("$v_2$", v2, NE);
label("$v_3$", v3, NW);
label("$e_1=e_1'$", v4, NW);

transform3 rotz = rotate(theta, O, v4);

triple v1p = rotz*v1;
triple v2p = rotz*v2;
triple v3p = rotz*v3;

label("$v_1'$", v1p, NW);
label("$v_2'$", v2p, NW);
label("$v_3'$", v3p, NW);

draw(O--v1p, Arrow3);
draw(O--v2p, Arrow3);
draw(O--v3p, Arrow3);

xaxis3("$x$",p=gray(0.7));
yaxis3("$y$",p=gray(0.7));
zaxis3("$z$",p=gray(0.7));

triple nut(triple a) {
  return (a.x, a.y, 0);
}

draw(v1--nut(v1), dotted);
draw(v2--nut(v2), dotted);
draw(v3--nut(v3), dotted);
draw(v1p--nut(v1p), dotted);
draw(v2p--nut(v2p), dotted);
draw(v3p--nut(v3p), dotted);
\end{asy}
\captionof{figure}{The $z$-axis remains stationary in a rotation of $\theta$ around the $z$-axis.}
\end{center}

\begin{inner_problem}
\item Find all three eigenvectors.
\end{inner_problem}

The determinant of the eigenvector matrix is, by the minors method:

$$0=\det \threemat{\cos\theta - \lambda}{-\sin\theta}{0}{\sin\theta}{\cos\theta- \lambda}{0}{0}{0}{1- \lambda} = (\cos\theta- \lambda) \det \twomat{\cos\theta- \lambda}{0}{0}{1- \lambda} - (-\sin\theta)\det \twomat{\sin\theta}{0}{0}{1-\lambda} + 0\left(\text{something}\right)$$
$$=(\cos\theta-\lambda)(\cos\theta-\lambda)(1-\lambda) + (\sin\theta)(\sin\theta)(1-\lambda)$$
$$=(1-\lambda)((\cos\theta-\lambda)^2+(\sin\theta)(\sin\theta))$$
$$=(1-\lambda)(lambda^2 + \cos^2\theta - 2\lambda\cos\theta + \sin^2\theta)$$
$$=(1-\lambda)(\lambda^2-2\lambda\cos\theta+\underbrace{\cos^2\theta+\sin^2\theta}_{\text{diff of squares}})$$
$$=(1-\lambda)(\lambda-(\cos \theta - i\sin\theta))(\lambda - (\cos\theta + i\sin\theta)).$$

This gives eigenvalues $1$, $\cos \theta - i\sin\theta$ and $\cos\theta + i\sin\theta$. We already knew about the first one.

We now compute the eigenvectors:

$1$:

$$\threemat{\cos\theta - \lambda}{-\sin\theta}{0}{\sin\theta}{\cos\theta- \lambda}{0}{0}{0}{1- \lambda}\threevec{x}{y}{z} = \threevec{0}{0}{0}$$
$$\threemat{\cos\theta - 1}{-\sin\theta}{0}{\sin\theta}{\cos\theta- 1}{0}{0}{0}{0}\threevec{x}{y}{z} = \threevec{(\cos\theta-1)x-(\sin\theta)y}{(\sin\theta)x+(\cos\theta-1)y}{0} = \threevec{0}{0}{0}.$$

Thus, $(\cos\theta-1)x-(\sin\theta)y=0$ and $(\sin\theta)x + (\cos\theta-1)y=0$. The first equation yields $x=\frac{\sin\theta}{\cos\theta - 1}y$. Substitution into the second equation yields

$$\frac{y\sin^2\theta}{\cos\theta - 1} + (\cos\theta - 1)y = 0$$
$$y\left(\frac{\sin^2 \theta + (\cos\theta - 1)^2}{\cos\theta-1}\right) = 0$$
$$y\left(\frac{\sin^2 \theta + \cos^2\theta - 2\cos\theta + 1}{\cos\theta - 1}\right) = 0$$
$$y\left(\frac{2-2\cos\theta}{\cos\theta-1}\right)=0$$
$$-2y=0$$
$$y=0.$$

Thus, $x=y=0$, or $\cos\theta-1 = 0$ (since then the first substitution is invalid). This makes sense! If $\cos\theta=1$, then it's a rotation by $0^\circ$, which has all vectors as eigenvectors. Anyway, this otherwise gives the family of eigenvectors $s\threevec{0}{0}{1}$.

$\cos\theta - i\sin\theta$:

$$\threemat{\cos\theta - \lambda}{-\sin\theta}{0}{\sin\theta}{\cos\theta- \lambda}{0}{0}{0}{1- \lambda}\threevec{x}{y}{z} = \threevec{0}{0}{0}$$
$$\threemat{i\sin\theta}{-\sin\theta}{0}{\sin\theta}{i\sin\theta}{0}{0}{0}{i\sin\theta - \cos\theta}\threevec{x}{y}{z}=\threevec{(i\sin\theta)x-(\sin\theta)y+1}{(\sin\theta)x+(i\sin\theta)y}{(i\sin\theta - \cos\theta+1)z} =\threevec{0}{0}{0}.$$

Since $i\sin\theta - \cos\theta + 1 \neq 0$ except for $\cos\theta=1$ (the rotation of $0$ again), this yields $z=0$ and $x=iy$, so the family of eigenvectors is $s\threevec{i}{1}{0}$.

$\cos\theta + i\sin\theta$:

$$\threemat{\cos\theta - \lambda}{-\sin\theta}{0}{\sin\theta}{\cos\theta- \lambda}{0}{0}{0}{1- \lambda}\threevec{x}{y}{z} = \threevec{0}{0}{0}$$
$$\threemat{-i\sin\theta}{-\sin\theta}{0}{\sin\theta}{-i\sin\theta}{0}{0}{0}{ \cos\theta - i\sin\theta + 1}\threevec{x}{y}{z}=\threevec{-(i\sin\theta)x-(\sin\theta)y}{(\sin\theta)x-(i\sin\theta)y}{(1-i\sin\theta-\cos\theta)z} =\threevec{0}{0}{0}.$$

For nonzero rotations, this yields $z=0$ and $x=-iy$, giving the family of eigenvectors $s=\threevec{-i}{1}{0}$.

Overall, the eigenvalue-eigenvector pairs are $\left\{1, \threevec{0}{0}{1}\right\}$, $\left\{\cos\theta - i\sin\theta, \threevec{i}{1}{0}\right\}$, and $\left\{\cos\theta + i\sin\theta, \threevec{-i}{1}{0}\right\}$.

\begin{outer_problem}
\item
\end{outer_problem}

\begin{inner_problem}
\item What should the absolute value of an eigenvalue of any rotation matrix be?
\end{inner_problem}

It should be $1$, since rotations don't stretch anything and doesn't change orientation. All distances are preserved. This is true of our eigenvectors.

\begin{inner_problem}
\item The complex eigenvalues relate to the angle of rotation. What is that relationship?
\end{inner_problem}

The complex eigenvalues are $\cos\theta + i\sin\theta=\cis\theta$ and $\cos\theta - i\sin\theta=\overline{\cis\theta}$, so they make an angle of $\theta$ with the real axis\footnote{Note that we shouldn't call it the $x$-axis, because this is a different set of axes than the $xyz$-axes we're considering in this problem.} in the complex plane. Furthermore, the angle between them is $2\theta$.

\begin{outer_problem}
\item In a right-handed coordinate system, rotations in three dimensions are performed by combinations of the three matrices
$$X=\threemat{1}{0}{0}{0}{\cos\alpha}{-\sin\alpha}{0}{\sin\alpha}{\cos\alpha},\, Y=\threemat{\cos\beta}{0}{\sin\beta}{0}{1}{0}{-\sin\beta}{0}{\cos\beta},\, Z=\threemat{\cos\gamma}{-\sin\gamma}{0}{\sin\gamma}{\cos\gamma}{0}{0}{0}{1}.$$
Each matrix $X,Y,Z$ rotates around the $x,y,z$ axes by $\alpha,\beta,\gamma$, respectively.

In 2D, rotations combine to make other rotations. Similarly, if we combine any number of these rotations, the net result will be a rotation about some axis---though not necessarily a \textit{coordinate} axis. Another way to picture this is that if we operate on an origin-centered sphere with these matrices, there will always be two opposite points on the sphere which have no net movement.

Try computing the following products.
\end{outer_problem}

\begin{inner_problem}[start=1]
\item $XY$
\end{inner_problem}

$$XY = \threemat{1}{0}{0}{0}{\cos\alpha}{-\sin\alpha}{0}{\sin\alpha}{\cos\alpha}\threemat{\cos\beta}{0}{\sin\beta}{0}{1}{0}{-\sin\beta}{0}{\cos\beta}$$
$$=\threemat{(1)(\cos\beta) + (0)(0) + (0)(-\sin\beta)}{(1)(0) + (0)(1) + (0)(0)}{(1)(\sin\beta) + (0)(0) + (0)(\cos\beta)}{(0)(\cos\beta) + (\cos\alpha)(0) + (-\sin\alpha)(-\sin\beta)}{(0)(0) + (\cos\alpha)(1) + (-\sin\alpha)(0)}{(0)(\sin\beta) + (\cos\alpha)(0) + (-\sin\alpha)(\cos\beta)}{(0)(\cos\beta) + (\sin\alpha)(0) + (\cos\alpha)(-\sin\beta)}{(0)(0) + (\sin\alpha)(1) + (\cos\alpha)(0)}{(0)(\sin\beta) + (\sin\alpha)(0) + (\cos\alpha)(\cos\beta)}$$
$$=\threemat{\cos\beta}{0}{\sin\beta}{\sin\alpha\sin\beta}{\cos\alpha}{-\sin\alpha\cos\beta}{-\cos\alpha\sin\beta}{\sin\alpha}{\cos\alpha\cos\beta}.$$

\begin{inner_problem}
\item $XZ$
\end{inner_problem}

$$XZ = \threemat{1}{0}{0}{0}{\cos\alpha}{-\sin\alpha}{0}{\sin\alpha}{\cos\alpha}\threemat{\cos\gamma}{-\sin\gamma}{0}{\sin\gamma}{\cos\gamma}{0}{0}{0}{1}$$
$$=\threemat{(1)(\cos\gamma) + (0)(\sin\gamma) + (0)(0)}{(1)(-\sin\gamma) + (0)(\cos\gamma) + (0)(0)}{(1)(0) + (0)(0) + (0)(1)}{(0)(\cos\gamma) + (\cos\alpha)(\sin\gamma) + (-\sin\alpha)(0)}{(0)(-\sin\gamma) + (\cos\alpha)(\cos\gamma) + (-\sin\alpha)(0)}{(0)(0) + (\cos\alpha)(0) + (-\sin\alpha)(1)}{(0)(\cos\gamma) + (\sin\alpha)(\sin\gamma) + (\cos\alpha)(0)}{(0)(-\sin\gamma) + (\sin\alpha)(\cos\gamma) + (\cos\alpha)(0)}{(0)(0) + (\sin\alpha)(0) + (\cos\alpha)(1)}$$
$$=\threemat{\cos\gamma}{-\sin\gamma}{0}{\cos\alpha\sin\gamma}{\cos\alpha\cos\gamma}{-\sin\alpha}{\sin\alpha\sin\gamma}{\sin\alpha\cos\gamma}{\cos\alpha}.$$

\begin{inner_problem}
\item $YX$
\end{inner_problem}

\iffalse
s=r"""\threemat{0}{0}{1}{0}{1}{0}{-1}{0}{0}\threemat{1}{0}{0}{0}{0}{-1}{0}{1}{0}""";m=map(lambda mtrx: mtrx.lstrip("{").rstrip("}").split("}{"), s.split(r"\threemat")[1:]);print( "\\threemat" + ''.join([item for sublist in (map(lambda i: map(lambda j: "{" + " + ".join(map(lambda c: "(%s)(%s)" % (m[0][3*i+c],m[1][3*c+j]), range(3))) + "}", range(3)), range(3))) for item in sublist]));
\fi

$$YX = \threemat{\cos\beta}{0}{\sin\beta}{0}{1}{0}{-\sin\beta}{0}{\cos\beta}\threemat{1}{0}{0}{0}{\cos\alpha}{-\sin\alpha}{0}{\sin\alpha}{\cos\alpha}$$

$$=\threemat{(\cos\beta)(1) + (0)(0) + (\sin\beta)(0)}{(\cos\beta)(0) + (0)(\cos\alpha) + (\sin\beta)(\sin\alpha)}{(\cos\beta)(0) + (0)(-\sin\alpha) + (\sin\beta)(\cos\alpha)}{(0)(1) + (1)(0) + (0)(0)}{(0)(0) + (1)(\cos\alpha) + (0)(\sin\alpha)}{(0)(0) + (1)(-\sin\alpha) + (0)(\cos\alpha)}{(-\sin\beta)(1) + (0)(0) + (\cos\beta)(0)}{(-\sin\beta)(0) + (0)(\cos\alpha) + (\cos\beta)(\sin\alpha)}{(-\sin\beta)(0) + (0)(-\sin\alpha) + (\cos\beta)(\cos\alpha)}$$
$$=\threemat{\cos\beta}{\sin\beta\sin\alpha}{\sin\beta\cos\alpha}{0}{\cos\alpha}{-\sin\alpha}{-\sin\beta}{\cos\beta\sin\alpha}{\cos\beta\cos\alpha}.$$

\begin{inner_problem}
\item $ZX$
\end{inner_problem}

$$ZX = \threemat{\cos\gamma}{-\sin\gamma}{0}{\sin\gamma}{\cos\gamma}{0}{0}{0}{1}\threemat{1}{0}{0}{0}{\cos\alpha}{-\sin\alpha}{0}{\sin\alpha}{\cos\alpha}$$
$$=\threemat{(\cos\gamma)(1) + (-\sin\gamma)(0) + (0)(0)}{(\cos\gamma)(0) + (-\sin\gamma)(\cos\alpha) + (0)(\sin\alpha)}{(\cos\gamma)(0) + (-\sin\gamma)(-\sin\alpha) + (0)(\cos\alpha)}{(\sin\gamma)(1) + (\cos\gamma)(0) + (0)(0)}{(\sin\gamma)(0) + (\cos\gamma)(\cos\alpha) + (0)(\sin\alpha)}{(\sin\gamma)(0) + (\cos\gamma)(-\sin\alpha) + (0)(\cos\alpha)}{(0)(1) + (0)(0) + (1)(0)}{(0)(0) + (0)(\cos\alpha) + (1)(\sin\alpha)}{(0)(0) + (0)(-\sin\alpha) + (1)(\cos\alpha)}$$
$$=\threemat{\cos\gamma}{-\sin\gamma\cos\alpha}{\sin\gamma\sin\alpha}{\sin\gamma}{\cos\gamma\cos\alpha}{-\cos\gamma\sin\alpha}{0}{\sin\alpha}{\cos\alpha}.$$

Interestingly, $ZX\neq XZ$. Indeed, while rotations commute in $2$ dimensions, they do not always commute in $3$ dimensions.

\begin{outer_problem}
\item
\end{outer_problem}

\begin{inner_problem}[start=1]
\item Without matrices, consider a cube with side length $2$ at the origin so its faces are perpendicular to the coordinate axes. Rotate it, first $90^\circ$ counterclockwise about the $y$ axis, then $90^\circ$ counterclockwise about the $x$ axis. Note that rotations are done facing from the ``positive side'' of the coordinate axis. The net result should leave two vertices fixed. Which two?
\end{inner_problem}

This requires a good amount of geometric visualization. The answer is the vertices $(1,1,1)$ and $(-1,-1,-1)$. Observe the figures below:

\begin{asydef}
import graph3;

triple A = (1,1,1);
triple B = (1,-1,1);
triple C = (-1,-1,1);
triple D = (-1,1,1);

transform3 sd = shift(0,0,-2);

triple E = sd*A;
triple F = sd*B;
triple G = sd*C;
triple H = sd*D;
\end{asydef}

\begin{minipage}{0.32\textwidth}
\begin{asy}[width=\textwidth]
import graph3;

label("$A$", A, N);
label("$B$", B, NW);
label("$C$", C, N);
label("$D$", D, NE);

label("$E$", E, S);
label("$F$", F, SW);
label("$G$", G, S);
label("$H$", H, SE);

draw(A--B--C--D--A);
draw(H--E--F);
draw(H--G--F, dashed);
draw(A--E);
draw(B--F);
draw(C--G, dashed);
draw(D--H);

xaxis3("$x$",p=gray(0.7));
yaxis3("$y$",p=gray(0.7));
zaxis3("$z$",p=gray(0.7));

\end{asy}
\captionof{figure}{The starting position of the cube.}
\end{minipage}
\begin{minipage}{0.32\textwidth}
\begin{asy}[width=\textwidth]
  import graph3;

  label("$D$", A, N);
  label("$C$", B, NW);
  label("$G$", C, N);
  label("$H$", D, NE);

  label("$A$", E, S);
  label("$B$", F, SW);
  label("$F$", G, S);
  label("$E$", H, SE);

  draw(A--B--C--D--A);
  draw(H--E--F);
  draw(H--G--F, dashed);
  draw(A--E);
  draw(B--F);
  draw(C--G, dashed);
  draw(D--H);

  xaxis3("$x$",p=gray(0.7));
  yaxis3("$y$",p=gray(0.2));
  zaxis3("$z$",p=gray(0.7));

  real r = 0.3;
  triple cntr = (0,1,0);

  draw((shift(-r,0,0)*cntr)..(shift(-r / sqrt(2), 0, r/sqrt(2))*cntr)..shift(0,0,r)*cntr, Arrow3);
  dot(cntr);
\end{asy}
\captionof{figure}{Rotation about the $y$-axis.}
\end{minipage}
\begin{minipage}{0.32\textwidth}
\begin{asy}[width=\textwidth]
  import graph3;

  label("$A$", A, N);
  label("$D$", B, NW);
  label("$H$", C, N);
  label("$E$", D, NE);

  label("$B$", E, S);
  label("$C$", F, SW);
  label("$G$", G, S);
  label("$F$", H, SE);

  draw(A--B--C--D--A);
  draw(H--E--F);
  draw(H--G--F, dashed);
  draw(A--E);
  draw(B--F);
  draw(C--G, dashed);
  draw(D--H);

  xaxis3("$x$",p=gray(0.2));
  yaxis3("$y$",p=gray(0.7));
  zaxis3("$z$",p=gray(0.7));

  real r = 0.3;
  triple cntr = (0,1,0);

  draw(rotate(-90, (0,0,0), (0,0,1))*(shift(-r,0,0)*cntr..shift(-r / sqrt(2), 0, r/sqrt(2))*cntr..shift(0,0,r)*cntr), Arrow3);
  dot(rotate(-90, (0,0,0), (0,0,1))*cntr);
\end{asy}
\captionof{figure}{Rotation about the $x$-axis.}
\end{minipage}

Indeed, $A$ and $G$ remain fixed. These are the vertices $(1,1,1)$ and $(-1,-1,-1)$.

\begin{inner_problem}
\item Write a vector for the axis of rotation.
\end{inner_problem}

The vector is any nonzero multiple of $\threevec{1}{1}{1}$. In the following figure, the axis of rotation is graphed.

\begin{center}
\begin{asy}[width=0.31\textwidth]
  import graph3;

  draw(A--B--C--D--A);
  draw(H--E--F);
  draw(H--G--F, dashed);
  draw(A--E);
  draw(B--F);
  draw(C--G, dashed);
  draw(D--H);

  xaxis3("$x$",p=gray(0.2));
  yaxis3("$y$",p=gray(0.7));
  zaxis3("$z$",p=gray(0.7));

  real disc = 2;

  draw(G--(disc,disc,disc), dotted);
  label("$l$", (disc,disc,disc), N);
\end{asy}
\captionof{figure}{The net axis of rotation is $\langle 1,1,1 \rangle$.}
\end{center}

\begin{inner_problem}
\item How many degrees do you think the net rotation of the cube is? Be careful; the answer is not $180^\circ$.
\end{inner_problem}

The rotation is $120^\circ$, because $E$ is going to $D$, $D$ is going to $B$ and $E$ is going to $D$, a cycle with period $3$.

\begin{inner_problem}
\item Let's check our answers using matrices. Write a matrix product that corresponds to a rotation of $90^\circ$ about the $y$ axis, followed by $90^\circ$ about the $x$ axis.
\end{inner_problem}

Rotation of $90^\circ$ about the $y$ axis: $Y=\threemat{\cos 90^\circ}{0}{\sin 90^\circ}{0}{1}{0}{-\sin 90^\circ}{0}{\cos 90^\circ}=\threemat{0}{0}{1}{0}{1}{0}{-1}{0}{0}$.

Rotation of $90^\circ$ about the $x$ axis: $X=\threemat{1}{0}{0}{0}{\cos 90^\circ}{-\sin 90^\circ}{0}{\sin 90^\circ}{\cos 90^\circ}=\threemat{1}{0}{0}{0}{0}{-1}{0}{1}{0}$.

As usual, matrix multiplication goes right-to-left, so the product is $XY$.

\begin{inner_problem}
\item Multiply out the matrix product.
\end{inner_problem}

$$XY = \threemat{1}{0}{0}{0}{0}{-1}{0}{1}{0}\threemat{0}{0}{1}{0}{1}{0}{-1}{0}{0}$$
$$=\threemat{(1)(0) + (0)(0) + (0)(-1)}{(1)(0) + (0)(1) + (0)(0)}{(1)(1) + (0)(0) + (0)(0)}{(0)(0) + (0)(0) + (-1)(-1)}{(0)(0) + (0)(1) + (-1)(0)}{(0)(1) + (0)(0) + (-1)(0)}{(0)(0) + (1)(0) + (0)(-1)}{(0)(0) + (1)(1) + (0)(0)}{(0)(1) + (1)(0) + (0)(0)}$$
$$=\threemat{0}{0}{1}{1}{0}{0}{0}{1}{0}.$$

Interesting!

\begin{inner_problem}
\item Remember that the real eigenvector in a rotation gives the axis of rotation, and the complex eigenvalues give information about the net rotation. Evaluate these and check your answers for (a) and (b).
\end{inner_problem}

We first find the eigenvalues:

$$\det \threemat{-\lambda}{0}{1}{1}{-\lambda}{0}{0}{1}{-\lambda}=0$$
$$-\lambda\cdot \det\twomat{-\lambda}{0}{1}{-\lambda} - 0\cdot(\text{something}) + 1\cdot \det\twomat{1}{-\lambda}{0}{1}=0$$
$$-\lambda^3 + (1 + 0\cdot -\lambda)=0$$
$$\lambda^3 = 1.$$

We let $\lambda = \cis\theta$:

$$\cis^3\theta=1 \Longrightarrow \theta = 0,\frac{2\pi}{3},\frac{4\pi}{3}.$$

Thus, $\lambda = 1, \cis \frac{2\pi}{3}, \cis \frac{4\pi}{3}$.

We now compute the eigenvector for the axis of rotation, which should correspond to $\lambda = 1$.

$$\threemat{-\lambda}{0}{1}{1}{-\lambda}{0}{0}{1}{-\lambda}\threevec{x}{y}{z} = \threevec{0}{0}{0}$$
$$\threemat{-1}{0}{1}{1}{-1}{0}{0}{1}{-1}\threevec{x}{y}{z} = \threevec{-x+z}{x-y}{y-z} = \threevec{0}{0}{0}.$$

These yields $x=y=z$ and the eigenvector family $s\threevec{1}{1}{1}$, confirming our previous result.

We can find the angle of rotation by the angle the complex eigenvalues make with the $x$-axis. These eigenvalues are $\cis \frac{2\pi}{3}, \cis \frac{4\pi}{3}$, which make a $\frac{2\pi}{3} = 120^\circ$ angle with the $x$-axis. Thus, the magnitude of the rotation is $120^\circ$, confirming our hypothesis.

\begin{outer_problem}
\item Here are two rotation matrices:

i. $\threemat{\frac{2}{3}}{-\frac{2}{3}}{-\frac{1}{3}}{\frac{1}{3}}{\frac{2}{3}}{-\frac{2}{3}}{\frac{2}{3}}{\frac{1}{3}}{\frac{2}{3}}$, ii. $\threemat{\frac{7}{9}}{\frac{4}{9}}{\frac{4}{9}}{-\frac{4}{9}}{-\frac{1}{9}}{\frac{8}{9}}{\frac{4}{9}}{-\frac{8}{9}}{\frac{1}{9}}$.
\end{outer_problem}

\begin{inner_problem}[start=1]
\item What is the determinant of each matrix? (Don't work, think!)
\end{inner_problem}

The determinant of each matrix is $1$, since rotation matrices have determinant $1$.

\begin{inner_problem}
\item What is true of each row and each column?
\end{inner_problem}

The sums of squares of each element in each row and column is $1$. Therefore, each row vector and column vector is a unit vector. As an example, consider the top row of (ii):

$$\left(\frac{7}{9}\right)^2+\left(\frac{4}{9}\right)^2+\left(\frac{4}{9}\right)^2 = \frac{81}{81} = 1.$$

\begin{inner_problem}
\item Find the axis of rotation associated with each matrix.
\end{inner_problem}

\begin{iinner_problem}[start=1]
\item $\threemat{\frac{2}{3}}{-\frac{2}{3}}{-\frac{1}{3}}{\frac{1}{3}}{\frac{2}{3}}{-\frac{2}{3}}{\frac{2}{3}}{\frac{1}{3}}{\frac{2}{3}}$
\end{iinner_problem}

We find the eigenvalues:

$$\det \threemat{\frac{2}{3}-\lambda}{-\frac{2}{3}}{-\frac{1}{3}}{\frac{1}{3}}{\frac{2}{3}-\lambda}{-\frac{2}{3}}{\frac{2}{3}}{\frac{1}{3}}{\frac{2}{3}-\lambda} = 0$$
$$\left(\frac{2}{3}-\lambda\right)\cdot\det \twomat{\frac{2}{3}-\lambda}{-\frac{2}{3}}{\frac{1}{3}}{\frac{2}{3}-\lambda} - \left(-\frac{2}{3}\right)\det \twomat{\frac{1}{3}}{-\frac{2}{3}}{\frac{2}{3}}{\frac{2}{3}-\lambda} - \frac{1}{3} \cdot \det \twomat{\frac{1}{3}}{\frac{2}{3}-\lambda}{\frac{2}{3}}{\frac{1}{3}}=0$$
$$-\lambda^3 + 2 \lambda^2 - 2 \lambda + 1=0$$
$$(\lambda - 1)(\lambda^2 - \lambda + 1)=0.$$

The real eigenvalue is $\lambda = 1$, so we find the corresponding eigenvector:

$$\threemat{\frac{2}{3}-\lambda}{-\frac{2}{3}}{-\frac{1}{3}}{\frac{1}{3}}{\frac{2}{3}-\lambda}{-\frac{2}{3}}{\frac{2}{3}}{\frac{1}{3}}{\frac{2}{3}-\lambda}\threevec{x}{y}{z} = \threevec{0}{0}{0}$$
$$\threemat{-\frac{1}{3}}{-\frac{2}{3}}{-\frac{1}{3}}{\frac{1}{3}}{-\frac{1}{3}}{-\frac{2}{3}}{\frac{2}{3}}{\frac{1}{3}}{-\frac{1}{3}}\threevec{x}{y}{z} = \threevec{-\frac{1}{3}x -\frac{2}{3}y - \frac{1}{3}z}{\frac{1}{3}x-\frac{1}{3}y-\frac{2}{3}z}{\frac{2}{3}x + \frac{1}{3}y - \frac{1}{3}z} = \threevec{0}{0}{0}.$$

Multiplying by $3$ on both sides yields the system of equations

$$\begin{cases} -x -2y -z = 0 \\x-y-2z = 0 \\ 2x + y - z = 0 \end{cases}.$$

The solution to this system of equations is $x=z=-y$. Thus, the eigenvector family is $s\threevec{1}{-1}{1}$, and the axis of rotation is the vector $\langle 1, -1, 1\rangle$.

\begin{iinner_problem}[start=1]
\item $\threemat{\frac{7}{9}}{\frac{4}{9}}{\frac{4}{9}}{-\frac{4}{9}}{-\frac{1}{9}}{\frac{8}{9}}{\frac{4}{9}}{-\frac{8}{9}}{\frac{1}{9}}$
\end{iinner_problem}

We find the eigenvalues:

$$\det \threemat{\frac{7}{9}-\lambda}{\frac{4}{9}}{\frac{4}{9}}{-\frac{4}{9}}{-\frac{1}{9}-\lambda}{\frac{8}{9}}{\frac{4}{9}}{-\frac{8}{9}}{\frac{1}{9}-\lambda} = 0$$
$$\left(\frac{7}{9}-\lambda\right) \cdot \det \twomat{-\frac{1}{9}-\lambda}{\frac{8}{9}}{-\frac{8}{9}}{\frac{1}{9}-\lambda} - \left(\frac{4}{9}\right) \cdot \det \twomat{-\frac{4}{9}}{\frac{8}{9}}{\frac{4}{9}}{\frac{1}{9}-\lambda} + \left(\frac{4}{9}\right) \cdot \det \twomat{-\frac{4}{9}}{-\frac{1}{9}-\lambda}{\frac{4}{9}}{-\frac{8}{9}} = 0$$
$$-\lambda^3 + \frac{7 \lambda^2}{9} - \frac{7 \lambda}{9} + 1=0$$
$$-\frac{1}{9} (\lambda - 1) (9 \lambda^2 + 2 \lambda + 9) = 0.$$

The real eigenvalue is $\lambda = 1$, so we find the corresponding eigenvector:

$$\threemat{\frac{7}{9}-\lambda}{\frac{4}{9}}{\frac{4}{9}}{-\frac{4}{9}}{-\frac{1}{9}-\lambda}{\frac{8}{9}}{\frac{4}{9}}{-\frac{8}{9}}{\frac{1}{9}-\lambda}\threevec{x}{y}{z} = \threevec{0}{0}{0}.$$
$$\threemat{-\frac{2}{9}}{\frac{4}{9}}{\frac{4}{9}}{-\frac{4}{9}}{-\frac{10}{9}}{\frac{8}{9}}{\frac{4}{9}}{-\frac{8}{9}}{\frac{8}{9}}\threevec{x}{y}{z} = \frac{1}{9} \threevec{-2x+4y+4z}{-4x-10y+8z}{4x-8y+8z} = \threevec{0}{0}.$$

The solution to this system of equations is $\langle x,y,z\rangle = s\langle -2, 0, 1\rangle$. This is the axis of rotation: $\langle -2,0,1\rangle$.

\begin{inner_problem}
\item Find the angle of rotation associated with each matrix.
\end{inner_problem}

\begin{iinner_problem}[start=1]
\item $\threemat{\frac{2}{3}}{-\frac{2}{3}}{-\frac{1}{3}}{\frac{1}{3}}{\frac{2}{3}}{-\frac{2}{3}}{\frac{2}{3}}{\frac{1}{3}}{\frac{2}{3}}$
\end{iinner_problem}

From the last time we dealt with this matrix, we found that the complex eigenvalues satisfy $\lambda^2 - \lambda + 1=0$. By the quadratic formula, the solutions to this quadratic are $\lambda = \frac{1\pm i\sqrt{3}}{2}$.

Since $\cis \pm 60^\circ = \frac{1}{2} \pm \frac{\sqrt{3}}{2}i = \lambda$, the rotation is $60^\circ$.

\begin{iinner_problem}[start=1]
\item $\threemat{\frac{7}{9}}{\frac{4}{9}}{\frac{4}{9}}{-\frac{4}{9}}{-\frac{1}{9}}{\frac{8}{9}}{\frac{4}{9}}{-\frac{8}{9}}{\frac{1}{9}}$
\end{iinner_problem}

Previously, we found that the complex eigenvalues of this matrix satisfy the polynomial equation $9 \lambda^2 + 2 \lambda + 9 = 0$. By the quadratic formula, the roots of this equation are

$$\frac{-2\pm \sqrt{2^2 - 4\cdot 9^2}}{18} = -\frac{1}{9}\pm\frac{4i\sqrt{5}}{9}.$$

The angle of rotation is given by

$$\tan^{-1} \frac{y}{x} = \frac{\pm\frac{4\sqrt{5}}{9}}{-\frac{1}{9}} = \pm\tan^{-1} 4\cdot\sqrt{5},$$

which has magnitude $\tan^{-1} (4\sqrt{5})$.

\end{document}

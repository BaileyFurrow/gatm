\documentclass[../gatm_answers.tex]{subfiles}

\begin{document}

\section{Your Daily Dose of Vitamin $i$}

\begin{outer_problem}[start=1]
\setcounter{store_outer_problem}{\value{outer_problemi}}
\item We will use complex numbers to find identities for $\cot$. Use Pascal's triangle to expand the following:
\end{outer_problem}

\begin{inner_problem}[start=1]
\item $(a+b)^3$
\end{inner_problem}
$$(a+b)^3 = a^3 + 3a^2b + 3ab^2 + b^3.$$
\begin{inner_problem}
\item $(a+b)^4$
\end{inner_problem}
$$(a+b)^4 = a^4 + 4a^3b + 6a^2b^2 + 4ab^3 + b^4.$$
\begin{inner_problem}
\item $(a+b)^5$
\end{inner_problem}
$$(a+b)^5 = a^5 + 5a^4b + 10a^3b^2 + 10a^2b^3 + 5ab^4 + b^5.$$
\begin{outer_problem}
\setcounter{outer_problemi}{\value{store_outer_problem}}
\item (cont.) Then substitute $b=i=\sqrt{-1}$ and expand:
\end{outer_problem}

\begin{inner_problem}
\item $(a+i)^3$
\end{inner_problem}
$$(a+b)^3 = a^3 + 3a^2b + 3ab^2 + b^3 = a^3 + 3a^2i - 3a - i.$$
\begin{inner_problem}
\item $(a+i)^4$
\end{inner_problem}
$$(a+b)^4 = a^4 + 4a^3b + 6a^2b^2 + 4ab^3 + b^4 = a^4 + 4a^3i - 6a^2 - 4ai + 1.$$
\begin{inner_problem}
\item $(a+i)^5$
\end{inner_problem}
$$(a+i)^5=a^5 + 5a^4b + 10a^3b^2 + 10a^2b^3 + 5ab^4 + b^5=a^5 + 5a^4i - 10a^3 - 10a^2i + 5a + i.$$
\begin{outer_problem}
\setcounter{outer_problemi}{\value{store_outer_problem}}
\item (cont.) Finally, substitute $a=\cot \theta$ and expand:
\end{outer_problem}

\begin{inner_problem}
\item $(\cot \theta +i)^3$
\end{inner_problem}
$$(\cot \theta +i)^3 = a^3 + 3a^2i - 3a - i = (\cot^3\theta - 3\cot\theta) + i(3\cot^2\theta - 1).$$
\begin{inner_problem}
\item $(\cot \theta +i)^4$
\end{inner_problem}
$$(\cot \theta +i)^4 = a^4 + 4a^3i - 6a^2 - 4ai + 1 = (\cot^4\theta - 6\cot^2\theta + 1) + (4\cot^3 \theta - 4\cot\theta).$$
\begin{inner_problem}
\item $(\cot \theta +i)^5$
\end{inner_problem}
$$(\cot \theta +i)^5 = a^5 + 5a^4i - 10a^3 - 10a^2i + 5a + i = (\cot^5\theta - 10\cot^3\theta + 5\cot\theta) + i(5\cot^4\theta - 10\cot^2\theta + 1).$$
\begin{outer_problem}
\setcounter{outer_problemi}{\value{store_outer_problem}}
\item (cont.) Consider $z=i+\cot\theta$.
\end{outer_problem}

\begin{inner_problem}
\item Use the above results to find identities for (i) $\cot 3\theta$, (ii) $\cot 4\theta$, and (iii) $\cot 5\theta$.
\end{inner_problem}

\begin{iinner_problem}[start=1]
\item $\cot 3\theta$
\end{iinner_problem}

Given the right triangle formed by $z=i+\cot\theta$ in Figure~\ref{fig:arg_z_is_theta}, we have $\tan(\Arg z) =\frac{1}{\cot\theta} = \tan\theta$, so $\Arg z = \theta$ and $z=r\cis \theta$.

\begin{center}
\begin{asy}[width=0.4\textwidth]
real theta = 5 * pi / 16;

pair z = (1 / tan(theta), 1);
pair O = (0,0);
pair zf = (z.x, 0);

dot(z);

draw(O--z);
draw(O--zf--z,dashed);

label("$1$", z--zf, E);

label("$z = i + \cot\theta $", z, NW);
label("$\cot\theta$", O--zf, S);

draw((-0.2,0)--(1,0),Arrow);
draw((0,-0.2)--(0,1.4),Arrow);

path arc1 = arc(O, (0.15,0), z);

draw(arc1);
label("$\theta$", arc1, expi(theta / 2));

real ras = 0.09;
draw(shift(zf) * ((-ras,0)--(-ras,ras)--(0,ras)));

\end{asy}
\label{fig:arg_z_is_theta}
\captionof{figure}{$\Arg (i + \cot\theta)=\theta$.}
\end{center}

Thus, we have

\begin{align*}
\cot 3\theta &= \frac{\cos 3\theta}{\sin 3\theta} \\
&= \frac{\Real(\cis 3\theta)}{\Imag(\cis 3\theta)} \\
&= \frac{\Real(r^3\cis 3\theta)}{\Imag(r^3\cis 3\theta)} \\
&= \frac{\Real(z^3)}{\Imag(z^3)}.
\end{align*}

We substitute in our expression for $z^3$, $(\cot^3\theta - 3\cot\theta) + i(3\cot^2\theta - 1)$:

$$\cot 3\theta = \frac{\cot^3\theta - 3\cot\theta}{3\cot^2\theta - 1}.$$

\begin{iinner_problem}[start=1]
\item $\cot 4\theta$
\end{iinner_problem}

We proceed in the same way as the last subproblem.

\begin{align*}
\cot 4\theta &= \frac{\cos 4\theta}{\sin 4\theta} \\ 
&= \frac{\Real(\cis 4\theta)}{\Imag(\cis 4\theta)} \\
&= \frac{\Real(r^4\cis 4\theta)}{\Imag(r^4\cis 4\theta)} \\
&= \frac{\Real(z^4)}{\Imag(z^4)} \\
\cot 4\theta &= \frac{\cot^4\theta - 6\cot^2\theta + 1}{4\cot^3 \theta - 4\cot\theta}. \\
\end{align*}

\begin{iinner_problem}[start=1]
\item $\cot 5\theta$
\end{iinner_problem}

We proceed in the same way as the last subproblem.

\begin{align*}
\cot 5\theta &= \frac{\cos 5\theta}{\sin 5\theta}  \\
&= \frac{\Real(\cis 5\theta)}{\Imag(\cis 5\theta)} \\
&= \frac{\Real(r^5\cis 5\theta)}{\Imag(r^5\cis 5\theta)} \\
&= \frac{\Real(z^5)}{\Imag(z^5)} \\
&= \frac{\cot^5\theta - 10\cot^3\theta + 5\cot\theta}{5\cot^4\theta - 10\cot^2\theta + 1}.
\end{align*}

\begin{inner_problem}
\item Graph $z$, $z^2$, $z^3$, $z^4$, and $z^5$, with $\theta \approx 75^\circ$. What is your solution method?
\end{inner_problem}

To graph these, I first calculated the approximate magnitude of $z$, which is how many times each subsequent power will be scaled by. We have $|1+\cot 75^\circ| \approx 1.268$, so we only need to scale by about $\frac{5}{4}$ each time. Of course, we rotate by about $75^\circ$ each time.

\begin{center}
\begin{asy}[width=0.4\textwidth]
import graph;

pair O = (0,0);
pair z = (1 / tan(75 * pi / 180), 1);
pair z2 = z * z;
pair z3 = z2 * z;
pair z4 = z3 * z;
pair z5 = z4 * z;

draw(O--z);
draw(O--z2);
draw(O--z3);
draw(O--z4);
draw(O--z5);

label("$z$", z, N);
label("$z^2$", z2, W);
label("$z^3$", z3, SW);
label("$z^4$", z4, SE);
label("$z^5$", z5, NE);

dot(z);
dot(z2);
dot(z3);
dot(z4);
dot(z5);

xaxis("$x$");
yaxis("$y$");
\end{asy}

\captionof{figure}{Graphs of $z$, $z^2$, $z^3$, $z^4$, and $z^5$.}
\end{center}

\begin{outer_problem}
\item Compute $(1+i)^n$ for $n=3,4,5,\ldots$. Can you find a general pattern?
\end{outer_problem}

We have

\begin{align*}
(1+i)^3 &= 1^3 + 3i - 3 - i &= -2-2i \\
(1+i)^4 &= 1^4 + 4i - 6 - 6i + 1 &= -4-2i \\
(1+i)^5 &= 1^5 + 5i - 10 - 10i + 5 + i &= -4-4i. \\
\end{align*}

We can find the pattern by representing $1+i = \sqrt{2} \cis 45^\circ$. This shows that it has period $8$ and let's us find an expression for $(1+1)^n$:

$$(1+i)^n = \left(\sqrt{2} \cis 45^\circ\right)^n = 2^{n/2} \cis \left(\frac{n\pi}{4}\right).$$

\begin{outer_problem}
\item Expand and graph $\cis^n \theta$ for $n=2,3,4,\ldots$.
\end{outer_problem}

Let $\cos\theta = c$ and $\sin\theta = s$. We have

\begin{align*}
(c+is)^2 &= c^2 + 2csi - s^2 = (c^2 - s^2) + i(2cs) \\\
(c+is)^3 &= c^3 + 3c^2si - 3cs^2 - s^3i = (c^3 - 3cs^2) + i(3c^2s - s^3) \\
(c+is)^4 &= c^4 + 4c^3si - 6c^2s^2 - 4cs^3i + s^4 = (c^4 - 6c^2s^2 + s^4) + i(4c^3s - 4cs^3) \\
(c+is)^5 &= c^5 + 5c^4si - 10c^3s^2 - 10c^2s^3i + 5cs^4 + s^5i = (c^5 - 10c^3s^2 + 5cs^4) + i(5c^4s - 10c^2s^3 + s^5). \\
\end{align*}

The graphs of $\cis^n\theta$ for $\theta \approx 50^\circ$ are shown in Figure~\ref{fig:cis_n_theta_graphs} below.

\begin{center}
\begin{asy}[width=0.4\textwidth]
import graph;

pair O = (0,0);
pair z = dir(49.5); // REEEEEE
pair z2 = z * z;
pair z3 = z2 * z;
pair z4 = z3 * z;
pair z5 = z4 * z;

draw(O--z);
draw(O--z2);
draw(O--z3);
draw(O--z4);
draw(O--z5);

label("$\cis \theta$", z, N);
label("$\cis^2 \theta$", z2, W);
label("$\cis^3 \theta$", z3, W);
label("$\cis^4 \theta$", z4, W);
label("$\cis^5 \theta$", z5, W);

dot(z);
dot(z2);
dot(z3);
dot(z4);
dot(z5);

xaxis("$x$");
yaxis("$y$");
\end{asy}
\captionof{figure}{Graphs of $\cis^n\theta$ for $\theta\approx 50^\circ$.}
\label{fig:cis_n_theta_graphs}
\end{center}

\begin{inner_problem}[start=1]
\item Why is the real part $\cos n\theta$ and the imaginary part $\sin n\theta$?
\end{inner_problem}

By DeMoivre's theorem, $\cis^n\theta = \cis n\theta$, which by definition has $\Imag(\cis n\theta) = \cos n\theta$ and $\Real(\cis n\theta) = \sin n\theta$.

\begin{inner_problem}
\item Use your results to write identities for $\cos n\theta$ and $\sin n\theta$ for $n=2,3,4,5$.
\end{inner_problem}

Here they are. Again, let $\cos\theta = c$ and $\sin\theta = s$:

\begin{align*}
\cos 2\theta = \Real(\cis 2\theta) &= c^2 - s^2 \\
\cos 3\theta = \Real(\cis 3\theta) &= c^3 - 3cs^2 \\
\cos 4\theta = \Real(\cis 4\theta) &= c^4 - 6c^2s^2 + s^4 \\
\cos 5\theta = \Real(\cis 5\theta) &= c^5 - 10c^3s^2 + 5cs^4 \\
\sin 2\theta = \Imag(\cis 2\theta) &= 2cs \\
\sin 3\theta = \Imag(\cis 3\theta) &= 3c^2s - s^3 \\
\sin 4\theta = \Imag(\cis 4\theta) &= 4c^3s - 4cs^3 \\
\sin 5\theta = \Imag(\cis 5\theta) &= 5c^4s - 10c^2s^3 + s^5. \\
\end{align*}

\newcommand{\cosdeg}[1] {\cos #1^\circ}

\begin{outer_problem}
\item Compute $\cosdeg{7} + \cosdeg{79} + \cosdeg{151} + \cosdeg{223} + \cosdeg{295}$ without a calculator. (Hint: what does this have to do with complex numbers?)
\end{outer_problem}

These numbers look random, but a closer inspection reveals they are in arithmetic progression, with starting term $7$ and increasing $72^\circ$ each time. That's the rotation of a pentagon!

We rewrite this as the real component of a sum of $\cis$es, then manipulate and evaluate:

\begin{align*}
\cosdeg{7} + \cosdeg{79} + \cosdeg{151} + \cosdeg{223} + \cosdeg{295} &= \Real(\cis 7^\circ + \cis 79^\circ +  \cis 151^\circ +  \cis 223^\circ +  \cis 295^\circ) \\
&= \Real((\cis 7^\circ)(\cis 0^\circ + \cis 72\circ + \cis 144^\circ + \cis 216^\circ + \cis 288^\circ)) \\
&= \Real((\cis 7^\circ)(0)) \\
&= \Real(0) \\
&= 0.
\end{align*}

\begin{outer_problem}
\item Factor the following:
\end{outer_problem}

\begin{inner_problem}[start=1]
\item $x^6-1$ as a difference of squares
\end{inner_problem}

We substitute $y=x^3$, giving $y^2-1=(y+1)(y-1)$. Substituting back in, we get $$(x^3+1)(x^3-1).$$

\begin{inner_problem}
\item $x^6-1$ as a difference of cubes
\end{inner_problem}

We now substitute $y=x^2$, giving $y^3-1 = (y-1)(y^2+y+1)$. Substituting back in, we get $$(x^2-1)(x^4+x^2+1)$$.

\begin{inner_problem}
\item $x^4+x^2+1$ over the real numbers
\end{inner_problem}

This one isn't as obvious. We substitute $y=x^2$ to get $y^2+y+1$ and find the quadratic's zeroes:

$y=\frac{-1\pm \sqrt{1 - 4}}{2} = \frac{-1\pm i\sqrt{3}}{2}.$

So it is irreducible over the reals.

\begin{inner_problem}
\item $x^6-1$ completely
\end{inner_problem}

We already broke it down into $(x^3+1)$ and $(x^3-1)$. Going further, we have $x^3+1 = (x+1)(x^2-x+1)$ and $x^3-1=(x-1)(x^2+x+1)$. To break apart the last two quadratics, we find their zeros:

$$x^2-x+1=0 \Longrightarrow x = \frac{1 \pm i\sqrt{3}}{2} \Longrightarrow \left(x - \frac{1 - i\sqrt{3}}{2}\right) \left(x - \frac{1 + i\sqrt{3}}{2}\right).$$

$$x^2+x+1=0 \Longrightarrow x = \frac{-1 \pm i\sqrt{3}}{2} \Longrightarrow \left(x + \frac{1 - i\sqrt{3}}{2}\right) \left(x + \frac{1 + i\sqrt{3}}{2}\right).$$

Combining all these, we get the complete factorization over the complex numbers:

$$x^6-1 = (x+1)\left(x - \frac{1 - i\sqrt{3}}{2}\right) \left(x - \frac{1 + i\sqrt{3}}{2}\right)(x-1)\left(x + \frac{1 - i\sqrt{3}}{2}\right) \left(x + \frac{1 + i\sqrt{3}}{2}\right).$$

\begin{inner_problem}
\item $x^4+x^2+1$ completely
\end{inner_problem}

We could do a lot of work again, or we could observe that $x^4+x^2+1 = \frac{x^6 - 1}{x^2 - 1} = \frac{x^6 - 1}{(x+1)(x-1)}$. Removing the denominator's terms from our factorization of $x^6-1$ we found in the last subproblem, we get

$$x^4+x^2+1 = \left(x - \frac{1 - i\sqrt{3}}{2}\right) \left(x - \frac{1 + i\sqrt{3}}{2}\right)\left(x + \frac{1 - i\sqrt{3}}{2}\right) \left(x + \frac{1 + i\sqrt{3}}{2}\right).$$

\begin{outer_problem}
\item Let $f(z)=\frac{z+1}{z-1}$.
\end{outer_problem}

\begin{inner_problem}[start=1]
\item Without a calculator, compute $f^{2014}(z)$.
\end{inner_problem}

This seems terrifying. Let's try computing $f^2(z)$ and perhaps $f^3(z)$.

$$f^2(z) = \frac{f(z)+1}{f(z)-1} = \frac{\frac{z+1}{z-1}+1}{\frac{z+1}{z-1}-1} = \frac{\frac{2z}{z-1}}{\frac{2}{z-1}} = z.$$

Oh.

Since $2014$ is even, we have $f^{2014}(z) = (f^2)^{1007}(z) = z$.

\begin{inner_problem}
\item What if you replace $2014$ with the current year?
\end{inner_problem}

Let $y$ be the current year. As I write this, it is $1492$.

If $y$ is even, then $f^y(z) = (f^2)^{y/2}(z) = z$. If $y$ is odd, then $f^y(z) = f((f^2)^{(y-1)/2}(z)) = f(z) = \frac{z+1}{z-1}$.

\begin{outer_problem}
\item Find $\Imag\left((\cis 12^\circ + \cis 48^\circ)^6\right)$.
\end{outer_problem}

These are some weird looking angles. Thinking back to some older problems, however, the resultant angle of the addition may be tractable. We draw a diagram, shown in Figure~\ref{fig:add_twelve_and_forty_eight}.

\begin{center}
\begin{asy}[width=0.6\textwidth]
import graph;

pair O = (0,0);
pair c1 = dir(12);
pair c2 = dir(48);
pair res = c1+c2;

draw(O--c2, Arrow);
draw(c2--res, Arrow);
label("$\cis 48^\circ$", O--c2, NW);
label("$\cis 12^\circ$", c2--res, N);

path arc1 = arc(O, (0.15, 0), c2);
draw(arc1, Arrow);
label("$48^\circ$", midpoint(arc1), dir(24));
// label("$48^\circ$", 

draw(O--res, dashed);

draw(c2--(c2+(0.8,0)), dotted);

path arc2 = arc(c2, c2 + (0.2,0), res);

draw(arc2);
label("$12^\circ$", midpoint(arc2), 5*dir(6));


xaxis("$x$");
yaxis("$y$");

path arc3 = arc(O, (0.3,0), c2);

label("$x$", point(arc3, 0.85), dir(36+9));
draw(arc3);
label("$48^\circ - x$", point(arc3, 0.4), dir(15));

path arc4 = arc(c2, point(c2--O, 0.15), c2+(0.15,0));

draw(arc4);
label("$132^\circ$", midpoint(arc4), SE);
\end{asy}
\captionof{figure}{Adding $\cis 12^\circ + \cis 48^\circ$.}
\label{fig:add_twelve_and_forty_eight}
\end{center}

Consider the isosceles triangle. The apex has angle measure $132^\circ + 12^\circ = 144^\circ$, so the base angles are each $x=\frac{180^\circ - 144^\circ}{2} = 18^\circ$. But $\Arg (\cis 12^\circ + \cis 48^\circ) = 48^\circ - x = 30^\circ$!

That's a familiar angle. Indeed, we have $z=\cis 12^\circ + \cis 48^\circ=r\cis 30^\circ$ for some $r$. It doesn't really matter which $r$, because

$$\Imag ((r\cis 30^\circ)^6) = \Imag (r^6 \cis 180^\circ) = \Imag(-r^6) = 0.$$

\begin{outer_problem}
\item Let $x$ satisfy the equation $x+\frac{1}{x}=2\cos\theta$.
\end{outer_problem}

\begin{inner_problem}[start=1]
\item Compute $x^2+\frac{1}{x^2}$ in terms of $\theta$.
\end{inner_problem}

Squaring the left hand side will get us some terms that look close to what we want.

$$\left(x + \frac{1}{x}\right)^2 = x^2 + 2 + \frac{1}{x^2}.$$

So $x^2 + \frac{1}{x^2} = (2\cos\theta)^2 - 2 = 4\cos^2\theta - 2 = 2 (2\cos^2\theta - 1) = 2\cos 2\theta$. Huh.

\begin{inner_problem}
\item Compute $x^n+\frac{1}{x^n}$ in terms of $n$ and $\theta$.
\end{inner_problem}

We conjecture that this is equal to $2\cos n\theta$. To do this, we let $x=\cis \frac{\theta}{n}$, so $x^n = \cis\theta$, and compute. That should give us some similar looking terms:

\begin{align*}
x^n + \frac{1}{x^n} &= \cis \theta + \frac{1}{\cis \theta} \\
&= \cis\theta + \cis (-\theta) \\
&= \cis\theta + \overline{\cis\theta} \\
&= 2 \Real(\cis \theta) \\
&= 2 \cos\theta.
\end{align*}

This proves the relationship.

\end{document}
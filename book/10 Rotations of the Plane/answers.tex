\documentclass[../key.tex]{subfiles}

\begin{document}

\section{Rotations of the Plane}

\newcommand{\Mat}{\operatorname{M}}

\begin{outer_problem}[start=1]
\item \label{prob:pr_start}
\end{outer_problem}

\begin{inner_problem}[start=1]
\item Which matrix changes nothing, so that the image is the same as the preimage?
\end{inner_problem}

We already found this; this is the identity matrix $I$. The $2\times 2$ identity matrix is

$$\begin{bmatrix} 1 & 0 \\ 0 & 1 \end{bmatrix}.$$

\begin{inner_problem}
\item Which complex number changes nothing?
\end{inner_problem}

$1$, since $1\cdot x = x\cdot 1 = x$. Although it might not seem complex at first sight, real numbers are complex too.

\begin{outer_problem}
\item
\end{outer_problem}

\begin{inner_problem}[start=1]
\item Which matrix doubles the length of every vector but leaves angles unchanged?
\end{inner_problem}

This is scaling up by a factor of $2$ in all directions, which is

$$\begin{bmatrix} 2 & 0 \\ 0 & 2 \end{bmatrix}.$$

\begin{inner_problem}
\item Which complex number corresponds to the same transformation?
\end{inner_problem}

This is just $2$, since multiplying $2$ produces the desired effect of scaling.

\begin{outer_problem}
\item Based on your answers to the previous problems, which matrix corresponds to the real number $r$? Let's call this $\Mat (r)$ for short.
\end{outer_problem}

It looks like

$$\Mat (r) = \begin{bmatrix} r & 0 \\ 0 & r \end{bmatrix}.$$

\begin{outer_problem}
\item Explain why $\Mat (u)+\Mat(v)=\Mat(u+v).$
\end{outer_problem}

From the perspective of complex numbers, the relationship $(u)+(v)=u+v$ holds. It also holds for matrices:

$$\Mat(u) + \Mat(v) = \begin{bmatrix} u & 0 \\ 0 & u \end{bmatrix} + \begin{bmatrix} v & 0 \\ 0 & v \end{bmatrix} = \begin{bmatrix} u+v & 0 \\ 0 & u+v \end{bmatrix} = \Mat(u+v).$$

\begin{outer_problem}
\item Under a $90^\circ$ counterclockwise rotation, what is the image of (a) $(1,0)$ and (b) $(0,1)$?
\end{outer_problem}

The image of $(1,0)$ is $(0,1)$, and the image of $(0,1)$ is $(-1,0)$. You can verify this by drawing it out if you want.

\begin{outer_problem}
\item \label{prob:pr_end}
\end{outer_problem}

\begin{inner_problem}[start=1]
\item Which matrix corresponds to a $90^\circ$ rotation?
\end{inner_problem}

This is just the rotation matrix $\left[\begin{smallmatrix} \cos \theta & -\sin\theta \\ \sin\theta & \cos\theta \end{smallmatrix}\right]$, with $\theta = 90^\circ = \frac{\pi}{2}$:

$$\begin{bmatrix} 0 & -1 \\ 1 & 0 \end{bmatrix}.$$

\begin{inner_problem}
\item Which complex number corresponds to the same rotation?
\end{inner_problem}

Multiplying by $i$ produces the same rotation.

\begin{outer_problem}
\item Based on your answers to Problems~\ref{prob:pr_start} to~\ref{prob:pr_end}, what matrix corresponds to the complex number $x+yi$? Let's extend our function $\Mat$ and call this $\Mat (x+yi)$ for short.
\end{outer_problem}

It looks like

$$\Mat(x+yi) = \begin{bmatrix} x & -y \\ y & x \end{bmatrix}.$$

\begin{outer_problem}
\item Check that $\Mat (a+bi)+\Mat (c+di)=\Mat((a+bi)+(c+di))$. That is, prove that $\Mat$ has the same addition rules as complex numbers.
\end{outer_problem}

We have

$$\Mat (a+bi)+\Mat (c+di) = \begin{bmatrix} a & -b \\ b & a \end{bmatrix} + \begin{bmatrix} c & -d \\ d & c \end{bmatrix} = \begin{bmatrix} a + c & -b-d \\ b+d & a+c\end{bmatrix} = \Mat((a+c) + (b+d)i) = \Mat((a+bi)+(c+di)).$$

\begin{outer_problem}
\item Check that $\Mat (a+bi)\Mat (c+di)=\Mat((a+bi)(c+di))$. That is, prove that $\Mat$ has the same multiplication rules as complex numbers.
\end{outer_problem}

We have

\begin{align*}
\Mat (a+bi)\Mat (c+di) &= \begin{bmatrix} a & -b \\ b & a \end{bmatrix} \begin{bmatrix} c & -d \\ d & c \end{bmatrix} \\
&= \begin{bmatrix} ac-bd & -ad-bc \\ ad+bc & ac-bd \end{bmatrix} \\
&= \Mat((ac-bd) + (ad+bc)i) \\
&= \Mat((a+bi)(c+di)).
\end{align*}

\begin{outer_problem}
\item Recall that multiplying by $\cis\theta$ rotates a complex number by $\theta$ radians.
\end{outer_problem}

\begin{inner_problem}[start=1]
\item Find $\Mat (\cis\theta)$.
\end{inner_problem}

Since $\cis\theta = \cos\theta + i\sin\theta$, we have

$$\Mat(\cis\theta) = \Mat(\cos\theta + i\sin\theta) = \begin{bmatrix} \cos\theta & -\sin\theta \\ \sin\theta & \cos\theta \end{bmatrix}.$$

\begin{inner_problem}
\item To prove that this matrix really does rotate by $\theta$:
\end{inner_problem}

\begin{iinner_problem}[start=1]
\item Check that the image and preimage have the same length;
\end{iinner_problem}

Let the preimage by $\begin{bmatrix} x \\ y \end{bmatrix}$, which has length $\sqrt{x^2+y^2}$. Then the image is

$$\begin{bmatrix} \cos\theta & -\sin\theta \\ \sin\theta & \cos\theta \end{bmatrix}\begin{bmatrix} x \\ y \end{bmatrix} = \begin{bmatrix} x\cos\theta - y\sin\theta \\ x\sin\theta + y\cos\theta \end{bmatrix},$$

which has length

\begin{align*}
\sqrt{(x\cos\theta - y\sin\theta)^2 + (x\sin\theta + y\cos\theta)^2} &= \sqrt{x^2\cos^2\theta - 2xy\cos\theta\sin\theta + y^2\sin^2\theta + x^2\sin^2\theta + 2xy\cos\theta\sin\theta + y^2\cos^2\theta} \\
&= \sqrt{x^2(\cos^2\theta + \sin^2\theta) + y^2(\sin^2\theta + \cos^2\theta)} \\
&= \sqrt{x^2 + y^2}.
\end{align*}

Indeed, this matches up with the length of the preimage.

\begin{iinner_problem}
\item Check that the angle of the image with the $x$-axis is $\theta$ more than the preimage.
\end{iinner_problem}

This is a bit unpleasant. The angle of the image with the $x$-axis is $\tan^{-1} \frac{y}{x}$... we can make this more pleasant by representing our preimage as

$$\begin{bmatrix}r\cos\phi \\ r\sin\phi \end{bmatrix},$$

which is a point $r$ away from the origin and making an angle of $\phi$ with the $x$-axis.

Then the image is

\begin{align*}
\begin{bmatrix} \cos\theta & -\sin\theta \\ \sin\theta & \cos\theta \end{bmatrix} \begin{bmatrix}r\cos\phi \\ r\sin\phi \end{bmatrix} &= \begin{bmatrix} r\cos\phi\cos\theta - r\sin\phi\sin\theta \\ r\sin\phi\cos\theta + r\cos\phi\sin\theta \end{bmatrix} \\
&= \begin{bmatrix} r(\cos\phi\cos\theta - \sin\phi\sin\theta) \\ r(\sin\phi\cos\theta + \cos\phi\sin\theta) \end{bmatrix} \\
&= \begin{bmatrix} r\cos(\phi + \theta) \\ r\sin(\phi + \theta) \end{bmatrix}. \\
\end{align*}

This is a point $r$ away from the origin and making an angle of $\theta + \phi$ with the $x$-axis, an angle $\theta$ more than the original $\phi$ as desired.

\begin{outer_problem}
\item
\end{outer_problem}

\begin{inner_problem}[start=1]
\item Find $\Mat (r\cis\theta)$.
\end{inner_problem}

$$\Mat (r\cis\theta) = \Mat(r\cos\theta + ir\sin\theta) = \begin{bmatrix} r\cos\theta & -r\sin\theta \\ r\sin\theta & r\cos\theta \end{bmatrix}.$$

\begin{inner_problem}
\item To prove that this matrix really does rotate by $\theta$ and stretch by $r$:
\end{inner_problem}

\begin{iinner_problem}[start=1]
\item Check that the length of the image is $r$ times the length of the preimage;
\end{iinner_problem}

Let the preimage be $\begin{bmatrix} x \\ y \end{bmatrix}$, which has length $\sqrt{x^2+y^2}$. Then the image is

$$\begin{bmatrix} r\cos\theta & -r\sin\theta \\ r\sin\theta & r\cos\theta \end{bmatrix}\begin{bmatrix} x \\ y \end{bmatrix} = r \underbrace{\begin{bmatrix} \cos\theta & -\sin\theta \\ \sin\theta & \cos\theta \end{bmatrix} \begin{bmatrix} x \\ y \end{bmatrix}}_{\Mat(\cis\theta)\cdot \left[\begin{smallmatrix} x \\ y \end{smallmatrix}\right]}.$$

This has length $r\sqrt{x^2+y^2}$ by the previous problem, as desired.

\begin{iinner_problem}
\item Check that the angle of the image with the $x$-axis is $\theta$ more than the preimage. (Hint: You may want to use the previous problem, or the tangent addition formulas.)
\end{iinner_problem}

Let our preimage be

$$\begin{bmatrix}\rho \cos\phi \\ \rho \sin\phi \end{bmatrix},$$

which makes an angle of $\phi$ with the $x$-axis. Then the image is

$$\begin{bmatrix} r\cos\theta & -r\sin\theta \\ r\sin\theta & r\cos\theta \end{bmatrix} \begin{bmatrix}\rho \cos\phi \\ \rho \sin\phi \end{bmatrix} = r\underbrace{\begin{bmatrix} \cos\theta & -\sin\theta \\ \sin\theta & \cos\theta \end{bmatrix} \begin{bmatrix}\rho \cos\phi \\ \rho \sin\phi \end{bmatrix}}_{\Mat(\cis\theta) \cdot \left[\begin{smallmatrix} \rho \cos\phi \\ \rho \sin\phi \end{smallmatrix} \right]},$$

which makes an angle of $\phi + \theta$ with the $x$-axis via the last problem, as desired.

\begin{outer_problem}
\item
\end{outer_problem}

\begin{inner_problem}[start=1]
\item What matrix reflects over the $x$-axis, taking $(x,y)\to (x,-y)$?
\end{inner_problem}

This is the matrix $\begin{bmatrix} 1 & 0 \\ 0 & -1 \end{bmatrix}$, since this flips the $y$ coordinate.

\begin{inner_problem}
\item What is the complex number operation equivalent to this transformation?
\end{inner_problem}

The equivalent operation is complex conjugation, denoted $\overline{a+bi} = a-bi$.

\begin{inner_problem}
\item Is there a complex number multiplication equivalent to this transformation? Justify your answer.
\end{inner_problem}

There is not. Suppose there was a complex number $r\cis\theta$ which satisfied

$$(a+bi)r\cis\theta = a-bi.$$

Then since $|a+bi| = |a-bi|$, $r=0$. So $(a+bi)\cis\theta = a-bi$. But the transformation described is a reflection, while this is a rotation! Thus, no such complex number exists.

\begin{outer_problem}
\item
\end{outer_problem}

\begin{inner_problem}[start=1]
\item What matrix reflects through the origin, taking $(x,y)\to (-x,-y)$?
\end{inner_problem}

This is the matrix $\begin{bmatrix} -1 & 0 \\ 0 & -1 \end{bmatrix}$, since this flips both the $x$ and $y$ coordinates.

\begin{inner_problem}
\item What is the complex number operation equivalent to this transformation?
\end{inner_problem}

This is negating the complex number: $f(z) = -z$.

\begin{inner_problem}
\item Is there a complex number multiplication equivalent to this transformation? Justify your answer.
\end{inner_problem}

Yes there is! $f(z) = -1\cdot z$ is equivalent; we have $-1 \cdot (a+bi) = -a-bi$ as desired.

\begin{outer_problem}
\item
\end{outer_problem}

\newcommand{\mtrxtbt}[4] {$\left[\begin{array}{cc}#1 & #2 \\ #3 & #4 \end{array}\right]$}

\begin{inner_problem}[start=1]
\item Which of the $16$ matrices on page~\pageref{prob:map_plane_sixteen_matrices}, for Problem~\ref{prob:map_plane_sixteen_matrices}, have corresponding complex numbers?
\end{inner_problem}

\begin{iinner_problem}[start=1]
\item \mtrxtbt{1}{0}{0}{-1}
\end{iinner_problem}

No, since $1\neq -1$.

\begin{iinner_problem}
\item \mtrxtbt{-1}{0}{0}{-1}
\end{iinner_problem}

Yes; the complex number is $-1$.

\begin{iinner_problem}
\item \mtrxtbt{2}{0}{0}{2}
\end{iinner_problem}

Yes; the complex number is $2$.

\begin{iinner_problem}
\item \mtrxtbt{0}{1}{-1}{0}
\end{iinner_problem}

Yes; the complex number is $-i$.

\begin{iinner_problem}
\item \mtrxtbt{0}{1}{1}{0}
\end{iinner_problem}

No, since $1\neq -1$.

\begin{iinner_problem}
\item \mtrxtbt{0}{0}{0}{0}
\end{iinner_problem}

Yes; the complex number is $0$.

\begin{iinner_problem}
\item \mtrxtbt{1}{0}{0}{1}
\end{iinner_problem}

Yes; the complex number is $1$.

\begin{iinner_problem}
\item \mtrxtbt{3}{0}{0}{1}
\end{iinner_problem}

No, since $3\neq 1$.

\begin{iinner_problem}
\item \mtrxtbt{1}{0}{-3}{1}
\end{iinner_problem}

No, since $3\neq 0$.

\begin{iinner_problem}
\item \mtrxtbt{2}{2}{-3}{-3}
\end{iinner_problem}

No, since $2\neq -3$.

\begin{iinner_problem}
\item \mtrxtbt{3}{2}{4}{-1}
\end{iinner_problem}

No, since $3\neq -1$.

\begin{iinner_problem}
\item \mtrxtbt{\frac{\sqrt{2}}{2}}{\frac{\sqrt{2}}{2}}{\frac{\sqrt{2}}{2}}{-\frac{\sqrt{2}}{2}}
\end{iinner_problem}

No, since $\frac{\sqrt{2}}{2} \neq -\frac{\sqrt{2}}{2}$ (comparing TR and BL corners).

\begin{iinner_problem}
\item \mtrxtbt{\frac{\sqrt{2}}{2}}{\frac{\sqrt{2}}{2}}{-\frac{\sqrt{2}}{2}}{\frac{\sqrt{2}}{2}}
\end{iinner_problem}

Yes; the complex number is $\frac{\sqrt{2}}{2} - \frac{\sqrt{2}}{2}i$.

\begin{iinner_problem}
\item \mtrxtbt{\frac{\sqrt{3}}{2}}{\frac{1}{2}}{\frac{1}{2}}{-\frac{\sqrt{3}}{2}}
\end{iinner_problem}

No, since $\frac{1}{2} \neq -\frac{1}{2}$.

\begin{iinner_problem}
\item \mtrxtbt{\frac{\sqrt{3}}{2}}{-\frac{1}{2}}{\frac{1}{2}}{\frac{\sqrt{3}}{2}}
\end{iinner_problem}

Yes; the complex number is $\frac{\sqrt{3}}{2} + \frac{1}{2}i$.

\begin{iinner_problem}
\item \mtrxtbt{\frac{\sqrt{3}}{2}}{\frac{1}{2}}{-\frac{1}{2}}{\frac{\sqrt{3}}{2}}
\end{iinner_problem}

Yes; the complex number is $\frac{\sqrt{3}}{2} - \frac{1}{2}i$.

\begin{inner_problem}
\item How can you tell algebraically?
\end{inner_problem}

The matrix must be of the form $\begin{bmatrix} a & -b \\ b & a \end{bmatrix}$ for real (but not necessarily positive) $a,b$.

\begin{inner_problem}
\item How can you tell geometrically?
\end{inner_problem}

If the matrix is purely a rotation and dilation, then it has an associated complex number. Note that the zero matrix is a dilation of $0$.

\begin{outer_problem}
\item Make multiplication tables with the set of matrices which correspond to the elements of the rotation group for the square (a $4\times 4$ table) and the equilateral triangle (a $3\times 3$ table).
\end{outer_problem}

Square: Define $r=\Mat(i) = \begin{bmatrix} 0 & -1 \\ 1 & 0 \end{bmatrix}$, which is a rotation $90^\circ$ counterclockwise. Then we have

$$r^2 = \begin{bmatrix} -1 & 0 \\ 0 & -1 \end{bmatrix};\qquad r^3 = \begin{bmatrix} 0 & 1 \\ -1 & 0 \end{bmatrix};\qquad I = \begin{bmatrix} 1 & 0 \\ 0 & 1 \end{bmatrix}.$$

The table is shown below.

$$\begin{array}{c|c|c|c|c|}
\cdot I & r & r^2 & r^3 \\ \hline
I & I & r & r^2 & r^3 \\ \hline
r & r & r^2 & r^3 & I \\ \hline
r^2 & r^2 & r^3 & I & r \\ \hline
r^3 & r^3 & I & r & r^2 \\ \hline
\end{array}$$

Equilateral triangle: Define $r=\Mat(\cis 120^\circ)=\begin{bmatrix} -\frac{1}{2} & -\frac{\sqrt{3}}{2} \\ \frac{\sqrt{3}}{2} & -\frac{1}{2}\end{bmatrix}$. Then we have

$$r^2 = \begin{bmatrix} -\frac{1}{2} & \frac{\sqrt{3}}{2} \\ -\frac{\sqrt{3}}{2} & -\frac{1}{2}\end{bmatrix};\qquad I = \begin{bmatrix} 1 & 0 \\ 0 & 1 \end{bmatrix}.$$

The table is shown below.

$$\begin{array}{c|c|c|c|}
\cdot & I & r & r^2 \\ \hline
I & I & r & r^2 \\ \hline
r & r & r^2 & I \\ \hline
r^2 & r^2 & I & r \\ \hline
\end{array}$$

\begin{outer_problem}
\item
\end{outer_problem}

\begin{inner_problem}[start=1]
\item Write a matrix for a rotation of $\theta$ around the origin followed by a translation by $(a,b)$.
\end{inner_problem}

Recall that for translations, we need $3\times 3$ matrices. The matrix is

$$\begin{bmatrix}1 & 0 & a \\ 0 & 1 & b \\ 0 & 0 & 1 \end{bmatrix} \begin{bmatrix} \cos\theta & -\sin\theta & 0 \\ \sin\theta & \cos\theta & 0 \\ 0 & 0 & 1 \end{bmatrix} = \begin{bmatrix} \cos\theta & -\sin\theta & a \\ \sin\theta & \cos\theta & b \\ 0 & 0 & 1 \end{bmatrix}.$$

\begin{inner_problem}
\item Write a matrix for a translation by $(a,b)$ followed by a rotation of $\theta$ around the origin.
\end{inner_problem}

The matrix is:

$$\begin{bmatrix} \cos\theta & -\sin\theta & 0 \\ \sin\theta & \cos\theta & 0 \\ 0 & 0 & 1 \end{bmatrix} \begin{bmatrix}1 & 0 & a \\ 0 & 1 & b \\ 0 & 0 & 1 \end{bmatrix} = \begin{bmatrix} \cos\theta & -\sin\theta & a\cos\theta - b\sin\theta \\ \sin\theta & \cos\theta & a\cos\theta + b\sin\theta \\ 0 & 0 & 1 \end{bmatrix}.$$

\begin{outer_problem}
\item Use matrix multiplication to find the image $(x',y')$ of a point $(x,y)$ rotated by $\theta$.
\end{outer_problem}

We represent $(x,y)$ as $\begin{bmatrix} x \\ y \end{bmatrix}$. With matrix multiplication, we get

$$\begin{bmatrix} \cos\theta & -\sin\theta \\ \sin\theta & \cos\theta \end{bmatrix} \begin{bmatrix} x \\ y \end{bmatrix} = \begin{bmatrix} x\cos\theta - y\sin\theta \\ x\sin\theta + y\cos\theta \end{bmatrix}.$$

Thus, $(x',y')=(x\cos\theta - y\sin\theta, x\sin\theta + y\cos\theta)$.

\begin{outer_problem}
\item
\end{outer_problem}

\begin{inner_problem}[start=1]
\item Given the parabola $x=t,y=t^2$, use matrix multiplication to rotate it by $45^\circ$.
\end{inner_problem}

Let the new axes be $x'$ and $y'$. Then we have

$$\begin{bmatrix} x' \\ y' \end{bmatrix} = \begin{bmatrix} \frac{\sqrt{2}}{2} & -\frac{\sqrt{2}}{2} \\ \frac{\sqrt{2}}{2} & \frac{\sqrt{2}}{2} \end{bmatrix} \begin{bmatrix} t \\ t^2 \end{bmatrix} = \begin{bmatrix} \frac{\sqrt{2}}{2}t - \frac{\sqrt{2}}{2}t^2 \\ \frac{\sqrt{2}}{2}t + \frac{\sqrt{2}}{2}t^2 \end{bmatrix}.$$

Thus, $x' = \frac{\sqrt{2}}{2}t - \frac{\sqrt{2}}{2}t^2$ and $y' = \frac{\sqrt{2}}{2}t + \frac{\sqrt{2}}{2}t^2$.

\begin{inner_problem}
\item Graph the new parametric equations on your calculator.
\end{inner_problem}

Here you go! We let $x=x'$ and $y=y'$ for graphing purposes, which rotates it.

\begin{figure}[h]
\begin{center}
\begin{asy}[width=0.5\textwidth]
import graph;

real x(real t) {return sqrt(2)/2 * t - sqrt(2)/2 * t*t;}
real y(real t) {return sqrt(2)/2 * t + sqrt(2)/2 * t * t;}

draw(graph(x,y,-3,3));

xaxis("$x$",BottomTop,LeftTicks);
yaxis("$y$",LeftRight,RightTicks(trailingzero));
\end{asy}
\end{center}
\caption{Rotated parabola.}
\end{figure}

Challenge: find the maximum $x$ value of this! Requires either some ingenuity or some calculus.

\begin{inner_problem}
\item Does it look like a rotation clockwise or counterclockwise? Why?
\end{inner_problem}

Looks like a rotation by $45^\circ$ counterclockwise, since that's what the matrix $\begin{bmatrix} \frac{\sqrt{2}}{2} & -\frac{\sqrt{2}}{2} \\ \frac{\sqrt{2}}{2} & \frac{\sqrt{2}}{2} \end{bmatrix}$ does!

\end{document}

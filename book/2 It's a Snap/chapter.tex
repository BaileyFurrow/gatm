\documentclass[../textbook.tex]{subfiles}

\begin{document}

\section{It's a Snap}

\newcommand\snap{\bullet} % Custom command for the "snap" operation

\begin{figure}[h]
	\begin{center}
		% Figure: side-by-side juxtaposition of all six elements of the three-post snap group, labeled with letters I through E
		\begin{minipage}[b]{\textwidth}
			\centering
			\begin{asy}[width=0.5\textwidth]
				size(250);

				for (int i = 0; i < 6; ++i) { // for every element...
					int[] to_draw = {standard_indices[i]};
					drawSnapElements((4 * i, 0), to_draw); // draw the snap element
					label("$" + standard_labels[i] + "$", (4 * i + 1, -2.5)); // label it with its letter
				}
			\end{asy}
		\end{minipage}
	\end{center}
	\vspace*{-2\baselineskip}
	\begin{center}
		\begin{minipage}[t]{\textwidth}
			\caption{The six possibilities for connections between two rows of three posts.}
			\label{fig:all_3_cols}
		\end{minipage}
	\end{center}

	\begin{center}
		\begin{minipage}[b]{0.45\textwidth}
			\centering
			% Figure: Two, chained snap group elements Y,X (they happen to be C and B) with Y on top of X. Y and X are labeled with arrows, there is a brace showing that the figure is 3 columns wide, and three white-filled dots indicate the middle row which is removed in the snap operation
			\begin{asy}[width=0.3\textwidth]
				size(0, 60);

				int[] elements = {2,5}; // C, then B

				drawSnapElements((0, 0), elements); // Draw them

				path top_brace = brace((-0.2, 0.6), (2.2, 0.6), .3);

				draw(top_brace, black+1bp); // Draw brace
				label("$3$", top_brace, N); // Label the brace "3"

				label("$\rightarrow Y$", (2.5, -0.75), E); // Top element is Y
				label("$\rightarrow X$", (2.5, -2.25), E); // Bottom element is X

				for (int i = 0; i < 3; ++i) // Color the middle dots white
					dot((i, -1.5), filltype=FillDraw(fillpen=white, drawpen=black));
			\end{asy}
		\end{minipage}
		\hfill
		\begin{minipage}[b]{0.45\textwidth}
			% Figure: Two chained snap group elements B,A are labeled with arrows, and are snapped to become E on the right. The arrow is thick and labeled "snap"
			\centering
			\begin{asy}[width=0.7\textwidth]
				size(0, 60);

				int[] starting_elements = {5, 1};

				drawSnapElements((0, 0), starting_elements); // Draw B, then A

				int[] new_elem = {3};

				drawSnapElements((5, 0), new_elem, 3, 1, 3); // Draw E, the result; note that yd is doubled
				label("$\stackrel{\mbox{\small snap!}}{\Longrightarrow}$", (3.5, -1.5));
				/* Looks like:
				snap!
				===>  */

				label("$B \leftarrow$", (-0.5, -0.75), W); // Label the elements as A and B
				label("$A \leftarrow$", (-0.5, -2.25), W);

				label("$E$", (7.5, -1.5), E); // Label the result as E on the right

				/* Phantom stuff to make it the same size as the figure on left. We draw a white brace and a white 3. Because I'm too lazy to have to align to the baseline. */
				path top_brace = brace((-0.2, 0.6), (2.2, 0.6), .3);

				draw(top_brace, white+1bp);
				label("$3$", top_brace, N, white);
			\end{asy}
		\end{minipage}
	\end{center}
	\vspace*{-2\baselineskip}
	\begin{center}
		\begin{minipage}[t]{0.45\textwidth}
			\caption{A grid with three strings.}
			\label{fig:n_rows_3_cols_ex}
		\end{minipage}
		\hfill
		\begin{minipage}[t]{0.45\textwidth}
			\caption{$A\snap B = E$.}
			\label{fig:snap_ex}
		\end{minipage}
	\end{center}
	\vspace*{-2\baselineskip}
\end{figure}

% Vocabulary: SNAP, GROUP, ELEMENT, SNAP GROUP, IDENTITY ELEMENT $A\snap B = E$.

\noindent We begin with a problem that ties together ideas from geometry, complex numbers, matrices, combinatorics, and group theory.
You have studied geometry, complex numbers, and combinatorics before, so you should have a basis to start this investigation, but the other two topics may be a bit unfamiliar at this point.

Figure~\ref{fig:all_3_cols} shows the six ways the posts in two rows, each row containing three posts, can be paired up. Convince yourself that these are the only six. Let's label them $I$, $A$, $B$, $C$, $D$, and $E$. For example, $I$ pairs every post in the top row with the post directly below it, while $A$ switches the pairings of the second and third posts.

Consider a grid of posts with three rows and three columns.
An elastic string is anchored to one post in the top row, looping around a post in the middle row, and finally descending to a post in the bottom row.
Two other strings are anchored and looped in the same way, with the condition that each post has exactly one string touching it.
An example is depicted in Figure~\ref{fig:n_rows_3_cols_ex}.

Notice that we can easily represent such a grid of posts with two of our six elements. You should have two stacked elements, $X$ and $Y$, with $Y$ on top of $X$, as shown in Figure~\ref{fig:n_rows_3_cols_ex}.
When you remove the middle posts (the posts indicated by $\circ$), the elastic string will \textbf{snap} to one of the six configurations we drew initially.
Let's call this operation ``snap,'' or $\snap$, so that $X\snap Y$ reads ``$X$ snap $Y$.''
Keep in mind that when evaluating the snap operation, the \textit{bottom} configuration $X$ goes first, and the \textit{top} configuration $Y$ goes last.
In Figure~\ref{fig:n_rows_3_cols_ex}, $X=B$ and $Y=C$, and $X\snap Y$ makes $E$, so $B\snap C = E$.
As another example, $A\snap B = E$, as shown in Figure~\ref{fig:snap_ex}.

Why do we put the bottom configuration first and the top configuration last? This choice is somewhat arbitrary, but there is a reason. Remember that when we compose functions, we write $(f \circ g)(x)=f(g(x))$. The right function, $g$, is evaluated first and used as an input to the left function, $f$. Similarly, when we write $X\snap Y$, the overall configuration (from top to bottom) first goes through $Y$, then through $X$. As we will see, these six elements often behave more like functions, rather than simply elements. Thus, it is natural to order them as if they were functions.

Some important terminology: these six configurations form a mathematical \textbf{group} under the operation $\snap$, and we say that each configuration is an \textbf{element} of our group.
More specifically, we will call this group the three-post \textbf{snap group}, or $S_3$. There are countless other mathematical groups, so we must be precise when we talk about a specific group. Note that this snap group is denoted $S_3$, not $S_6$, because the subscript $3$ is the number of posts in each row, \textit{not} the size of the group.
Groups are, unsurprisingly, the main objects studied in \textbf{group theory}.
Let's study this snap group and characterize its properties.

\begin{enumerate}
\item Fill out a $6\times 6$ table like the one in Figure~\ref{fig:sbstable}, showing the results of each of the $36$ possible snaps, where $X\snap Y$ is in $X$'s row and $Y$'s column. $A\snap B=E$ is done for you.
\item Which of the elements is the \textbf{identity element} $K$, such that $X\snap K = K\snap X = X$ for all $X$? \label{prob:group_definition_start}
\item Does every element have an inverse? In other words, can you get to the identity element from every element using only one snap?
\item \begin{enumerate}
\item Is the snap operation commutative (does $X \snap Y = Y \snap X$ for all $X,Y$)?
\item Is the snap operation associative (does $(X \snap Y) \snap Z = X \snap (Y\snap Z)$ for all $X,Y,Z$)?
\end{enumerate}
\item \begin{enumerate}
\item For any elements $X, Y$, is there always an element $Z$ so that $X\snap Z=Y$?
\item For (a), is $Z$ always unique?
\end{enumerate}
\newcounter{some_name}
\setcounter{some_name}{\value{enumi}}
\end{enumerate}% https://tex.stackexchange.com/questions/209092/how-do-i-make-every-column-the-same-width

\begin{figure}[h]
	\begin{center}
		\centering
		\begin{minipage}[b]{\textwidth}
			\centering
			\begin{tabular}{c|cccccc}
				\hline
				$\snap$ & $I$ & $A$ & $B$ & $C$ & $D$ & $E$ \\ \hline
				\rowcolor{light-gray}
				$I$    &   &   &   &   &   &   \\
				$A$    &   &   & $E$ &   &   &   \\
				\rowcolor{light-gray}
				$B$    &   &   &   &   &   &   \\
				$C$    &   &   &   &   &   &   \\
				\rowcolor{light-gray}
				$D$    &   &   &   &   &   &   \\
				$E$    &   &   &   &   &   &   \\ \hline
			\end{tabular}
			\vspace*{0.5\baselineskip}
		\end{minipage}
	\end{center}
	\vspace*{-2\baselineskip}
	\begin{center}
		\begin{minipage}[t]{\textwidth}
			\caption{Unfilled $3$-post snap group table.}
			\label{fig:sbstable}
		\end{minipage}
	\end{center}

	\begin{center}
		\begin{minipage}[b]{.45\textwidth}
			% Figure: E snap E snap E = I, so E has a period of 3. Shows three consecutive E elements, then a "snap" arrow and the I element.
			\centering
			\begin{asy}[width=0.4\textwidth]
			size(0, 60);

			int[] eggs = {3, 3, 3}; // 3 E elements

			drawSnapElements((0, 0), eggs); // Draw the Es

			int[] new_elem = {0};

			drawSnapElements((5, 0), new_elem, 3, 1, 4.5); // draw the identity element
			label("$\stackrel{\mbox{\small snap!}}{\Longrightarrow}$", (3.5, -2.25));
			\end{asy}
		\end{minipage}
		\begin{minipage}[b]{.45\textwidth}
			% Figure: Draw 8 elements from the 4-post snap group set, and arrange them in a 4 (wide) by 2 (high) rectangle
			\centering
			\begin{asy}[width=0.7\textwidth]
			size(0, 60);

			for (int i = 0; i < 24; i += 3) { // Draw elements 0, 3, ..., 24
			int[] elem = {i};
			drawSnapElements((5 / 3 * (i >= 12 ? (i - 12) : i), (i >= 12 ? 5 : 0)), elem, 4); // Expression used to make them into a 2x4 grid. Pretty sure I could have just used a modulo operator
			}
			\end{asy}
		\end{minipage}
	\end{center}
	\vspace*{-2\baselineskip}
	\begin{center}
		\begin{minipage}[t]{.45\textwidth}
			\caption{$E\snap E\snap E = I$; $E$ has period $3$.}
			\label{fig:eper3}
		\end{minipage}
		\begin{minipage}[t]{.45\textwidth}
			\caption{Some $4$-post group elements.}
			\label{fig:fpge}
		\end{minipage}
	\end{center}
	\vspace*{-2\baselineskip}
\end{figure}

\begin{enumerate}
\setcounter{enumi}{\value{some_name}}
\item If you constructed a $5\times 5$ table using only five of the snap elements, the table would not describe a group, because there would be entries in the table outside of those $5$. Therefore, a group must be \textbf{closed} under its operation. \textit{Some} subsets of our six elements, however, do happen to be closed. Write valid group tables using exactly one, two, and three elements from the snap group. These are known as \textbf{subgroups}.\label{prob:group_definition_end}
\item What do you guess is a good definition of a mathematical group? (Hint: consider your answers to Problems \ref{prob:group_definition_start}--\ref{prob:group_definition_end}.)
\item Notice that $E\snap E\snap E=I$ (see Figure~\ref{fig:eper3}). This means that $E$ has a \textbf{period} of $3$ when acting upon itself. Which elements have a period of
\begin{enumerate}
\begin{multicols}{3}
\item $1$?
\item $2$?
\item $3$?
\end{multicols}
\end{enumerate}
\item Answer the following with the one, two, and four-post snap groups $S_1$, $S_2$ and $S_4$. These are just the analogous groups for connections between one, two, and four posts. \begin{enumerate}
\item How many elements would there be?
\item Systematically draw and name them.
\item Make a group table of these elements. For four posts, instead of creating the massive table, give the number of elements that the table would have.
\item What is the relationship of your original table to this new table?
\end{enumerate}
\item Can you think of a shortcut to generate a snap group table without drawing every possible configuration?
\item \begin{enumerate}
\item How many elements would there be in the five-post snap group? \label{prob:five_post_snap_list_start}
\item How many entries would its table have?
\item What possible periods would its elements have? Make sure you include a period of six! \label{prob:five_post_snap_list_end}
\item Extend your answers for Problems \ref{prob:five_post_snap_list_start} through \ref{prob:five_post_snap_list_end} to $M$ posts per row.
\end{enumerate}
\item A \textbf{permutation} of a set of things is an order in which they can be arranged. What is the relationship between the set of permutations of $m$ things and the $m$-post snap group?
\end{enumerate}
\end{document}

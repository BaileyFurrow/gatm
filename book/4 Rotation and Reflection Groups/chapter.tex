\documentclass[../textbook.tex]{subfiles}

\begin{document}

\section{Rotation and Reflection Groups}

\newcounter{rrg_problem_i}

\begin{figure}[h]
	\begin{center}
		\begin{minipage}[b]{0.45\textwidth}
			\centering
			\begin{asy}[width=0.5\textwidth]
				size(0,80);

				pair offset = (0,0);
				string[] order = {"1", "2", "3"};

				// See main textbook.tex file for definition of drawTriangle
				drawTriangle(offset, 2, order);
			\end{asy}
		\end{minipage}
		\hfill
		\begin{minipage}[b]{0.45\textwidth}
			\centering
			\begin{asy}[width=0.7\textwidth]
				size(0,80);

				pair offset = (0,0);
				string[] order = {"", "", ""};

				drawTriangle(offset, 2, order, true);
			\end{asy}
		\end{minipage}
	\end{center}
	\vspace*{-2\baselineskip}
	\begin{center}
		\begin{minipage}[t]{0.45\textwidth}
			\caption{The paper triangle.}
			\label{fig:paper_triangle}
		\end{minipage}
		\hfill
		\begin{minipage}[t]{0.45\textwidth}
			\caption{Its axes of reflection.}
			\label{fig:triangle_reflections}
		\end{minipage}
	\end{center}
	\vspace*{-2\baselineskip}
\end{figure}

\noindent In the previous section, we started with the dihedral group of the equilateral triangle and discovered it had six elements: reflections about three different axes, rotations of $\pm 120^{\circ}$, and the identity transformation. We identified a subgroup consisting of the identity $I$ with two rotations $r$ and $r^2$, and three other subgroups of just the identity and a single reflection. The first subgroup---the one consisting of only rotations---is known as the \textbf{rotation group} of the equilateral triangle, or the \textbf{cyclic group} of order $3$, $C_3$.

\begin{enumerate}
\item Notice that the original dihedral group had twice as many elements as the rotation group. Why?
\item Make and justify a conjecture extending this observation to the dihedral groups of other shapes like rectangles, squares, and hexagons, as well as the symmetry group of the cube.
\item Let $r$ be a $180^{\circ}$ rotation, $x$ be a reflection over the $x$ axis, and $y$ be a reflection over the $y$ axis. Write a table for the dihedral group of the rectangle, recalling that the allowed isometries are reflections and rotations. How does this table differ from the dihedral group of the equilateral triangle? \label{prob:rectangle_group}
\item Write a table for the rotation group of the square, with $4$ elements and $16$ entries. Compare this table to Problem~\ref{prob:rectangle_group}.
\setcounter{rrg_problem_i}{\value{enumi}}
\end{enumerate}

\noindent We noticed that the rotation group for the equilateral triangle could be generated by just one of the elements, such as $r$---rotation by $120^{\circ}$ counterclockwise. Then $r^2$ is a rotation of $240^{\circ}$ counterclockwise, and $r^3=I$, the identity (see Figure~\ref{fig:successive_rotations}). Since we can generate the entire rotation group with a single element $r$, a natural question to ask is whether we can do the same with the dihedral group $D_3$. Clearly, we can't use the identity to do it, and a series of rotations always leaves us with a rotation, never a reflection. Also, a series of flips along one axis simply generates a two-member group with elements $I$, $f$ (see Figure~\ref{fig:flips}).

Let's try using two elements to generate our group, using the same definitions of $f$ and $r$ as in the previous section: a flip over the $A$ axis and rotation by $120^{\circ}$ counterclockwise, respectively. As we found, $fr$ is a flip over the $B$ axis and $rf$ is a flip over the $C$ axis. Consecutive powers of $r$ already got us the remaining elements, so ${r,f}$ generates the group.

We can also generate the group using two reflections, say $f$ and $f_B$ (flip over the $B$ axis, as shown in Figure~\ref{fig:triangle_reflections}). Notice that an even number of reflections always results in a rotation---even the identity element $I$ is just a rotation by $0$.\footnote{Any reflection group will include rotations, though they may be the identity.} We can think of this as the existence of a ``mirror world'' and its unmirrored counterpart, and each reflection takes us into or out of the mirror world.

\begin{figure}[h]
	\begin{center}
		\begin{minipage}[b]{0.45\textwidth}
			\centering
			\begin{asy}[width=\textwidth]
				size(0,60);

				drawTriangle((0, 0), 2, orders[4], false);
				label("$r$", (sqrt(3) / 2, -1.4), (0,0), basealign);
				drawTriangle((3, 0), 2, orders[5], false);
				label("$r^2$", (3 + sqrt(3) / 2, -1.4), (0,0), basealign);
				drawTriangle((6, 0), 2, orders[0], false);
				label("$I$", (6 + sqrt(3) / 2, -1.4), (0,0), basealign);

				label("$\stackrel{r}{\Longrightarrow}$", (2.4, 0.2));
				label("$\stackrel{r}{\Longrightarrow}$", (5.4, 0.2));
				label("$\stackrel{r}{\Longrightarrow}r$", (8.7, 0.2));
			\end{asy}
		\end{minipage}
		\hfill
		\begin{minipage}[b]{0.45\textwidth}
			\centering
			\begin{asy}[width=0.7\textwidth]
				size(0,60);

				drawTriangle((0, 0), 2, orders[1], false);
				label("$f$", (sqrt(3) / 2, -1.4), (0,0), basealign);
				drawTriangle((3, 0), 2, orders[0], false);
				label("$I$", (3 + sqrt(3) / 2, -1.4), (0,0), basealign);

				label("$\stackrel{f}{\Longrightarrow}$", (2.4, 0.2));
				label("$\stackrel{f}{\Longrightarrow}f$", (5.7, 0.1));
			\end{asy}
		\end{minipage}
	\end{center}
	\vspace*{-2\baselineskip}
	\begin{center}
		\begin{minipage}[t]{0.45\textwidth}
			\caption{${r}$ generates a three member group.}
			\label{fig:successive_rotations}
		\end{minipage}
		\hfill
		\begin{minipage}[t]{0.45\textwidth}
			\caption{${f}$ generates a two member group.}
			\label{fig:flips}
		\end{minipage}
	\end{center}
\end{figure}

\begin{figure}[h]
	\begin{center}
		\begin{minipage}[b]{\textwidth}
			\centering
			\begin{asy}[width=0.5\textwidth]
				import graph3;

				size(0,130);
				currentprojection = perspective(5, 5.8, 3.5);

				triple A = (0,0,0), B = (1,0,0), C = (1/2,sqrt(3)/2,0);
				triple D = (0,0,1), E = (1,0,1), F = (1/2,sqrt(3)/2,1);

				draw(A--B--C--cycle);
				draw(D--E--F--cycle);
				draw(A--D);
				draw(B--E);
				draw(C--F);

				label("$1$", (1/2,1,1));
				label("$2$", D, N);
				label("$3$", E, W);

				label("$1'$", (1/2,1,0));
				label("$2'$", A, S);
				label("$3'$", B, W);

				triple unitify3D(triple p) {
					return p / sqrt(p.x * p.x + p.y * p.y + p.z * p.z);
				}

				void draw3DExtLine(triple A, triple B, real extra = 0.2, string labelstr="", triple succ = (0,0,0)) {
					triple C = (A + extra * (A - B)), D = (B + extra * (B - A));
					draw(C--D, dashed);
					if (labelstr != "") label(labelstr, C, unitify3D(A - B) + succ);
				}

				draw3DExtLine((C+F)/2, (B+D)/2, 0.45, "$A$");
				draw3DExtLine((A+D)/2, (C+E)/2, 1.2, "$B$");
				draw3DExtLine((B+E)/2, (A+F)/2, 0.45, "$C$");

				// Drawing a curved (approximately circular) arrow, splinely interpolating between POINTS on the 3D ARQUE (arc)

				real[] x = {}; // cosine
				real[] y = {}; // stays constant
				real[] z = {}; // sine

				int arrow_pt_count = 10;
				real arrow_radius = 0.3;

				for (int i = 0; i < arrow_pt_count; ++i) {
					real angle = i / arrow_pt_count * 3.1416; // just pi because we want a 180 degree arrow
					x.push(arrow_radius * cos(angle) + 1/2);
					y.push(1.2);
					z.push(arrow_radius * sin(angle) + 1/2);
				}

				currentlight=(1,1,1);
				draw(graph(x,y,z,operator..),gray,Arrow3, currentlight);
			\end{asy}
		\end{minipage}
	\end{center}
	\vspace*{-2\baselineskip}
	\begin{center}
		\begin{minipage}[t]{\textwidth}
			\caption{Triangular prism's corresponding axes of rotation.}
			\label{fig:tri_prism_rot}
		\end{minipage}
	\end{center}
	\vspace*{-2\baselineskip}
\end{figure}

\noindent Moving to three dimensions, $D_3$ is isomorphic to the set of rotations of an equilateral triangular prism. The new rotation axes are coplanar with where the reflection axes used to be (see Figure~\ref{fig:tri_prism_rot}). Indeed, when you ``flipped'' your equilateral triangles, you were actually rotating a paper-thin triangular prism in the third dimension. Truly flipping the triangular prism using a rotation would require four spatial dimensions---something we cannot easily visualize.

You will next analyze the symmetries of a variety of objects under rotations or reflections. You will notice that the more symmetries an object has, the larger its symmetry group is. Indeed, group theory is the mathematics of symmetry \textit{par excellence}.

For each of the following groups, find the following:

\begin{enumerate}[label=(\alph*)]
\item The number of elements; this is known as the \textbf{order}. More formally known as \textbf{cardinality}
\item If order $< 10$, name the set of elements; otherwise, explain how you know the order
\item A smallest possible \textbf{generating set}; in other words, a list of elements which generate a group\footnote{There may be multiple generating sets of the same size.}
\item Whether the group is \textbf{commutative}; in other words, whether its operation $\cdot$ satisfies $X\cdot Y=Y\cdot X$ for all $X,Y$ %compare-books-disable
\end{enumerate}

\noindent If two problems have isomorphic groups, just write ``isomorphic to Problem N'' for the latter problem and move on.

\begin{enumerate}
\begin{multicols}{2}\raggedcolumns
\setcounter{enumi}{\value{rrg_problem_i}}
\item Rectangle under rotation
\item Rectangle under reflection
\item Square under rotation
\item Square under reflection
\item Square prism under rotation
\item Square prism under reflection
\item Regular pentagon under rotation
\item Regular pentagon under reflection
\item Regular pentagonal prism under rotation
\item Regular pentagonal prism under reflection
\item Regular pentagonal pyramid under rotation
\item Regular pentagonal pyramid under reflection
\item Regular tetrahedron (triangular pyramid) under rotation
\item Regular tetrahedron under reflection
\item Cube under rotation
\item Cube under reflection
\end{multicols}
\end{enumerate}

\end{document}

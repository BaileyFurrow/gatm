\documentclass[../gatm.tex]{subfiles}

\begin{document}

\newcounter{cg_problem_i} % for keeping track of which problem we're on

\section{Geometry of Complex Numbers}

\newcommand{\Arg} {\operatorname{Arg}}
\newcommand{\cis} {\operatorname{cis}}
\newcommand{\Real} {\operatorname{Re}}
\newcommand{\Imag} {\operatorname{Im}}

\textit{Thanks to Tristan Needham's \emph{Visual Complex Analysis} for many of the problems/examples and to Josh Zucker for most of the text.}

Last year, you became masters of the art of manipulating complex numbers. In this section, we will build on that background. Throughout the rest of the book, you can reinforce your skills with a dose of Vitamin $i$.

You should know at least two ways to think about the equation $x^3 = 1$. One way is to see that $x = 1$ is a solution, so $x - 1$ is a factor of $x^3-1$, and then proceed with synthetic or long division. The other way is to use \textbf{DeMoivre’s} beautiful theorem:

$$(r_1 (\cos \theta + i \sin \theta)) (r_2 (\cos \phi + i \sin \phi)) = (r_1r_2) (\cos(\theta + \phi) + i \sin(\theta + \phi)).$$

Like last year, we will often write $\cis \theta = \cos \theta + i \sin \theta$. Recall that $z=r\cis \theta$ is a complex number $r$ units away from the origin and making an angle of $\theta$ with the $x$ axis, taken counterclockwise. Let’s rewrite DeMoivre’s theorem using $\cis$:

$$(r_1 \cis \theta)(r_2 \cis \theta) = (r_1r_2)(\cis(\theta + \phi)).$$

In words: when you multiply complex numbers, the magnitudes are multiplied and the angles are added.\footnote{What else is added when you multiply? Exponents! In fact, $\cis \theta = e^{i\theta}$, but that’s another story.}

Through repeated application of this theorem, we know that $(r \cis \theta) ^ n = r^n \cis n*\theta$. If $x = r \cis \theta$, then $x^3 = r^3 \cis 3\theta$. Going back to our original $x^3 = 1$, since $1 = 1 \cis(2\pi k)$ for any integer $k$, we find that $r = 1$ and $3\theta = 2\pi k$. This yields three solutions: $1 \cis 0$, $1 \cis \frac{2\pi}{3}$, $1 \cis \frac{3\pi}{3}$. These correspond to $k = 0, 1, 2$; other values of $k$ produce coterminal angles, and are therefore duplicates.

You can bashfully prove DeMoivre’s theorem using the angle addition formulae for $\cos$ and $\sin$. But you can also understand it through pure geometry. Consider a complex number $z = a + bi$, being multiplied by $A = 4 + 3i$. $z$ forms an angle of $\theta$ with the x-axis, and $A$ forms an angle of $\phi$. In Figure 1, observe that $iz$ is perpendicular to $z$ for any $z$. In Figure 2 depicts the complex number $A$. Finally, in Figure 3, you see the multiplication carried out: $Az = (4+3i)z = 4z + 3iz$. These two components, $4z$ and $3iz$, are indicated.

\begin{asydef}


real distance(pair p1, pair p2) {
	return sqrt((p2.x - p1.x) * (p2.x - p1.x) + (p2.y - p1.y) * (p2.y - p1.y));
}

path mark_angle_label_precise(pair p1, pair p2, pair p3, string labelc = "", pair direction=(0,0), real r=-1) {
    if (r <= 0) {
    	r = min(distance(p1, p2), distance(p2, p3)) / 10;
	}
	
	path arc_a = arc(p2, point(p2--p1, r), point(p2--p3, r));
	label(labelc, midpoint(arc_a) + direction);
	
	return arc_a;
}

pair unitify(pair c) {
	real dis = sqrt(c.x * c.x + c.y * c.y);
	return (c.x / dis, c.y / dis);
}

path angle_marking(pair p1, pair p2, pair p3, string labelc = "", real label_dis, real r=-1) {
    return mark_angle_label_precise(p1, p2, p3, labelc, unitify((p1 + p3) / 2 - p2) * label_dis, r);
}

path right_angle(pair p1, pair p2, pair p3, real r = -1) {
	if (r <= 0) {
    	r = min(distance(p1, p2), distance(p2, p3)) / 10;
	}
	pair p1c = point(p2--p1, r);
	pair p3c = point(p2--p3, r);
	
	path r_angle = p3c -- (p1c + p3c - p2) -- p1c;
	return r_angle;
}

\end{asydef}

\begin{figure}
\begin{minipage}[b]{0.3\textwidth}
\begin{asy}[width=\textwidth]
real a = 3;
real b = 1;
pair z = (a,b);

draw((0,0)--(a,0)--z, dashed);
draw((0,0)--z);
dot(z);
dot((0,0));

label("$a$", (a/2, 0), S);
label("$b$", (a, b/2), E);

pair iz = (-b, a);

label("$a$", (-b, a/2), W);
label("$b$", (-b/2, 0), S);
draw((0,0)--(-b,0)--iz, dashed);
draw((0,0)--iz);

dot(iz);

draw((-2, 0) -- (4,0), Arrow);
draw((0, -2) -- (0,4), Arrow);

label("$z$", z, NE);
label("$iz$", iz, NW);
draw(angle_marking((a,0), (0,0), z, "$\theta$", 0.5));
draw(right_angle(z, (0,0), iz, 0.1));
\end{asy}
\end{minipage}
\begin{minipage}[b]{0.3\textwidth}
\begin{asy}[width=\textwidth]
real a = 4;
real b = 3;
pair z = (a,b);

draw((0,0)--(a,0)--z, dashed);
draw((0,0)--z);
dot(z);
dot((0,0));

draw(angle_marking((a,0), (0,0), z, "$\phi$", 0.3, 0.25));

label("$4$", (a/2, 0), S);
label("$3$", (a, b/2), E);

label("$A$", z, NE);

draw((-1, 0) -- (5,0), Arrow);
draw((0, -2) -- (0, 4), Arrow);
\end{asy}
\end{minipage}
\begin{minipage}[b]{0.4\textwidth}
\begin{asy}[width=\textwidth]
draw((-2, 0) -- (15,0), Arrow);
draw((0, -2) -- (0, 12), Arrow);

pair z = (3,1);
pair iz = (-1, 3);
pair A = (4,3);
pair O = (0,0);
pair z4 = 4 * z;
pair Az = z4 + 3 * iz;

for (int i = 0; i <= 4; ++i) {
	dot(i * z);
	if (i > 0) label("$z$", i * z - z / 2, (i == 1 || i == 2) ? (-.3, .9) / 2 : SE);
}

for (int i = 0; i <= 3; ++i) {
	dot(z4 + i * iz);
	if (i > 0) label("$iz$", z4 + i * iz - iz / 2, NE);
}

draw(O -- z4 -- Az, dashed);
draw(O -- Az);
label("$4z + 3iz = Az$", Az, NW);

draw(angle_marking(z4, O, Az, "$\phi$", 0.7, 0.9 / sqrt(10)));
draw(angle_marking((6,0), O, z4, "$\theta$", 0.7));

path z4brace = brace(z4 + (.3, -.9), O + (.3, -.9));
path iz4brace = brace(Az + (.9, .3), z4 + (.9, .3));

pen bracestyle = black+1.4bp;

draw(z4brace, bracestyle);
draw(iz4brace, bracestyle);

label("$4z$", midpoint(z4brace), SE);
label("$3iz$", midpoint(iz4brace), NE);
\end{asy}
\end{minipage}

\begin{minipage}[t]{0.2\textwidth}
\caption{$iz$ is perpendicular to $z$.}
\label{fig:izzperp}
\end{minipage}
\hspace{0.05\textwidth}
\begin{minipage}[t]{0.3\textwidth}
\caption{The complex number $A=4+3i$.}
\label{fig:lol}
\end{minipage}
\hspace{0.05\textwidth}
\begin{minipage}[t]{0.4\textwidth}
\caption{Breaking up $Az$ into its components, we can observe the geometry of complex multiplication.}
\end{minipage}
\end{figure}

Combining the observation in Figure 1 and knowledge from geometry, we know the triangles in Figure 2 and 3 are similar. Since the scaling is by a factor of $|z|$, multiplying $A$ by $z$ has the effect rotating $z$ by the angle of $A$, and multiplying it by the length of $A$.

Some notation: The angle $\theta$ of a complex number $z = a+bi$ is often called the argument, written as $\Arg z$. The real part of $z$ is written $\Real (z) = a$, and the imaginary part of $z$ is written $\Imag (z) = b$. Note that $\Imag (z)$ is a \textit{real} number, \textit{not} an  imaginary number. In particular, $\Imag (z) \neq bi$. Finally, the \textbf{complex conjugate} of $z$, where the imaginary part is negated, is written with a bar on top: $\overline{z} = a-bi$.

\begin{enumerate}
\item Explain why $iz$ is perpendicular to $z$, without using DeMoivre's theorem.
\item How does $\Arg \overline{z}$ relate to $\Arg z$? (Hint: symmetry!)
\item Compute $z\overline{z}$ and relate it to the $\cis$ form of $z$.
\item Explain, using a picture, why $\tan (\Arg z) = \frac{\Imag(z)}{\Real(z)}$.
\item Divide $\frac{a+bi}{c+di}$ by rationalizing the denominator.
\item Divide $\frac{r_1\cis \theta}{r_2\cis \phi}$ using DeMoivre's theorem.
\item Compare and contrast the methods of division in problems $5$ and $6$. Which is more convenient? Or does it depend on the circumstance?
\item \begin{enumerate}
\item If $z = r\cis \theta$, what is $\frac{1}{z}$?
\item Explain how this shows $\frac{1}{a+bi}=\frac{a-bi}{a^2+b^2}$, without having to rationalize the denominator. (Hint: use problems 3, 4, and 7.)
\end{enumerate}
\item Compute $(1+i)^{13}$; pencil, paper, and brains only. No calculators!
\item Compute $\frac{(1+i\sqrt{3})^3}{(1-i)^2}$ without a calculator.
\item Draw $\cis\left(\frac{\pi}{4}\right) + \cis\left(\frac{\pi}{2}\right)$. Use your picture to prove an expression for $\tan\left(\frac{3\pi}{8}\right)$. (Hint: add them as vectors.)
\item Solve $z^3 = 1$, and show that its solutions under the operation of multiplication form a group, isomorphic to the rotation group of the equilateral triangle. Write a group table!
\item \begin{enumerate}
\item Find multiplication groups of complex numbers which are isomorphic to the rotation groups for
\begin{multicols}{2}
\begin{enumerate}
\item a non-square rectangle, and
\item a regular hexagon.
\end{enumerate}
\end{multicols}
\item Make a table for each group.
\item Compare the regular hexagon's group to the dihedral group of the equilateral triangle, $D_3$. Consider: how are they the same? How are they different? Is the difference fundamental?
\end{enumerate}
\item Which of the following sets is a group under (i) addition and (ii) multiplication?
\begin{multicols}{3}
\begin{enumerate}
\item $\{0\}$
\item $\{1\}$
\item $\{0,1\}$
\item $\{-1,1\}$
\item $\{1, -1, i, -i\}$
\item $\{\text{naturals}\}$
\item $\{\text{integers}\}$
\item $\{\text{rationals}\}, \mathbb{Q}$
\item $\{\mathbb{Q}\text{ without zero}\}$
\item $\{\text{complex numbers}\}, \mathbb{C}$
\item $\{\mathbb{C}\text{ without zero}\}$
\end{enumerate}
\end{multicols}
\item For what values $a$ and $b$ does $\{a + b\sqrt{2}\}$ generate a group under (a) addition and (b) multiplication? Look for several answers.

\setcounter{cg_problem_i}{\value{enumi}}
\end{enumerate}

DeMoivre's theorem is the ``universal'' trig identity, in the sense that it can be used to calculate every other trig identity. For example, suppose you want an identity for $\cos 3\theta$. For convenience, let $c=\cos\theta$ and $s=\sin\theta$. Then we have

\begin{align*}
\cis 3\theta &= (\cis \theta)^3  & \qquad \text{[DeMoivre's Theorem]} \\
&= (c+is)^3 & \qquad \text{[Definition of } \cis \text{]} \\
&= c^3 + 3c^2s i - 3cs^2 - s^3 i & \qquad \text{[Binomial expansion]} \\
\cos 3\theta + i\sin 3\theta &= (c^3 - 3c^2) + i (3c^2 s - s^3) \qquad & \text{[Combining like terms]}\\
\end{align*}

Equating real parts on both sides, $\cos 3\theta = \cos^3\theta - 3\cos\theta\sin^2\theta$.

\begin{enumerate}
\setcounter{enumi}{\value{cg_problem_i}}
\item Prove that $(r_1\cis \theta)(r_2\cis \phi) = r_1r_2 \cis(\theta + \phi)$ using brute force and the angle-sum trig identities for $\cos$ and $\sin$. Do you prefer this method or the one on the previous page? Which method gives you a better understanding of why DeMoivre's works?
\item Find an identity for $\sin 3\theta$ as we have done for $\cos$. Most of the work is already done for you!
\item Your friend's textbook says $\cos 3\theta = 4\cos^3\theta - 3\cos \theta$, different from our identity. Who's right?

\setcounter{cg_problem_i}{\value{enumi}}
\end{enumerate}

Let’s apply complex numbers to a geometry problem: Prove that if we construct squares with centers $P$, $Q$, $R$, $S$ on the sides of any quadrilateral $ABCD$, as shown in Figure~\ref{fig:quad_square}, then (i) $\overline{PR} \perp \overline{QS}$ and (ii) $\overline{PR} \cong \overline{QS}$. In other words, segments joining centers of opposite squares are perpendicular and the same length.

\begin{asydef}
pair rotate(pair p, real degrees=0) {
	real rads = degrees * pi / 180;
	real c = cos(rads);
	real s = sin(rads);
	
	return (p.x*c-p.y*s,p.y*c+p.x*s);
}

path squareOn(pair A, pair B) {
	return A--B--(B+rotate(A-B,-90))--(A+rotate(B-A,90))--cycle;
}

pair A = (0,0);
pair B = (1,6);
pair C = (5.5,3);
pair D = (4,0);

\end{asydef}

\begin{figure}
\begin{minipage}{0.5\textwidth}
\begin{center}
\begin{asy}[width=\textwidth]
draw(A--B--C--D--cycle);

label("$0=A$", A, SW);
label("$B$", B, NNW);
label("$C$", C, NE+0.7*E);
label("$D$", D, 1.3 * SE + 0.6 * S);

draw(squareOn(A,B));
draw(squareOn(B,C));
draw(squareOn(C,D));
draw(squareOn(D,A));

draw((-6,0)--(9.5,0), Arrow);
draw((0,-5)--(0,11), Arrow);

pair a = (B-A)/2;
pair b = (C-B)/2;
pair c = (D-C)/2;
pair d = (A-D)/2;

pair P = a + rotate(a,90);
pair Q = B + b + rotate(b,90);
pair R = C + c + rotate(c,90);
pair SS = D + d + rotate(d,90);

dot(A);
dot(B);
dot(C);
dot(D);
dot(P);
dot(Q);
dot(R);
dot(SS);

draw(P--R);
draw(Q--SS);

label("$P$", P, W);
label("$R$", R, E);
label("$Q$", Q, N);
label("$S$", SS, S);

label("$m$", (1.5*P + R) / 2.5, S);
label("$n$", (1.5*Q + SS) / 2.5, W);

\end{asy}
\end{center}
\end{minipage}
\hfill
\begin{minipage}{0.28\textwidth}
\begin{center}
\begin{asy}[width=\textwidth]
label("$A$", A, SW);
label("$B$", B, NNW);
label("$C$", C, NE+0.7*E);
label("$D$", D, 1.3 * SE + 0.6 * S);

dot(A);
dot(B);
dot(C);
dot(D);

pair a = (B-A)/2;
pair b = (C-B)/2;
pair c = (D-C)/2;
pair d = (A-D)/2;

pair Aa = A+a;
pair Bb = B+b;
pair Cc = C+c;
pair Dd = D+d;

draw(A--Aa,Arrow);
draw(Aa--B,Arrow);
draw(B--Bb,Arrow);
draw(Bb--C,Arrow);
draw(C--Cc,Arrow);
draw(Cc--D,Arrow);
draw(D--Dd,Arrow);
draw(Dd--A,Arrow);

label("$a$",A+a/2,NW);
label("$a$",A+3*a/2,NW);
label("$b$",B+b/2,NE);
label("$b$",B+3*b/2,NE);
label("$c$",C+c/2,SE);
label("$c$",C+3*c/2,SE);
label("$d$",D+d/2,S);
label("$d$",D+3*d/2,S);

\end{asy}
\end{center}
\caption{$2(a+b+c+d)=0$.}
\label{fig:add_to_0}
\begin{center}
\begin{asy}[width=0.5\textwidth]

pair a = (B-A)/2;
draw(squareOn(A,B));

pair P = a + rotate(a,90);
dot(P);
label("$P$", P, W);

draw(A--a,Arrow);
draw(a--P,Arrow);
label("$a$",A+a/2,E);
label("$ia$",a+rotate(a,90)/2,N);

label("$A$", A, SW);
\end{asy}
\end{center}
\end{minipage}
\hfill
\begin{minipage}{0.5\textwidth}
\caption{The quadrilateral with four squares.}
\label{fig:quad_square}
\end{minipage}
\hfill
\begin{minipage}{0.28\textwidth}
\caption{$P=a+ia$.}
\label{fig:p_def_on_a}
\end{minipage}
\hfill
\end{figure}

We represent all points in the figure as complex numbers. For convenience, let $A=0$ be the origin. The edges of the quadrilateral can be thought of as vectors in the form of complex numbers, and are found using subtraction; for example, the edge from $A$ to $B$ is $B-A$. Similarly, the edge from $B$ to $C$ is $C-B$. Now, define complex numbers

$$a=\frac{B-A}{2},\, b=\frac{C-B}{2},\, c = \frac{D-C}{2},\, d = \frac{A-D}{2}.$$

$a$ is the vector halfway along $\overrightarrow{AB}$, $b$ is halfway along $\overrightarrow{BC}$, etc.; this is shown in Figure~\ref{fig:add_to_0}. We also have

$$a+b+c+d=\frac{B-A+C-B+D-C+A-D}{2}=\frac{0}{2}=0.$$
More geometrically, this is because $2(a+b+c+d)=2a+2b+2c+2d$ is the sum of the vectors $\overrightarrow{AB}, \overrightarrow{BC}, \overrightarrow{CD}, \overrightarrow{DA}$, which is just $\overrightarrow{AA}=\mathbf{0}$. This is shown in Figure~\ref{fig:add_to_0}.

$P,Q,R,S$ are also complex numbers. Let $m=R-P$ and $n = Q-S$, as the lengths of our two segments $\overline{PR}$ and $\overline{QS}$. To prove that they are perpendicular, recall that $z$ is perpendicular to $iz$ for any complex $z\neq 0$, so we just need to prove that $n=\pm im$.

We now need to relate $P,Q,R,S$ back to $a,b,c,d$. It is easy to see that $P = a+ia$, remembering that $a$ is the vector halfway along $\overrightarrow{AB}$. $a$ takes you from the origin $A$ to the midpoint of $\overline{AB}$, then $ia$ takes you to $P$. This shown in Figure~\ref{fig:p_def_on_a}. We can extend this logic to the other points, of course.

\newcounter{cg_problem_ii}

\begin{enumerate}
\setcounter{enumi}{\value{cg_problem_i}}
\item Now you can finish the rest of the proof.
\begin{enumerate}
\item Draw $a,b,c,d,m,n$ approximately for the quadrilateral on the previous page.
\item Why does showing $n=\pm im$ prove the segments are (i) perpendicular and (ii) the same length?
\item Explain why $Q=2a+b+ib$.
\item Find formulae for $R$ and $S$ in terms of $c$ and $d$.
\item Find $m$ and $n$ in terms of $a$, $b$, $c$, and $d$.
\item Check that $n-im=0$, using the fact that $a+b+c+d=0$.
\end{enumerate}
\item In the previous problem, we drew squares outside a quadrilateral and connected their centers. Conjecture what happens if we draw equilateral triangles outside a triangle and connect their centers. Prove your conjecture using complex numbers.
\item The hard way to find an identity for $\tan 3\theta$ is to divide the identity for $\sin$ and $\cos$ that we already found. Try this. Make sure your answer is in terms of $\tan$ only!
\item The easier way to get an identity for $\tan 3\theta$ starts with setting $z = 1 + i\tan\theta$.
\begin{enumerate}
\item Why is $\Arg z = \theta$?
\item Why is $\tan 3\theta = \frac{\Imag(z^3)}{\Real(z^3)}$?
\item Use (b) to find an identity for $\tan 3\theta$.
\end{enumerate}
\item Find multiplication groups of complex numbers isomorphic to rotation groups for the
\begin{enumerate}
\begin{multicols}{2}
\item regular octagon, and
\item regular pentagon.
\end{multicols}
\end{enumerate}
\item Make tables for
\begin{enumerate}
\begin{multicols}{2}
\item the rotation group of the regular octagon, and
\item the dihedral group of the square.
\end{multicols}
\end{enumerate}
Is the difference between them fundamental?

\item Which of the following tables defines a group? Why or why not?
\begin{multicols}{2}
\begin{enumerate}
\item \begin{tabular}{c|c|c|c|c|c|}
\$ & $I$ & $A$ & $B$ & $C$ & $D$ \\ \hline
$I$ & $I$ & $A$ & $B$ & $C$ & $D$ \\ \hline
$A$ & $A$ & $C$ & $D$ & $B$ & $I$ \\ \hline
$B$ & $B$ & $I$ & $C$ & $D$ & $A$ \\ \hline
$C$ & $C$ & $D$ & $A$ & $I$ & $B$ \\ \hline
$D$ & $D$ & $B$ & $I$ & $A$ & $C$ \\ \hline
\end{tabular}
\item \begin{tabular}{c|c|c|c|c|c|}
\# & $I$ & $A$ & $B$ & $C$ & $D$ \\ \hline
$I$ & $I$ & $A$ & $B$ & $C$ & $D$ \\ \hline
$A$ & $A$ & $B$ & $C$ & $D$ & $I$ \\ \hline
$B$ & $B$ & $C$ & $D$ & $I$ & $A$ \\ \hline
$C$ & $C$ & $D$ & $I$ & $A$ & $B$ \\ \hline
$D$ & $D$ & $I$ & $A$ & $B$ & $C$ \\ \hline
\end{tabular}
\end{enumerate}
\end{multicols}
\item Which subsets of the complex numbers are groups under multiplication? After you list a few of each type, generalize.
\item Prove with a diagram that if $|z|=1$, then $\Imag\left(\frac{z}{(z+1)^2}\right)=0$.
\item Prove geometrically that if $|z|=1$, then $|1-z|=\left|2\sin\left(\frac{\Arg z}{2}\right)\right|$.
\item \begin{enumerate}
\item Prove that if $(z-1)^{10}=z^{10}$, then $\Real (z) = \frac{1}{2}$. (Hint: if two numbers are equal, they have the same magnitude.)
\item How many solutions does this equation have?
\end{enumerate}
\item I claim that $e^{i\theta}=\cos\theta + i\sin\theta = \cis\theta$, for $\theta$ in radians.

\begin{multicols}{3}
\begin{enumerate}
\item Find $e^{-it}$. 
\item Find $\frac{e^{i\theta} + e^{-i\theta}}{2}$.
\item Find $\frac{e^{i\theta} - e^{-i\theta}}{2i}$.
\setcounter{cg_problem_ii}{\value{enumii}}
\end{enumerate}
\end{multicols}
Use your new, complex definitions for $\cos$ and $\sin$ to find:%
\begin{multicols}{2}
\begin{enumerate}
\setcounter{enumii}{\value{cg_problem_ii}}
\item $\cos^2\theta + \sin^2\theta$
\item $\tan\theta$
\item $\cos 2\theta$
\item $\sin 2\theta$
\item What kind of group is generated by $\{ e^{i\theta}, e^{-i\theta}\}$ under the operation of multiplication if $\theta$ is an integer? A rational multiple of $\pi$?
\end{enumerate}
\end{multicols}
\item You've used the quadratic equation throughout high school, but there's also a cubic equation that finds the roots of any cubic. Let's derive it, starting with the cubic $x^3+bx^2+cx+d=0$.
\begin{enumerate}
\item Make the substitution $x=y-\frac{b}{3}$. Combine like terms to create an equation of the form $y^3-3py-2q=0$, with $p,q$ in terms of $b,c$, and $d$.
\item Rearrange this equation as $y^3=3py+2q$.
\item Make the substitution $y=s+t$ into (b), and prove that $y$ is a solution of the cubic in part (a) if $st=p$ and $s^3+t^3=2q$.
\item Eliminate $t$ between these two equations to get a quadratic in $s^3$.
\item Solve this quadratic to find $s^3$. By symmetry, what is $t^3$?
\item Find a formula for $y$ in terms of $p$ and $q$. What about a formula for $x$?
\item What if we started with $ax^3+bx^2+cx+d=0$, with a coefficient in front of the $x^3$ term as well? Can you come up with a formula for $x$?
\end{enumerate}
\item Starting with the same cubic as in problem 31b.
\begin{enumerate}
\item Let $c=\cos\theta$. Remember that $\cos 3\theta=4c^3-3c$, as we proved. Substitute $y=2c\sqrt{p}$ into $y^3=3py+2q$ to obtain $4c^3-3c=\frac{q}{p^{3/2}}$.
\item Provided that $q^2\leq p^3$, show that $y=2\sqrt{p}\cos\left(\frac{1}{3}(\theta+2\pi n)\right)$, where $n$ is an integer. Why does this yield all three solutions?
\item Explain how you would find $\theta$ from $p$ and $q$, and how we would use what we have found to solve an arbitrary cubic $ax^3+bx^2+cx+d=0$.
\end{enumerate}
\end{enumerate}


\end{document}
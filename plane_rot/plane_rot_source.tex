\documentclass[../gatm.tex]{subfiles}

\begin{document}

\section{Rotations of the Plane}
\setcounter{problem_i}{0}

You've learned about matrix multiplication and about complex numbers. You may have guessed at the relationship between them, particularly now that we've spent some time seeing how $2\times 2$ matrices, certain $3\times 3$ matrices, and complex numbers all relate to the geometry of the plane. We will now make that connection explicit and relate it to some ideas that you worked with last year, such as rotation of axes.

Recall that a matrix $M$ acts on a column vector $v$ by the multiplication $Mv$, not $vM$; a complex number $z$ acts on the point $(x,y)$ by multiplication $z(x+yi)$ or $(x+yi)z$, since complex multiplication is commutative. We'll call $v$ or $(x,y)$ the preimage and $Mv$ or $z(x+yi)$ the image.

Some of the following problems are really trivial, so don't be alarmed if your answer takes only a few seconds. Some of them are fairly difficult and will take a bit of thought. Some of them are fairly tedious and will take some lengthy algebra but not much thought.

\newcommand{\Mat}{\operatorname{M}}
\newcommand{\cis}{\operatorname{cis}}

\begin{enumerate}
\item \label{prob:pr_start}\begin{enumerate}
\item Which matrix changes nothing, so that the image is the same as the preimage?
\item Which complex number changes nothing?
\end{enumerate}
\item \begin{enumerate}
\item Which matrix doubles the length of every vector but leaves angles unchanged?
\item Which complex number corresponds to the same transformation?
\end{enumerate}
\item Based on your answers to the previous problems, which matrix corresponds to the real number $r$? Let's call this $\Mat (r)$ for short.
\item Explain why $\Mat (u)+\Mat(v)=\Mat(u+v).$
\item Under a $90^\circ$ counterclockwise rotation, what is the image of (a) $(1,0)$ and (b) $(0,1)$?
\item \label{prob:pr_end}\begin{enumerate}
\item Which matrix corresponds to a $90^\circ$ rotation?
\item Which complex number corresponds to the same rotation?
\end{enumerate}
\item Based on your answers to Problems~\ref{prob:pr_start} to~\ref{prob:pr_end}, what matrix corresponds to the complex number $x+yi$? Let's extend our function $\Mat$ and call this $\Mat (x+yi)$ for short.
\item Check that $\Mat (a+bi)+\Mat (c+di)=\Mat((a+bi)+(c+di))$. That is, prove that $\Mat$ has the same addition rules as complex numbers.
\item Check that $\Mat (a+bi)\Mat (c+di)=\Mat((a+bi)(c+di))$. That is, prove that $\Mat$ has the same multiplication rules as complex numbers.
\item Recall that multiplying by $\cis\theta$ rotates a complex number by $\theta$ radians.
\begin{enumerate}
\item Find $\Mat (\cis\theta)$.
\item To prove that this matrix really does rotate by $\theta$:
\begin{enumerate}
\item Check that the image and preimage have the same length;
\item Check that the angle of the image with the $x$ axis is $\theta$ more than the preimage.
\end{enumerate}
\end{enumerate}
\item \begin{enumerate}
\item Find $\Mat (r\cis\theta)$.
\item To prove that this matrix really does rotate by $\theta$ and stretch by $r$:
\begin{enumerate}
\item Check that the length of the image is $r$ times the length of the preimage;
\item Check that the angle of the image with the $x$ axis is $\theta$ more than the preimage. (Hint: You may want to use the previous problem, or the tangent addition formulas.)
\end{enumerate}
\end{enumerate}
\setcounter{problem_i}{\value{enumi}}
\end{enumerate}

We've seen that there is a matrix for every complex number. These matrices have the same addition and multiplication rules as complex numbers. Furthermore, these matrices transform the plane in the same way as complex multiplication: a stretch by a factor of $r$ and a rotation by $\theta$. There are many matrices, however, that don't correspond to complex numbers.

\begin{enumerate}
\setcounter{enumi}{\value{problem_i}}
\item\begin{enumerate}
\item What matrix reflects over the $x$ axis, taking $(x,y)\to (x,-y)$?
\item What is the complex number operation equivalent to this transformation?
\item Is there a complex number multiplication equivalent to this transformation? Justify your answer.
\end{enumerate}
\item\begin{enumerate}
\item What matrix reflects through the origin, taking $(x,y)\to (-x,-y)$?
\item What is the complex number operation equivalent to this transformation?
\item Is there a complex number multiplication equivalent to this transformation? Justify your answer.
\end{enumerate}
\item \begin{enumerate}
\item Which of the $16$ matrices on page~\pageref{prob:map_plane_sixteen_matrices}, for Problem~\ref{prob:map_plane_sixteen_matrices}, have corresponding complex numbers? Which do not?
\item How can you tell algebraically?
\item How can you tell geometrically?
\end{enumerate}
\item Make multiplication tables with the set of matrices which correspond to the elements of the rotation group for the square (a $4\times 4$ table) and the equilateral triangle (a $3\times 3$ table).
\item \begin{enumerate}
\item Write a matrix for a rotation of $\theta$ around the origin followed by a translation by $(a,b)$.
\item Write a matrix for a translation by $(a,b)$ followed by a rotation of $\theta$ around the origin.
\end{enumerate}
\setcounter{problem_i}{\value{enumi}}
\end{enumerate}

Now that we know how to rotate with matrices or with complex numbers, we can revisit the topic of rotation of axes that you studied toward the end of last year.

\begin{enumerate}
\setcounter{enumi}{\value{problem_i}}
\item Use matrix multiplication to find the image $(x',y')$ of a point $(x,y)$ rotated by $\theta$.
\item \begin{enumerate}
\item Given the parabola $x=t,y=t^2$, use matrix multiplication to rotate it by $45^\circ$.
\item Graph the new parametric equations on your calculator.
\item Does it look like a rotation clockwise or counterclockwise? Why?
\end{enumerate}
\end{enumerate}
\end{document}
